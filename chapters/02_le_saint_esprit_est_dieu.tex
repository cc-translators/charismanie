\chapter{Le Esprit-Saint est Dieu}

\lettrine[lines=3]{P}{uisque} le but de ce livre est de vous amener dans une relation pleine, personnelle et sainement biblique avec Dieu l'Esprit-Saint, nous devons tout d'abord montrer que l'Esprit-Saint est une des trois personnes du Dieu Unique \NdT{le terme anglais \og Godhead \fg{} désigne à la fois la divinité et la tête de la Trinité dans le contexte chrétien. Il sera ici traduit par \og Dieu Unique \fg{}.}.

L'Église a accepté à travers son histoire qu'il y a un Dieu unique qui existe en trois personnes : le Père, le Fils et l'Esprit-Saint. Dans les Écritures, nous les trouvons travaillant ensemble en harmonie complète pour la rédemption de l'homme. Paul confessait à Timothée que le Dieu Unique était un grand mystère ; c'est un effort mental inutile pour nous que d'essayer de le comprendre complètement.

Beaucoup de groupes sectaires (comme les Témoins de Jéhovah) utilisent ce gouffre entre le fini et l'infini pour attaquer la trinité de Dieu en niant la déité de Jésus Christ et en faisant passer l'Esprit-Saint pour une essence. D'autres groupes nient l'existence du Père et de l'Esprit-Saint et disent que Jésus seul est Dieu. Une des caractéristiques communes à toute secte est le déni de la déité de Jésus Christ et de la personne de l'Esprit-Saint.

\section{Le Dieu trinitaire}

Parce que c'est un des domaines ou l'ennemi attaque constamment, nous devons affirmer non seulement le fait de la déité de l'Esprit-Saint, mais également pourquoi nous croyons dans sa déité. Le mot \og trinité \fg{} n'est pas présent dans la Bible, mais c'est un terme pratique que les théologiens utilisent pour décrire les trois personnes du Dieu unique. Le terme \og tri-unité \fg{} décrirait peut-être plus correctement Dieu. Il n'est pas $1 + 1 + 1 = 3$, mais $1 \times 1 \times 1 = 1$.

Dans \bibleverse{Gen}(1:1), nous lisons : \og Au commencement Dieu \fg{}. Le mot hébreu traduit par \og Dieu \fg{} est Elohim, qui est le pluriel de El (Dieu au singulier). En hébreu, il y a des cas singulier, double et pluriel. \og Dieu \fg{} au singulier est El, au double Elah, et au pluriel Elohim. Il est incontestable que le mot \og Elohim \fg{} suggère au moins une tri-unité de Dieu.

En continuant dans \bibleverse{Gen}(1:2), nous lisons : \og l'Esprit de Dieu planait au-dessus des eaux. \fg{} L'Esprit-Saint est la première personne du Dieu Unique à être identifiée séparément dans la Bible. Dans \bibleverse{Gen}(1:26), nous lisons : \og Puis Dieu dit : Faisons l'homme à notre image selon notre ressemblance. \fg{} Il n'a pas dit : \og Je vais faire l'homme à mon image. \fg{} En d'autres termes, les trois personnes du Dieu Unique parlaient conjointement.

\section{Les attributs de l'Esprit-Saint}

Pour établir que l'Esprit-Saint est Dieu, nous allons tout d'abord montrer que des attributs qui peuvent seulement désigner Dieu sont attribués à l'Esprit-Saint. Un des attributs divins est la nature éternelle de Dieu. Il a toujours existé. Dans \bibleverse{Heb}(9:14), nous lisons que Christ, par un esprit éternel, s'est offert lui-même sans tâche à Dieu. Si l'Esprit est éternel, et que c'est un attribut qui ne peut convenir qu'au divin, il en résulte que l'Esprit est Dieu. Notez également comment les trois personnes de la Trinité sont liées dans le verset.

Un autre attribut de Dieu est son omniscience. Dieu sait toute chose, comme Jacques le dit dans \bibleverse{Ac}(15:17-18) : \og [...], Dit le Seigneur, qui fait ces choses connues de toute éternité. \fg{} \NdT{La citation anglaise est issue de la traduction King James qui diffère notoirement des autres traductions sur ce verset. La version de la KJV pourrait se traduire par : \og Le Seigneur connaît toutes Ses œuvres depuis le commencement du monde \fg{}}. Cet attribut est également attribué à l'Esprit-Saint. Dans \bibleverse{ICo}(2:10-11), nous lisons : \og À nous, Dieu nous l'a révélé par l'Esprit. Car l'Esprit sonde tout, même les profondeurs de Dieu. Qui donc, parmi les hommes, sait ce qui concerne l'homme, si ce n'est l'esprit de l'homme qui est en lui ? De même, personne ne connaît ce qui concerne Dieu, si ce n'est l'Esprit de Dieu. \fg{}

Un autre attribut du divin est l'omniprésence. Dieu existe partout dans l'univers en même temps. Dans le \bibleverse{Ps}(139:7), David demande : \og Où irais-je loin de ton Esprit, Et où fuirais-je loin de ta face ? \fg{} Dieu existe dans les cieux, en enfer, et les plus profonds abîmes de la mer. L'Esprit est avec moi où je suis, et Il est en même temps avec vous, où que vous soyez en train de lire ce livre maintenant. Dieu est omnipotent. C'est un mot qui exprime qu'Il est tout-puissant. Quand Sarah rit en apprenant qu'elle va enfanter un fils dans son grand âge, l'ange du Seigneur lui demande : \og Y a-t-il rien qui soit étonnant de la part de l'Eternel ? \fg{} (\bibleverse{Gen}(18:14)). Jésus a dit : \og Tout est possible à Dieu. \fg{} (\bibleverse{Mc}(10:27)). Dans \bibleverse{Lc}(1:37), Il a dit : \og Rien n'est impossible à Dieu. \fg{} Quand Marie a demandé à l'ange comment elle, une vierge, pourrait porter un enfant, l'ange lui a répondu : \og Le Saint-Esprit viendra sur toi, et la puissance du Très-Haut te couvrira de son ombre. \fg{} (\bibleverse{Lc}(1:35)). Ici, l'Esprit-Saint et la puissance du Très-Haut sont utilisés comme synonymes.

