\chapter{Le Saint-Esprit est Dieu}

\lettrine[lines=3]{P}{uisque} le but de ce livre est de vous amener dans une relation pleine, personnelle et sainement biblique avec Dieu le Saint-Esprit, nous devons tout d'abord montrer que le Saint-Esprit est une des trois personnes du Dieu Unique \NdT{le terme anglais \og Godhead \fg{} désigne à la fois la divinité et la tête de la Trinité dans le contexte chrétien. Il sera ici traduit par \og Dieu Unique \fg{}}.

L'Église a accepté à travers son histoire qu'il y a un Dieu unique qui existe en trois personnes : le Père, le Fils et le Saint-Esprit. Dans les Écritures, nous les trouvons travaillant ensemble en harmonie complète pour la rédemption de l'homme. Paul confessait à Timothée que le Dieu Unique était un grand mystère ; c'est un effort mental inutile pour nous que d'essayer de le comprendre complètement.

Beaucoup de groupes sectaires (comme les Témoins de Jéhovah) utilisent ce gouffre entre le fini et l'infini pour attaquer la trinité de Dieu en niant la déité de Jésus Christ et en faisant passer l'Esprit Saint pour une essence. D'autres groupes nient l'existence du Père et du Saint-Esprit et disent que Jésus seul est Dieu. Une des caractéristiques communes à toute secte est le déni de la déité de Jésus Christ et de la personne de l'Esprit Saint.

\section{Le Dieu trinitaire}

Parce que c'est un des domaines ou l'ennemi attaque constamment, nous devons affirmer non seulement le fait de la déité de l'Esprit Saint, mais également pourquoi nous croyons dans sa déité. Le mot \og trinité \fg{} n'est pas présent dans la Bible, mais c'est un terme pratique que les théologiens utilisent pour décrire les trois personnes du Dieu unique. Le terme \og tri-unité \fg{} décrirait peut-être plus correctement Dieu. Il n'est pas $1 + 1 + 1 = 3$, mais $1 * 1 * 1 = 1$.

Dans Genèse 1:1, nous lisons : \og Au commencement Dieu \fg{}. Le mot hébreu traduit par \og Dieu \fg{} est Elohim, qui est le pluriel de El (Dieu au singulier). En hébreu, il y a des cas singulier, double et pluriel. \og Dieu \fg{} au singulier est El, au double Elah, et au pluriel Elohim. Il est incontestable que le mot \og Elohim \fg{} suggère au moins une tri-unité de Dieu.

En continuant dans Genèse 1:2, nous lisons : \og l'Esprit de Dieu planait au-dessus des eaux. \fg{} Le Saint-Esprit est la première personne du Dieu Unique à être identifiée séparément dans la Bible. Dans Genèse 1:26, nous lisons : \og Puis Dieu dit : Faisons l'homme à notre image selon notre ressemblance. \fg{} Il n'a pas dit : \og Je vais faire l'homme à mon image. \fg{} En d'autres termes, les trois personnes du Dieu Unique parlaient conjointement.
