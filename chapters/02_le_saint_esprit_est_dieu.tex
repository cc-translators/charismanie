\chapter{L'Esprit-Saint est Dieu}

\chlettrine{P}{uisque} le but de ce livre est de vous amener
 dans une relation pleine, personnelle et sainement biblique avec
 Dieu l'Esprit-Saint, nous devons tout d'abord montrer que l'Esprit-Saint
 est une des trois personnes du Dieu Unique
 \NdT{le terme anglais \og Godhead \fg{} désigne à la fois la divinité
 et la tête de la Trinité dans le contexte chrétien.
 Il sera ici traduit par \og Dieu Unique \fg{}.}.

L'Église a accepté à travers son histoire qu'il y a un Dieu unique
 qui existe en trois personnes\space: le Père, le Fils et le Saint-Esprit.
 Dans les Écritures, nous les trouvons travaillant ensemble en harmonie
 complète pour la rédemption de l'homme.
 Paul confessait à Timothée que le Dieu Unique était un grand mystère ;
 c'est un effort mental inutile pour nous que d'essayer de le comprendre
 complètement.

Beaucoup de groupes sectaires (comme les Témoins de Jéhovah) utilisent
 ce gouffre entre le fini et l'infini pour attaquer la trinité de Dieu
 en niant la déité de Jésus-Christ et en faisant passer l'Esprit-Saint
 pour une essence.
 D'autres groupes nient l'existence du Père et de l'Esprit-Saint et disent
 que Jésus seul est Dieu.
 Une des caractéristiques communes à toute secte est le déni de la déité
 de Jésus-Christ et de la personne de l'Esprit-Saint.

\section*{Le Dieu trinitaire}

Parce que c'est un des domaines ou l'ennemi attaque constamment, nous devons
 affirmer non seulement la déité de l'Esprit-Saint comme un fait,
 mais également pourquoi nous croyons dans sa déité.
 Le mot \og trinité \fg{} n'est pas présent dans la Bible, mais c'est un
 terme pratique que les théologiens utilisent pour décrire les trois personnes
 du Dieu unique.
 Le terme \og tri-unité \fg{} décrirait peut-être plus correctement Dieu.
 Il n'est pas $1 + 1 + 1 = 3$, mais $1 \times 1 \times 1 = 1$.

Dans \bibleverse{Gen}(1:1), nous lisons\space:
 \og Au commencement Dieu \fg{}.
 Le mot hébreu traduit par \og Dieu \fg{} est \emph{Elohim}, qui est
 le pluriel de \emph{El} (Dieu au singulier).
 En hébreu, il y a des cas singulier, double et pluriel.
 \og Dieu \fg{} au singulier est \emph{El}, au double \emph{Elah},
 et au pluriel \emph{Elohim}.
 Il est incontestable que le mot \emph{Elohim} suggère au moins
 une tri-unité de Dieu.

En continuant dans \bibleverse{Gen}(1:2), nous lisons\space:
 \og l'Esprit de Dieu planait au-dessus des eaux. \fg{}
 L'Esprit-Saint est la première personne du Dieu Unique à être identifiée
 séparément dans la Bible.
 Dans \bibleverse{Gen}(1:26), nous lisons\space:
 \og Puis Dieu dit\space: Faisons l'homme à notre image selon notre ressemblance. \fg{}
 Il n'a pas dit\space: \og Je vais faire l'homme à mon image. \fg{}
 En d'autres termes, les trois personnes du Dieu Unique parlaient conjointement.

\section*{Les attributs de l'Esprit-Saint}

Pour établir que l'Esprit-Saint est Dieu, nous allons tout d'abord montrer
 que des attributs qui peuvent seulement désigner Dieu sont attribués à
 l'Esprit-Saint.
 Un des attributs divins est la nature éternelle de Dieu.
 Il a toujours existé. Dans \bibleverse{Heb}(9:14), nous lisons que Christ,
 par un esprit éternel, s'est offert lui-même sans tâche à Dieu.
 Si l'Esprit est éternel, et que c'est un attribut qui ne peut convenir
 qu'au divin, il en résulte que l'Esprit est Dieu.
 Notez également comment les trois personnes de la Trinité sont liées
 dans le verset.

Un autre attribut de Dieu est son omniscience. Dieu sait toute chose, comme
 Jacques le dit dans \bibleverse{Acts}(15:17-18)\space:
 \og [...], Dit le Seigneur, qui fait ces choses connues de toute éternité. \fg{}
 \NdT{La citation anglaise est issue de la traduction King James qui diffère
 notoirement des autres traductions sur ce verset. La version \og King James
 Française \fg{} traduit \bibleverse{Acts}(15:18) par\space:
 \og Toutes ses œuvres sont connues à Dieu depuis le commencement du monde. \fg{}}.
 Cet attribut est également attribué à
 l'Esprit-Saint. Dans \bibleverse{ICo}(2:10-11), nous lisons\space: \og À nous, Dieu
 nous l'a révélé par l'Esprit. Car l'Esprit sonde tout, même les profondeurs de
 Dieu. Qui donc, parmi les hommes, sait ce qui concerne l'homme, si ce n'est
 l'esprit de l'homme qui est en lui ? De même, personne ne connaît ce qui
 concerne Dieu, si ce n'est l'Esprit de Dieu. \fg{}

Un autre attribut du divin est l'omniprésence. Dieu existe partout dans
 l'univers en même temps. Dans le \bibleverse{Ps}(139:7), David demande\space:
 \og Où irais-je loin de ton Esprit, Et où fuirais-je loin de ta face ? \fg{}
 Dieu existe dans les cieux, en enfer, et dans les plus profonds abîmes de
 la mer. L'Esprit est avec moi où je suis, et Il est en même temps avec vous,
 où que vous soyez en train de lire ce livre maintenant. Dieu est omnipotent.
 C'est un mot qui exprime qu'Il est tout-puissant. Quand Sarah rit en apprenant
 qu'elle va enfanter un fils dans son grand âge, l'ange du Seigneur lui demande\space:
 \og Y a-t-il rien qui soit étonnant de la part de l'Eternel ? \fg{}
 (\bibleverse{Gen}(18:14)). Jésus a dit\space: \og Tout est possible à Dieu. \fg{}
 (\bibleverse{Mc}(10:27)). Dans \bibleverse{Lc}(1:37), Il a dit\space:
 \og Rien n'est impossible à Dieu. \fg{} Quand Marie a demandé à l'ange comment
 elle, une vierge, pourrait porter un enfant, l'ange lui a répondu\space:
 \og Le Saint-Esprit viendra sur toi, et la puissance du Très-Haut te couvrira
 de son ombre. \fg{} (\bibleverse{Lc}(1:35)). Ici, l'Esprit-Saint et
 la puissance du Très-Haut sont utilisés comme synonymes.

\section*{Les œuvres de l'Esprit}


Non seulement les attributs divins s'appliquent à l'Esprit-Saint, mais
 également les œuvres divines.
 Une des œuvres divines est celle de la création.
 La Trinité entière était active dans la création.
 Dans \bibleverse{Gen}(1:1), nous lisons\space:
 \og Au commencement Dieu créa le ciel et la terre. \fg{}
 Dans \bibleverse{Jn}(1:1-3), nous lisons\space:
 \og Au commencement était la Parole, et la Parole était avec Dieu,
 et la Parole était Dieu. Elle était au commencement avec Dieu.
 Tout a été fait par elle, et rien de ce qui a été fait n'a été fait
 sans elle. \fg{}
 L'Esprit était également une force active dans la création.
 Dans \bibleverse{Gen}(1:2), l'Esprit est décrit comme planant au-dessus
 des eaux. L'Esprit était en communication avec le Père et le Fils quand
 Dieu a dit\space: \og Faisons l'homme à notre image \fg{} (\bibleverse{Gen}(1:26)).
 Dans \bibleverse{Ps}(104:30), nous lisons\space:
 \og Tu envoies ton souffle\space: ils sont créés \fg{}
 \NdT{le mot hébreu \emph{ruah} signifie souffle ou vent.
 C'est ce même mot qui est traduit par \og Esprit \fg{} dans
 \bibleverse{Gen}(1:1) et par \og souffle \fg{} dans \bibleverse{Ps}(104:30).}.

Une autre œuvre de Dieu est celle de donner la vie. Nous reconnaissons que Dieu
 est celui qui donne et soutiens la vie.
 Dans \bibleverse{IICo}(3:6), alors que Paul se référait à la lettre de la loi,
 il dit\space: \og La lettre tue, mais l'Esprit fait vivre. \fg{}
 Dans \bibleverse{Jn}(6:63), Jésus dit \og C'est l'Esprit qui vivifie.\fg{}

La Bible a été écrite sous l'inspiration de l'Esprit-Saint, pourtant nous
 désignons la Bible par la Parole de Dieu.
 \bibleverse{IIPet}(1:21) nous dit\space:
 \og Car ce n'est nullement par une volonté humaine qu'une prophétie a jamais
 été présentée, mais c'est poussés par le Saint-Esprit que des hommes ont parlé
 de la part de Dieu. \fg{}
 Dans \bibleverse{IITim}(3:16), Paul déclare\space:
 \og Toute Écriture est inspirée de Dieu. \fg{}
 Pierre dit que les auteurs étaient poussés par l'Esprit-Saint et Paul qu'ils
 étaient inspirés par Dieu. L'Esprit est donc reconnu comme Dieu.

C'est pourquoi beaucoup d'Écritures dans l'Ancien Testament qui déclarent que
 le Seigneur a parlé sont attribués à l'Esprit-Saint lorsqu'elles sont citées
 dans le Nouveau Testament.
 Dans \bibleverse{Is}(6:8-9), le prophète a dit\space: \og J'entendis la voix du
 Seigneur, disant\space: Qui enverrai-je, et qui marchera pour nous? Je répondis\space:
 Me voici, envoie-moi. Il dit (alors)\space: Va, tu diras à ce peuple\space:
 Écoutez toujours, mais ne comprenez rien ! Regardez toujours, mais n'en
 apprenez rien ! \fg{}
 Lorsque Paul a cité ce passage dans \bibleverse{Acts}(28:25-26), il a dit\space:
 \og C'est avec raison que le Saint-Esprit, parlant à vos pères par le
 prophète Esaïe, a dit\space: Va vers ce peuple, et dis\space: Vous entendrez bien et
 vous ne comprendrez point ; vous regarderez bien et vous ne verrez point. \fg{}
 Esaïe a dit que le Seigneur avait parlé ; Paul a dit que l'Esprit-Saint
 avait parlé.
 Ils peuvent tous deux avoir raison si l'Esprit-Saint et le Seigneur sont un.


\section*{La Trinité travaille ensemble}

Dans \bibleverse{Acts}(5:1-11), nous avons un témoignage intéressant de
 discipline dans l'Église primitive alors que Dieu essayait de préserver
 sa pureté.
 Motivés par l'amour, beaucoup de chrétiens ont tenté d'établir une
 communauté chrétienne en vendant tous leurs biens et en en faisant
 bénéficier les apôtres, afin que les chrétiens puissent avoir tout en
 commun. Un couple en particulier, Ananias et Saphira, ont vendu leur
 propriété mais ont ensemble décidé de garder une partie du prix pour
 eux-mêmes. Lorsqu'Ananias a apporté sa part à Pierre, Pierre lui a
 demandé\space: \og Pourquoi Satan a-t-il rempli ton coeur, au point de mentir
 à l'Esprit-Saint, et de retenir une partie du prix du champ ? Lorsque tu
 l'avais, ne demeurait-il pas à toi ? Et, après la vente le prix n'était-il
 pas à ta disposition ?
 Comment as-tu mis en ton cœur une pareille action ?
 Ce n'est pas à des hommes que tu as menti, mais à Dieu \fg{}
 (\bibleverse{Acts}(5:3-4)).
 Pierre a dit que Satan avait rempli le cœur d'Ananias pour mentir
 à l'Esprit-Saint, et a déclaré ensuite qu'il avait menti à Dieu, unifiant
 ainsi l'Esprit-Saint et Dieu.

À travers le Nouveau Testament, nous voyons la Trinité travaillant ensemble
 ou par paire. Lorsque Jésus a envoyé les disciples pour enseigner toutes
 les nations (\bibleverse{Mt}(28:19-20)), Il leur a dit de baptiser au nom
 du Père, du Fils et du Saint-Esprit. Ces trois noms distinguent les trois
 personnes du Dieu Unique.

Dans \bibleverse{IICo}(13:14), dans sa bénédiction apostolique, Paul a dit\space:
 \og Que la grâce du Seigneur Jésus-Christ, l'amour de Dieu et la communion
 du Saint-Esprit soient avec vous tous ! \fg{}
 Ici encore, les trois personnes du Dieu Unique sont liées entre elles.

Dans \bibleverse{ICo}(12:4-6), Paul dit\space:
 \og Il y a diversité de dons, mais le même Esprit ; diversité de services,
 mais le même Seigneur ; diversité d'opérations, mais le même Dieu
 qui opère tout en tous. \fg{}
 Au verset 4, il se réfère à l'Esprit, au verset 5 au Seigneur (Jésus)
 et au verset 6 à Dieu (le Père).
 Ainsi, malgré la diversité possible des dons et de leurs opérations
 et ministères, il y a une unité car Dieu est derrière tout.


\section*{Accès par l'Esprit}

À ce point, il est possible que vous vous demandiez\space:
 \og En fait, quelle différence cela fait-il si l'Esprit est Dieu
 ou simplement une essence de Dieu? \fg{}
 Parce que l'Esprit fait partie du Dieu Unique, il est bon de le louer,
 et nous avons raison de chanter\space:
 \og Louez le Père, le Fils et le Saint-Esprit \fg{}.
 Dieu a demandé que nous ayons une relation avec lui par l'intermédiaire
 de l'Esprit. C'est dans le domaine de l'Esprit que l'homme peut toucher Dieu.
 C'est mon esprit amené en union avec le Saint-Esprit. Jésus a dit\space:
 \og Dieu est esprit, et il faut que ceux qui l'adorent, l'adorent en esprit
 et en vérité \fg{} (\bibleverse{Jn}(4:24)).
 Paul a dit également\space: \og L'Esprit lui-même rend témoignage à notre esprit
 que nous sommes enfants de Dieu \fg{} (\bibleverse{Rom}(8:16)).
 Si je dois être en communion avec Dieu, je dois reconnaître l'Esprit-Saint
 et réaliser qu'il est celui qui rend cette relation possible.

L'homme n'a jamais eu d'accès direct au Père ; c'est un mensonge courant parmi
 les personnes qui oublient la sainteté extraordinaire de Dieu.
 Lorsque Dieu s'est manifesté sur la montagne sacrée au peuple juif
 (\bibleverse{Ex}(19:)), Il leur a fait poser des frontières autour de
 la montagne afin qu'ils ne s'approchent pas trop près de la manifestation
 de Dieu, ce qui leur aurait valu d'être mis à mort.
 Lorsque les gens ont vu de loin la démonstration extraordinaire de Dieu,
 ils ont dit à Moïse\space:
 \og Parles-nous et nous entendrons, mais ne laisse pas Dieu nous parler
 ou nous mourrons. \fg{}

Le voile dans le tabernacle démontrait la séparation qui doit exister entre
 le Dieu Saint et un peuple non saint.
 Ce voile pouvait seulement être traversé après une purification élaborée
 et des sacrifices par le grand-prêtre, et ce uniquement un jour par an,
 et par un seul homme, le grand-prêtre.

Jésus a dit\space: \og Nul ne vient au Père si ce n'est par moi \fg{}
 (\bibleverse{Jn}(14:6)).
 Jésus a dit aux juifs qu'ils ne connaissaient vraiment pas le Père.
 Il leur a également dit que Moïse témoignerait contre eux.
 Ils ne suivent pas le chemin indiqué vers Dieu, qui a été donné
 à Moïse par Dieu, mais ils cherchent aujourd'hui à s'approcher de Dieu
 sur la base de leurs bonnes œuvres sans sacrifices.
 Le péché a toujours été la barrière entre l'homme et Dieu, et à moins
 que quelque chose soit fait au sujet du péché de l'homme,
 il n'est pas possible de s'approcher de Dieu.
 Dans \bibleverse{Is}(59:1-2), nous lisons\space:
 \og Non, la main de l'Éternel n'est pas devenue trop courte pour sauver,
 ni son oreille trop dure pour entendre.
 Mais ce sont vos fautes qui mettaient une séparation entre vous et votre Dieu ;
 ce sont vos péchés qui vous cachaient (sa) face et l'empêchaient
 de vous écouter. \fg{}
 Jésus a donné un chemin pour nous purifier de nos péchés,
 rendant ainsi possible de s'approcher de Dieu.
 Par la foi en Jésus-Christ, mon esprit est rendu vivant, et je peux ainsi
 être uni à l'Esprit de Dieu. Ainsi, Dieu et l'homme sont unis par l'Esprit.

