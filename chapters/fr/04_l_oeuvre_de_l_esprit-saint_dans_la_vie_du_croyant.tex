\chapter{L'œuvre de l'Esprit-Saint dans la vie du croyant}

\chlettrine{Q}{uel} est le travail de l'Esprit attendu dans la vie du croyant ?
 Comme nous l'avons déjà noté dans \bibleverse{Jn}(14:), Son nom,
 \og consolateur \fg{}, indique Sa venue à nos côtés pour nous aider.
 Je ne trouve pas que ma démarche de chrétien soit facile.
 Je trouve que ma chair lutte avec moi tout le temps.
 Tout comme Pierre, j'ai moi aussi souvent découvert que l'esprit
 est bien disposé, mais que la chair est faible.
 Je comprend ce dont Paul parlait dans \bibleverse{Ga}(5:) lorsqu'il
 mentionnait la bataille entre la chair et l'esprit.
 Si Dieu a de l'aide pour moi, je suis prêt à la recevoir;
 je veux toute l'aide que je peux avoir !
 Je ne veux jamais mettre des limites à ce que Dieu veut me donner,
 ou à ce qu'Il veut faire dans ma vie. Je ne veux pas être coupable,
 comme les israélites dans le désert,
 de limiter Dieu (\bibleverse{Ps}(78:41)).
 De la même manière, je ne cherche pas l'expérience
 pour l'expérience en soit; je désire seulement l'œuvre véritable
 de l'Esprit-Saint, mais je la veux complètement.

\section*{Faire confiance à l'Esprit}

Dans le discours de Jésus à Ses disciples, à partir du chapitre 14
 de l'évangile de Jean, Il cherche à les préparer à Son départ.
 Il parle beaucoup du fait qu'Il va les quitter et retourner auprès du Père.
 Il leur parle également beaucoup des réserves que le Père et Lui
 ont faites pour eux par la puissance de l'Esprit-Saint.
 Il serait là pour les aider. Tout comme ils avaient appris à faire confiance
 à Jésus pour toute situation d'urgence qui pourrait arriver,
 ils doivent maintenant apprendre à faire confiance à l'Esprit-Saint.
 Il sera désormais leur soutien.

Jésus a passé Ses trois années avec Ses disciples à leur enseigner
 la vérité sur Dieu. Maintenant, leur enseignant les quitte pour retourner
 auprès du Père, mais les étudiants ne resteront pas seuls;
 ce soutien, l'Esprit-Saint, va maintenant leur enseigner toutes choses
 et leur rappeler toutes les choses que Jésus leur a dites
 (\bibleverse{Jn}(14:26)).

Peut-être avez-vous eu l'expérience de partager l'évangile avec quelqu'un,
 et il ou elle vous a posé une question qui vous a immédiatement bloqué;
 mais alors que vous avez commencé à répondre, les Écritures ont commencé
 à vous venir à l'esprit, et vous étiez content et satisfait de la réponse
 que vous avez donné à cette personne. Ceci est le travail de rappel de l'Esprit.

L'Esprit-Saint nous aide à comprendre les choses de Dieu.
 Bien souvent, j'ai été frustré dans ma tentative d'expliquer
 une vérité spirituelle à un non croyant.
 Cela paraît si clair et évident, et pourtant il ou elle ne semble pas
 pouvoir le comprendre.
 Si vous êtes dans ce type de situations au sujet de l'Esprit de Dieu,
 l'homme naturel \og ne reçoit pas les choses de l'Esprit de Dieu,
 car elles sont une folie pour lui, et il ne peut les connaître, 
 parce que c'est spirituellement qu'on en juge \fg{}
 (\bibleverse{ICo}(2:14)).


\section*{L'esprit mort et l'esprit vivant}

Lorsque j'étais à l'université, j'ai eu un professeur de sociologie
 qui croyait au dualisme de l'homme, et je croyais au trialisme de l'homme.
 Bien souvent, nous avons exprimé nos vues différentes l'un à l'autre.
 J'étais frustré qu'il ne puisse pas faire la distinction entre l'âme
 et l'esprit de l'homme, mais croyait qu'ils étaient synonymes.
 Un jour, alors que je quittais la classe frustré
 après une nouvelle discussion dans laquelle il paraissait
 avoir un aveuglement délibéré, ce fut comme si l'Esprit-Saint
 m'avait mis \bibleverse{ICo}(2:14) à l'esprit.
 Alors, j'ai réalisé qu'en tant que personne non régénérée,
 son esprit était mort, donc je lui parlais de mystères qu'il ne pouvait
 pas connaître.
 Il ne savait pas et ne pouvait pas savoir quoi que ce soit sur l'esprit
 de l'homme à moins d'être né de l'Esprit-Saint.
 Dans \bibleverse{ICo}(2:15), Paul dit~:
 \og L'homme spirituel, au contraire, juge de tout,
 et il n'est lui-même jugé par personne. \fg{}

Quiconque vit seulement au niveau de la conscience du corps
 vit au niveau animal de l'existence.
 Son esprit est dirigé et dominé par les besoins de son corps;
 il ne comprend pas les choses de l'Esprit, car son propre esprit est mort.
 Il n'est pas étonnant qu'il cherche à se rapprocher lui-même
 du reigne animal, car il vit comme un animal,
 une conscience dominée par le corps.
 Lorsqu'une personne est née de nouveau par l'Esprit,
 son propre esprit vient à la vie et, ainsi jointe à Dieu par l'Esprit,
 il est encouragé à travers la Parole à vivre une vie dominée par l'Esprit.
 Lorsqu'il le fait, il commence à avoir une conscience dominée par l'Esprit


\section*{Laisser l'Esprit diriger}

Alors que nous vivons selon l'Esprit, nos formes de pensée sont différentes
 car nous pensons maintenant à Dieu et comment nous pourrions
 Lui plaire et Le servir.
 Les tendances de l'Esprit, c'est la vie et la paix (\bibleverse{Rom}(8:6)).
 Quelle aide fantastique l'Esprit-Saint est pour nous alors qu'Il nous enseigne
 les choses de Dieu et nous aide à les comprendre !
 La Bible semble prendre vie avec signification et excitation
 alors qu'une Écriture après l'autre paraît presque jaillir
 hors de la page pour nous enseigner.

Dans \bibleverse{Jn}(16:13), Jésus a promis à Ses disciples
 que lorsque l'Esprit de vérité viendrait,
 Il les guiderait à travers toute vérité,
 et leur montrerait les choses à venir.
 Il est si nécessaire d'être guidé par l'Esprit-Saint
 à travers toute vérité.
 Jésus a mis en garde contre les faux prophètes qui seraient des loups,
 et pourtant paraîtraient habillés comme des agneaux (\bibleverse{Matt}(7:15)).
 Il y a des hommes qui viennent avec le troupeau de Dieu
 et paraissent en faire partie, mais leur principale motivation
 est de s'attaquer au troupeau.
 Ils apportent avec eux des hérésies de perdition
 et cherchent à conduire des hommes à leur suite.
 Dans \bibleverse{Acts}(20:29-30),
 Paul a mis en garde les anciens de l'église d'Éphèse~:
 \og Je sais que parmi vous, après mon départ,
 s'introduiront des loups redoutables qui n'épargneront pas le troupeau,
 et que du milieu de vous se lèveront des hommes
 qui prononceront des paroles perverses,
 pour entraîner les disciples après eux. \fg{}

Pierre a mis en garde dans \bibleverse{IIPet}(2:1-3)~:
 \og Il y a eu de faux prophètes parmi le peuple;
 de même il y a parmi vous de faux docteurs
 qui introduiront insidieusement des hérésies de perdition et qui,
 reniant le Maître qui les a rachetés,
 attireront sur eux une perdition soudaine.
 Beaucoup les suivront dans leurs dérèglements et, à cause d'eux,
 la voie de la vérité sera calomniée.
 Par cupidité, ils vous exploiteront au moyen de paroles trompeuses,
 mais depuis longtemps leur condamnation est en marche
 et leur perdition n'est pas en sommeil. \fg{}


\section*{Détecter les faux prophètes}

Remarquez une des caractéristiques d'un faux prophète~:
 \og vous utiliser au moyen de paroles trompeuses. \fg{}
 Je reçois régulièrement des lettres dactylographiées
 venant d'évangélistes de renom qui correspondent parfaitement
 à la description faite par Pierre.
 Ces lettres disent des choses comme~:
 \og Benny, Dieu t'as mis sur mon cœur ce matin, et j'ai prié pour toi.
 Je ne peux pas t'enlever de mon esprit, Benny.
 Est-ce que tout va bien? As-tu un besoin pour lequel je peux prier?
 Réponds-moi immédiatement, car je t'aime, Benny, et je veux t'aider!
 Il se trouve également que mon ministère traverse
 une de ses pires crises financières.
 Nous allons devoir fermer certaines de nos grandes œuvres pour Dieu
 à moins que tu puisses nous aider immédiatement.
 Si tu n'as pas 50 dollars à m'envoyer, peut-être peux-tu les emprunter
 ailleurs et m'aider à garder l'œuvre de foi en Dieu à continuer.
 Plante tes graines de foi aujourd'hui.
 Dieu t'aidera à rembourser l'emprunt que tu prends.
 Ton partenaire dans la foi. \fg{}

Mon nom n'est pas Benny, mais il semble que c'est ainsi que j'ai été noté
 dans leurs listes de diffusion.
 Ce type de lettres malhonnête cherchent seulement à exploiter les gens,
 et de par l'autorité de la Parole de Dieu,
 je n'hésite pas à désigner leurs auteurs de faux prophètes.

Il s'agit de charismanie dans une de ses formes les plus évidentes
 et elle est pratiquée par la plupart des évangélistes charismatiques,
 en particulier ceux qui mettent en avant la guérison divine.
 Je suis toujours émerveillé qu'ils puissent avoir une telle foi
 dans ma guérison et une si petite foi pour leur propres besoins financiers.

Il est beau de voir comment l'Esprit vous préviendra
 lorsque quelqu'un commence à diverger dans sa doctrine.
 Bien souvent, il ne vous est pas possible de pointer l'erreur immédiatement,
 mais vous savez que quelque chose ne va pas vraiment.
 L'Esprit a été donné au croyant pour le guider en tout vérité.


\section*{Apprendre les choses à venir}

L'Esprit-Saint nous montre également des choses à venir.
 Lorsque Daniel cherchait une plus grande compréhension des temps de la fin
 et des choses qu'il avait écrites, il lui fut ordonné de
 \og [tenir] secrètes ces paroles et [de sceller] le livre
 jusqu'au temps de la fin.
 Beaucoup alors le liront, et la connaissance augmentera. \fg{}
 (\bibleverse{Daniel}(12:4)).
 Alors que Daniel persistait à demander, à nouveau le Seigneur lui dit~:
 \og Va, Daniel, car ces paroles seront secrètes
 et scellées jusqu'au temps de la fin.
 Beaucoup seront purifiés, blanchis et épurés;
 les méchants feront le mal et aucun des méchants ne comprendra,
 mais ceux qui auront de l'intelligence comprendront. \fg{}
 (\bibleverse{Daniel}(12:9-10)).

C'est par l'aide de l'Esprit-Saint qu'une compréhension claire
 du retour de Jésus-Christ a été donnée à l'Église.
 L'Esprit-Saint a montré à l'apôtre Paul certaines des choses
 qui devaient arriver dans sa vie alors qu'il disait aux anciens d'Éphèse
 dans \bibleverse{Acts}(20:22-23)~:
 \og Et maintenant voici que lié par l'Esprit, je vais à Jérusalem,
 sans savoir ce qui m'y arrivera; seulement, de ville en ville,
 le Saint-Esprit atteste et me dit que des liens
 et des tribulations m'attendent. \fg{}
 Plus tard, alors que Paul continuait son voyage vers Jérusalem,
 le prophète Agabus prit la ceinture de Paul, s'attacha avec, et dit~:
 \og Voici ce que déclare le Saint-Esprit~:
 L'homme à qui cette ceinture appartient, les Juifs le lieront
 de cette manière à Jérusalem et le livreront
 entre les mains des païens. \fg{} (\bibleverse{Acts}(21:11)).
 Voici un exemple classique de l'Esprit-Saint montrant à Paul
 les choses qui devaient arriver dans sa vie.

Un autre exemple dans la vie de Paul de ce travail de l'Esprit
 se trouve dans \bibleverse{Acts}(27:21-24)~:
 \og On n'avait pas mangé depuis longtemps.
 Alors Paul, debout au milieu des hommes, leur dit~:
 Vous auriez dû m'obéir et ne pas repartir de Crète;
 vous auriez évité ce péril et ce dommage.
 Maintenant je vous exhorte à prendre courage;
 car aucun de vous ne perdra la vie, seul le navire sera perdu.
 Un ange du Dieu à qui j'appartiens et rends un culte,
 s'est approché de moi cette nuit et m'a dit~:
 Sois sans crainte, Paul; il faut que tu comparaisses devant César,
 et voici que Dieu t'accorde la grâce de tous ceux qui naviguent avec toi. \fg{}

\section*{La main puissante de Dieu}

Dans un petit groupe de prière, nous avons décidé de prier
 l'un pour l'autre.
 La personne pour qui nous devions prier s'asseyait sur une chaise
 au centre du groupe. Lorsque ce fut mon tour d'être le sujet de prière,
 quelqu'un a prononcé un mot de prophétie de la part de l'Esprit-Saint
 déclarant que la main de bénédiction de Dieu allait venir
 sur mon ministère de façon importante, que les gens viendraient écouter
 la Parole tellement nombreux qu'il n'y aurait plus de place dans l'église
 pour les contenir.
 La prophétie continuait en déclarant que je recevrais un nouveau nom
 qui signifiait berger, car le Seigneur alors faire de moi un berger
 pour de nombreux troupeaux.

Jusqu'à ce moment, j'avais lutté dans le ministère pendant près de 17 ans
 avec un succès tellement limité que j'envisageais de quitter le ministère
 pour faire un autre type de travail.
 L'église dont j'étais alors le pasteur rassemblait environ 100 personnes
 malgré tous nous efforts pour augmenter sa taille en donnant des hamburgers
 gratuits à tous ceux qui amèneraient un ami à l'école du dimanche.
 Alors que ces mots étaient prononcés, j'étais dans mon cœur comme l'homme
 sur qui le roi se reposait et qui, lorsqu'il entendit la promesse d'Élisée
 que des provisions abondantes viendraient sur les habitants affamés
 de Samarie, dit~:
 \og Même si le Seigneur envoyait du grain en perçant des trous dans
 la voûte du ciel, ce que tu viens de dire pourrait-il se réaliser ? \fg{}

Heureusement, Dieu fut miséricordieux envers moi,
 car mon destin ne devait pas être le même que celui de cet homme~;
 j'ai vu l'accomplissement de la prophétie et j'ai également pu y participer
 alors que nous voyons les locaux de l'église, fortement aggrandis,
 se remplir au-delà de leur capacité, non pas une fois, mais trois fois
 chaque dimanche matin, et alors que nous exerçons un ministère
 par cassettes audios et vidéos auprès de centaines de groupes
 d'étude biblique à travers le monde.

\section*{La puissance pour conquérir}

L'œuvre de l'Esprit-Saint dans votre vie si vous êtes un croyant
 est de vous donner la puissance d'être un témoin pour Jésus-Christ,
 de vous donner la puissance d'être tout ce que Dieu veut que vous soyez.
 Une des choses les plus frustrantes dans le monde est d'essayer
 de vivre la vie chrétienne par l'énergie de la chair.
 La Bible parle de la frustration dans \bibleverse{Rom}(7:),
 où Paul parle de comment, lorsqu'il essaya de garder la loi de Dieu
 et de faire le bien, il découvrit que \og moi qui veux faire le bien,
 je suis seulement capable de faire le mal. Et le bien que je veux faire,
 je ne le fais pas. Et ce que je ne veux pas, je le fais.
 Malheureux que je suis ! \fg{}
 Paul décrit dans \bibleverse{Gal}(5:) comment la chair lutte avec l'esprit
 et l'esprit contre la chair, et comment ils sont opposés l'un à l'autre.
 Jésus a dit à Pierre~:
 \og Restez éveillés et priez pour ne pas tomber dans la tentation.
 L'être humain est plein de bonne volonté, mais [la chair] est faible. \fg{}
 (\bibleverse{Matt}(26:41)).
 À cause de la faiblesse de notre chair, nous ne pouvons pas vivre
 le type de vie que le Seigneur voudrait que nous suivions
 et que nous-mêmes voudrions vivre à la face du monde.

Dieu désire que votre vie soit une vraie représentation de Lui dans ce monde.
 Dieu veut que le monde voie Jésus-Christ en vous.
 Il veut que vos actions et vos réactions Le reflètent.
 Il veut que vous soyez Son témoin, en Le représentant.
 Mais si vous tentez d'être Son témoin, si vous essayez de réagir comme Christ,
 vous verrez combien il est difficile et frustrant de le faire
 \ocadr en fait, combien c'est impossible à cause de la faiblesse de la chair.

\section*{Le témoin parfait}

Beaucoup de chrétiens se trouvent dans cette frustration de savoir
 ce qui est juste et de désirer ce qui est juste, mais de ne pas faire
 ce qui est juste.
 La Bible dit de Jésus-Christ qu'Il était le témoin véritable et fidèle ;
 Il était un témoin du Père.
 Si vous voulez savoir à quoi ressemble Dieu, regardez simplement
 Jésus-Christ, car Il est le Témoin véritable et fidèle.
 Lorsque Philippe s'est écrié~:
 \og Seigneur, montre-nous le Père et nous serons satisfaits \fg{},
 Jésus répliqua~:
 \og Il y a si longtemps que je suis avec vous et tu ne me connais pas encore,
 Philippe ? Celui qui m'a vu a vu le Père. Pourquoi donc dis-tu~:
 \og Montre-nous le Père \fg{} ?
 Ne crois-tu pas que je vis dans le Père et que le Père vit en moi ?
 Les paroles que je vous dis à tous ne viennent pas de moi.
 C'est le Père qui demeure en moi qui accomplit ses propres œuvres.
 Oui, je vous le déclare, c'est la vérité~:
 celui qui croit en moi fera aussi les œuvres que je fais \fg{}
 (\bibleverse{Jn}(14:9-12)).

Jésus représentait Dieu fidèlement dans toute action.
 Il nous a démontré que Dieu s'inquiète de la bonne santé physique,
 émotionnelle, et spirituelle de l'Homme.
 Dieu s'intéresse à vos souffrances ; Dieu s'intéresse à vos peines ;
 Dieu s'intéresse à vos faiblesses.
 Jésus n'est jamais arrivé dans une scène de peine sans y apporter
 de la victoire et de la joie. Il n'a jamais fait face à la faiblesse
 de l'humanité sans partager la force de Dieu.

\section*{Le grand assistant}

Dieu veut vous aider dans votre faiblesse,
 c'est pourquoi Il a envoyé un autre Consolateur,
 pour être à vos côtés et vous soutenir. Jésus a dit~:
 \og Vous recevrez une puissance,
 celle du Saint-Esprit survenant sur vous. \fg{}
 Vous recevrez cette dynamique. Quand je pense à la puissance
 de l'Esprit-Saint, c'est d'abord la puissance d'être ce que Dieu
 veut que je sois, et cela s'étend à toutes les parties de ma vie~:
 de la puissance dans ma vie de prière, de la puissance pour marcher
 dans la sainteté, de la puissance pour être et pour faire.
 Ici, c'est une promesse de puissance, et elle est liée au fait
 d'être un témoin de Jésus-Christ~: \og Vous serez mes témoins. \fg{}

Nous nous trompons lorsque nous pensons que témoigner
 est une chose que nous faisons.
 En réalité, c'est quelque chose que nous sommes.
 Si souvent, le témoignage est associé à la distribution de tracts
 au coin de la rue, ou au porte-à-porte pour annoncer l'évangile,
 ou encore au partage des quatre lois spirituelles avec notre voisin
 autour d'une tasse de café.
 Toutes ces actions sont des manières de partager notre foi,
 mais les effectuer ne fait pas de nous des témoins de Jésus-Christ.
 Être un témoin est plus que des mots ; c'est vivre une vie.
 Le mot \og témoin \fg{} vient du mot grec \emph{martus},
 qui nous a donné martyr en français.
 Nous pensons qu'un martyr est une personne qui meurt pour sa foi ;
 pourtant, c'est en fait quelqu'un dont la vie est tellement engagée
 pour sa foi, que rien ne peut l'en dissuader, pas même la menace de mort.
 Sa mort ne fait pas de lui un martyr ; elle ne fait que confirmer
 qu'il était vraiment un martyr.
 Bien des chrétiens témoignent de Jésus-Christ
 sans jamais être de vrais témoins.

\section*{Plus que de simples mots}

Ce qu'une personne dit est souvent sans importance
 car elle ne vit pas une vie en accord avec ce qu'elle dit.
 Si vous essayez de partager avec quelqu'un l'amour que Jésus donne,
 mais que votre vie est remplie de haine, de ressentiment et de jalousie,
 il ne réagira pas à ce que vous dites car votre vie est en contradiction
 avec vos paroles. Si vous déclarez~:
 \og Vous avez vraiment besoin de connaître la joie de Jésus-Christ.
 Il vous donnera une telle joie \fg{}, mais que vous êtes toujours déprimé
 et pessimiste, votre vie n'est pas un témoignage de ce que vous dites.
 Les gens verront votre dépression et décréditeront vos paroles.
 Si vous dites~: \og Vous avez besoin de connaître le Seigneur pour avoir
 une vraie paix dans votre cœur, la paix qui surpasse
 tout entendement humain \fg{}, mais que votre vie est déchirée
 et que vous êtes constamment nerveux, inquiet et plein d'anxiété,
 les gens regarderont votre anxiété et votre souci,
 et n'entendront pas ce que vous dites au sujet de la paix.
 Vos mots peuvent être complètement décrédités par vos actions.
 Il est plus important que les activités de votre vie soient
 un témoignage pour Jésus-Christ, et alors vos mots prendront un sens.
 Si vos mots ne sont pas en accord avec votre vie,
 vos mots n'ont vraiment aucun effet positif.

Beaucoup de gens pensent~:
 \og Je suis un témoin pour Jésus. Je vais à la plage
 et je distribue des tracts. Je partage les quatre lois spirituelles
 partout où je vais. \fg{}
 Cela ne fait pas de vous un témoin véritable et fidèle.
 Votre vie doit être en parfaite harmonie avec Dieu,
 pour que les gens voient votre vie et disent~:
 \og Il y a quelque chose de différent chez cette personne. \fg{}
 Le dire ne fait pas de vous un témoin ; le vivre, oui.

Le Seigneur veut vous donner la puissance d'être un témoin.
 Il vous donnera de la puissance à travers l'Esprit-Saint,
 car par nous-mêmes nous sommes faibles et impuissants.
 Dieu veut que nous soyons forts.
 Dieu veut que nous soyons des témoins pour Lui.

\section*{L'échec de Pierre}

Dans \bibleverse{Mc}(14:53-54), nous lisons~:
 \og Ils emmenèrent Jésus chez le souverain sacrificateur,
 où se réunirent tous les principaux sacrificateurs,
 les anciens et les scribes.
 Pierre le suivit de loin, jusque dans l'intérieur de la cour
 du souverain sacrificateur.
 Assis avec les gardes, il se chauffait près du feu. \fg{}
 Alors que nous suivons l'histoire au verset 66, nous lisons~:
 \og Pendant que Pierre était en bas dans la cour,
 il vint une des servantes du souverain sacrificateur.
 Elle vit Pierre qui se chauffait, le regarda en face et lui dit~:
 Toi aussi, tu étais avec Jésus de Nazareth. Il le nia en disant~:
 Je ne sais pas, je ne comprends pas ce que tu veux dire.
 Puis il sortit pour aller dans le vestibule.
 La servante le vit et se mit de nouveau à dire
 à ceux qui étaient présents~: Il est de ces gens-là.
 Il le nia de nouveau. Peu après, ceux qui étaient présents
 dirent encore à Pierre~: Certainement, toi aussi,
 tu es de ces gens-là ; car tu es Galiléen, [et tu parles comme eux] .
 Alors il se mit à faire des imprécations et à jurer~:
 Je ne connais pas l'homme dont vous parlez. Aussitôt pour la seconde fois
 la coq chanta, et Pierre se souvint de la parole que Jésus lui avait dite~:
 Avant que le coq chante deux fois, tu me renieras trois fois.
 Alors il se mit à pleurer \fg{} (versets 66-72).

Plus tôt ce soir là, Jésus avait annoncé~:
 \og Vous trouverez tous une occasion de chute ce soir. \fg{}
 Mais Pierre avait répondu~:
 \og Quand tous trouveraient une occasion de chute, moi pas. \fg{}
 Jésus lui avait alors répliqué~:
 \og En vérité, je te le dis, aujourd'hui, cette nuit même,
 avant que le coq chante deux fois, toi tu me renieras trois fois. \fg{}
 À ce moment-là, Pierre s'est vraiment énervé et a dit~:
 \og Quand il me faudrait mourir avec toi, je ne te renierai point. \fg{}

Pierre croyait être un vrai martyr,
 et je crois qu'il était parfaitement sincère.
 Je comprend vraiment ce qu'il a ressenti lorsqu'il s'est vanté
 auprès du Seigneur, car son esprit était volontaire et il sentait
 vraiment qu'il avait tout ce qu'il lui fallait pour mourir pour Jésus
 si nécessaire. Mais au moment crucial, Pierre n'était vraiment pas prêt.
 Lorsque cette jeune servante lui a demandé~:
 \og N'étais-tu pas avec Jésus? \fg{}, Pierre a répondu~:
 \og Je ne sais pas de quoi tu parles. \fg{}
 Plus tard, elle a dit à un groupe qui se tenait là~:
 \og Il est l'un d'entre eux \fg{}, et Pierre a renié Jésus à nouveau.
 Ensuite, ceux qui se tenaient prêt de lui ont dit~:
 \og Certainement, tu es l'un d'eux, tu as un accent galiléen. \fg{}
 Alors Pierre a commencé à faire des imprécations et à jurer, en disant~:
 \og Je ne connais pas cet homme! \fg{}
 Alors il s'est rappelé de ce que le Seigneur avait prédit~:
 le coq s'est mis à chanter.
 Lorsque Pierre l'a entendu, il est sorti et a pleuré.
 Combien de fois j'ai pleuré sur mes propres faiblesses
 et ma propre incapacité! Je ne voulais pas faire défaut au Seigneur.
 Je ne voulais pas Le laisser tomber. Je voulais vraiment être là pour Lui.
 Mais la pression était trop grande, je n'étais pas un témoin,
 et j'ai échoué. Que cet échec est amer!
 Qu'il est dur de prendre conscience~:
 \og Oh Seigneur, je t'ai encore fait défaut! \fg{}
 Nous arrivons au point où nous ne voulons même plus Lui promettre quoi
 que ce soit, car nous savons déjà que nous Lui ferons défaut à nouveau.

Je peux m'identifier avec Pierre.
 Je sais exactement comment il s'est senti lorsqu'il a entendu le coq chanter.
 Je connais très bien cet état lamentable~:
 \og Oh Dieu, je suis désolé de t'avoir fait défaut à nouveau. \fg{}
 Devons-nous toujours avancer dans notre expérience de chrétien
 en faisant défaut à notre Seigneur? Non.
 Grâce à Dieu, ce n'est pas une fatalité de Lui faire défaut.
 Il nous a promis la puissance d'être ce que nous ne pourrions
 jamais être par notre propre force ou la force de nos vœux.


