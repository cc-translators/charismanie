\chapter[L'\oe{}uvre de l'Esprit-Saint dans la vie du croyant]{L'\oe{}uvre de l'Esprit-Saint\\ dans la vie du croyant}

\chlettrine[lraise=0.30,loversize=0.15]{Q}{uel est le travail}
 de l'Esprit attendu dans la vie du croyant ?
 Comme nous l'avons déjà noté dans \ibibleverse{Jn}(14:), Son nom,
 \og consolateur \fg{}, indique Sa venue à nos côtés pour nous aider.
 Je ne trouve pas que ma démarche de chrétien soit facile.
 Je trouve que ma chair lutte avec moi tout le temps.
 Tout comme Pierre, j'ai moi aussi souvent découvert que l'esprit
 est bien disposé, mais que la chair est faible.
 Je comprend ce dont Paul parlait dans \ibibleverse{Ga}(5:) lorsqu'il
 mentionnait la bataille entre la chair et l'esprit.
 Si Dieu a de l'aide pour moi, je suis prêt à la recevoir;
 je veux toute l'aide que je peux avoir !
 Je ne veux jamais mettre des limites à ce que Dieu veut me donner,
 ou à ce qu'Il veut faire dans ma vie. Je ne veux pas être coupable,
 comme les israélites dans le désert,
 de limiter Dieu (\ibibleverse{Ps}(78:41)).
 De la même manière, je ne cherche pas l'expérience
 pour l'expérience en soit; je désire seulement l'œuvre véritable
 de l'Esprit-Saint, mais je la veux complètement.

\section*{Faire confiance à l'Esprit}

Dans le discours de Jésus à Ses disciples, à partir du chapitre
 \ibiblechvs{Jn}(14:) de l'Évangile de Jean,
 Il cherche à les préparer à Son départ.
 Il parle beaucoup du fait qu'Il va les quitter et retourner auprès du Père.
 Il leur parle également beaucoup des réserves que le Père et Lui
 ont faites pour eux par la puissance de l'Esprit-Saint.
 Il serait là pour les aider. Tout comme ils avaient appris à faire confiance
 à Jésus pour toute situation d'urgence qui pourrait arriver,
 ils doivent maintenant apprendre à faire confiance à l'Esprit-Saint.
 Il sera désormais leur soutien.

Jésus a passé Ses trois années avec Ses disciples à leur enseigner
 la vérité sur Dieu. Maintenant, leur enseignant les quitte pour retourner
 auprès du Père, mais les étudiants ne resteront pas seuls;
 ce soutien, l'Esprit-Saint, va maintenant leur enseigner toutes choses
 et leur rappeler toutes les choses que Jésus leur a dites
 (\ibibleverse{Jn}(14:26)).

Peut-être avez-vous eu l'expérience de partager l'Évangile avec quelqu'un,
 et il ou elle vous a posé une question qui vous a immédiatement bloqué;
 mais alors que vous avez commencé à répondre, les Écritures ont commencé
 à vous venir à l'esprit, et vous étiez content et satisfait de la réponse
 que vous avez donné à cette personne. Ceci est le travail de rappel de l'Esprit.

L'Esprit-Saint nous aide à comprendre les choses de Dieu.
 Bien souvent, j'ai été frustré dans ma tentative d'expliquer
 une vérité spirituelle à un non croyant.
 Cela paraît si clair et évident, et pourtant il ou elle ne semble pas
 pouvoir le comprendre.
 Si vous êtes dans ce type de situations au sujet de l'Esprit de Dieu,
 l'homme naturel \og ne reçoit pas les choses de l'Esprit de Dieu,
 car elles sont une folie pour lui, et il ne peut les connaître, 
 parce que c'est spirituellement qu'on en juge \fg{}
 (\ibibleverse{ICo}(2:14)).


\section*{L'esprit mort et l'esprit vivant}

Lorsque j'étais à l'université, j'ai eu un professeur de sociologie
 qui croyait au dualisme de l'homme, alors que je croyais au trialisme de l'homme.
 Bien souvent, nous avons exprimé nos vues différentes l'un à l'autre.
 J'étais frustré qu'il ne puisse pas faire la distinction entre l'âme
 et l'esprit de l'homme, mais croyait qu'ils étaient synonymes.
 Un jour, alors que je quittais la classe frustré
 après une nouvelle discussion dans laquelle il paraissait
 avoir un aveuglement délibéré, ce fut comme si l'Esprit-Saint
 m'avait mis \ibibleverse{ICo}(2:14) à l'esprit.
 Alors, j'ai réalisé qu'en tant que personne non régénérée,
 son esprit était mort, donc je lui parlais de mystères qu'il ne pouvait
 pas connaître.
 Il ne savait pas et ne pouvait pas savoir quoi que ce soit sur l'esprit
 de l'homme à moins d'être né de l'Esprit-Saint.
 Dans \ibibleverse{ICo}(2:15), Paul dit~:
 \og L'homme spirituel, au contraire, juge de tout,
 et il n'est lui-même jugé par personne. \fg{}

Quiconque vit seulement au niveau de la conscience du corps
 vit au niveau animal de l'existence.
 Son esprit est dirigé et dominé par les besoins de son corps;
 il ne comprend pas les choses de l'Esprit, car son propre esprit est mort.
 Il n'est pas étonnant qu'il cherche à se rapprocher lui-même
 du règne animal, car il vit comme un animal,
 une conscience dominée par le corps.
 Lorsqu'une personne est née de nouveau par l'Esprit,
 son propre esprit vient à la vie et, ainsi jointe à Dieu par l'Esprit,
 il est encouragé à travers la Parole à vivre une vie dominée par l'Esprit.
 Lorsqu'il le fait, il commence à avoir une conscience dominée par l'Esprit.


\section*{Laisser l'Esprit diriger}

Alors que nous vivons selon l'Esprit, nos formes de pensée sont différentes
 car nous pensons maintenant à Dieu et comment nous pourrions
 Lui plaire et Le servir.
 Les tendances de l'Esprit, c'est la vie et la paix (\ibibleverse{Rm}(8:6)).
 Quelle aide fantastique l'Esprit-Saint est pour nous alors qu'Il nous enseigne
 les choses de Dieu et nous aide à les comprendre !
 La Bible semble prendre vie avec signification et excitation
 alors qu'une Écriture après l'autre paraît presque jaillir
 hors de la page pour nous enseigner.

Dans \ibibleverse{Jn}(16:13), Jésus a promis à Ses disciples
 que lorsque l'Esprit de vérité viendrait,
 Il les guiderait à travers toute vérité,
 et leur montrerait les choses à venir.
 Il est si nécessaire d'être guidé par l'Esprit-Saint
 à travers toute vérité.
 Jésus a mis en garde contre les faux prophètes qui seraient des loups,
 et pourtant paraîtraient habillés comme des agneaux (\ibibleverse{Mt}(7:15)).
 Il y a des hommes qui viennent avec le troupeau de Dieu
 et paraissent en faire partie, mais leur principale motivation
 est de s'attaquer au troupeau.
 Ils apportent avec eux des hérésies de perdition
 et cherchent à conduire des hommes à leur suite.
 Dans \ibibleverse{Ac}(20:29-30),
 Paul a mis en garde les anciens de l'église d'Éphèse~:
 \og Je sais que parmi vous, après mon départ,
 s'introduiront des loups redoutables
 qui n'épargneront pas le troupeau,
 et que du milieu de vous se lèveront des hommes
 qui prononceront des paroles perverses,
 pour entraîner les disciples après eux. \fg{}

Pierre a mis en garde dans \ibibleverse{IIP}(2:1,3)~:
 \og Il y a eu de faux prophètes parmi le peuple;
 de même il y a parmi vous de faux docteurs
 qui introduiront insidieusement des hérésies de perdition et qui,
 reniant le Maître qui les a rachetés,
 attireront sur eux une perdition soudaine. [\dots{}]
 Par cupidité, ils vous exploiteront au moyen de paroles trompeuses,
 mais depuis longtemps leur condamnation est en marche
 et leur perdition n'est pas en sommeil. \fg{}


\section*{Détecter les faux prophètes}

Remarquez une des caractéristiques d'un faux prophète~:
 \og vous exploiter au moyen de paroles trompeuses. \fg{}
 Je reçois régulièrement des lettres dactylographiées
 venant d'évangélistes de renom qui correspondent parfaitement
 à la description faite par Pierre.
 Ces lettres disent des choses comme~:
 \og Benny, Dieu t'as mis sur mon cœur ce matin, et j'ai prié pour toi.
 Je ne peux pas t'enlever de mon esprit, Benny.
 Est-ce que tout va bien? As-tu un besoin pour lequel je peux prier?
 Réponds-moi immédiatement, car je t'aime, Benny, et je veux t'aider!
 Il se trouve également que mon ministère traverse
 une de ses pires crises financières.
 Nous allons devoir fermer certaines de nos grandes œuvres pour Dieu
 à moins que tu puisses nous aider immédiatement.
 Si tu n'as pas 50 dollars à m'envoyer, peut-être peux-tu les emprunter
 ailleurs et m'aider à garder l'œuvre de foi en Dieu à continuer.
 Plante tes graines de foi aujourd'hui.
 Dieu t'aidera à rembourser l'emprunt que tu prends.
 Ton partenaire dans la foi. \fg{}

Mon nom n'est pas Benny, mais il semble que c'est ainsi que j'ai été noté
 dans leurs listes de diffusion.
 Ce type de lettres malhonnête cherchent seulement à exploiter les gens,
 et de par l'autorité de la Parole de Dieu,
 je n'hésite pas à désigner leurs auteurs de faux prophètes.

Il s'agit de charismanie dans une de ses formes les plus évidentes
 et elle est pratiquée par la plupart des évangélistes charismatiques,
 en particulier ceux qui mettent en avant la guérison divine.
 Je suis toujours émerveillé qu'ils puissent avoir une telle foi
 dans ma guérison et une si petite foi pour leur propres besoins financiers.

Il est beau de voir comment l'Esprit vous préviendra
 lorsque quelqu'un commence à diverger dans sa doctrine.
 Bien souvent, il ne vous est pas possible de pointer l'erreur immédiatement,
 mais vous savez que quelque chose ne va pas vraiment.
 L'Esprit a été donné au croyant pour le guider en tout vérité.


\section*{Apprendre les choses à venir}

L'Esprit-Saint nous montre également des choses à venir.
 Lorsque Daniel cherchait une plus grande compréhension des temps de la fin
 et des choses qu'il avait écrites, il lui fut ordonné de
 \og [tenir] secrètes ces paroles et [de sceller] le livre
 jusqu'au temps de la fin.
 Beaucoup alors le liront, et la connaissance augmentera \fg{}
 (\ibibleverse{Dn}(12:4)).
 Alors que Daniel persistait à demander, à nouveau le Seigneur lui dit~:
 \og Va, Daniel, car ces paroles seront secrètes
 et scellées jusqu'au temps de la fin.
 Beaucoup seront purifiés, blanchis et épurés;
 les méchants feront le mal et aucun des méchants ne comprendra,
 mais ceux qui auront de l'intelligence comprendront \fg{}
 (\ibibleverse{Dn}(12:9-10)).

C'est avec l'aide de l'Esprit-Saint qu'une compréhension claire
 du retour de Jésus-Christ a été donnée à l'Église.
 L'Esprit-Saint a montré à l'apôtre Paul certaines des choses
 qui devaient arriver dans sa vie alors qu'il disait aux anciens d'Éphèse
 dans \ibibleverse{Ac}(20:22-23)~:
 \og Et maintenant voici que lié par l'Esprit, je vais à Jérusalem,
 sans savoir ce qui m'y arrivera; seulement, de ville en ville,
 le Saint-Esprit atteste et me dit que des liens
 et des tribulations m'attendent. \fg{}
 Plus tard, alors que Paul continuait son voyage vers Jérusalem,
 le prophète Agabus prit la ceinture de Paul, s'attacha avec, et dit~:
 \og Voici ce que déclare le Saint-Esprit~:
 L'homme à qui cette ceinture appartient, les Juifs le lieront
 de cette manière à Jérusalem et le livreront
 entre les mains des païens \fg{} (\ibibleverse{Ac}(21:11)).
 Voici un exemple classique de l'Esprit-Saint montrant à Paul
 les choses qui devaient arriver dans sa vie.

Un autre exemple dans la vie de Paul de ce travail de l'Esprit
 se trouve dans \ibibleverse{Ac}(27:21-24)~:
 \og On n'avait pas mangé depuis longtemps.
 Alors Paul, debout au milieu des hommes, leur dit~:
 Vous auriez dû m'obéir et ne pas repartir de Crète;
 vous auriez évité ce péril et ce dommage.
 Maintenant je vous exhorte à prendre courage;
 car aucun de vous ne perdra la vie, seul le navire sera perdu.
 Un ange du Dieu à qui j'appartiens et rends un culte,
 s'est approché de moi cette nuit et m'a dit~:
 Sois sans crainte, Paul; il faut que tu comparaisses devant César,
 et voici que Dieu t'accorde la grâce de tous ceux qui naviguent avec toi. \fg{}


\section*{La main puissante de Dieu}

Dans un petit groupe de prière, nous avons décidé de prier
 l'un pour l'autre.
 La personne pour qui nous devions prier s'asseyait sur une chaise
 au centre du groupe. Lorsque ce fut mon tour d'être le sujet de prière,
 quelqu'un a prononcé un mot de prophétie de la part de l'Esprit-Saint
 déclarant que la main de bénédiction de Dieu allait venir
 sur mon ministère de façon importante, que les gens viendraient écouter
 la Parole tellement nombreux qu'il n'y aurait plus de place dans l'église
 pour les contenir.
 La prophétie continuait en déclarant que je recevrais un nouveau nom
 qui signifiait berger, car le Seigneur alors faire de moi un berger
 pour de nombreux troupeaux.

Jusqu'à ce moment, j'avais lutté dans le ministère pendant près de 17 ans
 avec un succès tellement limité que j'envisageais de quitter le ministère
 pour faire un autre type de travail.
 L'église dont j'étais alors le pasteur rassemblait environ 100 personnes
 malgré tous nous efforts pour augmenter sa taille en donnant des hamburgers
 gratuits à tous ceux qui invitaient un ami à l'école du dimanche.
 Alors que ces mots étaient prononcés, j'étais dans mon cœur comme l'homme
 sur qui le roi se reposait et qui, lorsqu'il entendit la promesse d'Élisée
 que des provisions abondantes viendraient sur les habitants affamés
 de Samarie, a dit~:
 \og Même si le Seigneur envoyait du grain en perçant des trous dans
 la voûte du ciel, ce que tu viens de dire pourrait-il se réaliser ? \fg{}

Heureusement, Dieu a été miséricordieux envers moi,
 car mon destin ne devait pas être le même que celui de cet homme~;
 j'ai vu l'accomplissement de la prophétie et j'ai également pu y participer
 alors que nous voyons les locaux de l'église, fortement agrandis,
 se remplir au-delà de leur capacité, non pas une fois, mais trois fois
 chaque dimanche matin, et alors que nous exerçons un ministère
 par cassettes audios et vidéos auprès de centaines de groupes
 d'étude biblique à travers le monde.

\section*{La puissance pour conquérir}

L'œuvre de l'Esprit-Saint dans votre vie si vous êtes un croyant
 est de vous donner la puissance d'être un témoin pour Jésus-Christ,
 de vous donner la puissance d'être tout ce que Dieu veut que vous soyez.
 Une des choses les plus frustrantes dans le monde est d'essayer
 de vivre la vie chrétienne par l'énergie de la chair.
 La Bible parle de cette frustration dans \ibibleverse{Rm}(7:),
 où Paul parle de comment, lorsqu'il a essayé de garder la loi de Dieu
 et de faire le bien, il a découvert que \og moi qui veux faire le bien,
 je suis seulement capable de faire le mal. Et le bien que je veux faire,
 je ne le fais pas. Et ce que je ne veux pas, je le fais.
 Malheureux que je suis ! \fg{}
 Paul décrit dans \ibibleverse{Ga}(5:) comment la chair lutte avec l'esprit
 et l'esprit contre la chair, et comment ils sont opposés l'un à l'autre.
 Jésus a dit à Pierre~:
 \og Veillez et priez, afin de ne pas entrer en tentation ;
 l'esprit est bien disposé, mais la chair est faible \fg{}
 (\ibibleverse{Mt}(26:41)).
 À cause de la faiblesse de notre chair, nous ne pouvons pas vivre
 le type de vie que le Seigneur voudrait que nous suivions
 et que nous-mêmes, nous voudrions vivre à la face du monde.

Dieu désire que votre vie soit une vraie représentation de Lui dans ce monde.
 Dieu veut que le monde voie Jésus-Christ en vous.
 Il veut que vos actions et vos réactions Le reflètent.
 Il veut que vous soyez Son témoin, en Le représentant.
 Mais si vous tentez d'être Son témoin, si vous essayez de réagir comme Christ,
 vous verrez combien il est difficile et frustrant de le faire
 \ocadr en fait, combien c'est \emph{impossible}
 à cause de la faiblesse de la chair.

\section*{Le témoin parfait}

Beaucoup de chrétiens se trouvent dans cette frustration de savoir
 ce qui est juste et de désirer ce qui est juste, mais de ne pas faire
 ce qui est juste.
 La Bible dit de Jésus-Christ qu'Il était le témoin véritable et fidèle ;
 Il était un témoin du Père.
 Si vous voulez savoir à quoi ressemble Dieu, regardez simplement
 Jésus-Christ, car Il est le Témoin véritable et fidèle.
 Lorsque Philippe s'est écrié~:
 \og Seigneur, montre-nous le Père et nous serons satisfaits \fg{},
 Jésus répliqua~:
 \og Il y a si longtemps que je suis avec vous et tu ne me connais pas encore,
 Philippe ? Celui qui m'a vu a vu le Père. Pourquoi donc dis-tu~:
 \og Montre-nous le Père \fg{} ?
 Ne crois-tu pas que je vis dans le Père et que le Père vit en moi ?
 Les paroles que je vous dis à tous ne viennent pas de moi.
 C'est le Père qui demeure en moi qui accomplit ses propres œuvres.
 Oui, je vous le déclare, c'est la vérité~:
 celui qui croit en moi fera aussi les œuvres que je fais \fg{}
 (\ibibleverse{Jn}(14:9-12)).

Jésus représentait Dieu fidèlement dans toute action.
 Il nous a démontré que Dieu s'inquiète de la bonne santé physique,
 émotionnelle, et spirituelle de l'Homme.
 Dieu s'intéresse à vos souffrances ; Dieu s'intéresse à vos peines ;
 Dieu s'intéresse à vos faiblesses.
 Jésus n'est jamais arrivé dans une scène de peine sans y apporter
 de la victoire et de la joie. Il n'a jamais fait face à la faiblesse
 de l'humanité sans partager la force de Dieu.

\section*{Le grand assistant}

Dieu veut vous aider dans votre faiblesse,
 c'est pourquoi Il a envoyé un autre Consolateur,
 pour être à vos côtés et vous soutenir. Jésus a dit~:
 \og Vous recevrez une puissance,
 celle du Saint-Esprit survenant sur vous. \fg{}
 Vous recevrez cette dynamique. Quand je pense à la puissance
 de l'Esprit-Saint, c'est d'abord la puissance d'être ce que Dieu
 veut que je sois, et cela s'étend à toutes les parties de ma vie~:
 de la puissance dans ma vie de prière, de la puissance pour marcher
 dans la sainteté, de la puissance pour être et pour faire.
 Ici, c'est une promesse de puissance, et elle est liée au fait
 d'être un témoin de Jésus-Christ~: \og Vous serez mes témoins. \fg{}

Nous nous trompons lorsque nous pensons que témoigner
 est une chose que nous faisons.
 En réalité, c'est quelque chose que nous sommes.
 Si souvent, le témoignage est associé à la distribution de tracts
 au coin de la rue, ou au porte-à-porte pour annoncer l'Évangile,
 ou encore au partage des quatre lois spirituelles avec notre voisin
 autour d'une tasse de café.
 Toutes ces actions sont des manières de partager notre foi,
 mais les effectuer ne fait pas de nous des témoins de Jésus-Christ.
 Être un témoin est plus que des mots ; c'est vivre une vie.
 Le mot \og témoin \fg{} vient du mot grec \emph{martus},
 qui nous a donné martyr en français.
 Nous pensons qu'un martyr est une personne qui meurt pour sa foi ;
 pourtant, c'est en fait quelqu'un dont la vie est tellement engagée
 pour sa foi, que rien ne peut l'en dissuader, pas même la menace de mort.
 Sa mort ne fait pas de lui un martyr ; elle ne fait que confirmer
 qu'il était vraiment un martyr.
 Bien des chrétiens témoignent de Jésus-Christ
 sans jamais être de vrais témoins.

\section*{Plus que de simples mots}

Ce qu'une personne \emph{dit} est souvent sans importance
 car elle ne vit pas une vie en accord avec ce qu'elle dit.
 Si vous essayez de partager avec quelqu'un l'amour que Jésus donne,
 mais que votre vie est remplie de haine, de ressentiment et de jalousie,
 il ne réagira pas à ce que vous dites car votre vie est en contradiction
 avec vos paroles. Si vous déclarez~:
 \og Vous avez vraiment besoin de connaître la joie de Jésus-Christ.
 Il vous donnera une telle joie \fg{}, mais que vous êtes toujours déprimé
 et pessimiste, votre vie n'est pas un témoignage de ce que vous dites.
 Les gens verront votre dépression et décréditeront vos paroles.
 Si vous dites~: \og Vous avez besoin de connaître le Seigneur pour avoir
 une vraie paix dans votre cœur, la paix qui surpasse
 tout entendement humain \fg{}, mais que votre vie est déchirée
 et que vous êtes constamment nerveux, inquiet et plein d'anxiété,
 les gens regarderont votre anxiété et votre souci,
 et n'entendront pas ce que vous dites au sujet de la paix.
 Vos mots peuvent être complètement décrédités par vos actions.
 Il est plus important que les activités de votre vie soient
 un témoignage pour Jésus-Christ, et alors vos mots prendront un sens.
 Si vos mots ne sont pas en accord avec votre vie,
 vos mots n'ont vraiment aucun effet positif.

Beaucoup de gens pensent~:
 \og Je suis un témoin pour Jésus. Je vais à la plage
 et je distribue des tracts. Je partage les quatre lois spirituelles
 partout où je vais. \fg{}
 Cela ne fait pas de vous un témoin véritable et fidèle.
 Votre vie doit être en parfaite harmonie avec Dieu,
 pour que les gens voient votre vie et disent~:
 \og Il y a quelque chose de différent chez cette personne. \fg{}
 Le \emph{dire} ne fait pas de vous un témoin ; le \emph{vivre}, oui.

Le Seigneur veut vous donner la puissance d'être un témoin.
 Il vous donnera de la puissance à travers l'Esprit-Saint,
 car par nous-mêmes nous sommes faibles et impuissants.
 Dieu veut que nous soyons forts.
 Dieu veut que nous soyons des témoins pour Lui.

\section*{L'échec de Pierre}

Dans \ibibleverse{Mc}(14:53-54), nous lisons~:
 \og Ils emmenèrent Jésus chez le souverain sacrificateur,
 où se réunirent tous les principaux sacrificateurs,
 les anciens et les scribes.
 Pierre le suivit de loin, jusque dans l'intérieur de la cour
 du souverain sacrificateur.
 Assis avec les gardes, il se chauffait près du feu. \fg{}
 Alors que nous suivons l'histoire au verset \ibiblevs{Mc}(14:66),
 nous lisons~:
 \og Pendant que Pierre était en bas dans la cour,
 il vint une des servantes du souverain sacrificateur.
 Elle vit Pierre qui se chauffait, le regarda en face et lui dit~:
 Toi aussi, tu étais avec Jésus de Nazareth. Il le nia en disant~:
 Je ne sais pas, je ne comprends pas ce que tu veux dire.
 Puis il sortit pour aller dans le vestibule.
 La servante le vit et se mit de nouveau à dire
 à ceux qui étaient présents~: Il est de ces gens-là.
 Il le nia de nouveau. Peu après, ceux qui étaient présents
 dirent encore à Pierre~: Certainement, toi aussi,
 tu es de ces gens-là ; car tu es Galiléen, [et tu parles comme eux] .
 Alors il se mit à faire des imprécations et à jurer~:
 Je ne connais pas l'homme dont vous parlez. Aussitôt pour la seconde fois
 la coq chanta, et Pierre se souvint de la parole que Jésus lui avait dite~:
 Avant que le coq chante deux fois, tu me renieras trois fois.
 Alors il se mit à pleurer \fg{} (versets \ibiblevs{Mc}(14:66-72)).

Plus tôt ce soir là, Jésus avait annoncé~:
 \og Vous trouverez tous une occasion de chute ce soir. \fg{}
 Mais Pierre avait répondu~:
 \og Quand tous trouveraient une occasion de chute, moi pas. \fg{}
 Jésus lui avait alors répliqué~:
 \og En vérité, je te le dis, aujourd'hui, cette nuit même,
 avant que le coq chante deux fois, toi tu me renieras trois fois. \fg{}
 À ce moment-là, Pierre s'est vraiment énervé et a dit~:
 \og Quand il me faudrait \emph{mourir} avec toi, je ne te renierai point. \fg{}

Pierre croyait être un vrai martyr,
 et je crois qu'il était parfaitement sincère.
 Je comprend vraiment ce qu'il a ressenti lorsqu'il s'est vanté
 auprès du Seigneur, car son esprit était volontaire et il sentait
 vraiment qu'il avait tout ce qu'il lui fallait pour mourir pour Jésus
 si nécessaire. Mais au moment crucial, Pierre n'était vraiment pas prêt.
 Lorsque cette jeune servante lui a demandé~:
 \og N'étais-tu pas avec Jésus? \fg{}, Pierre a répondu~:
 \og Je ne sais pas de quoi tu parles. \fg{}
 Plus tard, elle a dit à un groupe qui se tenait là~:
 \og Il est l'un d'entre eux \fg{}, et Pierre a renié Jésus à nouveau.
 Ensuite, ceux qui se tenaient prêt de lui ont dit~:
 \og Certainement, tu es l'un d'eux, tu as un accent galiléen. \fg{}
 Alors Pierre a commencé à faire des imprécations et à jurer, en disant~:
 \og Je ne connais pas cet homme! \fg{}
 Alors il s'est rappelé de ce que le Seigneur avait prédit~:
 le coq s'est mis à chanter.
 Lorsque Pierre l'a entendu, il est sorti et a pleuré.
 Combien de fois j'ai pleuré sur mes propres faiblesses
 et ma propre incapacité! Je ne voulais pas faire défaut au Seigneur.
 Je ne voulais pas Le laisser tomber. Je voulais vraiment être là pour Lui.
 Mais la pression était trop grande, je n'étais pas un témoin,
 et j'ai échoué. Que cet échec est amer!
 Qu'il est dur de prendre conscience~:
 \og Oh Seigneur, je t'ai encore fait défaut! \fg{}
 Nous arrivons au point où nous ne voulons même plus Lui promettre quoi
 que ce soit, car nous savons déjà que nous Lui ferons défaut à nouveau.

Je peux m'identifier avec Pierre.
 Je sais exactement comment il s'est senti lorsqu'il a entendu le coq chanter.
 Je connais très bien cet état lamentable~:
 \og Oh Dieu, je suis désolé de t'avoir fait défaut à nouveau. \fg{}
 Devons-nous toujours avancer dans notre expérience de chrétien
 en faisant défaut à notre Seigneur? Non.
 Grâce à Dieu, ce n'est pas une fatalité de Lui faire défaut.
 Il nous a promis la puissance d'être ce que nous ne pourrions
 jamais être par notre propre force ou la force de nos vœux.


\section*{Pierre le témoin}

Quelques semaines plus tard, Pierre s'est retrouvé face au même groupe
 d'hommes qui avait incité au meurtre de Jésus~:
 \og Le lendemain, leurs chefs, ainsi que les anciens et les scribes,
 s'assemblèrent à Jérusalem, avec le souverain sacrificateur Anne,
 Caïphe, Jean, Alexandre, et tous ceux qui étaient de la famille
 des principaux sacrificateurs. Ils firent comparaître au milieu d'eux
 Pierre et Jean, et demandèrent~: Par quelle puissance ou par quel nom
 avez-vous fait cela ? Alors Pierre, \emph{rempli d'Esprit Saint}, leur dit~:
 Chefs du peuple, et anciens, puisque nous sommes interrogés aujourd'hui
 sur un bienfait accordé à un homme infirme, et sur la manière
 dont il a été guéri, sachez-le bien, vous tous, ainsi que tout le peuple
 d'Israël ! C'est par le nom de Jésus-Christ de Nazareth, que vous avez
 crucifié et que Dieu a ressuscité des morts, c'est par lui que cet homme
 se présente en bonne santé devant vous. C'est lui~: La pierre rejetée
 par vous, les bâtisseurs, et devenue la principale, celle de l'angle.
 Le salut ne se trouve en aucun autre ; car il n'y a sous le ciel aucun
 autre nom donné parmi les hommes, par lequel nous devions être
 sauvés \fg{} (\ibibleverse{Ac}(4:5-12)).

Lorsqu'ils virent l'assurance de Pierre, ils furent étonnés
 (\ibibleverse{Ac}(4:13)). Il s'agissait d'une personne différente
 de celle qui, quelques semaines plus tôt, se tenait à la porte du palais
 et avait renié son Seigneur. Quel homme différent!
 En lisant les deux témoignages, il serait difficile de croire
 qu'il s'agissait de la même personne. Qu'est-ce qui a fait la différence?
 La différence réside dans cette petite phrase~:
 \og rempli d'Esprit-Saint \fg{}. Jésus a dit à Ses disciples~:
 \og Ils vous amèneront devant les magistrats et les juges,
 et lorsqu'ils le feront, ne vous inquiétez pas de ce que vous direz.
 Ne préparez pas vos discours, car à ce moment l'Esprit-Saint viendra
 sur vous, et ce sera l'Esprit-Saint qui parlera par vous.
 Vous recevrez de la puissance, et vous serez témoins. \fg{}
 L'Esprit-Saint est une aide, Celui qui vous aide à être tout ce que Dieu
 veut que vous soyez~: un témoin véritable et fidèle pour Lui.


\section*{La seule source de puissance}

L'Esprit-Saint nous donne la puissance d'être un témoin véritable
 et fidèle de Jésus-Christ \ocadr la puissance de Le représenter
 véritablement au travail, à la maison, ou en classe \fcadr{}
 afin que lorsque les gens nous regardent, ils voient l'amour,
 la paix et la beauté de Jésus-Christ dans nos actions et nos attitudes.
 Ils verront une personne qui est en paix dans la tempête.
 C'est la puissance dont nous avons besoin si nous sommes Ses vrais témoins,
 car nous ne pouvons pas être un vrai témoin de Lui par notre propre force
 ou nos capacités ; sans la puissance de l'Esprit-Saint, nous échouerons
 chaque fois que le vrai problème apparaît et que la pression monte.
 C'est seulement lorsque nous apprenons à déprendre complètement
 de l'Esprit-Saint que nous faisons l'expérience de sa puissance.

Une des erreurs les plus fréquentes est que lorsque nous voyons une zone
 de faiblesse dans notre vie, nous essayons immédiatement de la compenser
 et de la corriger nous-mêmes. Nous disons~:
 \og Je suis désolé, Seigneur, je ne le ferai plus.
 Je te le promets, Seigneur. \fg{}
 Nous pensons ce que nous disons, mais nous recommençons.
 Le problème est que nous essayons de corriger le souci nous-mêmes,
 en pensant que d'une manière ou d'une autre, si nous faisons un petit effort
 supplémentaire ou si nous essayons une approche un peu différente,
 nous pouvons changer et corriger les faiblesses de notre caractère
 et de notre nature.

C'est seulement lorsque nous sommes arrivés dans un état de désespoir total
 dans l'aide possible que nous pouvons nous apporter, et que nous laissons
 tomber pour nous rendre, que nous connaissons la joie de \emph{Sa} victoire.
 C'est seulement lorsque Paul s'est écrié :
 \og Malheureux que je suis ! \fg{} qu'il a reconnu la vérité sur lui-même
 et qu'il a cessé de chercher \og qui a un autre programme à essayer ? \fg{},
 \og qui a une autre formule ? \fg{}.
 Paul a laissé tomber et a supplié de recevoir une puissance
 autre que la sienne~: \og Malheureux que je suis ! Qui me délivrera ?
 Je ne peux pas me délivrer moi-même. \fg{}
 Il a abandonné l'idée de se délivrer lui-même et a reconnu
 qu'il était misérable.

Puis, il a répondu a sa propre question~:
 \og Grâces soient rendues à Dieu par la promesse de Jésus Christ
 et la puissance de l'Esprit-Saint, Dieu a assuré ma victoire. \fg{}
 Alors que nous arrivons dans \ibibleverse{Rm}(8:),
 nous lisons toutes les choses au sujet d'une vie guidée par l'Esprit,
 remplie par l'Esprit, dirigée par l'Esprit, emplie de puissance par l'Esprit.
 Paul conclut le chapitre en disant que nous sommes \og plus que vainqueurs
 par celui qui nous a aimés. \fg{}

Quelle histoire différente de celle de la défaite et du triste désespoir
 de la faiblesse de notre chair au chapitre \ibiblechvs{Rm}(7:) !
 Quel cri de victoire glorieux~:
 \og Plus que vainqueurs par celui qui nous a aimés.
 Car je suis persuadé que ni la mort, ni la vie, ni les anges,
 ni les dominations, ni le présent, ni l'avenir, ni les puissances,
 ni les êtres d'en-haut, ni ceux d'en-bas, ni aucune autre créature
 ne pourra nous séparer de l'amour de Dieu en Christ-Jésus
 notre Seigneur. \fg{} (\ibibleverse{Rm}(8:37-39)).

Ce cri de victoire glorieux est possible grâce à la puissance de l'Esprit
 lorsque je lâche prise et que je me tourne vers Dieu pour recevoir
 cette puissance, cette dynamique de la part de Dieu. À ce moment là,
 j'autorise l'Esprit-Saint à effectuer Son travail au sein de ma vie,
 le travail que Dieu a prévu pour Lui.


\section*{Pas ma propre force}

La conséquence de ceci est que je ne peux pas me tenir devant vous
 et me vanter d'être la personne extraordinaire que je suis
 ou le témoin extraordinaire pour je suis pour le Seigneur,
 ou encore de la manière extraordinaire dont je réagis dans des
 situations difficiles.
 Toute vantardise est désormais dans le travail de Dieu par son Esprit.
 Je suis toujours un homme misérable, mais grâce à Dieu
 j'ai été délivré de ma misère par la puissance de l'Esprit-Saint.
 Maintenant, lorsque je me trouve dans une situation tendue
 et que les choses poussent de chaque côté, grâce à Dieu la pression
 ne monte même plus.
 C'est presque comme être assis dehors et regarder l'Esprit travailler
 au lieu d'être impliqué.
 Tout à coup, je dis~:
 \og Merci Dieu ! Ce n'est pas moi,
 ce n'est pas ma manière de réagir ! \fg{}

Un officier de la marine à la retraite a accepté Jésus comme son Seigneur
 il y a quelques temps.
 Il avait un parler cru, comme beaucoup de militaires.
 Après avoir accepté le Seigneur, il était vraiment motivé par Jésus.
 Après environ six mois dans le Seigneur, il était dehors dans la cour
 à tondre la pelouse avec sa tondeuse à gazon, sifflant en appréciant
 la joie du Seigneur.
 Alors qu'il était occupé à tondre et qu'il ne faisait pas très attention,
 il est passé sous un arbre et une grosse branche lui est tombée sur le front
 et l'a couché sur le dos.

Alors qu'il était couché sur le dos, il s'est tout à coup excité.
 Il a cour ru dans la maison, a attrapé sa femme et lui a dit~:
 \og Chérie, devine ce qui vient de m'arriver ! \fg{}
 Elle a regardé sa figure ensanglantée et a demandé~:
 \og Qu'est-ce qui t'es arrivé ? Qu'est-ce que tu as fait ? \fg{}
 Il a répondu~:
 \og Ce n'est pas ça, c'est ce que je \emph{n'ai pas} fait !
 Lorsque c'est arrivé, je n'ai pas juré ! Pas même un mot de travers ! \fg{}
 Elle a répondu~:
 \og Chéri, tu sais que je ne t'ai pas entendu jurer depuis six mois ? \fg{}
 Il a demandé~: \og Non ? \fg{}
 Le Seigneur lui avait retiré son mauvais langage sans qu'il s'en aperçoive,
 jusqu'à ce qu'une situation vienne mettre en valeur sa vieille nature
 pour lui faire soudainement prendre conscience
 que Dieu lui avait donné la victoire.


\section*{Changé de l'intérieur}

Voilà la belle façon d'œuvrer de l'Esprit-Saint;
 il travaille de telle manière que, bien souvent,
 le travail est déjà accompli, et nous n'en avons même pas conscience.
 Nous sommes changés de l'intérieur, c'est la méthode de l'Esprit.
 C'est le changement de l'intérieur qui ressort,
 qui est exactement l'opposé de la méthode par laquelle
 nous essayons d'y arriver.
 Nous avons essayé de forcer les changements de l'extérieur
 vers l'intérieur.
 Parfois, nous pouvons réussir à changer l'extérieur,
 mais si l'intérieur n'est pas changé,
 ce qui est \emph{au-dedans} va finir par \emph{ressortir}.

Il est important que l'Esprit opère le changement de \emph{l'intérieur}.
 Lorsque cela arrive, seul Dieu peut recevoir la gloire.
 Où est ma vantardise ? Elle est exclue.
 Je n'ai aucun moyen de me vanter, car je suis toujours qui j'étais.
 Mais je remercie Dieu pour sa grâce~: par la puissance de Son Esprit-Saint,
 je suis maintenant une nouvelle créature dans le Christ Jésus.
 Je considère la vieille nature comme morte.
 Est-ce que cela veut dire que je ne serai plus jamais en colère ?
 Non, j'aimerais que ce soit le cas.
 Voilà ce que cela signifie~: lorsqu'il arrive que je me mette en colère
 et que j'échoue, je dis~:
 \og Seigneur, que Ton Esprit soit à l'œuvre. Donne-moi la force,
 Seigneur ; je ne peux pas le faire. Il va falloir que tu le fasses,
 Seigneur ; donne-moi la force. \fg{}
 Une partie de ma vie après l'autre, lorsque je présente cette partie
 de faiblesse à la puissance de l'Esprit-Saint, je commence à faire
 l'expérience de changements réels alors que l'Esprit travaille en moi
 et me rend conforme à l'image de Christ.


\section*{Fermer la porte ?}

Dans le sermon sur la montagne, Jésus a fait une affirmation remarquable,
 qui doit avoir étonné ceux qui l'ont entendue.
 Dans \ibibleverse{Mt}(5:20), Il a dit~:
 \og Je vous l'affirme~:
 si vous n'êtes pas plus fidèles à la volonté de Dieu
 que les maîtres de la loi et les Pharisiens,
 vous ne pourrez pas entrer dans le Royaume des cieux. \fg{}
 C'est notre désir, notre but, et notre prière de pouvoir entrer
 dans le Royaume des cieux.
 Cependant, il semble que Jésus fermait la porte du Royaume des cieux
 plutôt qu'Il ne l'ouvrait lorsqu'Il a fait cette affirmation étonnante,
 car les Pharisiens pratiquaient le fait d'être juste.
 Ils passaient leurs vies à essayer d'interpréter la bonne action,
 puis à la mettre en pratique.
 Lorsque Jésus a dit à Ses disciples~:
 \og Si vous n'êtes pas plus fidèles à la volonté de Dieu que
 les maîtres de la loi et les Pharisiens, vous ne pourrez pas
 entrer dans le Royaume des cieux \fg{}, j'imagine quel soupir de tristesse
 a dû les traverser alors qu'ils abandonnaient
 l'idée de jamais y entrer.

Puis Jésus a fermé la porte encore plus,
 car Il a donné une série d'illustrations pour expliquer
 ce qu'Il voulait dire \ocadr des illustrations montrant comment la loi
 était mal interprétée par les scribes et les Pharisiens.
 Puis Il a mis cela en contraste avec la manière dont la loi
 devait être comprise à l'origine.
 Le problème majeur dans l'interprétation de la loi par les Pharisiens
 était qu'ils l'interprétaient dans le but de pouvoir l'accomplir
 et se sentir bien à son sujet.
 Ils interprétaient la loi pour vivre confortablement avec elle,
 mais on ne peut pas vivre confortablement avec la loi.
 Ils avaient commencé à sentir qu'ils avaient accompli la loi,
 et ils se promenaient en faisant leur petits actes de justice
 et en pensant qu'ils étaient justes.

Mais Jésus leur a montré que, bien que leurs actions
 aient été correctes, leurs attitudes étaient mauvaises,
 et ils étaient de ce fait des pécheurs,
 car la loi était spirituelle.
 La loi n'était pas faite pour s'occuper uniquement des actions extérieures
 de l'homme ; elle était faite pour s'occuper
 des \emph{attitudes intérieures} de l'homme.
 Lorsque la loi dit~: \og Tu ne tueras point \fg{},
 vous ne pouvez pas vraiment vous asseoir confortablement et vous vanter
 en vous-même en disant~:
 \og Eh bien, je n'ai jamais tué qui que ce soit. \fg{}
 Si vous vous sentez bien et juste d'avoir gardé cette loi,
 rappelez-vous ce que Jésus a dit~:
 \og Ce que Dieu a voulu dire, c'est que vous ne devez même pas
 haïr votre frère. \fg{}
 L'attitude de haine, nous a dit Jésus,
 était équivalente à l'action du meurtre,
 pour ce qui était de violer la loi.

Il pourrait donc sembler que Jésus fermait la porte du Royaume de Dieu.
 Finalement, nous arrivons au dernier verset de \ibibleverse{Mt}(5:),
 où il semble qu'Il l'a bloquée et verrouillée, car Il dit~:
 \og Soyez donc parfaits,
 tout comme votre Père qui est au ciel est parfait. \fg{}


\section*{Je laisse tomber}

Tout à coup, je réalise que je ne peux pas atteindre
 ce que Dieu demande de moi, car quels que soient mes efforts,
 je ne peux pas être parfait.
 J'ai échoué, et il n'y a aucun moyen pour moi d'accomplir
 les conditions posées par Dieu ou le commandement de Jésus-Christ.
 Ce n'est pas que je ne veux pas être parfait.
 Le Seigneur sait que j'aimerais beaucoup être parfait,
 surtout quand j'ai tort.
 Ce serait bien de faire toujours le bon choix,
 ce serait bien d'avoir toujours la bonne réaction,
 mais ce n'est pas le cas.
 Bien souvent, j'ai une réaction très mauvaise face aux événements,
 même lorsque j'aimerais avoir raison.

C'est ce que les psychologues appellent notre surmoi
 \ocadr l'image de notre être idéal, ce que nous voulons vraiment être,
 et ce que nous serions si les circonstances étaient différentes.
 À l'opposé, nous avons notre moi véritable
 \ocadr ce que nous sommes vraiment.
 Les psychologues nous disent que nos problèmes mentaux
 sont parfois causés par la disparité entre les deux.
 Si le vrai vous est tiraillé loin du vous idéal, alors il est probable
 que vous ayez des conflits mentaux importants.
 Plus vous rapprochez le vous idéal du vous réel,
 plus vous serez une personne équilibrée.

Si vous allez voir un psychologue parce que vous avez des troubles mentaux,
 il va essayer de découvrir ce que vous pensez que vous devriez être
 \ocadr le vous idéal \fcadr{}
 et où vous échouez dans votre personne réelle.
 Souvent, il va ensuite essayer de descendre votre idéal.
 Il va tenter de vous montrer que vos valeurs sont si hautes et pures
 qu'elles sont impossible à atteindre.
 Souvent, il va tenter de descendre vos valeurs
 afin de supprimer vos conflits internes.

Pourtant, lorsque le Seigneur travaille en nous, Il fait le contraire.
 Il essaie d'amener le vrai vous plus proche du vous idéal.
 Un homme travaillant sur le problème descendrait le vous idéal ;
 le Seigneur en travaillant sur le problème élèverait le vous réel
 pour le faire coïncider avec l'idéal.
 Mais Dieu nous demande ce que nous ne pouvons pas atteindre,
 ce que nous ne pouvons pas donner.


\section*{La provision de Dieu}

Il n'y a aucun moyen pour moi de remplir l'idéal divin
 pour ma vie, donc Dieu, réalisant cela, a mis de côté
 des ressources pour moi.
 Sachant que je ne peux pas atteindre Son idéal divin,
 Dieu a envoyé Son Fils unique pour prendre tous mes échecs,
 tous mes péchés, tous mes manquements, et pour accepter la responsabilité
 pour eux et mourir à ma place.
 Dieu, sachant que je ne peux pas remplir l'idéal divin,
 a inauguré un plan de substitution, et ainsi ce que Dieu
 me demande maintenant est seulement de croire en Son fils, Jésus-Christ.

Je peux faire cela!
 Bien que je ne puisse pas être parfait
 comme Dieu me l'a demandé \emph{dans l'idéal},
 je peux croire en Jésus-Christ,
 et c'est la condition \emph{réelle} de Dieu à mon égard.
 Vous voyez, Dieu a maintenant rendu le royaume de Dieu ouvert
 et accessible à nous tous, car tous ce que nous avons à faire
 est de croire en Jésus-Christ.
 Lorsque les gens sont venus vers Jésus et ont demandé~:
 \og Que devons-nous faire pour effectuer les œuvres de Dieu ?\fg{},
 Jésus a répondu~:
 \og C'est l'œuvre de Dieu que vous croyiez en Celui qu'Il a envoyé. \fg{}
 Vous ne pouvez pas vous tenir devant Dieu au jour du jugement
 et essayer de vous excuser en disant~:
 \og Eh bien, Dieu, je ne pouvais tout simplement pas être parfait.
 Je suis juste un être humain, j'avais tous ces défauts,
 et je ne pouvais pas remplir tes conditions,
 donc j'ai simplement laissé tomber car j'ai bien vu
 que ça ne valait pas le coup d'essayer. \fg{}
 Dieu rejettera votre excuse car Dieu vous a seulement demandé
 de croire en Jésus-Christ, la provision qu'Il a faite pour vos erreurs
 et votre personne pécheresse.
 Dieu a rendu le royaume des cieux accessible à tous.
 Vous n'avez pas besoin d'être parfait pour y aller.
 Tous ce que vous avez à faire est de croire en la provision
 de Dieu par Jésus-Christ.

Mais lorsque vous croyez en Jésus-Christ,
 et que vous ouvrez la porte de votre cœur
 et que vous l'invitez à entrer,
 alors l'Esprit de Dieu entre et commence à travailler en vous
 et à vous changer.
 La Bible dit~:
 \og Si quelqu'un est en Christ, il est une nouvelle créature.
 Les choses anciennes sont passées ; voici~: (toutes choses)
 sont devenues nouvelles \fg{} (\ibibleverse{IICo}(5:17)).
 L'Esprit de Dieu commence à travailler dans votre vie pour faire
 en vous ce que vous ne pouviez pas faire pour vous-même.
 L'Esprit de Dieu commence Son travail de changement en vous,
 vous renforçant, vous aidant, vous conformant à l'image de Jésus-Christ.


\section*{L'idéal de Dieu}

Lorsque nous regardons autour de nous aujourd'hui pour chercher
 à comprendre Dieu par sa création
 \ocadr le but de Dieu pour l'homme et son intention
 lorsqu'il a créé l'homme et l'a placé sur la terre \fcadr{}
 nous ne pouvons pas découvrir cette vérité,
 car nous ne voyons pas l'homme remplir cet idéal.
 Le seul endroit où nous pouvons découvrir ce que Dieu voulait vraiment
 pour l'homme est en Jésus-Christ.
 Il est ce que Dieu voulait lorsqu'au cours de ce concile divin, ils ont dit~:
 \og Faisons l'homme à notre image, selon notre ressemblance. \fg{}

Que voulait faire Dieu ?
 Regardez Jésus-Christ, et vous le saurez, car Jésus a dit~:
 \og Je fais toujours ce qui est agréable au Père. \fg{}
 Le Père a dit à propos du Christ~:
 \og Celui-ci est mon Fils bien-aimé, en qui j'ai mis toute mon affection. \fg{}
 Lorsque nous regardons Jésus-Christ,
 nous voyons ce que Dieu a voulu que l'homme soit.
 Nous ne pouvons pas regarder Adam, car Adam a failli,
 même si Dieu ne voulait pas que l'homme faillisse.
 Nous ne pouvons pas nous regarder nous-mêmes,
 car nous avons failli, même si Dieu n'a pas voulu que nous faillissions.
 Mais si nous regardons Jésus-Christ, nous voyons là l'idéal divin,
 ce que Dieu a voulu lorsqu'Il a créé l'homme.
 C'est le but de Dieu que, par le travail de son Esprit-Saint dans votre vie,
 et par la puissance de l'Esprit à vous apporter des changements,
 Il vous amène à être à la ressemblance ou à l'image de Jésus-Christ.

Dans \ibibleverse{Ep}(4:13), nous voyons ce que Dieu veut faire en nous.
 Paul déclare~:
 \og Jusqu'à ce que nous soyons tous parvenus à l'unité de la foi
 et de la connaissance du Fils de Dieu, à l'état d'homme fait,
 à la mesure de la stature parfaite du Christ. \fg{}
 C'est ce en quoi Dieu œuvre en nous aujourd'hui.
 C'est le travail que Dieu cherche à accomplir dans nos vies
 par Son Esprit-Saint, nous amenant à la perfection de l'homme,
 à la mesure de la stature de la plénitude du Christ.
 Dans \ibibleverse{Rm}(8:29), nous lisons
 ce qu'est le travail de l'Esprit-Saint en nous,
 comment il nous conforme à l'image du Fils de Dieu.
 C'est le but prédestiné de Dieu pour nous
 que, par l'Esprit-Saint, il nous conforme à l'image de Son Fils.


\section*{Comment ça marche}

Dans \ibibleverse{IICo}(3:6-18), Paul parle de la période
 de l'Ancien Testament, lorsque Dieu a donné la loi.
 Quand Moïse est descendu de la montagne après avoir séjourné
 dans la gloire de la présence de Dieu,
 sa figure brillait tellement qu'il devait mettre un voile sur son visage
 lorsqu'il parlait au peuple (verset \ibiblevs{IICo}(3:13)).
 Mais au verset \ibiblevs{IICo}(3:18), en contraste avec ce voile,
 Paul a dit~:
 \og Nous tous, qui le visage dévoilé, reflétons comme un miroir
 la gloire du Seigneur, nous sommes transformés en la même image,
 de gloire en gloire, comme par le Seigneur,
 l'Esprit. \fg{}

C'est lorsque je regarde l'idéal divin de Dieu en Jésus-Christ
 que l'Esprit-Saint travaille en moi, me changeant de gloire en gloire
 à l'image de Christ.
 Je crois que c'est un travail de toute une vie.
 L'Esprit-Saint est loin d'avoir terminé Son œuvre dans ma vie.
 Cependant, grâce à Dieu, Il est au travail, et grâce à Dieu,
 je ne suis plus ce que j'étais
 \ocadr je suis en train d'être changé!
 Ces changements prennent place, bien que je confesse qu'ils prennent place
 trop lentement à mon goût.
 J'aimerais beaucoup qu'ils soient terminés tous d'un coup.

Lorsqu'il arrive que l'Esprit-Saint me montre une zone qui a besoin de travail,
 lorsqu'Il allume la lumière de façon que je me voie tel que je suis,
 et à quel point je suis loin de ce que Dieu voudrait faire de moi,
 je pense immédiatement~:
 \og Oh, allons-y, gagnons du terrain. \fg{}
 J'y vais et j'essaie de faire de mon mieux,
 et je continue à essayer d'être meilleur,
 mais plus j'essaie et pire cela devient,
 jusqu'à ce que j'arrive à la défaite et que je laisse tomber.
 Alors, je m'écrie~:
 \og Oh Dieu, je suis si mauvais. Je ne peux pas le faire. \fg{}
 Il répond~:
 \og Bien. Maintenant, peux-tu t'écarter sur le côté et Me laisser travailler?
 Tu est resté sur Mon passage. \fg{}
 Ma justice personelle ne l'intéresse pas ;
 mon aide ne l'intéresse pas.
 Il veut faire Son travail en moi sans être entravé
 par mes efforts maladroits,
 car même si Il utilisait mes efforts maladroits
 pour m'aider vers la victoire,
 j'irais me vanter de ma maladresse plutôt que de mon Dieu.
 Dieu me laisse échouer jusqu'à ce que je crie à l'aide de désespoir.
 Lorsque je me rend à l'Esprit de Dieu et que je L'autorise
 à faire Son travail, Il me conforme à l'image de Jésus-Christ.


\section*{Je ne fais pas le poids}

Je dois être amené au point où je prend conscience et
 je reconnais que je ne peux pas me débarrasser de la chair
 ni de ses désirs et faiblesses.
 Je ne fais pas le poids. Tant que je lutte et que j'essaie,
 je ne peux pas y arriver ; j'échouerai.

Nous sommes des pécheurs ; nous devons reconnaître ce fait ;
 il n'y a rien que nous puissions faire à ce sujet par nous-mêmes.
 Nous devons faire appel à une puissance plus grande que la nôtre.
 C'est ce que faisait Paul dans \ibibleverse{Rm}(7:) lorsqu'il a dit~:
 \og Malheureux que je suis! Qui me délivrera de ce corps de mort \fg{}
 (verset \ibiblevs{Rm}(7:24)).
 Il a fait appel à une puissance plus grande que lui-même,
 et lorsqu'il l'a fait, il a trouvé la puissance.

Quand, le visage dévoilé, nous reflétons la gloire du Seigneur,
 nous sommes transformés de gloire en gloire.
 Dieu nous change, Il change nos attitudes.
 Par notre nature héritée d'Adam, nous sommes très égoïstes
 et centrés sur nous-mêmes.
 Cela commence très tôt dans la vie ; on peut le voir chez de jeunes enfants
 lorsqu'ils disent \og le mien \fg{}.
 C'est l'un des premiers mots qu'ils apprennent
 en dehors de \og maman \fg{} et \og papa \fg{}.
 On les voit s'attacher à leur possession, et on n'ose même pas
 essayer de leur prendre ou bien on peut être sûr qu'on va les entendre.
 Si vous prenez leur biberon, vous allez devoir vous battre,
 avec force cris, braillements et coups de pieds.
 Il est heureux qu'ils soient aussi petits et faibles qu'ils le sont,
 sinon ils réduiraient leur lit en pièces!
 Ce sont des petits enfants bénis,
 mais ils ont hérité de la nature d'Adam.

Tant que je suis égoïste et centré sur moi-même,
 je ne suis pas ce que Dieu veut que je sois.
 Dieu ne veut pas que je sois centré sur moi-même.
 Dieu ne veut pas que je sois intéressé par ma propre personne
 en premier lieu.
 Le Seigneur veut que je m'intéresse aux autres personnes et que
 je partage ce que j'ai avec eux en fonction de leur besoin.
 C'est ce dont Jésus parlait dans \ibibleverse{Mt}(5:) lorsqu'il a dit~:
 \og Soyez donc parfaits, comme votre Père céleste est parfait. \fg{}
 Cependant, cela n'est pas naturel, mais \emph{surnaturel}, et nous pouvons seulement
 l'atteindre par la puissance surnaturelle de l'Esprit-Saint
 qui vient et change notre attitude au sujet de notre propre personne
 et de nos possessions.

Non seulement Il change notre attitude (ce qui est la chose la plus importante),
 mais l'attitude changée modifie également l'action.
 Trop souvent, nous essayons de faire le contraire.
 Il semble que notre philosophie soit de changer les actions d'une personne
 et d'espérer qu'en changeant ses actions, nous allons changer son attitude.
 Les psychologues disent que si nous jouons une émotion, nous allons obtenir
 l'émotion correspondante.
 Dieu, toutefois, s'intéresse à changer réellement l'attitude du cœur,
 et cette attitude changée amène une action modifiée dans notre vie.


\section*{Le changement de l'intérieur}

L'Évangile et l'Esprit-Saint travaillent de l'intérieur vers l'extérieur.
 Mon cœur est changé et mon attitude est changée,
 et ainsi mes actions reflètent les attitudes changées à l'intérieur.
 L'Esprit-Saint qui travaille en moi me change de gloire en gloire,
 m'amenant à l'image de Jésus-Christ. Comment?
 Lorsque je Le contemple le visage dévoilé.
 Comment Le vois-je? Je peux seulement Le voir dans la Parole,
 et l'Esprit rend la Parole vivante à mon cœur.

Pierre nous dit dans son second épître que Dieu nous a donné
 des promesses extrêmement riches et précieuses,
 et que par elles nous prenons part à la nature divine.
 C'est là, dans le Livre, mais vous devez percevoir Jésus dans le Livre;
 vous devez L'y rechercher.
 Beaucoup de personnes lisent la Bible avec leur visage voilé.
 Il faut l'Esprit-Saint pour ouvrir la Bible, pour enlever le voile
 de leurs yeux afin qu'ils puissent comprendre.
 Le travail de l'Esprit-Saint en nous est si important.
 Nous ne pouvons pas être ce que Dieu veut que nous soyons
 à moins que l'Esprit-Saint ne travaille au sein de notre vie.

Personne ne me connaît aussi bien que moi-même, sinon le Seigneur,
 et Il me connaît mieux que je me connais moi-même.
 J'ai découvert que bien des choses que je pensais
 à mon sujet n'étaient pas vraies.
 Beaucoup de choses dans l'image idéale que je me faisais de moi-même
 se sont avérées différentes de ce que j'aurais pensé.
 Lorsque je m'observais au travers verres teintés en rose,
 j'avais un aspect très rose!
 Mais quand l'Esprit-Saint a cassé mes lunettes,
 j'ai été très surpris. Mais Il devait le faire.
 Il devait détruire les illusions que j'avais à mon sujet afin de s'occuper
 de ces parties de ma vie que je refusais de reconnaître devant Lui.
 Il devait les mettre en lumière, les révéler dans toute leur laideur
 afin qu'Il puisse ensuite travailler à me libérer d'elles.


\section*{Un fils de Dieu}

Je sais que je suis désormais un fils de Dieu,
 non pas grâce à ma justice personnelle,
 mais par ma foi en Jésus-Christ.
 \og À tous ceux qui l'ont reçue, elle a donné le pouvoir
 de devenir enfants de Dieu,
 à ceux qui croient en son nom \fg{} (\ibibleverse{Jn}(1:12)).
 Parce que j'ai cru en Jésus-Christ, Dieu m'a donné le pouvoir
 de devenir un fils de Dieu.
 Donc je sais maintenant que je suis un fils de Dieu,
 et ceci est une grande gloire pour moi.
 Si je suis un fils, alors je suis un héritier.
 Je suis un héritier de Dieu et un héritier conjoint avec Jésus-Christ,
 et je ne connais rien qui puisse être plus glorieux que cela.

Dieu travaille dans mon être intérieur par son Esprit-Saint,
 mais mon problème est que bien que je sois nouveau à l'intérieur,
 je suis toujours le vieux Chuck à l'extérieur.
 Mais le vieux Chuck est en fait mort, si bien que je dois traîner
 ce vieux corps partout jusqu'au jour où Dieu m'en délivrera enfin.
 Avec ma pensée et avec mon cœur, je sers le Seigneur,
 mais bien souvent avec mon corps,
 je suis contrôlé par mes propres désirs égoïstes.
 Ce vieux corps se fait lourd et difficile à porter.
 Il y a des fois où je me plaint, désirant être délivré,
 non pas pour être un esprit décharné, mais plutôt que je puisse
 être rhabillé avec ce nouveau corps qui vient du ciel,
 que je sois comme Lui, comme je Le vois tel qu'Il est.

Je suis un fils de Dieu. J'ai un esprit renouvelé et un corps non racheté.
 Dieu ne va pas prendre ce corps au ciel, gloire au Seigneur!
 \og Il faut en effet que ce (corps) corruptible
 revête l'incorruptibilité. \fg{}
 Un changement commence alors à prendre place~:
 \og Nous tous, qui le visage dévoilé,
 reflétons comme un miroir la gloire du Seigneur,
 nous sommes transformés. \fg{}
 Le mot qui est traduit par \og transformé \fg{}
 dans le texte grec est \emph{metamorphoo}.
 Ce mot est utilisé pour décrire un changement du corps,
 comme lorsqu'une chenille se change en papillon.
 Paul dit que toute création grogne et souffre ensemble
 jusqu'à ce jour, dans l'attente de la manifestation des fils de Dieu,
 c'est-à-dire de la rédemption de nos corps.


\section*{Comme Christ pour toujours}

Je ne devrais jamais me satisfaire de moi-même ou de mon état actuel
 de développement jusqu'à ce que je sois comme Jésus-Christ.
 David a dit~:
 \og Dès le réveil, je me rassasierai de ton image. \fg{}
 \NdT{La version anglaise KJV traduit~:
 \og Je serais satisfait lorsque je me réveillerais à ton image \fg{}}
 Le jour où je me réveillerai et je respirerai,
 et il n'y aura aucun brouillard, et je me sentirai si différent,
 et je réaliserai que ce corruptible a revêtu l'incorruptible,
 alors je serai satisfait, car je serai comme Lui, car je Le verrai
 comme tel qu'Il est. C'est ce vers quoi l'Esprit me guide,
 c'est le but du travail de l'Esprit-Saint dans ma vie.
 Il ne sera pas satisfait jusqu'à ce qu'Il ait fini de m'amener
 dans une conformité totale à l'image de Christ.

Le changement final prendra place au retour de Jésus-Christ pour moi,
 que ce soit par la mort ou par l'enlèvement de l'Église.
 À ce moment là, la vieille nature sera anéantie et le changement final
 sera effectué, mais je ne devrais pas attendre ce jour.
 Dès maintenant, lorsque je regarde vers Jésus, le processus de changement
 est en cours. Nous devrions être plus proche de l'image de Christ
 cette année que l'année dernière, et l'année prochaine plus encore
 que cette année, car nous grandissons dans la grâce et dans Sa connaissance,
 et l'Esprit qui travaille en nous devrait nous rendre
 de plus en plus à Son image.


\section*{Mature ou simplement vieux?}

J'aime vraiment être dans l'entourage de saints
 qui ont marché avec le Seigneur depuis 50, 60, voire 70 ans.
 Je veux dire ceux qui se sont vraiment développés dans leur démarche.
 Je sais qu'il y en a qui sont là depuis 50, 60 ou 70 ans mais qui sont
 toujours dans leur berceau spirituel, et c'est une chose tragique.
 Si vous voyez un enfant qui a tout juste quatre ou cinq mois,
 et qui agite ses bras d'excitation et dit~:
 \og Pa pa pa\dots{} \fg{}, vous vous dites~:
 \og C'est merveilleux, regardez comme il est intelligent,
 quel enfant magnifique! \fg{} 
 Mais si votre enfant avait 21 ans, et qu'en entrant dans sa chambre
 il était allongé sur le lit et se mettait à sourire en disant~:
 \og Pa pa \fg{}, ce ne serait plus une émotion excitante,
 ce serait tragique.

C'est la tragédie qui touche un si grand nombre de personnes
 dans l'Église aujourd'hui. Après 15 ou 20 ans,
 ils sont toujours au même niveau de développement.
 Ils sont toujours à s'agiter dans leur berceau.
 Ils ont toujours les mêmes petits problèmes.
 Ils s'énervent toujours à cause du message du dimanche précédent,
 et ils sont toujours divisés par leurs petites querelles de chapelles.
 Ils n'ont pas progressé du tout. Ce sont des monstruosités spirituelles
 car il n'ont jamais connu de développement, et le problème est qu'il y a
 tant de ce type de chrétiens que ce n'est même plus une curiosité.
 Ils sont partout. Ils n'ont simplement pas creusé dans la Parole de Dieu
 pour chercher la face du Seigneur. Ils n'ont pas autorisé la Parole de Dieu
 à les imprégner et l'Esprit de Dieu à les enseigner vraiment et
 à les instruire sur les choses du Seigneur,
 ou à leur révéler Jésus dans la Parole.


\section*{Jusqu'où aller?}

Oh, qu'il serait bon que nous nous soumettions à l'Esprit-Saint
 dès maintenant, pour qu'Il accomplisse son œuvre en nous,
 qu'Il nous conforme à l'image de Jésus-Christ!
 Jusqu'où doit-Il aller dans votre vie?
 Avez-vous déjà effectué un de ces petits tests de personnalité pour savoir
 si vous êtes plutôt attirant ou plutôt un cas social?
 Dans \ibibleverse{ICo}(13:), il y a un simple petit test d'auto-analyse
 que vous pouvez effectuer pour voir jusqu'où l'Esprit est allé
 dans votre vie dans un seul domaine~: celui de l'amour,
 un des domaines les plus importants.
 À partir du verset \ibiblevs{ICo}(13:4), la définition
 de ce mot \og amour \fg{} est donnée~:
 \og L'amour est patient, l'amour est serviable, il n'est pas envieux;
 l'amour ne se vante pas, il ne s'enfle pas d'orgueil,
 il ne fait rien de malhonnête, il ne cherche pas son intérêt,
 il ne s'irrite pas, il ne médite pas le mal, il ne se réjouit pas
 de l'injustice, mais il se réjouit de la vérité; il pardonne tout,
 il croit tout, il espère tout, il supporte tout.
 L'amour ne succombe jamais. \fg{}

Vous vous dites~:
 \og Qu'est-ce que ça a à voir avec moi ? \fg{}
 Enlevez le mot \og amour \fg{} et remplacez-le par votre nom,
 puis lisez la liste à nouveau.
 \og Chuck est patient, Chuck est serviable, il n'est pas envieux;
 Chuck ne se vante pas, il ne s'enfle pas d'orgueil, il ne fait rien
 de malhonnête, il ne cherche pas son intérêt, il ne s'irrite pas,
 il ne médite pas le mal, etc. \fg{}
 Tout ce qui paraît incorrect dans ce texte mesure à quel point j'ai échoué
 dans l'atteinte de ce que Dieu veut que j'atteigne.
 Dieu, aide-moi à me soumettre à l'Esprit-Saint afin qu'Il puisse accomplir
 Son œuvre en moi, afin que ce que j'ai essayé de faire sans succès
 \ocadr ce que je veux mais ne peux accomplir ou atteindre \fcadr
 soit accompli pour moi par Sa puissance.

