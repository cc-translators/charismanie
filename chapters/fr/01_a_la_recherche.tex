\chapter{\`A la recherche de la r\'eponse}

\lettrine{J}{'ai passé}
 une grande partie de mon enfance et de mon
 adolescence à tenter de prouver que j'étais normal même si je n'allais ni au
 cinéma, ni en boîte de nuit. Dans l'église pentecôtiste que je fréquentais,
 aller au cinéma ou aller danser étaient considérés comme de terribles péchés.

Comme je ne pouvais pas partager les activités profanes
 de mes amis, je leur demandais de venir avec moi à l'église,
 car on nous exhortait sans cesse à témoigner de Christ en amenant
 des amis à l'église. Le problème, c'était que, presque tous les dimanches,
 le pasteur mettait en garde contre les
 vices de Hollywood, la danse, la boisson, et la cigarette.
 Il disait\frcolon{} \Og Si Dieu voulait que l'homme fume, il lui aurait mis une
 cheminée sur la tête.\Fg{}
 De plus, le service était toujours interrompu par deux ou trois
 \Og messages en langues \Fg{} et leurs interprétations.

Bien souvent, lorsque j'étais assis avec les amis non sauvés que j'avais
 amenés à l'église, M\up{me}~Newman se mettait à respirer de façon bizarre.
 J'avais appris que cela indiquait qu'elle allait se mettre à parler en
 langues, et je me mettais donc rapidement à prier\frcolon{}
 \Og Oh Dieu, ne parle pas en langues aujourd'hui s'il te plaît.
 Mes amis ne comprendront pas. \Fg{}
 Soit Dieu ne m'entendait pas, soit M\up{me}~Newman n'écoutait pas Dieu,
 car elle se levait, toute tremblante, et déclamait le message de Dieu pour le jour
 d'une voix forte et haut perchée. Je mourais de honte tandis que mes amis
 gloussaient à côté de moi. J'espérais qu'ils ne commettaient pas le péché
 impardonnable.

J'étais toujours tendu après le culte, m'attendant à ce que mes amis me
 posent la question inévitable\frcolon{}
 \Og Qu'est-ce que c'était que ça ? \Fg{}
 J'avais du mal à l'expliquer parce que je ne le comprenais pas complètement
 moi-même.
\noclub[3]

Quand j'étais enfant, je ne pouvais pas m'empêcher de me poser des questions sur
 ces \Og messages en langues \Fg{} que j'entendais. Parfois, un message court
 était suivi d'une longue interprétation ou, au contraire, un message long était
 suivi d'une interprétation courte. D'autres fois, je remarquais des phrases
 répétées dans le message en langues et je me demandais pourquoi il n'y avait
 pas de phrases répétées correspondantes dans les interprétations.

\section{Les questions s'accumulent}

D'autres choses concernant l'église que je fréquentais me gênaient.
 Vu que nous étions l'église la plus spirituelle et la plus remplie
 de puissance de la ville,
 je me demandais pourquoi les autres églises avaient beaucoup plus
 de membres. On me répondait que la plupart des
 gens cherchent un chemin facile vers le paradis, et que les autres églises
 étaient plus grandes parce qu'elles disaient aux gens ce qu'ils voulaient
 entendre. Si notre église le faisait, elle aussi serait remplie,
 mais de personnes en partance pour l'enfer.

Un autre problème que j'avais avec notre église était son manque d'amour. Je
 savais que le fruit de l'Esprit est l'amour,
 \index{Esprit!fruit de l'\textasciitilde{}}                                    
 donc je ne pouvais pas comprendre
 pourquoi il y avait autant de divisions dans l'église. Il semblait
 qu'il y avait toujours quelques membres qui voulaient se débarrasser du pasteur et
 d'autres qui le soutenaient. Il était si fréquent que des gens quittent notre église que,
 si tous les anciens membres de l'assemblée étaient revenus en même temps, nous aurions
 eu la plus grande église de la ville!

D'une certaine manière, quitter notre église revenait un peu à quitter le
 Seigneur. Ceux qui étaient partis avaient certainement régressé dans la foi dans leur quête
 d'un chemin plus aisé vers le paradis. Pourtant, je me prenais souvent à
 souhaiter pouvoir aller à la \Og Community Church \Fg{} ou à l'église presbytérienne. Mais
 le dimanche soir, convaincu que ce désir constituait un péché de régression dans la foi,
 je~m'avançais vers l'autel pour être \Og sauvé \Fg{} à nouveau.
 
J'essayais de prouver que j'étais normal en excellant à l'école. Je travaillais
 pour être le gamin le plus intelligent de la classe, le coureur le plus rapide de
 l'école, et celui qui pouvait frapper la balle le plus loin au baseball.
 Malheureusement, la plupart des autres gamins de mon école du dimanche à l'église
 essayaient de prouver qu'ils étaient normaux en fumant, en buvant, et en
 traînant avec les bandes de durs à l'école. Très peu d'entre eux sont restés à
 l'école du dimanche après leur entrée au collège. Par la grâce de Dieu et avec
 l'aide de mes parents très engagés, j'ai survécu tant bien que mal.

\section{Les r\'esultats de ma qu\^ete}

Aussi étrange que cela puisse paraître, je suis aujour\-d'hui convaincu que
 l'orthodoxie morte d'un grand nombre d'églises pourrait être combattue par l'exercice des
 dons de l'Esprit-Saint à l'œuvre au sein du corps du Christ. Pas les excès non scriptuaire que
 j'avais observés étant enfant, mais les dons exercés d'une manière solide,
 scriptuaire, avec la Parole de Dieu comme autorité finale guidant notre foi et
 notre pratique.

Avec ceci en tête, je me suis mis à chercher dans les Écritures une
 approche saine et équilibrée de l'Esprit-Saint et de son œuvre dans l'Église
 aujourd'hui. Il devait bien y avoir une position inter\-mé\-di\-aire entre celle des
 pentecôtistes, qui accordent une trop grande importance à l'ex\-pé\-rience, et les
 fondamentalistes, qui, dans leur désir d'avoir absolument raison,
 en sont trop souvent arrivés à avoir raison \emph{jusqu'à en mourir}\NdT{Jeu de mots en anglais\frcolon{} \emph{to be dead right} veut dire
 \Og avoir parfaitement raison \Fg{},
 mais peut aussi suggèrer que cette position conduit à la mort (spirituelle).}.
 Les résultats de ma quête sont en partie relatés dans ce livre, et je prie que
 Dieu l'utilise pour vous amener à la plénitude de la vie remplie de
 l'Esprit.

Les \emph{charismes} sont une belle et naturelle onction de l'Esprit de Dieu
 sur la vie d'une personne, qui lui permet de faire l'œuvre~de Dieu. C'est par
 cette dynamique particulière de l'Esprit de Dieu que la gloire et l'amour de Dieu
 semblent rayonner d'une personne.

La \emph{charismanie} est une tentative de la chair pour simuler les charismes.
 C'est tout effort accompli pour faire l'œuvre de l'Esprit par les énergies ou les
 ressources de la chair (la vieille nature égoïste d'une personne). C'est
 une vogue soi-disant spirituelle qui substitue la transpiration à l'inspiration. C'est
 l'utilisation du génie, de l'énergie et des astuces de l'homme comme substitut
 à la sagesse et aux capacités de Dieu. Elle peut se manifester sous des formes
 aussi diverses que des sessions de planification et de stratégie, la mise en place de
 programmes pour amener la croissance de l'église, la levée de fonds
 pour le budget de l'église, ou les chahuts sauvages et désordonnés en langues
 qui interrompent le message du dimanche matin. Tout ce qui manque d'une saine base
 biblique et fait preuve d'un manque de confiance dans l'Esprit-Saint pour
 accomplir ses desseins dans l'Église sans faire appel aux attributs
 et aux capacités de l'homme, est l'œuvre de la chair.

\section{La position équilibrée}

Ce livre cherche à présenter une position équilibrée par rapport aux Écritures
 entre les détracteurs qui disent\frcolon{} \Og C'est le diable qui les fait agir \Fg{}
 et les fanatiques qui disent\frcolon{} \Og L'Esprit-Saint me l'a fait faire. \Fg{}
 Il montre également qui est l'Esprit-Saint et décrit Son propre travail dans
 le monde, l'Église et la vie du croyant.

Nous ne vous demandons pas d'accepter aveuglément toutes nos conclusions, mais nous
 vous encourageons à rechercher dans les Écritures pour voir si les choses sont
 bien ainsi. \Og Examinez toutes choses; retenez ce qui est bon \Fg{}
 (\ibibleverse{ITh}(5:21)).
% This page is full already, set the style only
%\closechapter
\thispagestyle{chapterend}

