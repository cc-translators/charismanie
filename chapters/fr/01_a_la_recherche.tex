\chapter{\`A la recherche de la r\'eponse}

\lettrine{J}{'ai passé}
 une grande partie de mon enfance et de mon
 adolescence à tenter de prouver que j'étais normal même si je n'allais ni au
 cinéma ni en boîte de nuit. Dans l'église pentecôtiste que je fréquentais,
 aller au cinéma ou aller danser étaient considérés comme de terribles péchés.

Étant donné que je ne pouvais pas partager les activités profanes
 de mes amis, je leur demandais de venir avec moi à l'église,
 puisque nous étions sans cesse exhortés à témoigner de Christ en amenant
 des amis à l'église. Le problème était que, presque tous les dimanches,
 le pasteur mettait en garde contre les
 maux de Hollywood, la danse, la boisson, et la cigarette.
 Il disait~: \og Si Dieu voulait que l'homme fume, il lui aurait mis une
 cheminée au-dessus de la tête.\fg{}
 En plus de cela, le service était sans cesse interrompu par deux ou trois
 \og messages en langues \fg{} et leurs interprétations.

Bien souvent, lorsque j'étais assis avec mes amis non sauvés que j'avais
 amenés à l'église, M\up{me}~Newman se mettait à respirer bizarrement.
 J'avais appris que cela arrivait quand elle allait se mettre à parler en
 langues, et je me mettais donc rapidement à prier~:
 \og Oh Dieu, ne parle pas en langues aujourd'hui s'il te plaît.
 Mes amis ne comprendront pas. \fg{}
 Soit Dieu ne m'entendais pas, soit M\up{me}~Newman n'écoutait pas Dieu,
 car elle se levait, toute tremblante, et transmettait le message du jour de Dieu
 d'une voix forte et haut perchée. Je mourais intérieurement tandis que mes amis
 gloussaient à côté de moi. J'espérais qu'ils ne commettaient pas le péché
 impardonnable.

J'étais toujours tendu après le service lorsque j'attendais que mes amis me
 posent la question inévitable~:
 \og Qu'est-ce que c'était que ça ? \fg{}
 J'avais du mal à l'expliquer parce que je ne le comprenais pas complètement
 moi-même.
\noclub[3]

Quand j'étais enfant, je ne pouvais pas m'empêcher de me poser des questions sur
 ces \og messages en langues \fg{} que j'entendais. Parfois, un message court
 était suivi d'une longue interprétation ou, au contraire, un message long était
 suivi d'une interprétation courte. D'autres fois, je remarquais des phrases
 répétées dans le message en langues et je me demandais pourquoi il n'y avait
 pas de phrases répétées correspondantes dans les interprétations.

\section{Les questions s'accumulent}

Il y avait d'autres choses qui me gênaient dans l'église que je
 fréquentais. Je me demandais pourquoi, si nous étions l'église la plus
 spirituelle de la ville et si nous avions la plus grande puissance, les autres
 églises avaient beaucoup plus de membres. On m'avait dit que la plupart des
 gens cherchaient un chemin facile vers le paradis, et que les autres églises
 étaient plus grandes parce qu'elles disaient aux gens ce qu'ils voulaient
 entendre. Si notre église avait fait cela, elle aurait été pleine également
 (de personnes condamnées à l'enfer).

Un autre problème que j'avais avec notre église était son manque d'amour. Je
 savais que le fruit de l'Esprit est l'amour,
 \index{Esprit!fruit de l'\textasciitilde{}}                                    
 donc je ne pouvais pas comprendre
 pourquoi il y avait autant de séparations dans l'église. Il semblait
 toujours y avoir des membres qui voulaient se débarrasser du pasteur et
 d'autres qui le soutenaient. Il était si fréquent que l'on quitte notre église que,
 si tous les anciens membres de l'assemblée étaient revenus, nous aurions eu la
 plus grande église de la ville!

D'une certaine manière, quitter notre église était un peu comme quitter le
 Seigneur. Ceux qui étaient partis étaient certainement retombés dans leur quête
 d'un chemin plus aisé vers le paradis. Pourtant, je me prenais souvent à
 espérer pouvoir aller à la Community Church ou à l'église presbytérienne. Mais
 le dimanche soir, je me sentais convaincu de mon désir de \og faire marche
 arrière \fg{}, et je~m'avançais vers l'autel pour être à nouveau
 \og sauvé \fg{}.

J'essayais de prouver que j'étais normal en excellant à l'école. Je travaillais
 pour être le gamin le plus intelligent de l'école, le coureur le plus rapide de
 l'école, et celui qui pouvait batter plus loin que tous les autres au baseball.
 Malheureusement, la plupart des autres gamins de mon école du dimanche
 essayaient de prouver qu'ils étaient normaux en fumant, en buvant, et en
 traînant avec les bandes de durs à l'école. Très peu d'entre eux sont restés à
 l'école du dimanche après leur entrée au collège. Par la grâce de Dieu et avec
 l'aide de mes parents très engagés, j'ai survécu tant bien que mal.

\section{Les r\'esultats de ma qu\^ete}

Aussi étrange que cela puisse paraître, je suis convaincu aujour\-d'hui que
 l'orthodoxie morte d'un grand nombre d'églises pourrait être améliorée par les
 dons de l'Esprit-Saint à l'œuvre au sein du corps du Christ. Pas l'excès non scripturaire que
 j'avais observé étant enfant, mais les dons exercés d'une manière solide,
 scripturaire, avec la Parole de Dieu comme autorité finale guidant notre foi et
 notre pratique.

En gardant cela à l'esprit, j'ai commencé à chercher dans les Écritures une
 approche saine et équilibrée de l'Esprit-Saint et de son travail dans l'Église
 aujourd'hui. Il devait y avoir une position inter\-mé\-di\-aire entre les
 pentecôtistes, avec leur emphase exagérée sur l'ex\-pé\-rience, et les
 fondamentalistes, qui, dans leur désir d'avoir absolument raison,
 en sont trop souvent arrivés à avoir raison \emph{à mort}
 \NdT{Jeu de mots en anglais~: \emph{to be dead right} veut dire
 \og avoir parfaitement raison \fg{},
 mais exprime aussi que cette position conduit à la mort (spirituelle).}.
 Les résultats de ma quête sont relatés en partie dans ce livre, et je prie que
 Dieu l'utilise pour vous amener dans la plénitude de la vie remplie de
 l'Esprit.

Les \emph{charismes} sont une bénédiction belle et naturelle de l'Esprit de Dieu
 sur la vie d'une personne, qui lui permet de faire l'œuvre~de Dieu. C'est par
 cette dynamique particulière de l'Esprit de Dieu qu'une personne paraît irradier de
 la gloire et de l'amour de Dieu.

La \emph{charismanie} est une tentative de la chair pour simuler les charismes.
 C'est tout effort pour faire l'œuvre de l'Esprit par les énergies ou les
 possibilités de la chair (la vieille nature égoïste d'une personne). C'est
 une tendance spirituelle qui substitue la transpiration à l'inspiration. C'est
 l'utilisation du génie, de l'énergie et des mimiques de l'homme comme substitut
 de la sagesse et des capacités de Dieu. On peut la trouver sous des formes
 aussi diverses que des sessions de planification et de stratégie, des
 programmes de planification pour la croissance de l'église, la levée de fonds
 pour le budget de l'église, ou les chahuts sauvages et désordonnés en langues
 qui interrompent le message du dimanche matin. Tout ce qui manque d'une base
 biblique saine et fait preuve d'un manque de confiance dans l'Esprit-Saint pour
 accomplir ses desseins dans l'Église sans faire appel aux attributs
 et aux capacités de l'homme, est l'œuvre de la chair.

\section{La position d'\'equilibre}

Ce livre cherchera à présenter une position équilibrée par rapport aux Écritures
 entre les détracteurs qui disent~: \og C'est le diable qui les fait agir \fg{}
 et les fanatiques qui disent~: \og L'Esprit-Saint me l'a fait faire. \fg{}
 Il montrera également qui est l'Esprit-Saint et décrira Son œuvre propre dans
 le monde, l'Église et la vie du croyant.

Nous ne vous demandons pas d'accepter aveuglément tous les prémices, mais nous
 vous encourageons à rechercher dans les Écritures pour voir si les choses sont
 bien ainsi. \og Mais examinez toutes choses; retenez ce qui est bon \fg{}
 (\ibibleverse{ITh}(5:21)).
% This page is full already, set the style only
%\closechapter
\thispagestyle{chapterend}

