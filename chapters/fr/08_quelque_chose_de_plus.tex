\chapter{Quelque chose de plus}

\lettrine{R}{écemment}, un jeune homme est venu me dire~:
 \og J'ai accepté Christ il y a plusieurs années,
 mais cela ne m'a jamais vraiment excité.
 Je trouvais que lire la Bible était inintéressant.
 En fait, mon esprit flânait, et je ne pouvais pas vraiment me concentrer
 sur la Parole. Je n'ai jamais vraiment su ce qu'était adorer Dieu,
 et ma vie de prière était erratique. Mais depuis que j'ai été rempli
 de l'Esprit il y a quelques mois, ma vie a complètement changé.
 J'ai un grand amour pour les choses de Dieu.
 Je semble ne jamais me rassasier de la Parole, et maintenant j'aime
 la communion fraternelle avec les croyants.
 Quel changement s'est produit dans ma vie depuis que j'ai été rempli
 de l'Esprit ! \fg{}

Cette histoire, avec des variations, m'a été racontée des centaines de fois
 par ceux qui ont découvert qu'il y a quelque chose de plus que de simplement
 avoir l'Esprit qui vient demeurer dans leur vie au moment de leur conversion.
 Nous reconnaissons effectivement que chaque croyant né de nouveau a l'Esprit
 qui demeure en lui. Paul déclare dans \ibibleverse{ICo}(6:19) que nos corps
 sont les temples de l'Esprit-Saint qui demeure en nous.
 Il déclare également dans \ibibleverse{ICo}(12:3) que vous ne pouvez pas
 appeler Christ Seigneur si ce n'est par l'Esprit.


\section*{L'Esprit et le croyant}

Il y a trois prépositions grecques utilisées dans le Nouveau Testament
 pour désigner les différentes relations de l'Esprit avec le croyant~:
 \emph{para}, \emph{en} et \emph{épi}.
 Dans \ibibleverse{Jn}(14:17), Jésus a dit à Ses disciples au sujet de
 l'Esprit-Saint~: \og Vous le connaissez, parce qu'il demeure près [para]
 de vous et qu'il sera en [en] vous. \fg{}
 Il y ici l'expression d'une relation à deux facettes~:
 \emph{para} (près) et \emph{en} (en).
 L'Esprit-Saint était avec nous avant notre conversion.
 Il est Celui qui nous a convaincu de péché et révélé Christ
 comme la solution. Lorsque nous avons accepté Jésus comme notre Sauveur
 et que nous l'avons invité dans nos vies, l'Esprit-Saint
 a commencé à demeurer en nous.

Mais Dieu a quelque chose en plus \ocadr la belle source de puissance
 par la relation \emph{épi}.
 Remarquez que c'est ce que Jésus promettait à Ses disciples juste avant
 Son ascension. Dans \ibibleverse{Lc}(24:49), Il a dit~:
 \og Et [voici] : j'enverrai sur [épi] vous ce que mon Père a promis \fg{} ou
 \og au-dessus de vous. \fg{}
 Dans \ibibleverse{Ac}(1:8), Il a dit~:
 \og Mais vous recevrez une puissance, celle du Saint-Esprit survenant
 sur [épi] vous. \fg{}

Nous lisons dans \ibibleverse{Ac}(10:44) que l'Esprit-Saint est descendu
 \og sur \fg{} les croyants des nations dans la maison de Cornélius~:
 \og Comme Pierre prononçait encore ces mots, le Saint-Esprit descendit
 sur [épi] tous ceux qui écoutaient la parole. \fg{}
 Dans \ibibleverse{Ac}(19:6), lorsque Paul a posé les mains sur les
 croyants Éphésiens, l'Esprit-Saint est venu sur [épi] eux.

Nous lisons dans \ibibleverse{Ac}(8:) que Philippe était allé à Samarie
 et leur préchait Christ ; beaucoup de gens ont cru le prêche de Philippe
 au sujet des choses du royaume de Dieu et du nom de Jésus-Christ,
 et ils ont été baptisés.
 S'il n'y a qu'un seul baptème (\ibibleverse{Ep}(4:5)),
 alors nous devons accepter à ce moment que les croyants Samaritains
 ont été baptisés par l'Esprit dans le corps de Christ
 (\ibibleverse{ICo}(12:13)), et que l'Esprit-Saint a commencé à demeurer
 en eux. Il est évident, cependant, qu'il y avait une relation plus poussée
 avec l'Esprit-Saint à recevoir, car lorsque l'église à Jérusalem a entendu
 que les Samaritains avaient reçu l'Évangile, ils ont envoyé Pierre et Jean
 vers eux afin qu'ils puissent prier pour qu'ils recoivent l'Esprit-Saint,
 car il n'était encore descendu sur [épi] aucun d'entre eux.

\section*{La vie qui déborde}

Quand Paul s'est rendu dans l'église d'Ephèse et a trouvé que les croyants
 y manquaient d'expérience, il leur a demandé, probablement par amour
 ou par joie et zèle~: \og Avez-vous reçu l'Esprit Saint
 quand vous avez cru ? \fg{}
 Si la relation pleine avec l'Esprit est atteinte en même temps
 que la conversion, la question n'a pas de sens.
 La question même supposait une relation plus profonde et au-delà
 de l'expérience de la conversion. Ce qui leur manquait était la relation
 \emph{épi} avec l'Es\-prit-Saint, car c'est ce qui a résulté lorsque
 Paul a apposé ses mains sur eux dans \ibibleverse{Ac}(19:6)~:
 \og et le Saint-Esprit vint sur [épi] eux. \fg{}

Être empli de l'Esprit ajoute de nouvelles dimensions d'amour,
 de joie et d'exubérance à la vie chrétienne.
 Si l'apôtre Paul vous rencontrait et commençait à partager les gloires
 de Christ avec vous, serait-il à même de demander~:
 \og Avez-vous reçu l'Esprit quand vous avez cru ? \fg{}
 Dieu veut que votre vie ne soit pas seulement une demeure
 pour l'Esprit ou même remplie de l'Esprit.
 Il veut que votre vie en déborde.


\section*{La fête de huit jours}

Dans \ibibleverse{Jn}(7:37), nous lisons~:
 \og Le dernier jour, le grand jour de la fête, Jésus debout s'écria~:
 Si quelqu'un a soif, qu'il vienne à moi et qu'il boive. \fg{}
 C'était la fête des Tentes, la fête où le peuple de Dieu se rappelait
 comment Il avait divinement préservé leurs pères lorsqu'ils avaient erré
 40~ans dans le désert.
 Dans \ibibleverse{Lv}(23:), nous lisons que quand ils célébraient cette fête,
 ils devaient faire de petites cabanes et devaient quitter leurs maisons
 pour séjourner dans ces cabanes pendant les huit jours de la fête.
 Comme le précisait la tradition, ils devaient laisser assez d'espace
 dans la tenture du toit afin de pouvoir observer les étoiles la nuit,
 pour se rappeler que leurs ancêtres avaient dormi à la belle étoile
 pendant 40~ans. De plus, suffisamment d'espace devait être laissé
 dans les murs pour que le vent puisse y souffler, afin qu'ils se rappellent
 que bien que leurs pères aient été exposés aux éléments pendant 40~ans,
 Dieu les avait miraculeusement protégés.

Au temple, chaque jour de la fête, les prêtres faisaient une procession
 jusqu'à la piscine de Siloé, où ils remplissaient de grandes jarres d'eau
 et montaient ensuite en procession les nombreuses marches du mont du temple.
 Alors que les gens chantaient les psaumes glorieux d'hallel,
 les prêtres versaient l'eau sur le sol pavé ; c'était pour rappeler
 aux adorateurs l'eau qui était sortie du rocher dans le désert lorsque Moïse
 l'avait frappé, et comment Dieu avait préservé de façon surnaturelle
 leurs pères dans ce désert aride. 

On dit que le huitième jour, le dernier jour (qui était connu comme
 le plus grand jour de la fête), les prêtres ne faisaient pas de procession
 à la piscine pour remplir les jarres d'eau.
 Ce jour-là, il n'y avait pas de versement d'eau sur le sol pavé.
 Cela avait également une signification, car c'était la reconnaissance
 que Dieu avait rempli Ses promesses ; Il les avait amenés dans le pays
 qui était bien irrigué et qui ruisselait de lait et de miel,
 et ils n'avaient plus besoin de la provision miraculeuse de l'eau
 qui sortait du rocher.

\begin{specialpar}{\tolerance=208}
C'était ce jour-là, le grand jour de la fête, que Jésus s'était levé
 et s'était écrié~:
 \og Si quelqu'un a soif, qu'il vienne à moi et qu'il boive. \fg{}
 Jésus parle de cette soif universelle dont tout le monde fait l'expérience.
\end{specialpar}


\section*{Les besoins élémentaires de l'homme}


\begin{specialpar}{\tolerance=470}
L'homme est une trinité composée du corps, de l'âme et de l'esprit.
 Il est difficile, voire impossible, pour un homme de se séparer
 en ces trois parties, car nous sommes totalement intégrés
 \ocadr corps, âme et\linebreak esprit \fcadr{} de sorte
 que tout ce qui m'affecte
 physiquement va m'affecter mentalement, et peut également m'affecter
 spirituellement.
 De plus en plus de psychologues découvrent la relation étroite
 entre nos émotions et notre santé physique.
 De la même manière, tout ce qui m'affecte spirituellement
 va également m'affecter émotionellement et physiquement,
 de sorte que quand une personne est née de nouveau,
 cela a un effet sur son être entier~: esprit, âme et corps.
\end{specialpar}

Abraham Maslow a identifié et répertorié par ordre d'importance
 les pulsions de notre corps, qui sont connues sous le nom d'homéo\-stasie.
 Ce sont les admirables mécanismes intégrés dans nos corps que Dieu
 a créés pour surveiller nos corps et conserver un équilibre correct,
 afin de soutenir et de perpétuer la vie.
 Maslow a identifié la plus importante de ces pulsions comme étant
 la pulsion d'air~: le corps surveille les niveaux d'oxygène dans le sang
 et exige que l'oxygène soit réapprovisionné quand il est trop bas.
 La réponse du corps est de commencer à haleter tandis que la fréquence
 des battements de cœur augmente.
 Vient ensuite la pulsion liée à la soif, puis la faim, puis la vessie,
 puis le sexe, \emph{et caetera} dans l'ordre décroissant.
 Ces pulsions sont toutes liées à des besoins physiologiques de l'homme.

Les sociologues ont également recensé ce qu'ils appellent nos pulsions
 sociologiques. L'homme a soif, ou a une pulsion, d'amour.
 Il y a également un besoin de sécurité. Et il y a le besoin d'être utile.

Au plus profond de l'homme, dans le domaine de son esprit,
 il y a également une très forte soif ou pulsion.
 C'est la soif de l'esprit de l'homme d'avoir une relation de sens avec Dieu.
 La tentative du psychologue de comprendre le comportement humain sera toujours
 bornée jusqu'à ce qu'il reconnaisse la dimension spirituelle de l'homme.
 La pulsion la plus forte et la plus profonde de l'homme est de connaître Dieu.
 Dans le \ibibleverse{Ps}(42:2-3), David a dit~:
 \og Comme une biche soupire après des courants d'eau,
 ainsi mon âme soupire après toi, ô Dieu !
 Mon âme a soif de Dieu, du Dieu vivant. \fg{}
 Paul a déclaré dans \ibibleverse{Ph}(3:7-8) que tout ce qui avait été
 important pour lui, il le considérait comme une perte à cause
 de l'excellence de la connaissance de Jésus-Christ.
 Paul explique dans \ibibleverse{Rm}(8:) comment Dieu a rendu l'homme
 sujet au vide ; il a été volontairement conçu ainsi afin que l'homme
 ne puisse jamais être complet sans Dieu.
 La nature tend à combler le vide, donc l'homme par nature a tenté
 de remplir ce vide spirituel avec une variété d'expériences physiques
 et émotionnelles.


\closechapter
