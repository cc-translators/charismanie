\chapter{Quelque chose de plus}

\lettrine{R}{écemment}, un jeune homme est venu me dire\frcolon{}
 \Og J'ai accepté Christ il y a plusieurs années,
 mais cela ne m'a jamais vraiment passionné.
 Je trouvais que lire la Bible était inintéressant.
 En fait, mes pensées vagabondaient, et je ne pouvais pas vraiment me concentrer
 sur la Parole. Je n'ai jamais vraiment su ce qu'était adorer Dieu,
 et ma vie de prière était irrégulière. Mais depuis que j'ai été rempli
 de l'Esprit il y a quelques mois, ma vie a complètement changé.
 J'ai un grand amour pour les choses de Dieu.
 Il me semble que je ne peux jamais me rassasier de la Parole, et maintenant j'aime
 la communion fraternelle avec les croyants.
 Quel grand changement s'est produit dans ma vie depuis que j'ai été rempli
 de l'Esprit ! \Fg{}

Cette histoire, avec des variations, m'a été racontée des centaines de fois
 par ceux qui ont découvert qu'il y a quelque chose de plus que de simplement
 avoir l'Esprit qui vient demeurer dans leur vie au moment de leur conversion.
 Nous reconnaissons effectivement que chaque croyant né de nouveau a l'Esprit
 qui demeure en lui. Paul déclare dans \ibibleverse{ICo}(6:19) que nos corps
 sont les temples de l'Esprit-Saint qui demeure en nous.
 Il déclare également dans \ibibleverse{ICo}(12:3) que vous ne pouvez pas
 appeler Christ Seigneur si ce n'est par l'Esprit.


\section{L'Esprit et le croyant}

Il y a trois prépositions grecques utilisées dans le Nouveau Testament
 pour désigner les différentes relations de l'Esprit avec le croyant\frcolon{}
 \emph{para}, \emph{en} et \emph{épi}.
 Dans \ibibleverse{Jn}(14:17), Jésus a dit à Ses disciples au sujet de
 l'Esprit-Saint\frcolon{} \Og Vous le connaissez, parce qu'il demeure près [para]
 de vous et qu'il sera en [en] vous. \Fg{}
 Il y a ici l'expression d'une relation à deux facettes\frcolon{}
 \emph{para} (près) et \emph{en} (en).
 L'Esprit-Saint était \emph{avec} nous avant notre conversion.
 Il est Celui qui nous a con\-vain\-cu de péché et qui a révélé Christ
 comme la solution. Lorsque nous avons accepté Jésus comme notre Sauveur
 et que nous l'avons invité dans nos vies, l'Esprit-Saint
 a commencé à demeurer \emph{en} nous.

Mais Dieu a quelque chose de plus \ocadr l'admirable source de puissance
 résultant de la relation \emph{épi}.
 Remarquez que c'est ce que Jésus promettait à Ses disciples juste avant
 Son ascension. Dans \ibibleverse{Lc}(24:49), Il a dit\frcolon{}
 \Og Et voici : J'enverrai sur [épi] vous ce que mon Père a promis \Fg{},
 ou \Og au-dessus de vous. \Fg{}
 Dans \ibibleverse{Ac}(1:8), Il a dit\frcolon{}
 \Og Mais vous recevrez une puissance, celle du Saint-Esprit survenant
 sur [épi] vous. \Fg{}

Nous lisons dans \ibibleverse{Ac}(10:44) que l'Esprit-Saint est descendu
 \Og sur \Fg{} les croyants des nations dans la maison de Cornélius\frcolon{}
 \Og Com\-me Pierre prononçait encore ces mots, le Saint-Esprit descendit
 sur [épi] tous ceux qui écoutaient la parole. \Fg{}
 Dans \ibibleverse{Ac}(19:6), lorsque Paul a imposé les mains sur les
 croyants Éphésiens, l'Esprit-Saint est venu sur [épi] eux.

Nous lisons dans \ibibleverse{Ac}(8:) que Philippe était allé à Samarie
 et y préchait Christ ; beaucoup de gens ont cru à la prédication de Philippe
 sur les choses du royaume de Dieu, et au nom de Jésus-Christ,
 ils ont été baptisés.
 S'il n'y a qu'un seul baptême (\ibibleverse{Ep}(4:5)),
 alors nous devons accepter qu'à cet instant les croyants samaritains
 ont été baptisés par l'Esprit dans le corps du Christ
 (\ibibleverse{ICo}(12:13)), et que l'Esprit-Saint a commencé à demeurer
 en eux. Il est évident, cependant, qu'il y avait une relation plus approfondie
 avec l'Esprit-Saint à recevoir, car lorsque l'église de Jérusalem a entendu
 que les Samaritains avaient reçu l'Évangile, ils ont envoyé Pierre et Jean
 vers eux afin qu'ils puissent prier pour qu'ils recoivent l'Esprit-Saint,
 car il n'était encore descendu sur [épi] aucun d'entre eux.

\section{Une vie qui d\'eborde}

Quand Paul s'est rendu dans l'église d'Éphèse et qu'il s'est aperçu
 qu'il manquait quelque chose à l'expérience des croyants, \index{expérience}
 il leur a demandé,
 probablement par amour, ou dans la joie et avec zèle\frcolon{}
 \Og Avez-vous reçu l'Esprit Saint quand vous avez cru ? \Fg{}
 Si la pleine relation avec l'Esprit est atteinte au moment de
 la conversion, la question n'a pas de sens.
 La question même supposait une relation plus profonde qui dépasse
 l'expérience \index{expérience} de la conversion.
 Ce qui leur manquait, c'était la relation
 \emph{épi} avec l'Es\-prit-Saint, car c'est ce qui
 est arrivé lorsque Paul a imposé les mains dans \ibibleverse{Ac}(19:6)\frcolon{}
 \Og et le Saint-Esprit vint sur~[épi]~eux. \Fg{}

Être rempli de l'Esprit ajoute de nouvelles dimensions d'amour,
 de joie et d'exubérance à la vie chrétienne.
 Si l'apôtre Paul vous rencontrait et commençait à partager les gloires
 de Christ avec vous,~aurait-il quelque raison de vous poser la question\frcolon{}
 \Og Avez-vous reçu l'Esprit quand~vous avez cru ? \Fg{}
 Dieu veut que votre vie ne soit pas seulement habitée
 par l'Esprit ou même remplie de l'Esprit.
 Il veut que votre vie en déborde.


\section{La f\^ete de huit jours}

Dans \ibibleverse{Jn}(7:37), nous lisons\frcolon{}
 \Og Le dernier jour, le grand jour de la fête, Jésus debout s'écria\frcolon{}
 Si quelqu'un a soif, qu'il vienne à moi et qu'il boive. \Fg{}
 C'était la fête des Huttes, la fête où le peuple de Dieu se rappelait
 comment Il avait divinement préservé leurs pères lorsqu'ils avaient erré
 quarante~ans dans le désert.
 Dans \ibibleverse{Lv}(23:), nous lisons que lorsqu'ils célébraient cette fête,
 ils devaient faire de petites huttes et devaient quitter leurs maisons
 pour séjourner dans ces~huttes pendant les huit jours de la fête.
 Selon la tradition qui s'était développée, ils devaient laisser assez d'espace
 dans les branchages du toit pour pouvoir observer les étoiles la nuit,
 pour se rappeler que leurs ancêtres avaient dormi à la belle étoile
 pendant quarante~ans. De plus, suffisamment d'espace devait être laissé
 dans les murs pour que le vent puisse y souffler, afin qu'ils se rappellent
 que bien que leurs pères aient été exposés aux éléments pendant quarante~ans,
 Dieu les avait miraculeusement préservés.

Au temple, chaque jour de la fête, les prêtres faisaient une procession
 jusqu'à la piscine de Siloé, où ils remplissaient de grandes jarres d'eau
 et montaient ensuite en procession les nombreuses marches du Mont du Temple.
 Pendant que le peuple chantait les glorieux cantiques des montées,
 les prêtres versaient l'eau sur le sol pavé ; c'était pour rappeler
 aux adorateurs l'eau qui était sortie du rocher dans le désert lorsque Moïse
 l'avait frappé, et comment Dieu avait préservé de façon surnaturelle
 leurs pères dans ce désert aride. 

On dit que le huitième jour, le dernier jour (qui était connu pour être
 le grand jour de la fête), les prêtres ne faisaient pas de procession
 à la piscine pour remplir les jarres d'eau.
 Ce jour-là, on ne versait pas d'eau sur le sol pavé.
 Cela avait aussi une signification symbolique, car c'était la reconnaissance
 que Dieu avait respecté Ses promesses ; Il les avait amenés dans un pays
 qui était bien irrigué et qui ruisselait de lait et de miel,
 et ils n'avaient plus besoin de la provision miraculeuse de l'eau
 qui sortait du rocher.

C'était ce jour-là, le grand jour de la fête, que Jésus s'était levé
 et s'était écrié\frcolon{}
 \Og Si quelqu'un a soif, qu'il vienne à moi et qu'il boive. \Fg{}
 Jésus parlait de cette soif spirituelle universelle dont toute personne fait l'expérience.
 \index{expérience}


\section{Les besoins \'el\'ementaires de l'homme}


L'homme est une trinité composée du corps, de l'âme et de l'esprit.
 Il est difficile, voire impossible pour un homme de se séparer
 en ces trois parties, car nous formons un tout
 \ocadr corps, âme et esprit \fcadr{} de sorte
 que tout ce qui m'affecte
 physiquement va m'affecter mentalement, et peut également m'affecter
 spirituellement. Tout ce qui m'affecte mentalement, m'affecte
 également physiquement.
 De plus en plus, les psychologues découvrent la relation étroite
 entre nos émotions et notre santé physique.
 De la même manière, tout ce qui m'affecte spirituellement
 va également m'affecter émotionellement et physiquement,
 de sorte que lorsqu'une personne est née de nouveau,
 cela a un effet sur son être entier\frcolon{} esprit, âme et corps.

Abraham Maslow a identifié et répertorié par ordre de puissance
 les pulsions vitales de notre corps, dont l'équilibre est connu sous le nom d'homéo\-stasie.
 Ce sont les admirables mécanismes intégrés à nos corps que Dieu
 a créés pour les surveiller et conserver un juste équilibre,
 afin de soutenir et de perpétuer la vie.
 Maslow a identifié la plus importante de ces pulsions comme étant
 la pulsion liée à la respiration\frcolon{} le corps surveille le niveau d'oxygène dans le sang
 et exige que l'oxygène soit réapprovisionné quand il est trop bas.
 La réponse du corps est de commencer à haleter et d'augmenter la fréquence cardiaque.
 Vient ensuite la pulsion liée à la soif, à la faim, à la vessie,
 puis au sexe, \emph{et cetera} dans l'ordre décroissant.
 Ces pulsions sont toutes liées à des besoins physiologiques de l'homme.

Les sociologues ont également recensé ce qu'ils appellent nos pulsions
 sociologiques. L'homme a soif, ou a une pulsion, d'amour.
 Il y a également un besoin de sécurité. Et il y a le besoin d'être utile.


Au plus profond de l'homme, dans le domaine de son esprit,
 il y a également une très forte soif ou pulsion.
 C'est la soif de l'esprit de l'homme d'avoir une relation significative avec Dieu.
 La tentative du psychologue de comprendre le comportement humain sera toujours
 limitée tant qu'il n'aura pas reconnu la dimension spirituelle de l'homme.
 La pulsion la plus forte et le besoin le plus profond de l'homme, c'est de connaître Dieu.
 Dans le \ibibleverse{Ps}(42:2-3), David a dit\frcolon{}
 \Og Comme une biche soupire après des courants d'eau,
 ainsi mon âme soupire après toi, ô Dieu !
 Mon âme a soif de Dieu, du Dieu vivant. \Fg{}
 En \ibibleverse{Ph}(3:7-8), Paul a déclaré que tout ce qui avait été
 important pour lui, il le considérait comme une perte à cause
 de l'excellence de la connaissance de Jésus-Christ.
 En \ibibleverse{Rm}(8:), Paul explique comment Dieu a rendu l'homme
 sujet au sentiment du vide ; il a été délibérément conçu ainsi afin qu'il
 ne puisse jamais être complet sans Dieu.
 La nature tend à combler le vide, et l'homme par nature a donc tenté
 de remplir ce vide spirituel avec une variété d'expériences \index{expérience}
 physiques et émotionnelles.


\section{Les besoins sont distincts}

\index{expérience|(}
Les soifs que nous ressentons sont dissociées et distinctes,
 et ainsi nous ne pouvons pas satisfaire une soif physique
 par une expérience émotionelle. Si vous étiez perdu dans le désert,
 traversant à pied les sables brûlants, et que le niveau d'hydratation
 de votre corps baisse dangereusement, vous ressentiriez une énorme soif
 physique. Au fur et à mesure que votre corps se déshydraterait,
 vous perdriez votre force. Imaginons que vous finissiez par vous
 retrouver couché sur le sable brûlant, creusant instinctivement
 pour tenter de trouver de l'eau, et que quelqu'un passe sur une dune
 de sable, vous repère, et s'écrie\frcolon{}
 \Og Oh, je sais qui tu es. Je veux que tu saches que je t'aime secrètement depuis longtemps.
 Je pense que tu es la personne la plus formidable au monde,
 et je t'aime profondément. \Fg{}


Bien que cette personne réponde probablement à votre besoin émotionnel
 d'amour, vous êtes en train de mourir de soif dans ses bras,
 car vous ne pouvez pas satisfaire une soif physique par une
 expérience émotionnelle. De la même manière, vous ne pouvez pas satisfaire une soif
 émotionnelle par une expérience physique,
 et c'est d'ailleurs une source de problèmes dans la société actuelle.


\section{Une r\'eponse bancale}

Nous vivons dans une culture où l'on s'occupe correctement de la plupart
 des besoins physiques de la personne. Pourtant, assez souvent,
 il y a un manque tragique de satisfaction des besoins émotionnels.
 Bien des fois, les parents ont du mal à comprendre les actes de rébellion
 de leurs enfants contre le foyer. J'en ai entendu dire\frcolon{}
 \Og J'ai tout donné à mon enfant. Je n'arrive pas à comprendre
 comment il peut faire ce qu'il fait. \Fg{}
 Quand ils déclarent\frcolon{} \Og Je lui ai tout donné \Fg{}, ils parlent en général
 de choses physiques\frcolon{} l'enfant a reçu plusieurs vélos
 et sa propre télévision, une chaîne hi-fi et~une~voiture.
 \nowidow[6]

Mais bien souvent, ces choses ont été données à l'enfant afin de le repousser.
 Le but était de distraire l'enfant avec ces choses afin de ne pas avoir
 à lui accorder de l'attention et du temps
 \ocadr du temps qui lui permettrait de ressentir la force
 de l'amour au sein de la famille. La mère dit bien souvent\frcolon{}
 \Og Pourquoi ne vas-tu pas regarder la télévision dans ta chambre ?
 Tu ne vois pas que tu me fatigues ? Ne poses pas autant de questions.
 Pourquoi ne vas-tu pas faire du vélo dehors ? \Fg{}
 L'enfant, assoiffé de se sentir aimé et en sécurité,
 est repoussé vers les choses matérielles, et un jour il finit
 par se rebeller contre le monde matériel, comme nous l'avons vu
 dans la révolution contre-culturelle des années 60, connue sous le nom de mouvement hippie.


Vous ne pouvez pas satisfaire une soif émotionnelle par une
 expérience physique. Il est également vrai que, au plus profond de lui-même,
 l'homme a une profonde soif spirituelle de Dieu.
 Un des problèmes que nous rencontrons aujourd'hui, c'est que l'homme
 s'efforce de satisfaire cette profonde soif de Dieu
 par des expériences physiques et émotionnelles.
 Cette profonde soif de Dieu est l'une des raisons derrière l'addiction
 au plaisir de notre monde actuel. Les gens essayent de satisfaire
 ce besoin profond de Dieu par des expériences émotionnelles et physiques.
 Cela explique aussi en partie la toxicomanie,
 car les gens font souvent des expériences pseudo-spirituelles
 par l'usage de drogues. Beaucoup des gens ayant pris du LSD
 pensent avoir fait des expériences authentiques avec~Dieu.


\section{La profonde soif universelle}

Quand Jésus a dit\frcolon{} \Og Si quelqu'un a soif \Fg{}, il se réferrait
 à cette profonde soif universelle de l'esprit de l'homme pour Dieu.
 Il est intéressant pour moi de voir que certains livres de psychologie
 identifient la frustration comme étant l'une des racines
 des comportements névrotiques. Ils déclarent que le problème
 d'une personne commence souvent par la frustration,
 ce sentiment que vous n'avez pas atteint le but ultime de la vie,
 qu'il doit y avoir quelque chose de plus dans la vie que ce dont
 vous avez fait l'expérience \ocadr mais de quoi s'agit-il, et comment
 l'atteindre ?
 Il s'agit de la tentative d'atteindre quelque chose dont je ne suis pas sûr,
 sans parvenir à trouver ce que j'espère. Qu'est-ce que cette frustration,
 sinon une soif, une soif spirituelle, cette profonde soif de l'esprit
 de l'homme pour Dieu ?
 \index{expérience|)}

Les livres de psychologie montrent comment la frustration con\-duit
 à un complexe d'infériorité, qui n'est rien de plus qu'un retour sur moi-même
 quand je me demande pourquoi je n'ai pas atteint cette satisfaction,
 cet épanouissement auquel j'aspire. Je me dis\frcolon{}
 \Og Si seulement j'avais de l'argent \Fg{}, ou\frcolon{}
 \Og Si seulement j'avais les yeux bleus plutôt que marrons \Fg{}, ou encore\frcolon{}
 \Og Si seulement j'avais fait des études plus avancées. \Fg{}
 Avec ces excuses, et des milliers d'autres, je m'explique la raison
 de ma frustration.


\section{Deux types d'\'evasions}

Si l'on en croit ces livres, je passe ensuite de mon complexe
 d'in\-fé\-rio\-ri\-té à une évasion.
 Il peut s'agir d'une extraversion ou d'une introversion.
 Les évasions de type introverties se manifestent par des tentatives
 de construction de murs autour de votre vrai moi.
 Vous allez souvent montrer aux autres personnes une façade qui est très
 différente de ce que vous êtes vraiment.
 Vous agissez comme si vous n'aviez pas mal alors que c'est faux ;
 vous montrez une grande confiance alors qu'en réalité vous avez peur.
 Vous commencez à garder les gens à distance ;
 vous avez peur qu'ils découvrent le vrai vous.
 Vous évitez les personnes quand vous sentez qu'elles s'approchent
 trop près de vous. Vous refusez de leur parler quand elles vous appellent.
 Vous allez jusqu'à ne pas répondre quand on sonne à votre porte.
 Dans sa forme terminale, l'évasion introvertie se manifeste par une vie
 d'ermite seul dans sa cabane au fond du désert, tirant des coups de semonce
 sur quiconque dépasserait les pancartes
 \Og Propriété Privée \ocadr Entrée interdite \Fg{} placardées sur son portail.

Les évasions de type extraverties se manifestent sous de nombreuses formes,
 comme l'alcoolisme, la toxicomanie, la boulimie, le jeu, le nomadisme,
 les aventures extraconjugales, etc.
 Je ne peux pas supporter de faire face à la réalité de mon échec dans ma quête
 d'épanouissement véritable, donc je m'échappe dans un monde virtuel.
 Ces évasions m'amènent ensuite à un complexe de culpabilité.
 Je sais que ce que je fais est mal. Je sais que cela me détruit
 ainsi que mon entourage qui m'aime, et pourtant il me semble ne pas avoir
 la capacité d'arrêter. Je commence à me haïr moi-même pour le mal que je me fais
 à moi-même et aux~autres.

Le complexe de culpabilité se transforme ensuite en désir subconscient
 de punition. Cela se manifeste généralement par un mode de comportement
 névrotique dont le but est d'amener la désapprobation de mes associés,
 que j'interpréte comme une punition, qui à son tour me soulage
 de mes sentiments de culpabilité. Quand j'étais enfant, mon père
 s'est occupé de mon complexe de culpabilité en me punissant.
 Dans mon cas, il s'agissait généralement de fessées.
 Une fois que j'avais été puni, je ne me sentais plus coupable
 et je pouvais reprendre ma place normale au sein de la famille.
 Avant la punition, je sentais que la relation était tendue et j'avais
 un sentiment d'aliénation.

En prenant de l'âge, nous nous libérons de toute autorité parentale,
 et pour nous libérer de la culpabilité, nous devons
 nous comporter d'une façon inacceptable afin d'amener la désapprobation
 ou le rejet, que nous interprétons alors comme une punition.
 Une fois punis, nous nous sentons libres de notre complexe de culpabilité
 et nous retournons alors à notre frustration et nous recommençons le cycle.
 Quand Jésus a dit\frcolon{} \Og Si quelqu'un a soif \Fg{},
 il faisait référence à cette soif de l'esprit de l'homme pour Dieu,
 que le psychologue répertorie comme une frustration.
 \nowidow[6]


\section{\'Etancher vraiment sa soif}

Quand Jésus a parlé à la Samaritaine, Il lui a demandé à boire,
 et elle a contesté sa demande, puisqu'Il était Juif
 et qu'elle était Samaritaine. Par tradition, il ne devait y avoir
 aucune forme de transaction entre eux. Jésus lui a répondu\frcolon{}
 \Og Si tu connaissais qui est celui qui te demande à boire,
 c'est toi qui lui aurais demandé à boire. \Fg{}
 Elle lui a répondu avec un brin d'impertinence\frcolon{}
 \Og Pourquoi te demanderais-je à boire alors que tu n'as rien pour puiser,
 et que ce puits est très profond ? \Fg{} Jésus lui a alors dit\frcolon{}
 \Og Quiconque boit de cette eau aura encore soif. \Fg{}
 \ibiblephantom{Jn}(4:7-14)Je crois que ce verset devrait être inscrit
 en lettres d'or au-dessus de tous les buts, toutes les
 ambitions ou toutes les quêtes de plaisir des hommes.
 Vous pouvez boire de cette eau, atteindre votre but, réaliser votre ambition,
 et combler vos fantasmes, mais vous aurez à nouveau soif.
 Cela ne va pas vous satisfaire, car tout au fond de vous,
 votre esprit a soif de Dieu, et rien ne peut satisfaire cette soif
 si ce n'est une relation authentique avec Dieu.

Quand Jésus a dit\frcolon{} \Og Si quelqu'un a soif, qu'il vienne à moi
 et qu'il boive \Fg{}, Il exprimait l'Évangile dans ses termes
 les plus simples. Il disait à toute l'humanité\frcolon{}
 \Og Tout au fond de votre vie, vous avez besoin de Dieu.
 Vous essayez d'atteindre une relation de sens avec Dieu.
 Venez à moi, et votre soif sera non seulement entièrement satisfaite
 et assouvie, mais de votre vie jailliront des torrents d'eau vive. \Fg{}
 Seul Christ peut satisfaire votre soif spirituelle,
 car Il vous amène dans une relation authentique avec Dieu.


\section{Des torrents d'eau}

Dans \ibibleverse{Jn}(7:38), Jésus a continué en disant\frcolon{}
 \Og Celui qui croit en moi, des fleuves d'eau vive couleront de son sein,
 comme dit l'Écriture. \Fg{}
 Les mots grecs utilisés ici sont un peu plus intenses que ce qui ressort
 de la version de la Bible à la Colombe. Le Seigneur déclare textuellement
 que, si une personne croit en Lui, \Og des torrents d'eau vive jailliront
 de son sein \Fg{} \ocadr pas juste un gentil petit ruisseau qui s'écoulerait,
 mais un puissant torrent, comme les cascades qui dévalent les ravins de montagne
 lors d'une averse torrentielle.

À quoi Jésus faisait-il référence lorsqu'Il parlait de
 \Og torrents d'eau vive \Fg{} s'écoulant de notre vie ?
 Quand Jean a écrit cet Évangile, c'était plusieurs années après les faits.
 Son livre était l'un des derniers du Nouveau Testament à être rédigé,
 et Jean écrivait avec le bénéfice du recul. Au moment précis où Jésus parlait
 des torrents d'eau vive, Jean était probablement confus sur ce que
 Jésus voulait dire ou sur ce qu'Il promettait aux gens.
 Mais parce que Jean a écrit l'Évangile avec la compréhension gagnée grâce
 au bénéfice du recul, il a ajouté son propre commentaire exprimé entre
 parenthèses au verset \ibiblevs{Jn}(7:39), dans lequel il explique que Jésus
 parlait de l'Esprit-Saint, \Og qu'allaient recevoir ceux qui croiraient
 en lui ; car l'Esprit n'était pas encore [donné], parce que Jésus n'avait
 pas encore été glorifié. \Fg{}
 Ainsi donc, Jésus parlait de la puissance conférée à la vie du croyant
 par l'Esprit-Saint.


\section{Ce que Dieu d\'esire pour vous}

Je crois qu'il nous faut admettre sans hésitation qu'il s'agit là de bien
 plus que de la simple présence de l'Esprit venant demeurer dans la vie
 du croyant à sa conversion. C'est une chose que d'avoir l'Esprit-Saint
 qui demeure dans votre vie, c'en est une toute autre que d'avoir cette puissance
 glorieuse et dynamique de l'Esprit de Dieu qui jaillit de votre vie
 comme un torrent d'eau vive.

Dieu a une relation plus riche pour vous que la seule présence de l'Esprit
 venu demeurer en vous. Dieu désire que Son Esprit jaillisse
 de votre vie. Le nom que vous donnez à ce phénomène importe peu.
 Certains parlent du baptême de l'Esprit-Saint,
 d'autres d'être rempli de l'Esprit-Saint, et d'autres encore du don
 de puissance de l'Esprit-Saint. Le nom que vous lui donnez n'a vraiment
 pas d'im\-por\-tan\-ce ; ce qui est important, c'est que l'Esprit
 déborde glorieusement de votre vie.

Dieu considère toujours l'homme de deux façons. D'abord, Dieu le regarde
 \emph{subjectivement}, lorsqu'Il cherche à effectuer Son œuvre
 dans votre vie.
 Mais les desseins de Dieu ne se limitent jamais à Son œuvre subjective.
 Dieu considère également l'œuvre \emph{objective} qu'Il souhaite opérer
 par vous.
 Il œuvre en vous \emph{subjectivement} afin de pouvoir œuvrer par vous
 \emph{objectivement}. Il désire opérer une œuvre \emph{en} vous
 et \emph{pour} vous afin
 qu'Il puisse œuvrer \emph{par} vous pour toucher d'autres personnes.
 Notre relation à l'Esprit n'est jamais complète quand Il ne fait que
 demeurer en nous. Nous sommes plus que des récipients contenant
 l'Esprit de Dieu. Dieu désire que nous soyons des \emph{canaux} à travers
 lesquels Son Esprit puisse s'écouler.


\section{La puissance en action}


Lorsque vous considérez votre propre expérience \index{expérience}
 et votre relation \index{relation}
 avec l'Esprit-Saint, si vous ne pouvez pas dire que la force puissante
 de l'Esprit de Dieu jaillit de votre vie comme un fleuve ou comme
 des torrents d'eau vive, alors Dieu désire amener dans votre vie une relation
 beaucoup plus riche avec Son Esprit,
 et je voudrais vous encourager à rechercher cette puissance de l'Esprit
 de Dieu jusqu'à ce qu'elle jaillisse de votre vie.
 Il y a autour de nous un monde de misère qui a besoin d'être touché
 par la puissance de l'Esprit de Dieu jaillissant de nous.
 Si vous refusez d'appeler cela le baptême de l'Esprit-Saint,
 appelez-le ce que vous voulez, mais ce que Jésus décrit est bien plus
 que la simple présence de l'Esprit-Saint dans la vie du croyant
 qui intervient au moment de sa conversion. Ce flot plein de beauté de l'Esprit
 qui jaillit de la vie d'une personne est le vrai \emph{charisme}.
%\closechapter
% page too short to use \closechapter
\thispagestyle{chapterend}

