\chapter{Quelque chose de plus}

\lettrine{R}{écemment}, un jeune homme est venu me dire~:
 \og J'ai accepté Christ il y a plusieurs années,
 mais cela ne m'a jamais vraiment excité.
 Je trouvais que lire la Bible était inintéressant.
 En fait, mon esprit flânait, et je ne pouvais pas vraiment me concentrer
 sur la Parole. Je n'ai jamais vraiment su ce qu'était adorer Dieu,
 et ma vie de prière était erratique. Mais depuis que j'ai été rempli
 de l'Esprit il y a quelques mois, ma vie a complètement changé.
 J'ai un grand amour pour les choses de Dieu.
 Je semble ne jamais me rassasier de la Parole, et maintenant j'aime
 la communion fraternelle avec les croyants.
 Quel changement s'est produit dans ma vie depuis que j'ai été rempli
 de l'Esprit ! \fg{}

Cette histoire, avec des variations, m'a été racontée des centaines de fois
 par ceux qui ont découvert qu'il y a quelque chose de plus que de simplement
 avoir l'Esprit qui vient demeurer dans leur vie au moment de leur conversion.
 Nous reconnaissons effectivement que chaque croyant né de nouveau a l'Esprit
 qui demeure en lui. Paul déclare dans \ibibleverse{ICo}(6:19) que nos corps
 sont les temples de l'Esprit-Saint qui demeure en nous.
 Il déclare également dans \ibibleverse{ICo}(12:3) que vous ne pouvez pas
 appeler Christ Seigneur si ce n'est par l'Esprit.


\section*{L'Esprit et le croyant}

Il y a trois prépositions grecques utilisées dans le Nouveau Testament
 pour désigner les différentes relations de l'Esprit avec le croyant~:
 \emph{para}, \emph{en} et \emph{épi}.
 Dans \ibibleverse{Jn}(14:17), Jésus a dit à Ses disciples au sujet de
 l'Esprit-Saint~: \og Vous le connaissez, parce qu'il demeure près [para]
 de vous et qu'il sera en [en] vous. \fg{}
 Il y ici l'expression d'une relation à deux facettes~:
 \emph{para} (près) et \emph{en} (en).
 L'Esprit-Saint était \emph{avec} nous avant notre conversion.
 Il est Celui qui nous a convaincu de péché et révélé Christ
 comme la solution. Lorsque nous avons accepté Jésus comme notre Sauveur
 et que nous l'avons invité dans nos vies, l'Esprit-Saint
 a commencé à demeurer en nous.

Mais Dieu a quelque chose en plus \ocadr la belle source de puissance
 par la relation \emph{épi}.
 Remarquez que c'est ce que Jésus promettait à Ses disciples juste avant
 Son ascension. Dans \ibibleverse{Lc}(24:49), Il a dit~:
 \og Et [voici] : j'enverrai sur [épi] vous ce que mon Père a promis \fg{} ou
 \og au-dessus de vous. \fg{}
 Dans \ibibleverse{Ac}(1:8), Il a dit~:
 \og Mais vous recevrez une puissance, celle du Saint-Esprit survenant
 sur [épi] vous. \fg{}

Nous lisons dans \ibibleverse{Ac}(10:44) que l'Esprit-Saint est descendu
 \og sur \fg{} les croyants des nations dans la maison de Cornélius~:
 \og Comme Pierre prononçait encore ces mots, le Saint-Esprit descendit
 sur [épi] tous ceux qui écoutaient la parole. \fg{}
 Dans \ibibleverse{Ac}(19:6), lorsque Paul a posé les mains sur les
 croyants Éphésiens, l'Esprit-Saint est venu sur [épi] eux.

Nous lisons dans \ibibleverse{Ac}(8:) que Philippe était allé à Samarie
 et leur préchait Christ ; beaucoup de gens ont cru le prêche de Philippe
 au sujet des choses du royaume de Dieu et du nom de Jésus-Christ,
 et ils ont été baptisés.
 S'il n'y a qu'un seul baptème (\ibibleverse{Ep}(4:5)),
 alors nous devons accepter à ce moment que les croyants Samaritains
 ont été baptisés par l'Esprit dans le corps de Christ
 (\ibibleverse{ICo}(12:13)), et que l'Esprit-Saint a commencé à demeurer
 en eux. Il est évident, cependant, qu'il y avait une relation plus poussée
 avec l'Esprit-Saint à recevoir, car lorsque l'église à Jérusalem a entendu
 que les Samaritains avaient reçu l'Évangile, ils ont envoyé Pierre et Jean
 vers eux afin qu'ils puissent prier pour qu'ils recoivent l'Esprit-Saint,
 car il n'était encore descendu sur [épi] aucun d'entre eux.

\section*{La vie qui déborde}

Quand Paul s'est rendu dans l'église d'Ephèse et a trouvé que les croyants
 y manquaient d'expérience, il leur a demandé, probablement par amour
 ou par joie et zèle~: \og Avez-vous reçu l'Esprit Saint
 quand vous avez cru ? \fg{}
 Si la relation pleine avec l'Esprit est atteinte en même temps
 que la conversion, la question n'a pas de sens.
 La question même supposait une relation plus profonde et au-delà
 de l'expérience de la conversion. Ce qui leur manquait était la relation
 \emph{épi} avec l'Es\-prit-Saint, car c'est ce qui a résulté lorsque
 Paul a apposé ses mains sur eux dans \ibibleverse{Ac}(19:6)~:
 \og et le Saint-Esprit vint sur [épi] eux. \fg{}

Être empli de l'Esprit ajoute de nouvelles dimensions d'amour,
 de joie et d'exubérance à la vie chrétienne.
 Si l'apôtre Paul vous rencontrait et commençait à partager les gloires
 de Christ avec vous, serait-il à même de demander~:
 \og Avez-vous reçu l'Esprit quand vous avez cru ? \fg{}
 Dieu veut que votre vie ne soit pas seulement une demeure
 pour l'Esprit ou même remplie de l'Esprit.
 Il veut que votre vie en déborde.


\section*{La fête de huit jours}

Dans \ibibleverse{Jn}(7:37), nous lisons~:
 \og Le dernier jour, le grand jour de la fête, Jésus debout s'écria~:
 Si quelqu'un a soif, qu'il vienne à moi et qu'il boive. \fg{}
 C'était la fête des Tentes, la fête où le peuple de Dieu se rappelait
 comment Il avait divinement préservé leurs pères lorsqu'ils avaient erré
 40~ans dans le désert.
 Dans \ibibleverse{Lv}(23:), nous lisons que quand ils célébraient cette fête,
 ils devaient faire de petites cabanes et devaient quitter leurs maisons
 pour séjourner dans ces cabanes pendant les huit jours de la fête.
 Comme le précisait la tradition, ils devaient laisser assez d'espace
 dans la tenture du toit afin de pouvoir observer les étoiles la nuit,
 pour se rappeler que leurs ancêtres avaient dormi à la belle étoile
 pendant 40~ans. De plus, suffisamment d'espace devait être laissé
 dans les murs pour que le vent puisse y souffler, afin qu'ils se rappellent
 que bien que leurs pères aient été exposés aux éléments pendant 40~ans,
 Dieu les avait miraculeusement protégés.

Au temple, chaque jour de la fête, les prêtres faisaient une procession
 jusqu'à la piscine de Siloé, où ils remplissaient de grandes jarres d'eau
 et montaient ensuite en procession les nombreuses marches du mont du temple.
 Alors que les gens chantaient les psaumes glorieux d'hallel,
 les prêtres versaient l'eau sur le sol pavé ; c'était pour rappeler
 aux adorateurs l'eau qui était sortie du rocher dans le désert lorsque Moïse
 l'avait frappé, et comment Dieu avait préservé de façon surnaturelle
 leurs pères dans ce désert aride. 

On dit que le huitième jour, le dernier jour (qui était connu comme
 le plus grand jour de la fête), les prêtres ne faisaient pas de procession
 à la piscine pour remplir les jarres d'eau.
 Ce jour-là, il n'y avait pas de versement d'eau sur le sol pavé.
 Cela avait également une signification, car c'était la reconnaissance
 que Dieu avait rempli Ses promesses ; Il les avait amenés dans le pays
 qui était bien irrigué et qui ruisselait de lait et de miel,
 et ils n'avaient plus besoin de la provision miraculeuse de l'eau
 qui sortait du rocher.

\begin{specialpar}{\tolerance=208}
C'était ce jour-là, le grand jour de la fête, que Jésus s'était levé
 et s'était écrié~:
 \og Si quelqu'un a soif, qu'il vienne à moi et qu'il boive. \fg{}
 Jésus parle de cette soif universelle dont tout le monde fait l'expérience.
\end{specialpar}


\section*{Les besoins élémentaires de l'homme}


\begin{specialpar}{\tolerance=470}
L'homme est une trinité composée du corps, de l'âme et de l'esprit.
 Il est difficile, voire impossible, pour un homme de se séparer
 en ces trois parties, car nous sommes totalement intégrés
 \ocadr corps, âme et\linebreak esprit \fcadr{} de sorte
 que tout ce qui m'affecte
 physiquement va m'affecter mentalement, et peut également m'affecter
 spirituellement.
 De plus en plus de psychologues découvrent la relation étroite
 entre nos émotions et notre santé physique.
 De la même manière, tout ce qui m'affecte spirituellement
 va également m'affecter émotionellement et physiquement,
 de sorte que quand une personne est née de nouveau,
 cela a un effet sur son être entier~: esprit, âme et corps.
\end{specialpar}

Abraham Maslow a identifié et répertorié par ordre d'importance
 les pulsions de notre corps, qui sont connues sous le nom d'homéo\-stasie.
 Ce sont les admirables mécanismes intégrés dans nos corps que Dieu
 a créés pour surveiller nos corps et conserver un équilibre correct,
 afin de soutenir et de perpétuer la vie.
 Maslow a identifié la plus importante de ces pulsions comme étant
 la pulsion d'air~: le corps surveille les niveaux d'oxygène dans le sang
 et exige que l'oxygène soit réapprovisionné quand il est trop bas.
 La réponse du corps est de commencer à haleter tandis que la fréquence
 des battements de cœur augmente.
 Vient ensuite la pulsion liée à la soif, puis la faim, puis la vessie,
 puis le sexe, \emph{et caetera} dans l'ordre décroissant.
 Ces pulsions sont toutes liées à des besoins physiologiques de l'homme.

Les sociologues ont également recensé ce qu'ils appellent nos pulsions
 sociologiques. L'homme a soif, ou a une pulsion, d'amour.
 Il y a également un besoin de sécurité. Et il y a le besoin d'être utile.

Au plus profond de l'homme, dans le domaine de son esprit,
 il y a également une très forte soif ou pulsion.
 C'est la soif de l'esprit de l'homme d'avoir une relation de sens avec Dieu.
 La tentative du psychologue de comprendre le comportement humain sera toujours
 bornée jusqu'à ce qu'il reconnaisse la dimension spirituelle de l'homme.
 La pulsion la plus forte et la plus profonde de l'homme est de connaître Dieu.
 Dans le \ibibleverse{Ps}(42:2-3), David a dit~:
 \og Comme une biche soupire après des courants d'eau,
 ainsi mon âme soupire après toi, ô Dieu !
 Mon âme a soif de Dieu, du Dieu vivant. \fg{}
 Paul a déclaré dans \ibibleverse{Ph}(3:7-8) que tout ce qui avait été
 important pour lui, il le considérait comme une perte à cause
 de l'excellence de la connaissance de Jésus-Christ.
 Paul explique dans \ibibleverse{Rm}(8:) comment Dieu a rendu l'homme
 sujet au vide ; il a été volontairement conçu ainsi afin que l'homme
 ne puisse jamais être complet sans Dieu.
 La nature tend à combler le vide, donc l'homme par nature a tenté
 de remplir ce vide spirituel avec une variété d'expériences physiques
 et émotionnelles.


\section*{Les besoins sont distincts}


Les soifs dont nous faisons l'expérience sont séparées et distinctes,
 et ainsi nous ne pouvons pas satisfaire une soif physique
 par une expérience émotionelle. Si vous étiez perdu dans le désert,
 traversant à pied les sables brûlants, et que le niveau d'hydratation
 de votre corps baisse dangereusement, vous ressentiriez une énorme soif
 physique. Tandis que votre corps se déshydraterait,
 vous perdriez votre force. Imaginons que vous finissiez par vous
 retrouver couché sur le sable brûlant, creusant instinctivement
 pour tenter de trouver de l'eau, et que quelqu'un passe sur une dune
 de sable, vous remarque, et s'écrie~:
 \og Oh, je sais qui tu es. Je veux que tu saches que je t'aime secrètement.
 Je pense que tu es la meilleure personne au monde,
 et je t'aime profondément. \fg{}

Bien que cette personne réponde probablement à votre besoin émotionnel
 d'amour, vous êtes en train de mourir de soif dans ses bras,
 car vous ne pouvez pas satisfaire une soif physique par une expérience
 émotionnelle. De la même manière, vous ne pouvez pas satisfaire une soif
 émotionnelle par une expérience physique,
 et c'est d'ailleurs la source de problèmes dans notre société d'aujourd'hui.


\section*{Une satisfaction unilatérale}

Nous vivons dans une culture où l'on s'occupe correctement de la plupart
 des besoins physiques d'une personne. Pourtant, assez souvent,
 il y a un manque tragique de satisfaction des besoins émotionnels.
 Bien des fois, les parents ont du mal à comprendre les actes de rébellion
 de leurs enfants contre le foyer. J'en ai entendu dire~:
 \og J'ai tout donné à mon enfant. Je n'arrive pas à comprendre
 comment il peut faire ce qu'il fait. \fg{}
 Quand ils déclarent~: \og Je lui ai tout donné \fg{}, ils parlent en général
 de choses physiques~: l'enfant a reçu plusieurs vélos
 et sa propre télévision, une chaîne hi-fi et une voiture.

Mais bien souvent, ces choses ont été données à l'enfant afin de le repousser.
 Le but était de garder l'enfant occupé avec ces choses afin de ne pas avoir
 à lui accorder du temps avec ses parents
 \ocadr du temps qui lui permettrait de ressentir la relation de proximité
 de l'amour au sein de la famille. La mère dit bien souvent~:
 \og Pourquoi ne vas-tu pas regarder la télévision dans ta chambre ?
 Tu ne vois pas que tu me stresses ? Ne poses pas autant de questions.
 Pourquoi ne vas-tu pas faire du vélo dehors ? \fg{}
 L'enfant, assoiffé d'amour et de se sentir en sécurité,
 est repoussé vers les choses matérielles, et un jour il finit
 par se rebeller contre le monde matériel, comme nous l'avons vu
 dans la révolution contre-culturelle, connue sous le nom de mouvement hippie.

\begin{specialpar}{\tolerance=330}
Vous ne pouvez pas satisfaire une soif émotionnelle par une expérience
 physique. Il est également vrai que, au plus profond de lui-même,
 l'homme a une profonde soif spirituelle de Dieu.
 Un des problèmes que nous rencontrons aujourd'hui est que l'homme
 s'est mis dans l'idée de satisfaire cette profonde soif de Dieu
 par des expériences physiques et émotionnelles.
 Cette profonde soif de Dieu est l'une des raisons derrière la manie
 du plaisir de notre monde actuel. Les gens essayent de satisfaire
 ce besoin profond de Dieu par des expériences émotionnelles et physiques.
 Cela explique aussi en partie les abus de drogues,
 car les gens font souvent des expériences pseudo-spirituelles
 par la prise de drogues. Beaucoup des gens ayant pris du LSD
 pensent avoir eu de vraies expériences avec Dieu.
\end{specialpar}


\section*{La profonde soif universelle}

Quand Jésus a dit~: \og Si quelqu'un a soif \fg{}, il se réferrait
 à cette profonde soif universelle de l'esprit de l'homme pour Dieu.
 Il est intéressant pour moi que certains livres de psychologie
 identifient la frustration comme étant la cause centrale
 d'un comportement névrotique. Ils déclarent que le problème
 d'une personne commence souvent par la frustration,
 ce sentiment que vous n'avez pas atteint le but ultime de la vie,
 qu'il doit y avoir quelque chose de plus dans la vie que ce dont
 vous avez fait l'expérience \ocadr mais de quoi s'agit-il, et comment
 l'atteindre ?
 Il s'agit de la tentative d'atteindre quelque chose dont je ne suis pas sûr,
 et ne pas trouver ce que j'espère. Qu'est-ce que la frustration,
 sinon une soif, une soif spirituelle, cette profonde soif de l'esprit
 de l'homme pour Dieu ?

Les livres de psychologie montrent comment la frustration conduit
 à un complexe d'infériorité, qui n'est rien de plus que mon raisonnement
 sur moi-même quand je contaste que je n'ai pas atteint cette satisfaction,
 cet épanouissement que je désire. Je dis~:
 \og Si seulement j'avais de l'argent \fg{}, ou~:
 \og Si seulement j'avais les yeux bleus plutôt que marrons \fg{}, ou encore~:
 \og Si seulement j'avais fait des études plus élevées. \fg{}
 Avec ces excuses, et des milliers d'autres, je m'explique la raison
 de ma frustration.


\section*{Deux types d'évasions}

\begin{specialpar}{\tolerance=400}
Si l'on en croit ces livres, je passe ensuite de mon complexe
 d'in\-fé\-rio\-ri\-té à une évasion.
 Il peut s'agir d'une extraversion ou d'une introversion.
 Les évasions de type introverties se manifestent par des tentatives
 de construction de murs autour de votre vrai moi.
 Vous allez souvent montrer aux autres personnes une façade qui est très
 différente de ce que vous êtes vraiment.
 Vous agissez comme si vous n'aviez pas mal alors que c'est faux ;
 vous montrez une grande confiance alors qu'en réalité vous avez peur.
 Vous commencez à garder les gens à distance ;
 vous avez peur qu'ils découvrent le vrai vous.
 Vous évitez les personnes quand vous sentez qu'elles s'approchent
 trop près de vous. Vous refusez de leur parler quand elles vous appellent.
 Vous allez jusqu'à ne pas répondre quand on sonne à votre porte.
 Dans sa forme finale, l'évasion introvertie se manifeste par une vie
 d'hermite seul dans sa cabane dans le désert, tirant des coups de semonce
 sur quiconque dépasserait les signes
 \og Éloignez-vous \ocadr Entrée interdite \fg{} sur sa porte.
\end{specialpar}

\begin{specialpar}{\tolerance=400}
Les évasions de type extraverties se manifestent sous de nombreuses formes,
 comme l'alcoolisme, l'abus de drogues, la boulimie, le jeu, le nomadisme,
 les affaires extra-maritales, etc.
 Je ne peux pas supporter de faire face à la réalité de mon échec à trouver
 un vrai sens, donc je m'échappe dans l'irréel.
 Ces évasions m'ammènent ensuite à un complexe de culpabilité.
 Je sais que ce que je fais est mal. Je sais que cela me détruit
 ainsi que mon entourage qui m'aime, et pourtant je ne semble pas avoir
 la capacité d'arrêter. Je commence à me haïr moi-même pour ce que je me fais,
 à moi-même et aux autres.
\end{specialpar}

Le complexe de culpabilité se transforme ensuite en désir subconscient
 de punition. Cela se manifeste généralement par un mode de comportement
 névrotique dont le but est d'amener la désapprobation de mes associés,
 que j'interpréte comme une punition, qui à son tour me soulage
 de mes sentiments de culpabilité. Quand j'étais enfant, mon père
 s'est occupé de mon complexe de culpabilité en me punissant.
 Dans mon cas, il s'agissait généralement de la fessée.
 Une fois que j'avais été puni, je ne me sentais plus coupable
 et je pouvais prendre ma place correcte comme membre de la famille.
 Avant la punition, je sentais que la relation était tendue et j'avais
 un sentiment d'aliénation.

Au fur et à mesure que nous grandissons, il n'y a plus d'autorité parentale
 au-dessus de nous, et pour nous libérer de la culpabilité, nous devons donc
 nous comporter de façon inacceptable afin d'amener la désapprobation
 ou le rejet, que nous interprétons comme une punition.
 Une fois punis, nous nous sentons libres de notre complexe de culpabilité
 et nous retournons alors à notre frustration et recommençons le cycle
 à nouveau. Quand Jésus a dit~: \og Si quelqu'un a soif \fg{},
 il faisait référence à cette soif de l'esprit de l'homme pour Dieu,
 que le psychologue répertorie comme une frustration.


\section*{Étancher vraiment sa soif}

Quand Jésus parlait à la Samaritaine, Il lui a demandé à boire,
 et elle l'a mis au défi de lui demander, puisqu'Il était Juif
 et qu'elle était Samaritaine. Par tradition, il ne devait y avoir
 aucune forme de transaction entre eux. Jésus lui a répondu~:
 \og Si tu connaissais qui est celui qui te demande à boire,
 c'est toi qui lui aurais demandé à boire. \fg{}
 Elle lui a répondu plutôt intelligemment~:
 \og Pourquoi te demanderais-je à boire alors que tu n'as rien pour puiser,
 et que ce puits est très profond ? \fg{} Jésus lui a alors dit~:
 \og Quiconque boit de cette eau aura encore soif. \fg{}
 \ibiblephantom{Jn}(4:7-14)Je crois que ce verset devrait être inscrit
 au-dessus de tout but,
 ambition ou quête de plaisir qu'un homme ait.
 Vous pouvez boire de cette eau, atteindre votre but, réaliser votre ambition,
 et combler vos fantasmes, mais vous aurez à nouveau soif.
 Cela ne va pas vous satisfaire, car tout au fond de vous,
 votre esprit a soif de Dieu, et rien ne peut satisfaire cette soif
 si ce n'est une relation de sens avec Dieu.

Quand Jésus a dit~: \og Si quelqu'un a soif, qu'il vienne à moi
 et qu'il boive \fg{}, Il exprimait l'Évangile dans ses termes
 les plus simples. Il disait à toute l'humanité~:
 \og Tout au fond de votre vie, vous avez besoin de Dieu.
 Vous essayez d'atteindre une relation de sens avec Dieu.
 Venez à moi, et votre soif sera non seulement entièrement satisfaite
 et assouvie, mais de votre vie jailliront des torrents d'eau vive. \fg{}
 Seul Christ peut satisfaire votre soif spirituelle,
 car Il vous amène dans une relation de sens avec Dieu.


\section*{Des torrents d'eau}

Dans \ibibleverse{Jn}(7:38), Jésus a continué en disant~:
 \og Celui qui croit en moi, des fleuves d'eau vive couleront de son sein,
 comme dit l'Écriture. \fg{}
 Les mots grecs utilisés ici sont un peu plus intenses que ce qui ressort
 de la version de la Bible à la Colombe. Le Seigneur déclare textuellement
 que, si une personne croit en Lui, \og des torrents d'eau vive jailliront
 de son sein \fg{} \ocadr pas juste un gentil petit ruisseau qui s'écoulerait,
 mais un puissant torrent, comme ceux qui dévalent les ravins de montagne
 en cascades durant la saison des orages.

À quoi Jésus faisait-il référence quand Il a parlé de
 \og torrents d'eau vive \fg{} s'écoulant de notre vie ?
 Quand Jean a écrit cet Évangile, c'était plusieurs années après les faits.
 Son livre était l'un des derniers du Nouveau Testament à être rédigé,
 et Jean écrivait avec le bénéfice du recul. Au moment où Jésus parlait
 des torrents d'eau vive, Jean était probablement confus au sujet de ce que
 Jésus voulait dire ou de ce qu'Il promettait aux gens.
 Mais parce que Jean a écrit l'Évangile avec la compréhension gagnée grâce
 au bénéfice du recul, il a ajouté son propre commentaire exprimé entre
 parenthèses au verset \ibiblevs{Jn}(7:39), dans lequel il explique que Jésus
 parlait de l'Esprit-Saint, \og qu'allaient recevoir ceux qui croiraient
 en lui ; car l'Esprit n'était pas encore [donné], parce que Jésus n'avait
 pas encore été glorifié. \fg{}
 Jésus parlait donc de la puissance offerte à la vie du croyant
 par l'Esprit-Saint.


\section*{Ce que Dieu désire pour vous}

Je crois que nous devons admettre sans hésitation qu'il s'agit là de bien
 plus que de la simple présence de l'Esprit venant demeurer dans la vie
 du croyant à sa conversion. C'est une chose d'avoir l'Esprit-Saint
 demeurant dans votre vie, c'en est une autre d'avoir cette puissance
 glorieuse et dynamique de l'Esprit de Dieu s'épandant de votre vie
 comme un torrent d'eau vive.

Dieu a une relation plus riche pour vous que la seule venue de l'Esprit
 pour demeurer en vous. C'est le désir de Dieu que l'Esprit s'épande
 de votre vie. Le nom que vous donnez à ce phénomène importe peu.
 Certains l'appellent le baptème de l'Esprit-Saint,
 d'autres l'emplissage de l'Esprit-Saint, et d'autres encore le don
 de puissance de l'Esprit-Saint. Le nom que vous lui donnez n'a vraiment
 pas d'im\-por\-tan\-ce ; ce qui est important, c'est que vous ayez ce débordement
 glorieux de l'Esprit s'épandant de votre vie.

Dieu regarde toujours l'homme de deux façons. D'abord, Dieu le regarde
 \emph{subjectivement}, lorsqu'Il cherche à effectuer Son œuvre
 dans votre vie.
 Mais les desseins de Dieu ne se limitent jamais à Son œuvre subjective.
 Dieu considère également l'œuvre \emph{objective} qu'Il souhaite opérer
 par vous.
 Il œuvre en vous \emph{subjectivement} afin de pouvoir œuvrer par vous
 \emph{objectivement}. Il désire opérer une œuvre \emph{en} vous
 et \emph{pour} vous afin
 qu'Il puisse œuvrer \emph{par} vous pour toucher d'autres personnes.
 Notre relation à l'Esprit n'est jamais complète quand Il ne fait que
 demeurer en nous. Nous sommes plus que des récipients pour contenir
 l'Esprit de Dieu. Dieu désire que nous soyons des \emph{canaux} à travers
 lesquels Son Esprit puisse s'écouler.


\section*{La puissance en action}


Lorsque vous regardez votre propre expérience et votre relation
 avec l'Esprit-Saint, si vous ne pouvez pas dire que la dynamique puissante
 de l'Esprit de Dieu jaillit de votre vie comme un fleuve ou comme
 des torrents d'eau vive, alors Dieu a une relation plus pleine
 avec Son Esprit qu'Il désire amener dans votre vie,
 et je voudrais vous encourager à rechercher cette puissance de l'Esprit
 de Dieu jusqu'à ce qu'elle s'épande de votre vie.
 Il y a autour de nous un monde misérable qui a besoin d'être touché
 par la puissance de l'Esprit de Dieu s'épandant de nous.
 Si vous refusez d'appeler cela le baptème de l'Esprit-Saint,
 appelez-le ce que vous voulez, mais ce que Jésus décrit est bien plus
 que la simple présence de l'Esprit-Saint dans la vie du croyant
 vécue au moment de sa conversion. La beauté de l'Esprit qui se dégage de la vie
 d'une personne est le vrai \emph{charisme}.


\closechapter
