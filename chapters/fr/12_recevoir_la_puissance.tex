\chapter{Recevoir la puissance}

\lettrine{I}{l y a souvent} des obstacles à la réception de cette onction
 spéciale ou puissance de l'Esprit-Saint dans votre vie qui doivent
 être surmontés ! Tout d'abord, il y a une impression générale
 de ne pas la mériter. Satan, \index{Satan} remplissant son rôle d'accusateur
 des frères, va essayer d'exagérer nos échecs et nos erreurs,
 et il laisse entendre que nous ne sommes pas dignes de recevoir quoi que ce soit
 de la part de Dieu. Dans un sens, cela est vrai ;
 cependant, Dieu ne nous donne pas Ses dons comme une récompense
 pour notre bonne conduite, mais pour nous permettre de vivre une vie
 qui Lui plaise. La puissance de l'Esprit-Saint est précisément
 la puissance dont j'ai besoin pour m'aider à vivre une vie victorieuse
 en Christ. De plus, dans la mesure où c'est un don de Dieu,
 Il me le donne sur la base de Sa grâce et non de mon mérite.

Un autre obstacle est constitué par les attentes non scripturaires
 que nous pourrions avoir développées à partir d'idées préconçues
 qui sont souvent semées dans nos esprits par le témoignage de l'expérience \index{expérience}
 d'une autre personne. Pendant des années, j'ai pensé que
 quand je recevrais la puissance de l'Esprit, je tomberais dans un état
 d'inconscience ou dans une sorte de transe.
 J'avais entendu les témoignages de ceux qui avaient été remplis de l'Esprit,
 et ils déclaraient souvent\frcolon{} \Og [\dots{}] et quand j'ai repris mes esprits,
 j'ai été surpris d'apprendre que cela faisait quatre heures
 que j'étais là. \Fg{}
 Ainsi, en attendant l'Esprit-Saint, j'attendais souvent en vain
 de glisser dans un état d'inconscience.
 D'autres témoignaient de sensations variées, telles que
 \Og les dix-mille volts d'éléctricité qui ont traversé mon corps \Fg{} ou
 \Og de la sensation de chaleur qui m'a submergé. \Fg{}
 D'autres encore décrivaient les vagues continues de gloire déferlant
 sur eux ou la sensation de picotements en bas de leur colonne vertébrale.
 Certains évoquaient des pleurs incontrôlables,
 tandis que d'autres faisaient mention de violents tremblements.

Tout ceci peut constituer des réactions valides à l'œuvre
 ou à la puissance de l'Esprit dans la vie d'une personne,
 mais leur grande variété montre seulement que Dieu n'est pas lié
 à un seul mode opératoire quand il envoie le don de l'Esprit-Saint sur nos vies.
 Nous ne devrions pas nous attendre à un quelconque type de sensation
 comme une preuve que Dieu nous a rempli de Son Esprit,
 en dehors d'un débordement d'amour, car le fruit de l'Esprit est l'amour.
 \index{Esprit!fruit de l'\textasciitilde{}}

Bien souvent, si je m'attend à un type spécial de réaction ou de sensation,
 je suis déçu quand je ne le reçois pas, et j'ai l'impression que Dieu
 n'a pas envoyé Son don sur moi. J'ai tendance à douter de la validité
 de ma propre expérience, \index{expérience}
 voire de son existence, et à considérer cela
 comme un refus de Dieu de me bénir.


\section{Demandez et recevez}

Si nous voulons recevoir le don de l'Esprit-Saint,
 il nous faut le demander. Dans \ibibleverse{Lc}(11:13), Jésus a dit\frcolon{}
 \Og À combien plus forte raison le Père céleste donnera-t-il
 l'Esprit Saint à ceux qui le lui demandent. \Fg{}
 Demander est une part très importante de la réception.
 \ibibleverse{Jc}(4:2) nous dit que nous ne possédons pas,
 parce que nous ne demandons pas. Beaucoup de gens n'ont pas
 la puissance de l'Esprit de Dieu dans leurs vies aujourd'hui
 simplement parce qu'ils ne l'ont jamais demandée.
 Dans \ibibleverse{Jn}(15:16), Jésus a dit à ses disciples\frcolon{}
 \Og Ce n'est pas vous qui m'avez choisi, mais moi,
 je vous ai choisis et je vous ai établis, afin que vous alliez,
 que vous portiez du fruit, et que votre fruit demeure,
 pour que tout ce que vous demanderez au Père en mon nom,
 il vous le donne\NdT{ou \Og il vous l'accordera \Fg{}, selon la TOB}. \Fg{}
 Notez que Jésus dit littéralement\frcolon{} \Og pour qu'Il puisse vous le donner \Fg{} et non
 \Og Il vous le donnera \Fg{}.
 Autrement dit, c'est quelque chose que Dieu a déjà résolu de faire,
 et le demander ne fait que Lui ouvrir la porte pour qu'Il fasse
 ce qu'Il aspire à faire pour vous.

Dans \ibibleverse{Jn}(16:24), Jésus a dit\frcolon{}
 \Og Demandez et vous recevrez, afin que votre joie soit complète. \Fg{}
 La vie remplie de l'Esprit est pleine de joie.
 La joie est le premier mot que Paul utilise pour définir l'amour,
 qui est le fruit de l'Esprit (\ibibleverse{Ga}(5:22)).
 \index{Esprit!fruit de l'\textasciitilde{}}
 \ibibleverse{IJn}(5:14-15) nous dit que si nous demandons quelque chose
 selon sa volonté, Il nous écoute, et s'Il nous écoute,
 alors nous possédons ce que nous Lui avons demandé.
 Est-ce la volonté de Dieu que nous soyons remplis de l'Esprit?
 Nous savons que c'est bien le cas, car Dieu a donné ce commandement dans
 \ibibleverse{Ep}(5:18)\frcolon{} \Og Soyez remplis de l'Esprit. \Fg{}
 Quand je demande que Dieu me remplisse de l'Esprit,
 j'ai cette assurance de savoir que je demande selon Sa volonté.

Tout ce que je demande à Dieu, je dois le demander par la foi,
 croyant que Dieu va l'exaucer. Ma prochaine étape doit donc être
 l'étape de foi ; je dois croire que Dieu l'a exaucé.
 La foi, c'est la réalité de ce qu'on espère,
 l'attestation de choses qu'on ne voit pas.
 \ibiblephantom{He}(11:1)La foi est la seule preuve dont vous avez besoin ;
 croyez que vous le recevez, et vous le recevrez.
 Nous ne devons pas rechercher des signes immédiats, comme les langues,
 des bouffées de chaleur ou~des~vagues de gloire.
 Ces choses peuvent se produire, mais pas obligatoirement,
 et je ne dois pas rechercher des sentiments comme preuves
 que Dieu a accordé la réponse à ma prière. Notre foi doit toujours se reposer
 sur la Parole de Dieu qui est certaine, et jamais sur un sentiment.
 Nos sentiments changent souvent, mais la Parole de Dieu, jamais.

Paul a demandé aux croyants Galates\frcolon{}
 \Og Est-ce en pratiquant la loi que vous avez reçu l'Esprit,
 ou en écoutant avec foi ? \Fg{}
 \ibiblephantom{Ga}(3:2)Cela est également vrai dans nos vies.
 Être rempli de l'Esprit-Saint n'est pas une récompense que Dieu
 me donne pour service méritoire, mais simplement un pur don de Sa grâce.
 Dans \ibibleverse{Rm}(4:20), nous lisons que, parce qu'Abraham avait
 une foi solide, il a rendu gloire à Dieu.
 Demandez à Dieu dès maintenant de vous remplir de Son Esprit-Saint,
 et commencez à exercer votre foi en louant Dieu dès maintenant
 pour cette nouvelle dynamique d'amour qu'Il déverse dans votre vie.
\closechapter

