\chapter{Une voie plus excellente}

\lettrine{N}{ous lisons dans \ibibleverse{Ga}(5:)}
 au sujet de la bataille spirituelle
 qui prend place dans la vie de chaque chrétien.
 Cette bataille n'a pas lieu dans la vie du non-chrétien ;
 il n'en sait rien, car l'esprit du non-chrétien est en sommeil.

Mais une fois que l'esprit a pris vie, une fois que vous êtes nés de nouveau,
 il y a un conflit interne. Dans \ibibleverse{Ga}(5:17), Paul dit~:
 \og Car la chair a des désirs contraires à l'Esprit,
 et l'Esprit en a de contraires à la chair ; ils sont opposés l'un à l'autre,
 afin que vous ne fassiez pas ce que vous voudriez. \fg{}
 La bataille qui a lieu est celle de l'esprit contre la chair ;
 la chair vous empêche de faire les choses que vous souhaitez
 faire pour le Seigneur.
 \og Or, les œuvres de la chair sont évidentes, c'est-à-dire inconduite,
 impureté, débauche, idolâtrie, magie, hostilités, discorde, jalousie,
 fureurs, rivalités, divisions, partis-pris, envie, ivrognerie, orgies,
 et choses semblables. Je vous préviens comme je l'ai déjà fait~:
 ceux qui se livrent à de telles pratiques n'hériteront pas
 du royaume de Dieu \fg{} (versets \ibiblevs{Ga}(5:19-21)).

En constrate marqué avec les œuvres de la chair, on trouve les résultats
 de l'Esprit~: \og Mais le fruit de l'Esprit est [l'] amour [agapé] \fg{}
 (verset \ibiblevs{Ga}(5:22)). La méthode de Dieu réside dans le fruit
 plutôt que les œuvres. Toutes les fois que vous entrez dans le domaine
 des \emph{œuvres}, vous entrez dans le domaine de la chair.
 Mais le \emph{fruit}, qui nous indique la méthode de Dieu, est la conséquence
 naturelle d'une relation. On ne voit pas un abricotier lutter
 et se démener pour produire des abricots, ni des abricots lutter
 et se démener pour mûrir ; ça n'est qu'un processus naturel.


\section{Comment porter du fruit}

La méthode que Dieu a pour vous est simplement le processus naturel
 de Dieu pour œuvrer dans votre vie : ce n'est pas quelque chose
 que vous pouvez faire, et vous n'avez pas besoin de vous démener
 ou de lutter pour essayer de le développer.
 Dès l'instant où vous commencez à vous démener ou à lutter,
 vous êtes à nouveau dans le domaine des œuvres.
 La méthode de Dieu est le fruit, et le fruit est la conséquence
 naturelle de la relation positionnelle en Christ.

Jésus a dit~: \og Je suis le cep et vous les sarments.
 Toute branche qui demeure en moi et porte du fruit,
 le Père l'émonde [ou la nettoie, la lave] afin qu'elle porte
 encore plus de fruit. Maintenant , vous êtes lavés par la parole
 que je vous ai dite ; demeurez en moi et laissez mes paroles
 demeurer en vous. \fg{}
 Lorsque que vous demeurez en Christ, vous allez porter du fruit.
 Jésus a dit~: \og Vous ne pouvez pas porter de fruit à moins
 de prendre part au cep. \fg{} \ibiblephantom{Jn}(15:2-7)La branche
 ne peut pas produire de fruit par elle-même.
 Vous devez prendre part au cep si vous voulez porter du fruit.
 Jésus a dit également~: \og Sans moi, vous ne pouvez rien faire. \fg{}
 Mais lorsque vous demeurez en Christ, le résultat naturel
 est que Son amour va commencer à s'écouler de votre vie.
 La méthode de Dieu est le fruit \ocadr la conséquence très simple,
 très naturelle d'une juste position en Christ.

Le mot \og Esprit \fg{} dans \og le fruit de l'Esprit \fg{}
 nous indique la dynamique de Dieu \ocadr l'œuvre de l'Esprit-Saint
 dans notre vie de croyant. Ce n'est pas par hasard que le
 chapitre~\ibiblechvs{ICo}(13:) de \bibleverse{ICo} est un encart
 au milieu de la discussion de Paul sur les dons de l'Esprit
 aux chapitres~\ibiblechvs{ICo}(12:) et \ibiblechvs{ICo}(14:).
 Au chapitre~\ibiblechvs{ICo}(12:), il liste les nombreux dons de l'Esprit ;
 au chapitre~\ibiblechvs{ICo}(14:), il décrit comment certains de ces dons
 opèrent, et le but de certains de ces dons.
 Mais à la fin du chapitre~\ibiblechvs{ICo}(12:), il dit~:
 \og Je vais encore vous montrer une voie par excellence, \fg{}
 quelque chose d'encore mieux que de posséder ces merveilleux dons.


\section{La voie par excellence}

Souvent, nous disons~:
 \og Oh Dieu, je veux avoir le don d'opérer des miracles \fg{} ou
 \og Je veux avoir le don de la foi \fg{} ou
 \og Je veux avoir le don de guérison \fg{} ou
 \og Je veux avoir le don de discernement des esprits \fg{} ou
 \og Je veux avoir le don des paroles de connaissance. \fg{}
 Nous voudrions avoir ces dons surnaturels à l'œuvre dans nos vies.
 Mais Paul a dit~:
 \og Je vais encore vous montrer une voie par excellence. \fg{}
 Plus encore que d'avoir des dons surnaturels à l'œuvre dans ma vie,
 il est préférable d'avoir l'amour de Dieu s'écoulant de ma vie,
 et si cet amour ne s'écoule pas, ces dons surnaturels perdent leur intérêt.

\og Le fruit de l'Esprit \fg{} nous indique la dynamique de Dieu.
 Jésus a dit~: \og Vous recevrez une puissance [\emph{dunamis}],
 celle du Saint-Esprit survenant sur vous. \fg{}
 L'Esprit-Saint est la dynamique de Dieu dans nos vies.
 Il est cette puissance en nous qui nous donne la capacité
 d'être ce que nous ne pourrions pas être sans Lui,
 d'avoir ce que nous ne pouvons pas avoir sans Lui,
 et de faire ce que nous ne pourrions pas faire sans Lui.
 Vous ne pouvez pas avoir l'amour agapé sans l'Esprit-Saint,
 et vous ne pouvez pas exprimer l'agapé sans l'Esprit-Saint.
 Le fruit de l'Esprit de Dieu dans votre vie est que cet amour vienne.
 L'écoulement naturel de l'Esprit de Dieu en vous sera cet amour,
 car l'Esprit de Dieu est la puissance dynamique de Dieu
 en vous qui produit cet agapé à partir de votre vie.
 La méthode de Dieu est le fruit ; la dynamique de Dieu est l'Esprit.


\section{Le v\'eritable fruit de l'Esprit}

Vous avez probablement entendu qu'il y a neuf fruits de l'Esprit.
 J'ai entendu des enseignements sur les neuf dons de l'Esprit
 et les neuf fruits de l'Esprit. Mais je veux que vous lisiez
 attentivement \ibibleverse{Ga}(5:22)~:
 \og Le fruit de l'Esprit est l'\emph{agapé}. \fg{}
 De la manière dont je comprend la grammaire anglaise
 et de la manière dont je comprend la grammaire grecque,
 s'il y avait neuf fruits, Paul aurait dit~:
 \og Mais les \emph{fruits} de l'Esprit \emph{sont} l'amour,
 la joie, la paix, etc. \fg{}
 \nowidow

Mais ce n'est pas ce qu'il a dit. Il a utilisé le singulier~:
 \og Le fruit de l'Esprit est l'amour. \fg{}
 Alors que sont ces autres choses citées dans le verset ?
 Qu'en est-il de la joie, de la paix, de la patience, de la bonté,
 de la bienveillance, de la fidélité, de la douceur
 et de la maîtrise de soi ? Ils définissent tous l'agapé.
 Notre utilisation du mot \og amour \fg{} en français est si médiocre
 qu'elle peut vouloir dire presque n'importe quoi.
 C'est pourquoi Paul définit l'agapé en utilisant ces autres mots.

La \emph{joie} est la conscience de l'amour.
 Avez-vous déjà vu une
 personne vraiment amoureuse ?
 La caractéristique principale d'une telle personne est la joie
 qu'elle possède. Oh, quelle joie il y a dans l'amour véritable !
 Vous pouvez faire face à des situations très difficiles
 et rester dans une joie véritable. Vous pouvez accomplir
 des tâches misérables, mais s'il y a un amour véritable,
 il y a une joie glorieuse. En parlant avec une jeune fille
 il y a quelques temps, je lui ai demandé~:
 \og Eh bien, comment vas-tu ? \fg{} Elle m'a répondu~:
 \og Oh, je vais très bien! Je viens de me marier et je n'ai plus besoin
 de travailler. \fg{}
 Elle voulait dire qu'elle n'avait plus à travailler de~8~h~à~17~h,
 assise derrière un bureau. Elle faisait probablement plus de travail
 qu'auparavant, mais il y avait maintenant un tel amour
 qu'elle ne considérait même pas cela comme du travail.
 L'amour change toute tâche en plaisir. Quand vous aimez vraiment,
 vous ne vous crispez pas à cause des choses que vous faites
 pour ceux que vous aimez. Vous y prenez plaisir.
 La joie est la conscience de l'amour.

La seconde caractéristique de l'agapé de l'Esprit est la \emph{paix}.
 Il n'y a pas de vraie paix en dehors de l'amour agapé.
 Quelqu'un a dit~: \og Désormais, le Moyen-Orient est en paix. \fg{}
 Ne le croyez pas ! Il n'y a pas du tout de paix.
 Il y a tant de haine ; il y a tant d'amertume.
 Il n'y a aucune paix réelle là-bas. En un instant, tout ça pourrait
 exploser en une guerre à grande échelle.
 La seule vraie base pour la paix est l'amour.
 Vous pouvez avoir un arrêt des hostilités ; vous pouvez avoir des accords ;
 mais la seule vraie base de la paix est l'amour véritable.
 Quand je vous aime tellement que je ne voudrais pas faire
 quoi que ce soit qui vous fasse du mal, alors il y a la paix entre nous.

La \emph{patience}. Paul a utilisé ce mot dans la définition de l'agapé
 dans \ibibleverse{ICo}(13:)~:
 \og L'amour est patient, l'amour est serviable. \fg{}
 Si vous aimez véritablement quelqu'un, vous ne comptez pas
 le nombre de fois où il vous a offensé. Vous êtes patient.
 Vous supportez, vous supportez, et vous supportez encore,
 et pour finir vous êtes serviable.

Une autre caractéristique de l'agapé est sa \emph{bonté}~:
 Oh, que l'amour est bon ! Quelle belle qualité,
 qu'elle admirable qualité que la bonté de l'amour véritable !

Puis vient la \emph{bienveillance}. Je crois que l'amour est la seule
 vraie motivation de la bienveillance.
 Beaucoup de gens sont bienveillants seulement parcequ'ils ont peur
 des conséquences d'une mauvaise attitude.
 Mais ce n'est pas la vraie bienveillance.
 \og J'aimerais bien te tuer, mais je finirais en prison. \fg{}
 \og J'aimerais bien dévaliser cette banque mais je risque
 de me faire prendre. \fg{}
 Beaucoup de gens sont empêchés d'agir mal seulement par peur
 des conséquences. Cela n'est pas de la bienveillance.
 La seule vraie motivation de la bienveillance est l'amour.
 À cause de l'amour, je ne voudrais pas faire de mal
 et je ne voudrais pas offenser. Je ne ferais rien qui cause
 la chute d'une personne, parce que j'aime cette personne.
 Voilà la vraie motivation de la bienveillance.

Une autre caractéristique de l'agapé est la \emph{fidélité}
 (ou la \emph{foi})
 \NdT{La Bible King James, tout comme la Nouvelle Bible Segond et la TOB,
 traduit le mot grec \emph{pistis} par
 \og foi \fg{}, alors que la Colombe utilise le mot \og fidélité \fg{}.
 La Bible Parole de Vie parle quant à elle de \og confiance
 dans les autres \fg{}.}.
 Il ne s'agit pas de la même foi que celle que nous trouvons citée
 comme un don de l'Esprit dans \ibibleverse{ICo}(12:),
 mais c'est une fidélité ou une confiance dans les \emph{gens}.
 Il s'agit simplement d'une forme de confiance. Si vous dites~:
 \og Je ne fais confiance à personne \fg{}, vous dites en fait~:
 \og Je n'aime personne. \fg{}
 Si vous aimez vraiment quelqu'un, vous allez lui faire confiance,
 parce que la confiance fait partie des qualités de l'amour agapé.

Enfin, il y a la \emph{douceur} (ou \emph{soumission})
 \NdT{La plupart des Bibles francophones traduisent le mot grec
 \emph{praotes} par  \og douceur \fg{} alors que la Bible King James
 utilise le mot \emph{meekness}, qui signifie \og soumission \fg{}.}.
 Le véritable amour ne recherche pas son propre intérêt,
 ne se vante pas et ne s'enfle pas d'orgueil.
 Une des caractéristiques principales de Jésus-Christ
 est Son admirable soumission. Il aurait pu frimer quand Il était là.
 Après tout, regardez qui Il était. Souvent, Jésus parlait de Lui-même
 comme du \og Fils de l'Homme \fg{}.
 Il avait de nombreux titres glorieux qu'Il aurait pu porter~:
 le Fils de la Justice, le Fils de la Gloire, l'Étoile Brillante du Matin,
 le Plus Beau Entre Dix-Mille, le Lis des Vallées,
 le Conseiller Merveilleux, le Dieu Puissant, le Père Éternel,
 le Prince de Paix. Jésus aurait pu porter tous ces titres
 de manière appropriée. Il aurait pu dire~:
 \og L'Étoile Brillante du Matin vous dit \fg{} ou
 \og L'Oint de Dieu déclare \fg{}, mais au lieu de cela,
 Il s'est souvent réferré à Lui-même comme au Fils de l'Homme~:
 \og Le Fils de l'homme est venu chercher et sauver
 ce qui était perdu. \fg{} \ibiblephantom{Lc}(19:10)

Beaucoup de gens aujourd'hui s'honorent les uns les autres
 avec des titres sophistiqués. Que les gens aiment les titres !
 Mais Jésus a dédaigné les titres ; Il a rabaissé ceux qui aimaient
 se tenir sur la place centrale en se faisant appeler~:
 \og Rabbi, Rabbi ! \fg{}
 Quelqu'un a dit un jour que les titres ne sont que des distinctions
 pour différencier un ver de terre d'un autre. Que suis-je ?
 \emph{Rien}, si je n'ai pas Dieu. La caractéristique de l'amour
 est celle de la soumission.

Enfin, la \emph{maîtrise de soi}. La meilleure manière que je puisse imaginer
 pour définir la maîtrise de soi est de donner son opposé~: \emph{l'excès}.
 Malheureusement, nous savons tous trop bien de quoi il s'agit,
 et c'est l'opposé de la maîtrise de soi.
 La maîtrise de soi est la modération, le fait de ne pas faire de folies.
 C'est l'admirable régularité de l'amour.


\section{Le fruit dans votre vie}

L'amour agapé est ce que l'Esprit-Saint cherche à apporter dans votre vie ;
 l'agapé est le véritable fruit de l'Esprit.
 Il sera le résultat final de la présence de l'Esprit de Dieu en vous.
 Lorsque l'Esprit de Dieu œuvre en vous et que vous cédez à l'Esprit,
 le fruit de l'Esprit est l'amour agapé. Le but de l'œuvre de
 l'Esprit-Saint dans votre vie de croyant est de faire pour vous
 ce que vous ne pouvez pas faire pour vous-même~:
 vous donner l'amour agapé de Dieu envers la famille de Dieu.

Ce sera un signe par lequel le monde saura que vous êtes un disciple
 de Christ, et un signe par lequel vous saurez que vous êtes passé
 de la mort à la vie. C'est un signe parce que vous verrez
 l'amour de Dieu à l'œuvre dans votre vie.

Nous avons besoin d'être remplis par l'Esprit.
 Nous avons besoin de céder à l'Esprit afin que Son fruit soit produit
 en abondance dans nos vies. Alors Son amour agapé nous dirigera,
 s'écoulant de nos vies comme un torrent d'eau vive.
\closechapter

