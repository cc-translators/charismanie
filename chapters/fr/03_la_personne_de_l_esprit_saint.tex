\chapter{La personne de l'Esprit-Saint}

\lettrine{P}{uisque} nous voulons avoir une rencontre personnelle avec
 l'Esprit-Saint, nous allons maintenant montrer que les Écritures enseignent
 que l'Esprit-Saint est une personne, plutôt qu'une simple essence, une force, ou
 encore une puissance. Vous pouvez avoir une puissance brute sans personnalité,
 comme l'électricité, mais il est difficile d'avoir une relation intime et
 proche avec une telle puissance impersonnelle.

Le mot \emph{esprit} en grec est \emph{pneuma}, qui est du genre neutre.
 À cause de cela, dans l'Église primitive, un brillant théologien du nom
 d'Arius a commencé à promouvoir l'idée que Jésus était moins que Dieu,
 puisqu'il avait été créé par Dieu, et que l'Esprit-Saint était seulement
 une \Og essence \Fg{} de Dieu. Cela devint connu sous le nom d'Hérésie
 Arienne; elle existe toujours et continue d'attirer de nombreux adeptes. Le concile de
 Nicée radia Arius de ses fonctions et déclara ses enseignements hérétiques.
 L'Esprit-Saint est plus qu'une simple essence ou force ; Il est
 une personne. Vous ne devriez pas adorer une force ou une essence.
 Pouvez-vous vous imaginer chantant la doxologie\frcolon{}
 \Og Louez le Père, le Fils et l'Essence \Fg{} ?
 Il est une personne, et en tant qu'une des personnes
 de la Divinité, Il est digne de louange. Si nous ne croyons pas que le
 Saint-Esprit est une personne, nous Lui refusons la louange et l'adoration qui lui
 reviennent. Si nous ne réalisons pas que le Saint-Esprit est une personne,
 nous nous trouvons dans une position où nous cherchons à établir un
 rapport avec une force ou une essence. Nous dirions\frcolon{} \Og J'ai besoin de livrer
 ma vie à ça \Fg{}, ou\frcolon{} \Og J'ai besoin de plus de ça dans~ma~vie.\Fg


\section{Conna\^itre, agir, ressentir}

Le fait qu'Il est une personne est clairement montré dans les Écritures.
 Certaines des caractéristiques qui lui sont assignées ne peuvent être assignées
 qu'à des personnes. Une personne est un être qui possède une raison,
 une volonté et des sentiments.
 Si, dans les Écritures, ces caractéristiques sont assignées à
 l'Esprit-Saint, alors nous devons conclure que l'Esprit est une personne.
 Dans \ibibleverse{ICo}(2:10-11), nous lisons\frcolon{} \Og À nous, Dieu nous l'a
 révélé par l'Esprit. Car l'Esprit sonde tout, même les profondeurs de Dieu.
 Qui donc, parmi les hommes, sait ce qui concerne l'homme, si ce n'est
 l'esprit de l'homme qui est en lui ? De même, personne ne connaît ce qui
 concerne Dieu, si ce n'est l'Esprit de Dieu. \Fg{} Ici il est fait référence
 à l'Esprit possédant une connaissance. Une force brute ou une puissance ne
 possède pas de connaissance. Il serait absurde de remplacer le mot
 \Og Esprit \Fg{} par \Og essence \Fg{} dans le texte, car vous auriez alors
 une \Og essence \Fg{} qui sonde tout!

Dans \ibibleverse{Rm}(8:27), Paul dit\frcolon{} \Og Et celui qui sonde les cœurs
 connaît quelle est l'intention de l'Esprit\frcolon{} c'est selon Dieu qu'il
 intercède en faveur des saints. \Fg{} Ici, il est fait référence
 à la \emph{raison} de l'Esprit, une caractéristique qui n'est pas associée à
 une simple essence. Dans \ibibleverse{ICo}(12:11), Paul, à propos des dons de
 l'Esprit, dit\frcolon{} \Og Un seul et même Esprit opère toutes ces choses, les
 distribuant à chacun en particulier comme il veut. \Fg{} L'Esprit-Saint
 possède donc une \emph{volonté}, un trait associé à une personne.

Dans \ibibleverse{Rm}(15:30), Paul associe l'émotion de l'amour avec l'Es\-prit.
 Une force ou une puissance ne peut pas aimer. On n'associe pas l'amour à autre
 chose qu'une personne. Il est intéressant que, bien que j'aie lu ou
 entendu un grand nombre de sermons sur l'amour de Dieu, et l'amour de
 Jésus-Christ pour nous, je n'ai encore jamais entendu de sermon sur l'amour de
 l'Esprit-Saint. Et pourtant cela doit être l'une des principales
 caractéristiques de l'Esprit, puisque c'est le fruit qu'Il produit dans nos
 vies.\index{Esprit!fruit de l'\textasciitilde{}}
 L'Esprit-Saint possède réellement des émotions et peut être attristé,
 car Paul dans \ibibleverse{Ep}(4:30) exhorte l'Église à ne pas attrister
 l'Esprit-Saint de Dieu. Réfléchissez à quel point il semblerait aberrant
 de dire que vous avez attristé l'essence.


\section{Les pronoms personnels}

Tout au long des Écritures, des pronoms personnels sont utilisés pour se
 référer à l'Esprit-Saint. Dans \ibibleverse{Jn}(14:16), Jésus a dit\frcolon{}
 \Og Et moi, je prierai le Père, et il vous donnera un autre Consolateur
 qui soit éternellement avec vous. \Fg{}
 Ici, le pronom \Og qui \Fg{} est utilisé pour désigner l'Esprit\NdT{L'auteur insiste ici sur le fait que la Bible utilise un pronom personnel
 non neutre pour désigner l'Esprit-Saint.
 Là où la version à la Colombe utilise \Og qui \Fg{} dans le verset cité, d'autres versions
 font usage d'un pronom personnel pour désigner l'Esprit-Saint.
 Par exemple, la Nouvelle Bible Segond traduit\frcolon{}
 \Og Moi, je demanderai au Père de vous donner
 un autre défenseur pour qu'il soit avec vous pour toujours. \Fg{}
 La Bible Parole de Vie accentue encore l'aspect personnel de l'Esprit-Saint
 dans sa traduction de ce verset en utilisant le mot \Og quelqu'un \Fg{}\frcolon{}
 \Og Et moi, je prierai le Père.
 Et il vous donnera quelqu'un d'autre pour vous aider,
 quelqu'un qui sera avec vous pour toujours. \Fg{}}.
 Si vous croyez en un Dieu personnel, vous devriez également croire en
 un Esprit personnel. Dans ce même passage, Jésus a continué en disant
 que le monde ne pouvait pas recevoir l'Esprit car il ne \emph{Le} voyait ni
 ne \emph{Le} connaissait. Jésus a dit que vous \emph{Le} connaissez,
 car \emph{Il} habite en vous.

Remarquez le nombre de fois où Jésus utilise un pronom personnel pour
 désigner l'Esprit-Saint. Dans \ibibleverse{Jn}(16:7-14), Jésus utilise
 plusieurs fois ce pronom personnel pour se référer à l'Esprit-Saint.
 \Og Cependant, je vous dis la vérité\frcolon{} il est avantageux pour vous
 que je parte, car si je ne pars pas, le Consolateur ne viendra pas
 \emph{vers} vous ; mais si je m'en vais, je vous \emph{l'}enverrai.
 Et quand \emph{il} sera venu, il convaincra le monde de péché,
 de justice et de jugement\frcolon{} de péché, parce qu'ils ne croient pas en moi ;
 de justice, parce que je vais vers le Père, et que vous ne me verrez plus ;
 de jugement, parce que le prince de ce monde est jugé.
 J'ai encore beaucoup de choses à vous dire, mais vous ne pouvez pas
 les comprendre maintenant. Quand \emph{il} sera venu, \emph{lui},
 l'Esprit de vérité, \emph{il} vous conduira dans toute la vérité ;
 car ses paroles ne viendront pas de \emph{lui-même}, mais il parlera de
 tout ce qu'\emph{il} aura entendu et vous annoncera les choses à venir.
 \emph{Lui} me glorifiera, parce qu'\emph{il} prendra de ce qui est à moi
 et vous l'annoncera. \Fg{}
 Dans le texte grec, le pronom personnel \Og il \Fg{} ou \Og lui \Fg{}
 est utilisé pour l'Esprit maintes et maintes fois dans les Écritures.

\section{L'Esprit en action}

Dans les Écritures, des actes réservés à des personnes sont attribués à l'Esprit-Saint.
 Dans \ibibleverse{Ac}(13:2), nous lisons\frcolon{} \Og Le Saint-Esprit dit\frcolon{}
 Mettez-moi à part Barnabas et Saul pour l'œuvre à laquelle je les ai
 appelés. \Fg{}
 Ici aussi, insérer \Og puissance \Fg{} ou \Og essence \Fg{} à la place
 de l'Esprit est incompréhensible. Comment une essence ou une puissance
 peut-elle parler ? Dans \ibibleverse{Rm}(8:26), on nous dit que l'Esprit-Saint
 Lui-même intercède pour nous par des soupirs inexprimables.
 Encore une fois, essayez de concevoir une simple force qui intercéderait !
 Si l'Esprit-Saint n'était qu'une essence ou une force, vous devriez pouvoir
 insérer les mots \Og force \Fg{} ou \Og essence \Fg{} lorsqu'Il est mentionné
 dans les Écritures, et la signification du texte n'en serait pas altérée pour
 autant. Mais une telle chose est bien évidemment impossible, car
 l'Esprit-Saint est une personne. L'Esprit-Saint témoigne de Jésus-Christ dans
 \ibibleverse{Jn}(15:26), et Il enseigne les croyants et leur rappelle des choses
 dans \ibibleverse{Jn}(14:26). Dans \ibibleverse{Ac}(16:2-7), l'Esprit-Saint a
 empêché Paul et ses compagnons d'aller en Asie et ne les a pas laissés aller
 en Bithynie. Dans \ibibleverse{Gn}(6:3), nous voyons que l'Esprit-Saint lutte
 avec l'homme.

L'Esprit-Saint peut être traité comme une \emph{personne}. Il peut être offensé.
 Il est impossible de concevoir d'offenser \Og la puissance \Fg{} ou
 \Og le souffle \Fg{}. Votre souffle peut-être offensif, mais vous ne pouvez
 pas offenser votre souffle ! Dans \ibibleverse{Ep}(4:30), Paul exhorte\frcolon{}
 \Og N'attristez pas le Saint-Esprit de Dieu. \Fg{}
 Il est possible de mentir à l'Esprit-Saint. Voici l'accusation que Pierre a
 formulée à l'encontre d'Ananias\frcolon{}
 \Og Vous avez menti à l'Esprit-Saint. \Fg{}
 Il est également possible de blasphémer contre l'Esprit-Saint.
 Jésus a dit que c'était un péché tellement horrible qu'il ne pouvait pas
 être pardonné à la personne qui l'avait commis. Il a dit\frcolon{}
 \Og Vous pouvez blasphémer contre moi et être pardonné, mais pas contre
 l'Esprit-Saint. \Fg{}
 Ici, Jésus fait une distinction claire entre Lui-même et l'Esprit-Saint.

L'Esprit-Saint est identifié à des personnes. Paul a dit\frcolon{}
 \Og Cela parût bon au Saint-Esprit et à nous \Fg{}
 (\ibibleverse{Ac}(15:28)).
 Essayez de remplacer par vent ou puissance dans ce verset et voyez si cela
 a toujours du sens !

L'Esprit-Saint est une personne ; il n'est pas simplement l'essence de Dieu.
 Vous avez besoin d'entrer dans une relation personnelle avec Lui pour
 commencer à faire l'expérience de Son amour et de Sa puissance à l'œuvre
 dans votre vie en Le laissant vous guider dans votre cheminement spirituel.
 \index{expérience}


\section{La puissance de l'Esprit}

Vous est-il déjà arrivé de sentir que vous deviez faire part à quelqu'un
 du besoin d'accepter Jésus-Christ, mais sans avoir le cran d'aborder le sujet?
 Vous est-il déjà arrivé, en passant devant une université,
 d'observer des centaines d'étudiants, de prendre conscience que la plupart d'entre eux sont perdus,
 et de vous demander s'ils pourraient être gagnés à Christ ?
 Vous arrive-t-il de penser aux milliards de gens à qui l'on n'a jamais
 annoncé la vérité de l'Évangile, et vous êtes-vous ensuite demandé
 comment cela pourrait être accompli?

Pour Pierre, qui a renié son Seigneur en présence d'une seule servante,
 et pour le reste des disciples (qui se sont enfuis quand les choses se sont gâtées),
 le commandement de Jésus d'aller dans le monde entier et de prêcher
 l'Évangile à toute créature doit avoir paru complètement inapplicable voire
 impossible, et c'était bien le cas.
 Il n'était pas possible que onze hommes insignifiants venant de Galilée
 puissent toucher le monde pour Jésus-Christ.
 C'est pour cela que Jésus leur a dit d'attendre à Jérusalem jusqu'à ce
 qu'ils reçoivent la \emph{puissance de l'Esprit-Saint}, car c'était par
 sa puissance qu'ils pourraient témoigner jusqu'aux extrémités de la terre.

\index{expérience}
Dieu réservait-il cette expérience de la puissance du Saint-Esprit à la seule
 Église primitive? Les Écritures indiquent-elles qu'il viendrait
 un temps où nous n'aurions plus besoin de dépendre de la puissance
 de l'Esprit, mais où nous pourrions, par notre connaissance parfaite
 des Écritures, faire l'œuvre de Dieu par nous-mêmes? L'Église qui a été
 commencée dans l'Esprit doit-elle maintenant être perfectionnée dans
 la chair?
 Quelle est la réponse à l'impuissance de l'Église?
 Pourquoi l'Église n'a-t-elle pas réussi à arrêter la folle chute
 du monde corrompu qui nous entoure?

Paul nous prévient dans \ibibleverse{He}(4:) de craindre de ne pas
 recevoir la promesse de Dieu d'entrer dans Son repos.
 N'est-il pas également approprié pour nous de craindre que, si Dieu
 nous a fait une promesse de puissance dans nos vies personnelles et
 dans le corps de l'Église tout entier, nous en soyons dépourvus ?


\section{La promesse du P\`ere}

Dans \bibleverse{Ac}(1:), nous lisons que les disciples étaient avec
 Jésus à Béthanie, d'où Il allait bientôt les quitter pour monter
 aux cieux.
 Les nuages allaient Le recevoir alors hors de leur vue.
 Il leur donnait ses instructions finales, qui étaient donc de la plus
 grande importance.
 Dans \ibibleverse{Ac}(1:4), Jésus leur a dit de \Og ne pas s'éloigner
 de Jérusalem, mais d'attendre la \emph{promesse} du Père dont, leur dit-il,
 vous m'avez entendu parler. \Fg{}
 Dans \ibibleverse{Lc}(24:49), Jésus a dit\frcolon{}
 \Og Et voici, j'enverrai sur vous ce que mon Père a \emph{promis}, mais vous,
 restez dans la ville, jusqu'à ce que vous soyez revêtus de la puissance
 d'en haut.\Fg{}

Dans ces deux passages, Jésus se référait à la promesse du Père, qui est
 sans nul doute une référence à \ibibleverse{Jl}(3:1-2), où Dieu a promis\frcolon{}
 \Og Après cela, je répandrai mon Esprit sur toute chair ;
 vos fils et vos filles prophétiseront, vos anciens auront des songes,
 et vos jeunes gens des visions.
 Même sur les serviteurs et sur les servantes,
 en ces jours-là, je répandrai mon Esprit. \Fg{}
 Cela est confirmé dans le deuxième chapitre des Actes,
 \ibiblephantom{Ac}(2:12,16-17)lorsque la foule qui
 s'était assemblée à cause du phénomène surnaturel qui accompagnait l'envoi
 de l'Esprit-Saint se demandait\frcolon{}
 \Og Qu'est-ce que cela signifie ? \Fg{}
 Pierre a répondu en expliquant\frcolon{}
 \Og C'est ce qui a été dit par le prophète Joël \Fg{},
 et il a cité la prophétie de Joël.
 La promesse de Dieu était qu'un jour viendrait où Il enverrait Son Esprit,
 pas seulement sur des individus en particulier, mais sur toute chair.

\section{La promesse du Sauveur}

Jésus a également promis son Esprit à Ses disciples dans
 \BRallowhypbch\ibibleverse{Jn}(14:16-17)\BRforbidhypbch, quand Il a dit\frcolon{}
 \Og Et moi, je prierai le Père, et il vous donnera un autre Consolateur
 qui soit éternellement avec vous, l'Esprit de vérité, que le monde ne peut
 pas recevoir, parce qu'il ne le voit pas et ne le connaît pas ; mais vous,
 vous le connaissez, parce qu'il demeure près de vous
 et qu'il sera en vous. \Fg{}
 Lorsque Jésus a promis l'Esprit-Saint, Il se référait à Lui comme à
 \Og un autre Consolateur \Fg{}.
 Le mot traduit par \Og consolateur \Fg{} vient du grec \emph{parakletos},
 qui signifie littéralement \Og celui qui vient aux côtés pour aider \Fg{}.
 C'est le ministère de base de l'Esprit-Saint auprès du croyant.
 Il est là pour nous aider. Jusqu'à ce moment-là, Jésus avait été aux côtés
 de Ses disciples, les aidant. Ils étaient, avec raison, devenus dépendants
 de Son aide. Il maîtrisait toutes les situations.

Lorsque la tempête a menacé de couler leur petite embarcation, Jésus
 a interpellé le vent et les vagues, et il s'est fait un grand calme.
 Lorsque le collecteur d'impôts exigeait des taxes injustes,
 Jésus a dit à Pierre d'aller pêcher un poisson et de prendre
 la pièce dans sa bouche afin de payer les taxes.
 Quelle que soit la situation, Jésus était toujours prêt à aider.

Mais maintenant, Il leur a dit qu'Il allait les quitter.
 Il ne serait plus avec eux comme par le passé.
 Leurs cœurs devaient être troublés par Ses mots,
 et ils avaient peur d'envisager l'avenir sans Lui.
 C'est pourquoi Il leur a promis qu'Il ne les laisserait pas sans aide,
 qu'Il demanderait au Père de leur envoyer un autre Aide ou Consolateur
 pour rester avec eux éternellement\frcolon{}
 l'Esprit de vérité.
 Pour notre démarche de chrétiens, nous sommes complètement dépendants de
 l'aide de l'Esprit-Saint.
 Il est impossible de faire quoi que ce soit de~valable dans le service
 chrétien sans Son aide.

\section{L'attente \`a J\'erusalem}

À cause des mots \Og restez dans la ville \Fg{} utilisés dans
 l'Évangile
 selon Luc, beaucoup de pentecôtistes ont établi des \Og réunions
 d'attente \Fg{} comme moyen par lequel la puissance de l'Esprit-Saint est
 reçue dans la vie du croyant. Notez que le commandement était de \Og rester
 à Jérusalem\NdT{La Bible à la Colombe traduit \Og restez dans la
 ville \Fg{}, alors que la version Ostervald précise
 \Og dans la ville de Jérusalem \Fg{}.} \Fg{},
 donc pour être vraiment scripturaire, les réunions d'attente devraient toutes
 avoir lieu à Jérusalem!

Il est clair que Jésus n'établissait pas ainsi une méthode universelle par
 laquelle l'Esprit-Saint serait conféré aux croyants de tous les temps. Il les
 encourageait seulement à attendre quelques jours à Jérusalem jusqu'à ce
 qu'Il envoie l'Esprit-Saint comme don à l'Église. Une fois l'Esprit-Saint
 donné le jour de la Pentecôte, il n'allait plus jamais être nécessaire de
 l'attendre à nouveau, et nous ne trouvons pas dans
 le livre des Actes de réunion d'attente, et elles ne sont~pas non plus
 recommandées dans le Nouveau Testament comme la méthode pour recevoir le don de
 l'Esprit-Saint.


\section{Une puissance dynamique à votre disposition}

Dans \ibibleverse{Ac}(1:8), Jésus a promis à ses disciples
 qu'ils recevraient une puissance lorsque l'Esprit-Saint serait
 sur eux, et qu'à travers cette puissance ils seraient témoins
 du Christ jusqu'aux extrémités de la terre.
 Le mot grec traduit par \Og puissance \Fg{} est \emph{dunamis}.
 Le mot français \Og dynamique \Fg{} vient directement de ce mot,
 et il décrit que l'Esprit-Saint est en nous \ocadr la dynamique par laquelle
 nous vivons et servons Dieu. Sans cette dynamique, la vie chrétienne est
 impossible et le service ne porte pas de fruit.
 Quelles nouvelles dimensions glorieuses la puissance de l'Esprit-Saint
 apporte dans la vie du croyant \ocadr la puissance d'être et de faire
 tout ce que Dieu~veut !

Ce n'est pas la volonté de Dieu que votre vie en Christ soit ennuyeuse
 et terne, ou que votre service soit une corvée.
 Dieu veut que votre marche avec Lui soit pleine de joie.
 Il veut que vous ayez la puissance et la victoire dans votre vie.
 Si votre vie en Christ n'est pas dynamique et victorieuse,
 Dieu a quelque chose de plus pour vous.\ibiblephantom{Ac}(2:39)
 La promesse du don de l'Esprit-Saint est \Og pour vous, pour vos enfants,
 et pour tous ceux qui sont au loin,
 en aussi grand nombre que le Seigneur notre Dieu les appellera \Fg{}
 (\bibleverse{Ac}(2:39)).
\closechapter
