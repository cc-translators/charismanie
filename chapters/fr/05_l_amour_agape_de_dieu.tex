\chapter{L'amour agapé de Dieu}

\chlettrine{S}{i une personne} n'accédait pas au paradis,
 elle pourrait accuser beaucoup de gens
 ou de circonstances, mais il y a une personne
 qu'elle ne pourra jamais accuser, et c'est Dieu.
 Quelqu'un pourrait dire~:
 \og C'est la faute de l'Église \ocadr j'ai essayé l'Église mais
 elle n'a vraiment servi à rien pour moi. \fg{}
 Elle pourrait accuser un mauvais exemple de christanisme
 dont elle a été témoin~:
 \og Bon, il a dit qu'il était chrétien, mais j'ai vu la manière
 dont il vivait, et j'ai décidé que je ne voulais rien avoir à faire
 avec ça. \fg{}

Mais une personne que vous ne pourrez jamais accuser, c'est Dieu.
 Quand je pense à tout ce que Dieu a fait pour nous donner le salut,
 je réalise combien il faut se battre pour ne pas être sauvé.
 La Bible dit~:
 \og Si Dieu est pour nous, qui sera contre nous? \fg{}
 Dieu a tant aimé le monde qu'Il a donné Son fils unique,
 puis Dieu a envoyé Son Esprit dans le monde pour nous convaincre
 de nos péchés et nous amener à Jésus-Christ.
 Il nous montre notre propre désespoir, et nous dirige ensuite
 vers Jésus-Christ comme la Solution \ocadr le Chemin, la Vérité et la Vie.

\section*{Commencer le vrai travail}

Une fois que nous avons été amenés au point de venir à Jésus
 et que nous disons~:
 \og Ok Seigneur, prends ma vie \fg{}
 \ocadr au moment où nous nous rendons à Dieu \fcadr
 l'Esprit-Saint commence sérieusement Son travail en nous.
 Au moment où la porte de notre cœur est ouverte pour recevoir le salut,
 l'Esprit-Saint vient dans notre vie et Il commence à effectuer
 les changements nécessaires à l'intérieur, nous conformant
 à l'image de Christ et nous donnant la puissance pour être
 le type de personne que le Seigneur voudrait que nous soyons.
 Il nous donne la connaissance et la compréhension des choses de Dieu,
 afin que la Bible devienne tout à coup un livre complètement
 nouveau pour nous. Lorsque nous commençons à lire,
 elle prend vie, car l'Esprit commence à ouvrir notre compréhension
 et à emplir nos cœurs de l'amour \emph{agapé} de Dieu.
 Mais il faut d'abord que nous venions à Christ
 et que nous soumettions nos vies à Lui.

Jésus a dit dans \bibleverse{Ap}(3:20)~:
 \og Voici~: je me tiens à la porte et je frappe.
 Si quelqu'un entend ma voix et ouvre la porte,
 j'entrerai chez lui, je souperai avec lui et lui avec moi. \fg{}
 Cela veut en fait dire~: \og manger le dîner avec lui. \fg{}
 En Orient, la meilleure méthode de communion avec une personne
 était de manger avec elle. En mangeant avec une personne,
 vous créez une unité, un lien.
 Puisque vous avez partagé la même nourriture, elle devient une partie
 de vous deux, et ainsi vous devenez une partie l'un de l'autre.
 Les peuples d'Orient donnaient une grande signification à la rupture
 du pain ensemble et au fait de boire dans la même coupe,
 car cela créait une affinité, une unité.
 Il est intéressant de noter que Jésus a toujours aimé dîner avec les gens.
 Il appréciait cette unité, cette identité avec les personnes.


Il est significatif que Jésus dise~:
 \og Je me tiens à la porte et je frappe.
 Si vous ouvrez la porte, j'entrerai et je souperai avec vous. \fg{}
 Il viendra dans votre vie et vous pourrez commencer cette relation belle
 et intime avec Lui dans laquelle vous devenez une partie l'un de l'autre.
 Tout cela est fait par le travail de l'Esprit-Saint.
 Au moment où j'ouvre la porte et où je crois en Jésus,
 l'Esprit fait un travail merveilleux pour moi et en moi.
 Dans \bibleverse{Ep}(1:13) et suivants, nous lisons au sujet de ce travail
 de l'Esprit-Saint qui consiste à sceller le croyant.
 Paul y décrit pour nous toutes les bénédictions extraordinaires
 que nous avons en tant qu'enfants de Dieu.
 Il commence au verset 1.3 en disant~:
 \og Béni soit le Dieu et Père de notre Seigneur Jésus-Christ,
 qui nous a bénis de toute bénédiction spirituelle
 dans les lieux célestes en Christ. \fg{}
 Puis, il commence à lister certaines de ces bénédictions glorieuses
 dont Dieu nous a béni.


\section*{Bénis au-delà de toute description}

Le chrétien est en fait la personne la plus bénie au monde.
 Dieu nous a tout simplement béni jusqu'à ce que nous ne puissions plus
 en recevoir davantage.
 Et quand il aura fini de nous bénir ici, Il va encore nous recevoir
 dans sa gloire éternelle, où Il va nous bénir pour toujours.
 Paul parle des bénédictions de Dieu~: Il nous a choisi, prédestiné,
 accepté, racheté, pardonné, nous a fait connaître
 le mystère de Sa volonté, et nous a donné un héritage.
 Dans \bibleverse{Ep}(1:13), Paul dit~:
 \og En lui, vous aussi, après avoir entendu la parole de la vérité,
 l'Évangile de votre salut, en lui, vous avez cru et vous avez été scellés
 du Saint-Esprit qui avait été promis. \fg{}

Tout commence lorsque vous entendez \og la parole de la vérité,
 l'Évangile de votre salut. \fg{} Paul a dit~:
 \og Comment croiront-ils en celui dont ils n'ont pas entendu parler ? \fg{}
 Il est nécessaire pour que vous croyiez que vous entendiez d'abord le message
 que Dieu vous aime d'un amour éternel, et que parcequ'Il vous aime,
 Il a envoyé Son Fils pour prendre votre péché et mourir à votre place,
 afin que ce qui vous éloignait de Dieu soit mis de côté et que rien
 ne vienne entraver votre relation avec Dieu.

Dieu a dit par le prophète Esaïe~:
 \og Non, la main de l'Éternel n'est pas devenue trop courte pour sauver,
 ni son oreille trop dure pour entendre \fg{} (\bibleverse{Is}(59:1-2)).
 C'est toujours le résultat tragique du péché~: la séparation d'avec Dieu.
 Le péché dans ma vie me sépare de Dieu.
 Dieu n'a pas voulu cette séparation, mais il fallait s'occuper du péché,
 donc Dieu a envoyé Son Fils pour prendre mes péchés
 \ocadr pour mourir à ma place \ocadr afin que je n'ai pas
 à être séparé de Lui, et que je puisse revenir
 dans une relation avec Dieu.

Si vous êtes né de nouveau, vous avez entendu
 la bonne nouvelle de Dieu pour vous, et vous avez fait confiance
 après avoir entendu. D'abord, vous avez \emph{entendu},
 puis vous avez \emph{cru} ce que Dieu disait.
 Puis, après avoir cru, vous avez ouvert la porte,
 et vous avez été \og scellé du Saint-Esprit qui avait été promis. \fg{}


\section*{Le sceau de propriété de Dieu}

Le sceau était utilisé dans les temps anciens principalement
 comme une marque de propriété.
 La ville d'Éphèse était un port important où des marchandises
 venant d'Asie étaient importées puis envoyées dans d'autres lieux,
 y compris Rome.
 Les marchands de Rome venaient à Éphèse pour y acheter leurs marchandises.
 Ensuite, ils scellaient ces marchandises en apposant leur bague-cachet
 dans la cire.
 Lorsque le navire arrivait à Pouzzoles
 (le port où arrivaient les marchandises à destination de Rome),
 le marchand allait récupérer ses marchandises,
 en prouvant sa propriété par son sceau personnel.
 Si quelqu'un prétendait posséder les marchandises,
 le marchand pouvait dire~:
 \og Elles sont à moi ; elles portent mon sceau. \fg{}

La vérité biblique dans toute sa beauté est que, une fois que j'ai cru,
 Dieu m'a scellé par Son propre cachet de propriété.
 Il m'a en fait déclaré comme étant Sa possession, afin que l'ennemi
 ne puisse pas prétendre me posséder, car Dieu dirait alors~:
 \og  Enlève tes mains de là, il m'appartient! \fg{}
 Ce sceau de propriété de Dieu est l'Esprit-Saint.
 Lorsque vous croyez en Jésus-Christ, l'Esprit-Saint vient dans votre vie,
 et le fait que l'Esprit-Saint demeure dans votre vie
 constitue le sceau de Dieu, la marque de propriété de Dieu
 par laquelle Il déclare que vous Lui appartenez.


\section*{Chéri par Dieu}

Je ne comprend pas pourquoi Dieu me considère d'une si grande valeur,
 mais c'est le cas.
 Dans \bibleverse{Ep}(1:18), Paul dit~:
 \og \dots{} afin que vous sachiez quelle est l'espérance
 qui s'attache à son appel, quelle est la glorieuse richesse
 de son héritage au milieu des saints. \fg{}
 En d'autres termes, il dit~:
 \og Que Dieu ouvre vos yeux pour que vous réalisiez
 à quel point Il vous chérit ! \fg{}
 Je prie que Dieu ouvre mes yeux par l'Esprit afin que je réalise
 combien Dieu me chérit. C'est une gloire pour moi que Dieu me chérisse.
 Pour Ses propres raisons, Dieu nous chérit,
 et Il a placé Son sceau sur nous.
 L'Esprit-Saint en nous est le sceau de Dieu.
 On nous le dit également dans \bibleverse{IICo}(1:22)~:
 \og Il nous a aussi marqués de son sceau et a mis dans nos cœurs
 les arrhes de l'Esprit. \fg{}
 Dans \bibleverse{Ep}(4:30), nous recevons le commandement suivant~:
 \og N'attristez pas le Saint-Esprit de Dieu, par lequel
 vous avez été scellés pour le jour de la rédemption. \fg{}


\section*{La revendication future}

Dieu a placé sur vous son sceau de propriété dès maintenant
 car Il vous déclare comme Sa propriété, bien que votre rédemption
 ne soit pas encore complète.
 C'est pour cette raison que les marchands apposaient leur sceau de propriété
 sur leurs marchandises, car ils ne les avaient pas encore revendiquées
 dans leur port de destination.
 De la même manière, Dieu a placé son sceau de propriété sur vous,
 bien qu'Il ne vous ait pas encore revendiqué comme Sa possession acquise.
 Notre rédemption n'est pas encore complète,
 mais l'Esprit-Saint est ce sceau et \og ces arrhes \fg{}.

Dans \bibleverse{Ep}(1:14), Paul déclare~:
 \og \dots{} et qui constitue le gage de notre héritage,
 en vue de la rédemption de ceux que Dieu s'est acquis
 pour célébrer sa gloire. \fg{}
 L'Esprit-Saint n'est pas seulement le \emph{sceau} de propriété de Dieu,
 mais Il est également le \emph{gage}, c'est-à-dire la caution, l'acompte.
 Dieu a toutes les intentions de terminer votre rédemption.
 Il a donné une caution ou un acompte~: l'Esprit-Saint.
 Dieu déclare Son intention de compléter Sa transaction pour vous.

Cette rédemption ne sera complète que lorsque nous serons libérés
 de ce corps.
 Ce corps est la chose qui continue à nous tirer vers le bas.
 Paul a dit~:
 \og Nous qui sommes dans ce corps, nous gémissons. \fg{}
 Dans \bibleverse{Rm}(8:22), il décrit comment nous
 \og [soupirons] et [souffrons] les douleurs de l'enfantement. \fg{}
 La création toute entière soupire et souffre
 les douleurs de l'enfantement jusqu'à ce jour.
 \bibleverse{Rm}(8:23) dit~:
 \og Bien plus~: nous aussi, qui avons les prémices de l'Esprit,
 nous aussi nous soupirons en nous-mêmes, en attendant l'adoption
 [litéralement: être placés comme fils],
 la rédemption de notre corps. \fg{}


\section*{La fin du vieux corps}

Robert Service, dans son poème \emph{La crémation de Sam Magee},
 parle de se dépêcher \og avec un corps à moitié caché
 dont il ne pouvait pas se débarasser. \fg{}
 Il était arnaché à un traîneau, mais il ne pouvait pas s'en débarasser
 car il avait fait cette promesse le soir de Noël.
 C'est comme cela que nous sommes en tant que chrétiens.
 Nous avons un esprit rachété qui est vivant pour Dieu,
 mais nous devons traîner notre vieux corps.
 Il est là avec nous où que nous allions, jusqu'au jour où nous pourrons
 finalement nous débarasser de notre charge. Paul a dit~:
 \og Aussi nous gémissons dans cette tente, désireux de revêtir
 notre domicile céleste par-dessus l'autre, si du moins nous sommes
 trouvés vêtus et non pas nus \fg{} (voir \bibleverse{IICo}(5:1-4)).
 Ce sera la complétion de notre rédemption. Voilà ce que j'attend.

Certaines personnes sont troublées
 à l'idée qu'ils vont se débarasser de ce corps. Cela ne me trouble pas.
 L'apôtre Paul a dit~:
 \og  Nous savons, en effet, que si notre demeure terrestre,
 qui n'est qu'une tente, est détruite, nous avons dans les cieux
 un édifice qui est l'ouvrage de Dieu, une demeure éternelle
 qui n'a pas été faite par la main des hommes. \fg{}
 Ce corps que je possède maintenant, je l'ai reçu de mes ancêtres ;
 tous les gênes sont passés de génération en génération,
 et je suis un produit composé de mes ancêtres.
 J'ai récupéré toutes les caractéristiques héritées de l'échec humain,
 et c'est pourquoi je suis ici dans mon corps qui gémit.



