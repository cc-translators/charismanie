\chapter{L'amour agap\'e de Dieu}

\lettrine{S}{i une personne} n'accédait pas au paradis,
 elle pourrait accuser beaucoup de gens
 ou de circonstances, mais il y a une personne
 qu'elle ne pourra jamais accuser~: c'est Dieu.
 Quelqu'un pourrait dire~:
 \og C'est la faute de l'Église~: j'ai essayé l'Église mais
 elle ne m'a vraiment servi à rien. \fg{}
 Elle pourrait accuser un mauvais exemple de christianisme
 dont elle a été témoin~:
 \og Bon, il a dit qu'il était chrétien, mais j'ai vu la manière
 dont il vivait, et j'ai décidé que je ne voulais rien avoir à faire
 avec ça. \fg{}

Mais il y a une personne que vous ne pourrez jamais accuser, c'est Dieu.
 Quand je pense à tout ce que Dieu a fait pour nous donner le salut,
 je réalise combien il faut se battre pour ne pas être sauvé.
 La Bible dit~:
 \og Si Dieu est pour nous, qui sera contre nous? \fg{}
 \ibiblephantom{Rm}(8:31)Dieu a tant aimé le monde qu'Il a donné
 Son fils unique, \ibiblephantom{Jn}(3:16)puis Dieu a envoyé Son Esprit
 dans le monde pour nous convaincre
 de nos péchés et nous amener à Jésus-Christ.
 Il nous montre notre propre désespoir, et nous dirige ensuite
 vers Jésus-Christ comme la Solution \ocadr le Chemin, la Vérité et la Vie.

\section{Commencer le vrai travail}

Une fois que nous avons été amenés au point de venir à Jésus
 et que nous disons~:
 \og Ok Seigneur, prends ma vie \fg{}
 \ocadr au moment où nous nous rendons à Dieu \fcadr
 l'Esprit-Saint commence sérieusement Son œuvre en nous.
 Au moment où la porte de notre cœur est ouverte pour recevoir le salut,
 l'Esprit-Saint vient dans notre vie et Il commence à effectuer
 les changements nécessaires à l'intérieur, nous conformant
 à l'image de Christ et nous donnant la puissance pour être
 le type de personne que le Seigneur voudrait que nous soyons.
 Il nous donne la connaissance et la compréhension des choses de Dieu,
 afin que la Bible devienne tout à coup un livre complètement
 nouveau pour nous. Lorsque nous commençons à lire,
 elle prend vie, car l'Esprit commence à ouvrir notre compréhension
 et à emplir nos cœurs de l'amour \emph{agapé} de Dieu.
 Mais il faut d'abord que nous venions à Christ
 et que nous Lui soumettions nos vies.

Jésus a dit dans \ibibleverse{Ap}(3:20)~:
 \og Voici~: je me tiens à la porte et je frappe.
 Si quelqu'un entend ma voix et ouvre la porte,
 j'entrerai chez lui, je souperai avec lui et lui avec moi. \fg{}
 Cela veut en fait dire~: \og manger le dîner avec lui. \fg{}
 En Orient, la meilleure méthode de communion avec une personne
 était de manger avec elle. En mangeant avec une personne,
 vous créez une unité, un lien.
 Puisque vous avez partagé la même nourriture, elle devient une partie
 de vous deux, et ainsi vous devenez une partie l'un de l'autre.
 Les peuples d'Orient donnaient une grande signification à la rupture
 du pain ensemble et au fait de boire dans la même coupe,
 car cela créait une affinité, une unité.
 Il est intéressant de noter que Jésus a toujours aimé dîner avec les gens.
 Il appréciait cette unité, cette identité avec les personnes.


Il est significatif que Jésus dise~:
 \og Je me tiens à la porte et je frappe.
 Si vous ouvrez la porte, j'entrerai et je souperai avec vous. \fg{}
 Il viendra dans votre vie et vous pourrez commencer cette relation belle
 et intime avec Lui dans laquelle vous devenez une partie l'un de l'autre.
 Tout cela est fait à travers l'œuvre de l'Esprit-Saint.
 Au moment où j'ouvre la porte et où je crois en Jésus,
 l'Esprit effectue une œuvre merveilleuse pour moi et en moi.
 Les versets \ibibleverse{Ep}(1:13) et suivants nous enseignent
 au sujet de cette œuvre de l'Esprit-Saint qui consiste à sceller le croyant.
 Paul y décrit pour nous toutes les bénédictions extraordinaires
 que nous avons en tant qu'enfants de Dieu.
 Il commence au verset \ibiblechvs{Ep}(1:3) en disant~:
 \og Béni soit le Dieu et Père de notre Seigneur Jésus-Christ,
 qui nous a bénis de toute bénédiction spirituelle
 dans les lieux célestes en Christ. \fg{}
 Puis, il commence à lister certaines de ces bénédictions glorieuses
 dont Dieu nous a béni.


\section{B\'enis au-del\`a de toute description}

Le chrétien est en fait la personne la plus bénie au monde.
 Dieu nous a tout simplement béni jusqu'à ce que nous ne puissions plus
 en recevoir davantage.
 Et quand il aura fini de nous bénir ici, Il va encore nous recevoir
 dans Sa gloire éternelle, où Il va nous bénir pour toujours.
 Paul parle des bénédictions de Dieu~: Il nous a choisi, prédestiné,
 accepté, racheté, pardonné, nous a fait connaître
 le mystère de Sa volonté, et nous a donné un héritage.
 Dans \ibibleverse{Ep}(1:13), Paul dit~:
 \og En lui, vous aussi, après avoir entendu la parole de la vérité,
 l'Évangile de votre salut, en lui, vous avez cru et vous avez été scellés
 du Saint-Esprit qui avait été promis. \fg{}

Tout commence lorsque vous entendez \og la parole de la vérité,
 l'Évangile de votre salut. \fg{} Paul a dit~:
 \og Comment croiront-ils en celui dont ils n'ont pas entendu parler ? \fg{}
 Il est nécessaire pour que vous croyiez que vous entendiez d'abord le message
 que Dieu vous aime d'un amour éternel, et que parcequ'Il vous aime,
 Il a envoyé Son Fils pour prendre votre péché et mourir à votre place,
 afin que ce qui vous éloignait de Dieu soit mis de côté et que rien
 ne vienne entraver votre relation avec Dieu.

Dieu a dit par le prophète Ésaïe~:
 \og Non, la main de l'Éternel n'est pas devenue trop courte pour sauver,
 ni son oreille trop dure pour entendre. Mais ce sont vos fautes qui mettaient
 une séparation entre vous et votre Dieu \fg{} (\ibibleverse{Is}(59:1-2)).
 C'est toujours le résultat tragique du péché~: la séparation d'avec Dieu.
 Le péché dans ma vie me sépare de Dieu.
 Dieu n'a pas voulu cette séparation, mais il fallait s'occuper du péché,
 donc Dieu a envoyé Son Fils pour prendre mes péchés
 \ocadr pour mourir à ma place \ocadr afin que je n'ai pas
 à être séparé de Lui, et que je puisse revenir
 dans une relation avec Dieu.

Si vous êtes né de nouveau, vous avez entendu
 la bonne nouvelle de Dieu pour vous, et vous avez fait confiance
 après avoir entendu. D'abord, vous avez \emph{entendu},
 puis vous avez \emph{cru} ce que Dieu disait.
 Puis, après avoir cru, vous avez ouvert la porte,
 et vous avez été \og scellé du Saint-Esprit qui avait été promis. \fg{}


\section{Le sceau de propri\'et\'e de Dieu}

\begin{specialpar}{\tolerance=300}
Le sceau était utilisé dans les temps anciens principalement
 comme une marque de propriété.
 La ville d'Éphèse était un port important où des marchandises
 venant d'Asie étaient importées puis envoyées dans d'autres lieux,
 y compris Rome.
 Les marchands de Rome venaient à Éphèse pour y acheter leurs marchandises.
 Ils scellaient ensuite ces marchandises en apposant leur bague-cachet
 dans la cire.
 Lorsque le navire arrivait à Pouzzoles
 (le port où arrivaient les marchandises à destination de Rome),
 le marchand allait récupérer ses marchandises,
 en prouvant qu'il en était le propriétaire par son sceau personnel.
 Si quelqu'un prétendait posséder les marchandises,
 le marchand pouvait dire~:
 \og Elles sont à moi ; elles portent mon sceau. \fg{}
\end{specialpar}

La vérité biblique dans toute sa beauté est que, une fois que j'ai cru,
 Dieu m'a scellé par Son propre cachet de propriété.
 Il m'a en fait déclaré comme étant Sa possession, afin que l'ennemi
 ne puisse pas prétendre me posséder, car Dieu dirait alors~:
 \og  Enlève tes mains de là, il m'appartient! \fg{}
 Ce sceau de propriété de Dieu est l'Esprit-Saint.
 Lorsque vous croyez en Jésus-Christ, l'Esprit-Saint vient dans votre vie,
 et le fait que l'Esprit-Saint demeure dans votre vie
 constitue le sceau de Dieu, la marque de propriété de Dieu
 par laquelle Il déclare que vous Lui appartenez.


\section{Ch\'eri par Dieu}

Je ne comprend pas pourquoi Dieu me considère d'une si grande valeur,
 mais c'est le cas.
 Dans \ibibleverse{Ep}(1:18), Paul dit~:
 \og [\dots{}] afin que vous sachiez quelle est l'espérance
 qui s'attache à son appel, quelle est la glorieuse richesse
 de son héritage au milieu des saints. \fg{}
 En d'autres termes, il dit~:
 \og Que Dieu ouvre vos yeux pour que vous réalisiez
 à quel point Il vous chérit ! \fg{}
 Je prie que Dieu ouvre mes yeux par l'Esprit afin que je réalise
 combien Dieu me chérit. C'est une gloire pour moi que Dieu me chérisse.
 Pour des raisons qui Lui sont propres, Dieu nous chérit,
 et Il a placé Son sceau sur nous.
 L'Esprit-Saint en nous est le sceau de Dieu.
 On nous le dit également dans \ibibleverse{IICo}(1:22)~:
 \og Il nous a aussi marqués de son sceau et a mis dans nos cœurs
 les arrhes de l'Esprit. \fg{}
 Dans \ibibleverse{Ep}(4:30), nous recevons le commandement suivant~:
 \og N'attristez pas le Saint-Esprit de Dieu, par lequel
 vous avez été scellés pour le jour de la rédemption. \fg{}


\section{La revendication future}

Dieu a placé sur vous Son sceau de propriété dès maintenant
 car Il vous déclare comme Sa propriété, bien que votre rédemption
 ne soit pas encore complète.
 C'est pour cette raison que les marchands apposaient leur sceau de propriété
 sur leurs marchandises, car ils ne les avaient pas encore revendiquées
 dans leur port de destination.
 De la même manière, Dieu a placé Son sceau de propriété sur vous,
 bien qu'Il ne vous ait pas encore revendiqué comme Sa possession acquise.
 Notre rédemption n'est pas encore complète,
 mais l'Esprit-Saint est ce sceau et \og ces arrhes \fg{}.

Dans \ibibleverse{Ep}(1:14), Paul déclare~:
 \og [\dots{}] et qui constitue le gage de notre héritage,
 en vue de la rédemption de ceux que Dieu s'est acquis
 pour célébrer sa gloire. \fg{}
 L'Esprit-Saint n'est pas seulement le \emph{sceau} de propriété de Dieu,
 mais Il est également le \emph{gage}, c'est-à-dire la caution, l'acompte.
 Dieu a toutes les intentions de terminer votre rédemption.
 Il a versé une caution ou un acompte~: l'Esprit-Saint.
 Dieu déclare Son intention de compléter Sa transaction pour vous.

Cette rédemption ne sera complète que lorsque nous serons libérés
 de ce corps.
 Ce corps est la chose qui continue à nous tirer vers le bas.
 Paul a dit~:
 \og Nous qui sommes dans ce corps, nous gémissons. \fg{}
 Dans \ibibleverse{Rm}(8:22), il décrit comment nous
 \og [soupirons] et [souffrons] les douleurs de l'enfantement. \fg{}
 La création toute entière soupire et souffre
 les douleurs de l'enfantement jusqu'à ce jour.
 \ibibleverse{Rm}(8:23) dit~:
 \og Bien plus~: nous aussi, qui avons les prémices de l'Esprit,
 nous aussi nous soupirons en nous-mêmes, en attendant l'adoption
 [littéralement~: d'être placés comme fils],
 la rédemption de notre corps. \fg{}


\section{La fin du vieux corps}

Robert Service, dans son poème \emph{La crémation de Sam Magee},
 parle de se dépêcher \og avec un corps à moitié caché
 dont il ne pouvait pas se débarrasser. \fg{}
 Il était harnaché à un traîneau, mais il ne pouvait pas s'en débarrasser
 car il avait fait cette promesse le soir de Noël.
 C'est comme cela que nous sommes en tant que chrétiens.
 Nous avons un esprit racheté qui est vivant pour Dieu,
 mais nous devons traîner notre vieux corps.
 Il est là avec nous où que nous allions, jusqu'au jour où nous pourrons
 finalement nous débarrasser de notre charge. Paul a dit~:
 \og Aussi nous gémissons dans cette tente, désireux de revêtir
 notre domicile céleste par-dessus l'autre, si du moins nous sommes
 trouvés vêtus et non pas nus \fg{} (voir \ibibleverse{IICo}(5:1-4)).
 Ce sera la complétion de notre rédemption. Voilà ce que j'attends.

Certaines personnes sont troublées
 à l'idée qu'ils vont se débarrasser de ce corps. Cela ne me trouble pas.
 L'apôtre Paul a dit~:
 \og  Nous savons, en effet, que si notre demeure terrestre,
 qui n'est qu'une tente, est détruite, nous avons dans les cieux
 un édifice qui est l'ouvrage de Dieu, une demeure éternelle
 qui n'a pas été faite par la main des hommes. \fg{}
 Ce corps que je possède maintenant, je l'ai reçu de mes ancêtres ;
 tous les gênes sont passés de génération en génération,
 et je suis un produit composé de mes ancêtres.
 J'ai récupéré toutes les caractéristiques héritées de l'échec humain,
 et c'est pourquoi je suis ici dans mon corps qui gémit.


\section{Le nouveau corps c\'eleste}

Le nouveau corps que je vais avoir ne sera pas hérité d'un homme déchu ;
 il me sera donné directement par Dieu.
 Il ne sera pas sujet à la douleur, à la fatigue, à des genoux abîmés
 par le sport ou à tant d'autres choses dont j'ai fait l'expérience
 \index{expérience} dans ce corps. Il viendra directement de Dieu.
 Paul appelle ce corps présent une \og tente \fg{} dans \ibibleverse{IICo}(5:4).
 On ne pense jamais à une tente comme à un endroit permanent pour vivre.
 Si vous devez vivre dans une tente, ça va pour quelques semaines
 dans la montagne pendant les vacances, mais vous n'aimez pas l'idée
 que cela soit votre lieu de résidence permanent.
 Il est bien préférable de sortir de la tente pour habiter dans une vraie maison.
 Dieu prévoit de nous racheter complètement.

La rédemption inclut non seulement le corps \emph{racheté}, mais également
 un \emph{nouveau} corps. Avec mon intelligence, je veux faire la volonté de Dieu.
 Avec mon intelligence, je veux tout donner à Dieu.
 Complètement et entièrement, je veux vivre le style de vie que Dieu
 veut que je vive. Il n'y a aucun problème avec mon cœur ou mon intelligence.
 Mon problème est que mon \emph{corps} continue à me tirer vers le bas.
 Il continue à me ramener vers le bas, et ainsi je ne fais pas toujours
 ces choses que je veux faire. Je suis tiré vers le bas par les appétits
 de mon corps. Je ne peux pas être tout ce que je voudrais être,
 donc je gémis. Toute la création autour de nous gémit,
 dans l'attente du jour de la rédemption, lorsque Dieu clamera ce qu'Il est.
 Il a apposé Son sceau de propriété sur elle, et un jour Il descendra
 et dira : \og Voilà. \fg{}
 Il libérera mon âme et mon esprit de mon corps et les incorporera
 immédiatement dans un nouveau corps céleste.


\section{Le monde souffrant}

Cela est également vrai pour ce monde.
 En ce moment, le monde entier souffre à cause du péché~:
 \og Toute la création gémit dans les douleurs de l'enfantement. \fg{}
 Chaque épine, dit-on, est une floraison à venir.
 Les épines sont là à cause de la malédiction, et une épine n'est qu'une 
 marque de la création gémissant, désirant fleurir, mais incapable
 d'y parvenir. Toute la création souffre sous la malédiction du péché,
 dans l'attente du jour de la délivrance, dans l'attente du jour
 où Dieu rachètera ce qu'Il a acquis.

Jésus a acquis le monde, mais Il ne l'a pas encore réclamé.
 Il Lui appartient, mais Il ne l'a pas encore réclamé.
 Le monde est toujours sous l'influence de Satan.
 \index{Satan}
 Cependant, un jour proche viendra où Il reviendra prendre possession
 de ce qu'Il a acquis.
 Le chapitre~\ibiblechvs{Ap}(5:) de l'Apocalypse mentionne ce sujet.
 Il y a un livre dans la main droite de Celui qui est assis sur le trône.
 Un ange déclare d'une voix forte~:
 \og Qui est digne d'ouvrir le livre et d'en rompre les sceaux ? \fg{}
 Jean répond~: \og Je me suis mis à pleurer,
 car personne ne fut trouvé digne. \fg{}
 Les anciens rétorquent~: \og Ne pleure pas, Jean ;
 le lion de la tribu de Juda a vaincu pour ouvrir
 le livre et ses sept sceaux. \fg{}
 Alors Jean dit~: \og Je me suis tourné et je l'ai vu comme un Agneau debout,
 qui semblait immolé. Il vint recevoir le livre de la main droite
 de celui qui était assis sur le trône. \fg{}
 Ce livre est le titre de propriété de la terre.
 Qui est digne de le prendre ? Qui est digne de le réclamer ?
 Personne, sinon Jésus, car Il l'a acquis sur la Croix, et Il revient
 réclamer la possession qu'Il a acquise.
 Et c'est \emph{moi} qui suis Sa possession acquise!


\section{Mon assurance}

Le travail de l'Esprit-Saint aujourd'hui dans ma vie est de m'avoir scellé.
 Sa présence en moi me donne une réelle assurance.
 Lorsque Satan \index{Satan} vient et commence à me harceler à cause
 de la faiblesse de ma chair et de mes échecs,
 et qu'il commence à me dire que Dieu
 ne s'intéresse pas à moi et que Dieu ne m'aime pas et qu'Il ne va pas
 me sauver, je dis~:
 \og Satan, tu te \emph{trompes} ! J'ai le sceau de Dieu ; Il m'a marqué ;
 Il a apposé sur moi Son sceau de propriété. L'Esprit-Saint demeure en moi.
 Dieu m'a scellé ! Il m'a versé les arrhes, et Il vient récupérer
 ce qu'Il a racheté. \fg{}

Quand nous arriverons au ciel (soit quand Jésus viendra nous chercher,
 ou quand nous mourrons), notre rédemption sera complète.
 L'œu\-vre de Christ sera achevée en nous, et nous partagerons pour
 toujours le paradis glorieux de Dieu sans aucune des limitations
 de ce corps. Nous serons capables d'aimer, de partager, de donner
 et d'être en relation les uns avec les autres sans aucune restriction
 ou limitation.

Quel jour glorieux! Quel œuvre glorieuse de Dieu que de nous sceller
 et de nous donner le trésor de l'Esprit jusqu'au jour de la rédemption
 de la possession acquise!


\section{L'amour agap\'e de Christ}

Une autre œuvre de l'Esprit-Saint dans la vie du croyant
 est de nous donner l'amour agapé de Jésus-Christ.
 Jésus a dit à ses disciples~:
 \og À ceci tous connaîtront que vous êtes mes disciples,
 si vous avez de l'\emph{amour} les uns pour les autres. \fg{}
 Le mot traduit par \og amour \fg{} est le mot grec \emph{agapé},
 qui est rarement trouvé en grec en dehors de la Bible.
 C'est un mot qui était utilisé par notre Seigneur Jésus-Christ
 pour définir une qualité d'amour supérieure à l'expérience habituelle
 \index{expérience}
 de l'amour.
 La langue anglaise est limitée par certains côtés,
 et peut-être encore plus limitée dans sa capacité à exprimer l'amour.
 Les français disent~:
 \og Vous les anglais, vous n'avez qu'une seule façon de dire à une femme
 que vous l'aimez. Nous en avons cent. \fg{}
 Ils expriment à quel point la langue française est plus libre et plus riche
 que la langue anglaise dans ce domaine.
 Dans la langue grecque, il y a plusieurs mots pour l'amour,
 mais nous sommes limités au seul mot \emph{aimer} en français.
 J'\emph{aime} les cacahuètes. J'\emph{aime} les M\&Ms,
 et j'\emph{aime} ma femme.
 Je dois utiliser le même mot pour décrire mes sentiments
 vis-à-vis de glaces nappées de chocolat fondu et ceux que j'ai
 pour mes enfants. Et pourtant, ce que je ressens pour les glaces nappées
 au chocolat fondu est entièrement différent de ce que je ressens
 pour mes enfants ou ma femme.
 Je suis coincé avec ce seul mot \og aimer \fg{}
 \NdT{Malgré l'argumentaire de l'auteur concernant la richesse relative
 de la langue française en matière d'amour,
 nous sommes tout aussi limités que les anglophones
 en comparaison de la richesse qu'offre le grec.}.


\section{Les trois mots de l'amour}

Dans la langue grecque, il y a un mot pour l'amour sur le plan physique,
 le mot \emph{éros}.
 Il est facile de voir les mots français qui dérivent de ce mot,
 tels que \og érotique \fg{}.
 Il s'agit de l'amour sur le plan purement physique.
 Ce mot est devenu très en vogue chez les jeunes d'aujourd'hui,
 et également chez les plus âgés je suppose. Ils disent~:
 \og Faisons l'amour \fg{}, et par cela ils se réfèrent
 à une expérience \index{expérience} d'éros, qui n'implique par forcément
 qu'il y ait un amour véritable.

Les Grecs ont un deuxième mot pour l'amour,
 un amour sur un plan plus élevé que le physique, le plan intellectuel,
 une relation émotionnelle. Ce mot est \emph{philéo}.
 Il est beaucoup plus profond que éros, car il implique une interaction
 plus profonde avec l'autre personne.
 Philéo se construit en discutant et en découvrant que l'on aime
 les mêmes choses. Nous avons beaucoup de choses en commun,
 nous nous apprécions mutuellement, et à travers un échange mutuel~nous faisons
 l'expérience \index{expérience} d'un amour philéo.

L'amour \emph{agapé} est l'amour total.
 C'est l'amour sur le plan le plus profond, c'est le vrai amour spirituel.
 Éros n'est pas le véritable amour. Si je dis~: \og Je t'aime \fg{}
 dans le domaine de l'éros, ce que je dis vraiment est~:
 \og Je \emph{m'}aime, et je te veux car je suis amoureux de moi-même,
 et j'ai besoin de toi. \fg{}
 Si quelqu'un dit~: \og Je ne peux pas vivre sans toi \fg{},
 ça n'est pas une expression d'amour profond envers vous.
 Cela montre seulement qu'il pense à lui-même.
 Éros est extrêmement égoïste. C'est de l'amour propre.

L'amour philéo est réciproque~: \og Je t'aime car tu m'aimes ;
 je t'aime car tu ris de mes blagues ;
 je j'aime car nous aimons tant de choses en commun ;
 je t'aime car nous allons bien ensemble et nous nous amusons beaucoup
 lorsque nous sommes ensembles. Je t'aime car tu m'aimes,
 et tu es une personne agréable, et nous passons de bons moments ensemble. \fg{}


\section{Le plus grand amour}

L'amour agapé continue d'aimer même si il n'y a pas d'amour en retour.
 C'est un amour profond qui donne et ne demande rien~en~retour.
 Cet amour est si profond et si grand qu'il se contente
 de continuer à donner.
 En fait, c'est la préoccupation principale de l'agapé~: donner.
 Le mot \emph{agapé} est un mot tellement vaste et large,
 qu'il est même difficile pour nous de le définir dans la langue française.
 Il nous est impossible de le comprendre en dehors de l'Esprit de Dieu
 et de Sa révélation à nos cœurs, car ce n'est pas un amour naturel ;
 c'est un amour surnaturel.
 La Bible dit~: \og Dieu est \emph{agapé}. \fg{}
 C'est un amour divin, surnaturel, et sa meilleure définition pour nous
 est probablement dans \ibibleverse{ICo}(13:).

Tout d'abord, Paul met en avant la suprématie de cet amour agapé.
 Il est plus important que vous ayez ce type d'amour
 plutôt que des dons spirituels. Paul a dit~:
 \og Quand je parlerais les langues des hom\-mes et des anges,
 si je n'ai pas l'amour (agapé), je suis du bronze qui résonne
 ou une cymbale qui retentit. \fg{}
 Vous pouvez avoir une grande puissance d'oraison ;
 vous pouvez avoir une langue d'argent ;
 il est possible que vous puissiez vous exprimer extrêmement bien.
 Cependant, si vous n'avez pas l'amour agapé, cela ne vaut pas mieux
 que de frapper une cymbale. C'est un son sans signification.

L'agapé est plus important que les dons de prophétie,
 la parole de connaissance ou le don de la foi,
 car \og quand j'aurais (le don) de prophétie,
 la science de tous les mystères et toute la connaissance,
 quand j'aurais même toute la foi jusqu'à transporter des montagnes,
 si je n'ai pas l'agapé, je ne suis rien. \fg{} Paul continue~:
 \og Et quand je distribuerais tous mes biens
 pour la nourriture (des pauvres)\dots{} \fg{},
 l'amour agapé est encore plus important que le sacrifice.
 Vous pourriez vendre tout ce que vous avez, nourrir les pauvres
 et donner votre corps pour qu'il soit brûlé, accomplissant le sacrifice
 suprême, mais si vous n'avez pas l'amour agapé,
 cela ne vous sert à rien.
 \nowidow[6]
 

Paul continue en définissant ce type d'amour.
 \og L'amour est patient, l'amour est serviable. \fg{}
 Cela signifie que l'amour agapé supporte les abus et souffre sans broncher.
 Il supporte continuellement et est toujours serviable à la fin.
 Vous avez entendu des gens dire~:
 \og Bon, j'ai supporté assez longtemps, maintenant ça suffit!
 Maintenant je vais prendre ma revanche. \fg{}
 Ce n'est pas de l'agapé. L'agapé supporte encore et encore
 et reste toujours serviable. Il ne cherche pas la vengeance.
 \og L'amour n'est pas envieux ; l'amour ne se vante pas,
 il ne s'enfle pas d'orgueil, il ne fait rien de malhonnête,
 il ne cherche pas son intérêt, il ne s'irrite pas.
 L'amour ne médite pas le mal, il ne se réjouit pas de l'injustice,
 mais il se réjouit de la vérité ; il pardonne tout, il croit tout,
 il espère tout, il supporte tout. \fg{}
 L'amour agapé \og n'échoue jamais. \fg{}


\section{Les deux signes de l'amour}

Jésus a dit~:
 \og À ceci tous connaîtront que vous êtes mes disciples,
 si vous avez de l'amour [agapé] les uns pour les autres. \fg{}
 Cela devient vraiment l'évidence la plus manifeste pour le monde
 que nous sommes les disciples de notre Seigneur Jésus-Christ.
 Cet amour agapé devrait être à l'œuvre dans nos vies et nous rendre un
 les uns avec les autres, nous donnant la priorité les uns aux autres,
 plutôt que de nous exalter nous-mêmes ou de former des cliques,
 mais simplement de partager l'unité de l'amour qui nous rend
 tous un ensemble.
 Nous devrions partager les uns avec les autres la bonté et la grâce de Dieu,
 donnant gratuitement comme nous avons gratuitement reçu l'amour
 et la grâce de Dieu.
 Au fur et à mesure que cet amour agapé travaille dans nos vies,
 il devient le signe pour le monde que nous sommes vraiment
 des disciples du Christ.

Dans \ibibleverse{IJn}(3:14), on peut lire~:
 \og Nous savons que nous sommes passés de la mort à la vie,
 parce que nous \emph{aimons} les frères. \fg{}
 À nouveau, le mot \emph{agapé} est utilisé.
 C'est non seulement un signe pour le \emph{monde}
 que nous sommes des disciples du Christ, mais c'est un signe pour nous
 que nous sommes passés de la mort à la vie.
 Lorsque l'amour de Dieu commence à \oe{}uvrer dans ma vie,
 il devient un signe pour \emph{moi} que je suis passé de la mort à la vie,
 car j'ai cet amour pour les frères, pour ceux qui sont
 dans le corps du Christ.


\section{La source du v\'eritable amour}

Puisque c'est un amour divin, sa source est en Dieu.
 Ce n'est pas quelque chose que je peux générer.
 Ce n'est pas quelque chose que je peux faire en moi-même.
 C'est une des difficultés que la communauté chrétienne a rencontrées~:
 savoir que nous devons aimer tous les croyants, mais également savoir
 qu'il y en a aussi certains que nous n'aimons pas vraiment.
 C'est pourquoi nous essayons de fabriquer un amour artificiel.
 Nous essayons de nous convaincre de son existence.
 Mais l'amour agapé n'a pas son origine en moi ;
 l'origine de l'amour agapé est en Dieu. Dieu est agapé.
 Je ne peux pas le forger ; c'est quelque chose qui doit venir à moi
 comme une œuvre de Dieu dans ma vie.
 Si je me rend compte que je manque de cet amour, je ne peux pas vraiment
 y remédier par moi-même ; je dois juste confesser ce manque à Dieu
 et Lui demander de planter cet agapé en moi.

Beaucoup de chrétiens ont été très frustrés car ils ont essayé
 de produire cet agapé.
 Ils ont tenté avec beaucoup d'ardeur d'aimer de cet amour divin,
 mais ça ne leur est pas possible. Son origine est en Dieu,
 et il doit venir de Dieu comme un don qui vous est fait,
 et ensuite il émane de votre vie. Si vous trouvez que vous manquez
 de cet agapé, la seule chose que vous puissiez faire est de demander à Dieu
 de remplir votre cœur d'agapé par l'Esprit-Saint.
 Ne vous laissez pas abattre et décourager
 dans votre démarche spirituelle
 sous prétexte que vous vous rendez compte que vous n'avez pas
 cet agapé alors que vous le devriez ;
 demandez-le plutôt au Seigneur.
\closechapter

