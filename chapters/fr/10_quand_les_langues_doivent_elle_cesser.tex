\chapter{Quand les langues doivent-elles cesser ?}

\lettrine{L}{es doctrines religieuses} qui interdisent le parler en langues
 se réfèrent en général à \ibibleverse{ICo}(13:8), où il est dit\frcolon{}
 \Og les langues, elles cesseront \Fg{} comme base de leur interdiction.
 Le moment où les langues cesseront, cependant, dépend de la manière
 dont la phrase\frcolon{} \Og ce qui est parfait sera venu \Fg{} est interprétée.
 Ceux qui utilisent ce verset pour interdire les langues interprètent
 \Og ce~qui est parfait \Fg{} comme une référence à la révélation complète
 du Canon des Écritures, qui se conclut par l'Apocalypse,
 la révélation de Jésus-Christ faite à Jean. Leur argument part en général du principe que,
 jusqu'à ce que le Canon des Écritures soit complet, ces dons étaient
 utilisés pour instruire l'Église primitive. Mais une fois que les Écritures
 ont été achevées, on n'avait plus besoin de dépendre de ces dons.
 C'est pourquoi les langues ont cessé lorsque les Écritures
 ont été achevées.


\section{R\'eponses \`a cet argument}

Cet argument paraît plausible de prime abord ; cependant, il n'est rien
 de plus qu'une spéculation hypothétique, et il n'est pas seulement dénué
 d'une base scripturaire, mais il semblerait contredire tous les usages
 scripturaires du don dans le Nouveau Testament.
 Pas une fois nous ne trouvons d'exemple du parler en langues utilisé
 pour instruire les croyants dans le Nouveau Testament. Au contraire,
 nous lisons dans \ibibleverse{ICo}(14:2) que ceux qui parlaient en langues
 ne parlaient pas aux hommes mais à Dieu. Il ne devaient pas non plus parler
 en langues dans l'église à moins qu'une personne possédant le don
 d'interprétation ne soit présente, afin que toutes les personnes présentes
 puissent dire \Og Oui \Fg{} et \Og Amen \Fg{} aux bénédictions
 et aux actions de grâces offertes à Dieu.

L'usage du don des langues dans les Écritures n'a jamais été associé
 à l'enseignement de la vérité de Dieu à l'Église.
 Il ne pourrait donc pas y avoir de relation entre la cessation
 du don des langues et l'achèvement du Canon des Écritures.
 Une des règles d'or de l'interprétation des Écritures est d'examiner
 le texte à la lumière~de son contexte. Pour connaître le texte,
 lisez le contexte. Le contexte de \bibleverse{ICo}(13:) est la suprématie
 de l'amour. Il est suprême sur la pratique du don des langues,
 qui sont nulles sans l'amour (versets~\ibiblevs{ICo}(13:1-3)).
 L'amour est ensuite défini aux versets~\ibiblevs{ICo}(13:4-7),
 puis le fait que l'amour ne succombe jamais est déclaré aux
 versets~\ibiblevs{ICo}(13:8-12), montrant qu'il va durer plus longtemps
 que les langues, les prophéties et la connaissance.
 Enfin, au verset~\ibiblevs{ICo}(13:13), la trilogie perpétuelle de la foi,
 de l'espérance et de l'amour est présentée, précisant que l'amour
 est suprême. Le contexte immédiat est la nature indéfectible de l'amour
 en contraste avec les prophéties (qui seront abolies), les langues
 (qui vont cesser) et la connaissance (qui sera abolie).
 Les prophéties et la connaissance sont partielles, mais quand ce qui
 est parfait sera venu, nous n'aurons plus de vision confuse,
 mais nous verrons face à face. Notre connaissance ne sera plus
 une connaissance partielle, mais complète, parce que nous connaîtrons
 alors comme nous sommes connus.

L'idée que le mot grec \emph{téléios}, traduit par le mot \Og parfait \Fg{},
 se référait au Canon des Écritures complètement achevé n'est pas venu
 à l'idée de quelques uns des plus grands érudits de grec du siècle dernier.
 Il s'agit plutôt d'une invention ou d'une création récente pour contrer
 le mouvement moderne des langues. Thayer, dans son
 \emph{Lexique grec-anglais},
 dit de \emph{téléios} tel qu'utilisé dans \ibibleverse{ICo}(13:10)\frcolon{}
 \Og L'état parfait des choses qui accompagneront le retour du Christ
 des cieux. \Fg{} Alford, dans son \emph{Nouveau Testament pour le lecteur
 anglophone}, en dit\frcolon{} \Og Au moment de la venue du Seigneur et par la suite. \Fg{}
 Quand la seule base scripturaire pour le rejet de la validité des langues
 repose sur une interprétation si douteuse et fragile du mot grec
 \emph{téléios}, arraché du contexte où il est utilisé, il faut vraiment
 remettre en question l'honnêteté d'analyse de cette érudition.
 Pour être bienveillant, je dirais que dans le meilleur des cas il s'agit
 d'un aveuglement préjudiciable \ocadr pas du tout académique ou concluant.

Il faut aussi noter qu'il était dit dans \ibibleverse{ICo}(13:8)
 qu'en même temps que la cessation des langues,
 les prophéties et la connaissance seraient elles aussi abolies.
 Y a-t-il quelqu'un qui soit prêt à admettre
 que Dieu ne parle plus à l'Église pour l'édifier, l'exhorter ou la consoler ?
 La connaissance a-t-elle disparu ? Les Écritures déclarent que nous
 connaissons partiellement. Certains prétendent avoir une connaissance
 parfaite, mais je doute sérieusement de leurs prétentions.
 Nous ne connaîtrons pas tels que nous sommes connus jusqu'à ce que Christ
 revienne.


\section{L'Esprit dans l'histoire de l'\'Eglise}

Puisqu'il n'y a aucune base scripturaire solide pour nier la validité
 du parler en langues aujourd'hui, quelles autres bases avons-nous
 pour remettre en question l'exercice de ce don ?
 On peut toujours mentionner
 l'absence présumée de son usage dans l'histoire de l'Église.
 Ceci n'est pas vrai, cependant, car au cours de l'histoire de~l'Église,
 la question semble avoir surgi de temps à autre.
 Il y a des témoignages de parler en langues au sein de groupes radicaux
 au cours de l'histoire de l'Église. Son absence de pratique apparente
 pendant la majorité de l'histoire de l'Église n'est pas un témoignage
 fort contre sa validité.

Je ne suis personnellement pas fier de l'histoire de l'Église traditionnelle.
 Il me semble que c'est une histoire pleine d'échecs. L'Église du Nouveau Testament
 a prospéré pendant la période apostolique ; Paul a pu annoncer aux
 Colossiens que la vérité de l'Évangile était parvenue au monde entier
 et qu'elle portait des fruits (\ibibleverse{Col}(1:6)).
 Sous la direction et le pouvoir de l'Esprit, ils ont pu apporter l'Évangile
 au monde entier au cours du premier siècle. C'est une prouesse
 que l'Église traditionnelle n'a pas été capable d'égaler au cours
 des siècles qui ont suivi. Il est dramatique que beaucoup de gens
 cherchent à reléguer le pouvoir particulier de l'Esprit-Saint à la seule période
 apostolique, et lui ont maintenant substitué le génie et les programmes
 humains pour accomplir la grande mission confiée par Christ.
 Le résultat a été l'échec lamentable de l'Église.
 Il faut sérieusement se demander si c'était le plan de Dieu ou l'orgueil
 de l'homme de mettre de côté la dépendance à la direction et au pouvoir
 de l'Esprit-Saint pour atteindre le monde perdu pour Jésus-Christ.

Paul a dit aux Galates\frcolon{} \Og Êtes-vous tellement insensés ?
 Après avoir commencé par l'Esprit, allez-vous maintenant finir
 par la chair ? \Fg{} (\ibibleverse{Ga}(3:3)).
 C'est précisément ce qui est déclaré par ceux qui relèguent
 les opérations des dons de l'Esprit à la seule période apostolique.
 L'Église, disent-ils, a été commencée dans l'Esprit pour l'aider
 à surmonter tous les obstacles du monde païen et hostile.
 Mais une fois les séminaires et les structures organisationnelles
 mises en place, elle n'a plus eu besoin de la puissance de l'Esprit.
 L'Église pouvait alors être rendue parfaite par les hommes instruits.
 Un regard honnête sur l'histoire de l'Église devrait dissiper
 ce faux raisonnement une fois pour toutes.


\section{La promesse de Jo\"el}

Lorsque nous considérons la promesse de l'Esprit dans \ibibleverse{Jl}(2:28),
 et que nous lisons le contexte de cette promesse dans son intégralité,
 nous voyons qu'il se référait aux derniers jours.
 La prophétie nous amène en fait en plein dans la période de la tribulation,
 lorsque le soleil devient ténèbre et que la lune se change en sang,
 et au grand jour de la venue du Seigneur, quand ce qui est parfait sera venu.
 Ce qui a commencé à la Pentecôte devait évidemment continuer jusqu'au retour
 de Jésus-Christ. Pierre l'a confirmé quand il a parlé à la multitude curieuse
 qui lui demandait au jour de la Pentecôte\frcolon{} \Og Que ferons-nous ? \Fg{}
 Il a ordonné\frcolon{} \Og Repentez-vous, et que chacun de vous soit baptisé
 au nom de Jésus-Christ, pour le pardon de vos péchés ;
 et vous recevrez le don du Saint-Esprit. Car la promesse
 [la promesse de Joël] est pour vous, pour vos enfants,
 et pour tous ceux qui sont au loin, en aussi grand nombre
 que le Seigneur notre Dieu les appellera \Fg{} (\ibibleverse{Ac}(2:38-39)).
 Il n'est pas fait mention d'une date limite à la fin de la période
 apostolique. Cette idée est une invention des hommes pour excuser
 le manque de puissance dans leurs églises et dans leurs vies aujourd'hui.

Nous ne préconisons certes pas que tous parlent en langues.
 Paul, par sa question réthorique\frcolon{} \Og Tous parlent-ils en langues ? \Fg{}
 (\ibibleverse{ICo}(12:30)), attendait un \Og Non \Fg{} en réponse,
 de la même manière que tous n'ont pas les dons de guérison.
 D'un autre côté, je sens qu'il est mauvais d'interdire,
 ou même de décourager l'usage du parler en langues par ceux
 qui veulent utiliser ce don pour les assister dans leur vie
 de prière ou leurs temps personnel avec Dieu.
\closechapter

