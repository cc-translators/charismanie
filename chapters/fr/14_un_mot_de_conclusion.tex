\chapter{Un mot de conclusion}

\lettrine{U}{n des tristes résultats} de la \emph{charismanie}
 est la réaction de rejet
 qu'elle suscite chez tant de fervents saints de Dieu assoiffés
 qui ont besoin d'une expérience \index{expérience}
 plus profonde de la puissance de Dieu
 dans leurs vies et la recherchent.

Nous lisons dans \ibibleverse{IIR}(4:) qu'Elisée, pendant la famine,
 a cherché à nourrir les fils des prophètes en disposant devant eux
 une grande marmite dans laquelle ils avaient placé les herbes
 aromatiques qu'ils avaient trouvées.
 L'un des jeunes prophètes avait coupé en morceaux des coloquintes
 sauvages et les avait ajoutées dans le potage et quand ils sont venus
 satisfaire leur faim en mangeant, ils se sont écriés~:
 \Og La mort est dans la marmite ! \Fg{}

C'est ce qui se passe souvent quand une personne,
 assoiffée de la plénitude de Dieu, fait l'expérience \index{expérience}
 de l'embrasement
 vireux des \emph{charismaniaques}. Elle conclue malheureusement
 que l'œuvre véritable de l'Esprit de Dieu n'existe pas
 dans l'Église d'aujourd'hui, et elle continue à lutter
 dans sa démarche chrétienne sans l'aide et la pleine dynamique
 de l'Esprit-Saint.

Le plan de Dieu n'était pas que les enfants d'Israël meurent
 dans le désert, mais plutôt qu'ils arrivent dans l'abondance débordante
 de la Terre Promise. Le plan de Dieu n'est pas que votre démarche
 avec Lui soit une expérience \index{expérience} de désert sec et aride,
 mais Il désire
 que vous découvriez les pleines richesses de cette vie
 qui nous a été promise dans l'Esprit-Saint.

Ne laissez pas les excès non scripturaires de ceux qui pratiquent 
la \emph{charismanie} vous décourager de rechercher tout ce dont Dieu
 veut que vous fassiez l'expérience \index{expérience} dans l'amour,
 la joie et la puissance de la vie vécue dans la plénitude de l'Esprit.
 Nous n'avons pas encore fait l'expérience \index{expérience}
 de la gamme complète
 et riche de l'œuvre véritable de l'Esprit dans nos vies ;
 c'est pourquoi nous devons toujours rester ouverts
 à tout ce que Dieu désire nous accorder.
 Il y a tant de choses dans les Écritures dont nous n'avons pas encore
 fait l'expérience, \index{expérience}
 que nous n'avons certainement pas besoin
 d'aller au-delà des Écritures.

Paul a exprimé sa joie que l'église de Corinthe n'ait manqué
 d'aucun don spirituel dans leur attente de la venue de Jésus.
 Ma prière est donc que nous fassions
 nous aussi l'expérience \index{expérience}
 de la plénitude de l'Esprit-Saint et des dons qu'Il désire
 nous accorder tandis que nous attendons le retour du Seigneur.
\closechapter


