\chapter{Le parler en langues}

\begin{specialpar}{\tolerance=240}
\lettrine{U}{n des domaines} où la controverse est la plus vive au sein
 du corps du Christ aujourd'hui implique le fait de parler dans d'autres
 langues, ou \og glossolalie \fg{}.
 À un extrême, il y a des personnes qui considèrent tout exercice
 des langues comme satanique.
 À l'autre extrême, il y a des personnes qui affirment que vous n'êtes pas
 remplis ou baptisés de l'Esprit-Saint à moins que vous ne parliez
 dans d'autres langues. Ils affirment que le parler en langues
 est l'évidence initiale du baptème de l'Esprit-Saint.
 Dans \ibibleverse{ICo}(13:1), Paul déclare que les langues en elles-mêmes
 ne sont pas une évidence valide de la présence de l'Esprit-Saint dans la vie
 du croyant, car \og quand je parlerais les langues des hommes et des anges,
 si je n'ai pas l'amour (agapé), je suis du bronze qui résonne ou une cymbale
 qui retentit. \fg{}
 En d'autres termes, les langues ne sont que des sons sans signification
 et n'ont aucune validité si elles ne sont pas accompagnées de l'amour agapé.
\end{specialpar}


\closechapter
