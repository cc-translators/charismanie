\chapter{Le parler en langues}

\begin{specialpar}{\tolerance=240}
\lettrine{U}{n des domaines} où la controverse est la plus vive au sein
 du corps du Christ aujourd'hui implique le fait de parler dans d'autres
 langues, ou \og glossolalie \fg{}.
 À un extrême, il y a des personnes qui considèrent tout exercice
 des langues comme satanique.
 À l'autre extrême, il y a des personnes qui affirment que vous n'êtes pas
 remplis ou baptisés de l'Esprit-Saint à moins que vous ne parliez
 dans d'autres langues. Ils affirment que le parler en langues
 est l'évidence initiale du baptème de l'Esprit-Saint.
 Dans \ibibleverse{ICo}(13:1), Paul déclare que les langues en elles-mêmes
 ne sont pas une évidence valide de la présence de l'Esprit-Saint dans la vie
 du croyant, car \og quand je parlerais les langues des hommes et des anges,
 si je n'ai pas l'amour (agapé), je suis du bronze qui résonne ou une cymbale
 qui retentit. \fg{}
 En d'autres termes, les langues ne sont que des sons sans signification
 et n'ont aucune validité si elles ne sont pas accompagnées de l'amour agapé.
\end{specialpar}


\section*{Les langues ou non ?}

Dans le livre des Actes, le parler en langues est souvent accompagné
 de la relation \emph{épi} avec l'Esprit-Saint.
 C'est le cas dans Actes chapitres \ibiblechvs{Ac}(2:), \ibiblechvs{Ac}(10:)
 et \ibiblechvs{Ac}(19:). Cependant, dans le huitième chapitre des Actes, \ibiblephantom{Ac}(8:)quand les croyants Samaritains reçoivent
 l'Esprit-Saint, il n'est pas dit qu'ils ont parlé en langues.
 Pourtant, il est évident qu'il a dû y avoir un phénomène qui a accompagné
 le moment où ils ont reçu l'Esprit-Saint, car Simon le magicien cherchait
 à acheter la puissance que Pierre et Jean possédaient ;
 il désirait pouvoir également imposer les mains sur les gens
 afin qu'ils reçoivent l'Esprit-Saint. Il est évident qu'un phénomène
 accompagnait le moment où ils ont reçu l'Esprit car Simon voulait en acheter
 la puissance afin de pouvoir en dupliquer le prodige.
 Plus tard, dans \ibibleverse{Ac}(9:17), quand Ananias a imposé les mains
 sur Saul (Paul) pour qu'il reçoive le don de l'Esprit-Saint,
 il n'y a aucune mention de Paul parlant en langues.
 Cependant, nous savons que plus tard, alors que Paul écrivait aux
 Corinthiens, il remerciait Dieu qu'il parlait en langues plus que tous
 les Corinthiens.\ibiblephantom{ICo}(14:18)Le moment où Paul a fait pour la première fois
 l'expérience du don des langues n'est pas indiqué.

Nous devons attirer l'attention sur le fait qu'une personne qui parle
 en langues et n'a pas d'amour agapé fait preuve d'un témoignage
 moins valide de la présence ou de l'emplissage de l'Esprit dans sa vie
 qu'une personne qui n'a jamais parlé en langues mais fait preuve d'amour
 et d'autres qualités dynamiques de l'Esprit de Dieu.
 Je ne peux pas nier la validité de vies remplies de l'Esprit de beaucoup
 de responsables dynamiques et de laïcs dans l'Église aujourd'hui
 qui n'ont jamais goûté à l'expérience du parler en langues,
 et je préfère leur compagnie à celle de beaucoup de ceux qui promeuvent
 le parler en langues comme la seule vraie preuve d'une vie remplie
 de l'Esprit, mais dont les vies personnelles sont entâchées de discorde
 ou d'orgueil, et souvent même d'hérésie.

Quand Paul a écrit aux Galates, il a déclaré~:
 \og Le fruit de l'Esprit est l'amour. \fg{}\ibiblephantom{Ga}(5:22)La vraie preuve que l'Esprit de Dieu remplit la vie d'une personne
 est l'amour. L'amour est l'évidence la plus valide qu'un homme
 est vraiment rempli de l'Esprit, et les langues sans l'amour
 ne sont que des sons sans signification.


\closechapter
