\chapter{Le parler en langues}

\lettrine{U}{n des domaines} où la controverse est la plus vive au sein
 du corps du Christ aujourd'hui implique le fait de parler dans d'autres
 langues, ou \og glossolalie \fg{}.
 À un extrême, il y a des personnes qui considèrent tout exercice
 des langues comme satanique.
 À l'autre extrême, il y a des personnes qui affirment que vous n'êtes pas
 remplis ou baptisés de l'Esprit-Saint à moins~que~vous ne parliez
 dans d'autres langues. Ils affirment que le parler en langues
 est l'évidence initiale du baptème de l'Esprit-Saint.
 Dans \ibibleverse{ICo}(13:1), Paul déclare que les langues en elles-mêmes
 ne sont pas une évidence valide de la présence de l'Esprit-Saint dans la vie
 du croyant, car \og quand je parlerais les langues des hommes et des anges,
 si je n'ai pas l'amour (agapé), je suis du bronze qui résonne ou une cymbale
 qui retentit. \fg{}
 En d'autres termes, les langues ne sont que des sons sans signification
 et n'ont aucune validité si elles ne sont pas accompagnées de l'amour agapé.


\section{Les langues ou non ?}

Dans le livre des Actes, le parler en langues est souvent accompagné
 de la relation \emph{épi} avec l'Esprit-Saint.
 C'est le cas dans Actes chapitres \ibiblechvs{Ac}(2:), \ibiblechvs{Ac}(10:)
 et \ibiblechvs{Ac}(19:). Cependant, dans le huitième chapitre des Actes,~\ibiblephantom{Ac}(8:)quand
 les croyants Samaritains reçoivent
 l'Esprit-Saint, il n'est pas dit qu'ils ont parlé en langues.
 Pourtant, il est évident qu'il a dû y avoir un phénomène accompagnant 
 le moment où ils ont reçu l'Esprit-Saint, car Simon le magicien cherchait
 à acheter la puissance que Pierre et Jean possédaient ;
 il désirait pouvoir également imposer les mains sur les gens
 afin qu'ils reçoivent l'Esprit-Saint. Il est évident qu'un phénomène
 accompagnait le moment où ils ont reçu l'Esprit car Simon voulait en acheter
 la puissance afin de pouvoir en dupliquer le prodige.
 Plus tard, dans \ibibleverse{Ac}(9:17), quand Ananias a imposé les mains
 sur Saul (Paul) pour qu'il reçoive le don de l'Esprit-Saint,
 il n'y a aucune mention de Paul parlant en langues.
 Cependant, nous savons que plus tard, lorsque Paul écrivait aux
 Corinthiens, il remerciait Dieu qu'il parlait en langues plus que tous
 les Corinthiens.\ibiblephantom{ICo}(14:18)
 Le moment où Paul a fait pour la première fois
 l'expérience \index{expérience} du don
 des langues \index{langues!parler en \textasciitilde{}} n'est pas indiqué.

Nous devons attirer l'attention sur le fait qu'une personne qui parle
 en langues et n'a pas d'amour agapé fait preuve d'un témoignage
 moins valide de la présence ou du remplissage de l'Esprit dans sa vie
 qu'une personne qui n'a jamais parlé en langues mais fait preuve d'amour
 et d'autres qualités dynamiques de l'Esprit de Dieu.
 Je ne peux pas nier la validité de vies remplies de l'Esprit de~beaucoup
 de responsables et de laïcs entreprenants dans l'Église aujourd'hui
 qui n'ont jamais goûté à l'expérience \index{expérience}
 du parler en langues\index{langues!parler en \textasciitilde{}},
 et je préfère leur compagnie à celle de beaucoup de ceux qui promeuvent
 le parler en langues comme la seule vraie preuve d'une vie remplie
 de l'Esprit, mais dont les vies personnelles sont entâchées de discorde
 ou d'orgueil, et souvent même d'hérésie.

Quand Paul a écrit aux Galates, il a déclaré~:
 \og Le fruit de l'Esprit est l'amour. \fg{}\ibiblephantom{Ga}(5:22)
 \index{Esprit!fruit de l'\textasciitilde{}}
 La vraie preuve que l'Esprit de Dieu remplit la vie d'une personne
 est l'amour. L'amour est l'évidence la plus valide qu'un homme
 est vraiment rempli de l'Esprit, et les langues sans l'amour
 ne sont que des sons sans signification.


\section{\'Edifier, adorer, prier}

Le parler en langues est une expérience \index{expérience}
 très édifiante pour le croyant.
 Paul nous dit dans \ibibleverse{ICo}(14:4)~:
 \og Celui qui parle en langues s'édifie lui-même. \fg{}
 Le terme \og édifier \fg{} signifie construire, et il est utilisé
 dans le Nouveau Testament pour parler de la constitution de Christ
 dans la vie de l'Église ou du croyant. Le but du rassemblement de l'Église
 est de s'édifier en Christ, et quand je suis à l'église,
 je devrais chercher à édifier le corps du Christ dans son entier.
 Les dévotions individuelles ont pour but de m'édifier moi-même en Christ ;
 quand je parle en langues dans mes dévotions personnelles,
 c'est une des manières par lesquelles Christ est édifié~en~moi.

Le parler en langues est également un excellent moyen d'adorer le Seigneur.
 Je me rend souvent compte que j'ai du mal à exprimer à Dieu les sentiments
 que j'ai en moi. Dieu a été si bon, et m'a tant béni, que simplement dire~:
 \og Ô Dieu, je Te remercie pour tout ce que Tu as fait \fg{}
 est bien loin de mon sentiment de gratitude profonde et d'adoration.
 J'ai du mal à exprimer ces sentiments profonds de mon esprit.
 Il est merveilleux d'être capable, par l'Esprit, d'exprimer mes louanges
 à Dieu sans être obligé de me limiter au canal étriqué
 de mon propre intellect. Paul nous dit que, quand nous parlons en langues,
 nous bénissons Dieu par l'Esprit. Cependant, si nous le faisons
 dans l'église sans interprète, la personne qui est parmi les simples auditeurs
 ne peut pas dire \og Amen \fg{} en réponse à mon action de grâces,
 car elle ne comprend pas ce que je dis. Paul fait remarquer~:
 \og Tu rends, il est vrai, d'excellentes actions de grâces. \fg{}
 \ibiblephantom{ICo}(14:17)En d'autres termes,
 Paul déclare que c'est une bonne manière
 de rendre grâces à Dieu et de Lui exprimer votre adoration et vos louanges.

Écrivant aux Éphésiens au sujet du combat spirituel, Paul parle de l'armure
 que nous devons porter. Puis il continue en indiquant comment
 nous devons livrer bataille à l'ennemi~:
 \og Priez en tout temps par l'Esprit, avec toutes sortes de prières
 et de supplications \fg{} (\ibibleverse{Ep}(6:18)).
 Dans les versets~20~et~21\ibiblephantom{Ju}(20-21:) de son épître,
 Jude nous exhorte à demeurer dans l'amour de~Dieu.
 Il nous dit que l'une des manières de demeurer dans Son amour est de prier
 par l'Esprit-Saint. Dans \ibibleverse{Rm}(8:26), Paul nous dit que l'une
 des faiblesses dont nous faisons l'expérience \index{expérience}
 dans notre démarche
 chrétienne vient de notre vie de prière, car nous ne savons pas toujours
 comment nous devrions prier dans une situation donnée.
 Bien souvent, nous ne connaissons pas la volonté de~Dieu.
 Je veux prier selon la volonté de Dieu car je sais que la prière en dehors
 de la volonté de Dieu n'a aucune valeur. Nous savons que si nous demandons
 quoi que ce soit selon Sa volonté, Il nous entend.
 Mais c'est là que le problème émerge, et c'est là notre faiblesse~:
 nous ne savons pas toujours quelle est la volonté de Dieu.

Paul nous dit dans \ibibleverse{Rm}(8:) que l'Esprit-Saint vient au secours
 de notre faiblesse quand nous ne savons pas comment nous devrions prier,
 car l'Esprit lui-même intercédera pour nous par des soupirs inexprimables.
 Il recherche le cœur et Il sait quelle est la pensée de l'esprit,
 parce qu'Il intercède pour les saints selon la volonté de Dieu.

Par conséquent, quand je ne sais pas comment prier pour un problème en particulier,
 je peux simplement soupirer en mon esprit, et bien que je ne comprenne pas
 les soupirs, Dieu les interprète lui-même comme une intercession
 et une prière selon Sa volonté pour cette personne ou cette situation
 particulière pour laquelle je soupire. Maintenant, si Dieu comprend
 les soupirs inexprimables de l'esprit comme une intercession
 et une prière selon Sa volonté, certainement ces mots articulés
 dans une autre langue, bien que cette langue me soit inconnue,
 sont toutefois compréhensibles à Dieu.


\section{Les langues en priv\'e}

Nous ne pouvons pas remettre en question l'affirmation de Paul lorsqu'il
 a remercié Dieu de parler en langues plus que tous les croyants Corinthiens.
 Pourtant, Paul déclare que, quand il était à l'église, il préférait dire
 cinq mots dans une langue connue plutôt que \numprint{10 000}~mots
 dans une langue inconnue. Il y a ceux qui déclarent~que, puisque les dons
 de l'Esprit ont été donnés au bénéfice de tout le corps, comme le déclare
 Paul dans \ibibleverse{ICo}(12:7), toute utilisation privée
 de l'un de ces dons de l'Esprit est interdite et mauvaise.
 Puisque Paul exerçait le don des langues plus que tous les Corinthiens
 (\ibiblechvs{ICo}(14:19)), il faut supposer qu'il priait et chantait
 par l'Esprit dans ses dévotions privées.

Puisque le don du parler en langues édifie le croyant qui exerce ce don,
 et qu'il est préférable qu'il n'exerce pas ce don dans une assemblée publique
 et qu'il est même interdit de le faire en l'absence d'un interprète,
 la seule place restante pour l'exercice de ce don est dans ses propres
 dévotions personnelles. Paul a dit~:
 \og Qu'on parle à soi-même et à Dieu \fg{}, et il est donc correct
 d'exercer ce don pour sa propre édification personnelle, comme Paul l'a fait.

Au fur et à mesure que vous êtes édifiés en Christ, vous deviendrez
 un instrument par lequel le corps entier pourra être édifié.
 En effet, quand~un membre du corps est honoré, tous les membres
 se réjouissent avec lui.


\section{L'abus des langues}

Certaines personnes disent n'avoir aucun contrôle sur leurs effusions
 en langues, et bien souvent, elles se mettent à parler en langues
 pendant un service public, interrompant le sermon.
 Parfois, ces effusions erratiques éclatent au milieu de conversations
 avec des amis. Une femme m'a dit que lorsqu'elle a reçu le don des langues,
 elle ne pouvait pas le contrôler. Le jour suivant, quand l'employé de son
 fournisseur de gaz est venu relever son compteur, elle est sortie
 lui parler d'un problème avec le service, et elle s'est alors mise
 à lui~parler en langues. Il l'a regardée assez bizarrement,
 et a fini par se retourner et se dépêcher de quitter la cour.
 Elle m'a dit qu'elle ne pouvait pas contrôler son parler en langues.
 Les Écritures nous disent dans \ibibleverse{ICo}(14:32) que les esprits
 des prophètes sont soumis aux prophètes. Je crois que cela affirme
 que nous avons toujours le contrôle de nous-mêmes lorsque nous exerçons
 tout don de l'Esprit. Paul nous instruit dans \ibibleverse{ICo}(14:28)
 que si aucun interprète n'est présent, la personne doit garder
 le silence dans l'église, et parler à soi-même et à Dieu.
 Paul appelle au contrôle sur le don ; il dit qu'une personne n'est pas
 obligée de s'exprimer à voix haute, qu'il lui est possible de parler
 seulement à elle-même et à Dieu. Au verset~\ibiblevs{ICo}(14:15),
 Paul affirme aussi qu'il prie par l'Esprit, et qu'il prie également
 avec compréhension, montrant que le parler en langues était en fait
 contrôlé par l'exercice de sa propre volonté. Quand il le souhaitait,
 il pouvait parler en langues ; quand il le souhaitait, il~pouvait parler
 dans une des langues qu'il comprenait et connaissait.

Dans certaines églises, le sermon est interrompu par des paroles en langues.
 Mais il n'y a vraiment aucune base scripturaire soutenant ce type
 d'interruptions. En fait, Paul a dit~: \og Mais que tout se fasse
 avec bienséance et avec ordre. \fg{}\ibiblephantom{ICo}(14:40)
 Je ne vois jamais aucun ordre dans ce type d'interruptions.
 Elles sont, d'un autre côté, très malpolies et gênantes.
 Il n'y a vraiment aucun besoin pour l'Esprit-Saint d'apporter une parole
 en langues pendant le ministère de la Parole de Dieu, car le ministre
 du culte devrait lui-même parler sous l'onction de l'Esprit-Saint
 et exercer, pour ainsi dire, le don de prophétie lorsqu'il enseigne
 la vérité de Dieu aux gens. Quand une personne se lève et interrompt
 le messager de Dieu, elle met l'Esprit-Saint dans une situation
 incomfortable de s'interrompre Lui-même pour interposer une autre pensée
 ou idée. Un tel usage non scripturaire du don des langues est une autre
 forme de \emph{charismanie}.


\section{Langues et interpr\'etations}

Il est incontestable que Paul cherche à restreindre l'utilisation
 du don des langues dans l'église. Il s'est développé ce qui me semble
 être un faux concept des \og messages \fg{} en langues,
 comme si Dieu avait un message spécial pour l'église qui devait être délivré
 par les langues et l'interprétation. C'est ainsi qu'on s'y réfère
 généralement lorsqu'ils sont prononcés dans l'église
 \ocadr comme à un message en langues.
 Il n'y a pas un seul épisode dans le Nouveau Testament
 qui puisse être utilisé comme exemple de Dieu parlant à quiconque
 par les langues et les interprétations, ou simplement par les langues
 elles-mêmes.

Le plus souvent, lorsqu'il y a une parole en langues qui doit être suivie
 d'une interprétation, il est rare qu'une véritable interprétation
 des langues soit donnée.

J'ai grandi dans une église pentecôtiste, et je suis convaincu que pendant
 toutes mes années dans l'église, j'ai rarement entendu une véritable
 interprétation de la multitude de paroles en langues dont j'ai été témoin.
 Si jamais ça m'est arrivé au cours de ces jeunes années, je n'en ai pas conscience.
 Il y avait de longues paroles en langues suivies de courtes interprétations.
 Il y avait de courtes paroles en langues suivies de longues interprétations.
 On m'a toujours expliqué qu'il y a une différence entre l'interprétation
 et la traduction, ce que j'accepte volontiers. Cependant, je noterais que,
 dans les paroles en langues, il s'agissait bien souvent d'une seule phrase
 répétée en boucle. Pourtant, l'interprétation supposée ne contenait aucune phrase
 répétée.


\section{\`A qui s'adressent les langues ?}


Dans \ibibleverse{ICo}(14:2), Paul nous dit que celui qui parle dans une
 langue inconnue \og ne parle pas aux hommes, mais à Dieu, car personne
 ne le comprend, et c'est en esprit qu'il dit des mystères
 [ou des secrets divins]. \fg{} Il fait ici remarquer que les langues
 sont manifestement adressées à Dieu. Dans tous les cas d'utilisation
 des langues dans le Nouveau Testament, nous voyons qu'elles sont adressées
 à Dieu. Au jour de la Pentecôte, dans \ibibleverse{Ac}(2:11),
 ceux qui pouvaient comprendre les langues remarquaient que ces gens
 déclaraient les merveilleuses œuvres de Dieu. Ils n'utilisaient pas
 les langues pour prêcher, mais pour glorifier Dieu
 en déclarant Ses glorieuses œuvres.
 Dans \ibibleverse{ICo}(14:14), Paul déclare que les langues sont utilisées
 dans ses prières à Dieu. En \ibiblechvs{ICo}(14:16), il déclare
 qu'elles sont utilisées pour bénir Dieu, et pour finir,
 pour rendre grâces à Dieu.

Mais il n'y a pas une seule référence à une utilisation du don
 pour s'adresser aux hommes, que ce soit sous forme de prêche
 ou d'enseignement ; nous voyons qu'elles sont toujours adressées à Dieu,
 et il en résulte nécessairement qu'une interprétation véritable devrait
 également s'adresser à Dieu. L'interprétation devrait être de l'ordre
 de la prière, de l'action de grâces, de la louange ou de la déclaration
 de la gloire de Dieu. Elle devrait souvent sonner comme l'un des psaumes
 de David déclarant la gloire de Dieu. Paul a dit~:
 \og Si tu dis une parole en langues et qu'il n'y a pas d'interprète,
 comment celui qui est assis parmi les simples auditeurs
 répondra-t-il~: Amen! à ton action de grâces, puisqu'il ne sait pas
 ce que tu dis ? \fg{}

Notez que Paul déclare que vous rendez grâces à Dieu, pas que vous délivrez
 un message à l'église ; mais je ne peux même pas dire \og Amen \fg{}
 à votre action de grâces si je ne comprend pas ce que vous dites.
 C'est pourquoi l'interprétation est nécessaire s'il y a une manifestation
 publique des langues, afin que le corps entier soit édifié.


\section{Langues ou proph\'eties ?}

En contraste avec cela, Paul nous dit que celui qui prophétise
 parle aux hommes pour l'édification, l'exhortation et la consolation
 (\ibibleverse{ICo}(14:3)). Lorsque j'ai étudié cette définition
 de la prophétie, j'en ai conclu que la plupart des soi-disant
 interprétations que l'on entend dans les services pentecostaux
 ou charismatiques relèvent en fait de l'exercice du don de prophétie,
 car elles sont souvent dans le style de~: \og Ainsi parle le Seigneur~:
 \og Mes petits enfants, appelez mon nom. \fg{} ou \og Adorez-moi ! \fg{}
 Elles exhortent les gens à adorer, à rendre grâces, à louer,
 ou bien elles réconfortent les gens dans la bonté et la grâce de Dieu.
 Quand les mots sont adressés à l'église pour édifier, cela tombe
 dans la catégorie de la prophétie plutôt que de l'interpréation des langues.

J'en ai conclu que quand une personne prononce une parole en langues,
 plutôt que de prier qu'il y ait une interprétation, bien souvent
 la prière est~: \og Ô Dieu, parle-nous. \fg{}
 Si Dieu nous parle à travers un don de l'Esprit, c'est normalement
 à travers le don de prophétie, des paroles de sagesse ou de connaissance.
 Nous constatons bien souvent que la manifestation des langues donne de la foi
 à la personne qui a un don de prophétie, et celle-ci se lève alors et exerce son don
 de prophétie plutôt que de donner une interprétation de ce qui a été dit
 en langues.

Paul déclare que, si tous parlent en langues dans l'église et qu'un simple
 auditeur survienne, il dira qu'ils sont tous fous.
 Paul restreint également l'usage des langues dans l'église à deux ou trois
 manifestations au maximum, et chacune à son tour.
 Si personne avec le don d'interprétation n'est présent, Paul interdit
 complètement l'usage des langues, disant à la personne qu'elle devrait
 garder le silence et parler à elle-même et à Dieu,
 ce qui implique également que la personne doit pouvoir contrôler
 l'exercice du don.


\section{Une cons\'equence des langues}

Il y a plusieurs années, lorsque l'église Calvary Chapel à Costa Mesa était
 relativement petite, nous nous retrouvions les dimanches soirs dans un
 club-house. Un dimanche soir en particulier (qui était le dimanche
 de la Pentecôte), à la fin de la leçon pendant que nous adorions Dieu
 doucement ensemble, j'ai demandé à l'un des femmes du groupe si elle
 souhaitait adorer Dieu par l'Esprit, et je savais que lorsqu'elle parlait
 en langues, elle parlait habituellement en français.
 Lorsqu'elle~a commencé d'adorer Dieu, je pouvais comprendre suffisamment
 son français pour savoir qu'elle remerciait Dieu pour sa nouvelle vie
 en Christ et la belle nouvelle chanson d'amour qu'Il lui avait offerte.
 J'ai trouvé cela particulièrement beau, car elle était chanteuse de boite
 de nuit avant sa conversion. À la fin de son adoration dans l'Esprit,
 ma femme a commencé à donner l'interprétation au groupe, et sachant
 qu'elle ne connaît pas le français, j'étais particulièrement béni
 d'entendre la précision avec laquelle l'adoration par l'Esprit
 était interprétée pour le groupe.

Après la réunion, un des jeunes hommes du groupe a amené une jeune fille
 juive de Palm Springs pour se faire conseiller. Quand nous nous sommes
 assis ensemble, elle a dit~: \og Avant que nous en venions à mes problèmes,
 expliquez-moi ce qui s'est passé ici ce soir. Pourquoi la première
 femme a-t-elle parlé à Dieu en français, et l'autre femme a-t-elle
 traduit au groupe ce qu'elle a dit ? \fg{}
 J'ai dit~: \og Le croiriez-vous si je vous disais qu'aucune de ces femmes
 ne connaît le français ? \fg{} Je lui ai dit que je savais avec certitude
 que ni l'une ni l'autre ne connaissaient le français, car l'une d'elles
 était une amie proche, et l'autre était ma femme. Puis je lui ai montré
 \bibleverse{ICo} où il est question du don des langues
 et de leur in\-ter\-pré\-ta\-tion. Elle m'a alors dit qu'elle avait vécu en France
 pendant six ans, et que l'accent du français qui avait été parlé était
 un parfait exemple de ce qu'elle appelait le français aristocratique.
 Puis elle a dit~: \og Je dois accepter Jésus-Christ maintenant,
 avant d'aller plus loin. \fg{}

J'ai eu la joie de la voir trouver son Messie et devenir membre du corps
 du Christ. Il y avait eu une démonstration du don des langues,
 suivie par une interprétation véritable, qui était une louange glorieuse
 et une adoration de Dieu. Le résultat était l'édification de l'Église,
 et dans ce cas la conversion de cette jeune fille juive.
\closechapter

