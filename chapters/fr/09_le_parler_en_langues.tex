\chapter{Le parler en langues}

\begin{specialpar}{\tolerance=240}
\lettrine{U}{n des domaines} où la controverse est la plus vive au sein
 du corps du Christ aujourd'hui implique le fait de parler dans d'autres
 langues, ou \og glossolalie \fg{}.
 À un extrême, il y a des personnes qui considèrent tout exercice
 des langues comme satanique.
 À l'autre extrême, il y a des personnes qui affirment que vous n'êtes pas
 remplis ou baptisés de l'Esprit-Saint à moins que vous ne parliez
 dans d'autres langues. Ils affirment que le parler en langues
 est l'évidence initiale du baptème de l'Esprit-Saint.
 Dans \ibibleverse{ICo}(13:1), Paul déclare que les langues en elles-mêmes
 ne sont pas une évidence valide de la présence de l'Esprit-Saint dans la vie
 du croyant, car \og quand je parlerais les langues des hommes et des anges,
 si je n'ai pas l'amour (agapé), je suis du bronze qui résonne ou une cymbale
 qui retentit. \fg{}
 En d'autres termes, les langues ne sont que des sons sans signification
 et n'ont aucune validité si elles ne sont pas accompagnées de l'amour agapé.
\end{specialpar}


\section*{Les langues ou non ?}

Dans le livre des Actes, le parler en langues est souvent accompagné
 de la relation \emph{épi} avec l'Esprit-Saint.
 C'est le cas dans Actes chapitres \ibiblechvs{Ac}(2:), \ibiblechvs{Ac}(10:)
 et \ibiblechvs{Ac}(19:). Cependant, dans le huitième chapitre des Actes, \ibiblephantom{Ac}(8:)quand les croyants Samaritains reçoivent
 l'Esprit-Saint, il n'est pas dit qu'ils ont parlé en langues.
 Pourtant, il est évident qu'il a dû y avoir un phénomène qui a accompagné
 le moment où ils ont reçu l'Esprit-Saint, car Simon le magicien cherchait
 à acheter la puissance que Pierre et Jean possédaient ;
 il désirait pouvoir également imposer les mains sur les gens
 afin qu'ils reçoivent l'Esprit-Saint. Il est évident qu'un phénomène
 accompagnait le moment où ils ont reçu l'Esprit car Simon voulait en acheter
 la puissance afin de pouvoir en dupliquer le prodige.
 Plus tard, dans \ibibleverse{Ac}(9:17), quand Ananias a imposé les mains
 sur Saul (Paul) pour qu'il reçoive le don de l'Esprit-Saint,
 il n'y a aucune mention de Paul parlant en langues.
 Cependant, nous savons que plus tard, alors que Paul écrivait aux
 Corinthiens, il remerciait Dieu qu'il parlait en langues plus que tous
 les Corinthiens.\ibiblephantom{ICo}(14:18)Le moment où Paul a fait pour la première fois
 l'expérience du don des langues n'est pas indiqué.

Nous devons attirer l'attention sur le fait qu'une personne qui parle
 en langues et n'a pas d'amour agapé fait preuve d'un témoignage
 moins valide de la présence ou de l'emplissage de l'Esprit dans sa vie
 qu'une personne qui n'a jamais parlé en langues mais fait preuve d'amour
 et d'autres qualités dynamiques de l'Esprit de Dieu.
 Je ne peux pas nier la validité de vies remplies de l'Esprit de beaucoup
 de responsables dynamiques et de laïcs dans l'Église aujourd'hui
 qui n'ont jamais goûté à l'expérience du parler en langues,
 et je préfère leur compagnie à celle de beaucoup de ceux qui promeuvent
 le parler en langues comme la seule vraie preuve d'une vie remplie
 de l'Esprit, mais dont les vies personnelles sont entâchées de discorde
 ou d'orgueil, et souvent même d'hérésie.

Quand Paul a écrit aux Galates, il a déclaré~:
 \og Le fruit de l'Esprit est l'amour. \fg{}\ibiblephantom{Ga}(5:22)La vraie preuve que l'Esprit de Dieu remplit la vie d'une personne
 est l'amour. L'amour est l'évidence la plus valide qu'un homme
 est vraiment rempli de l'Esprit, et les langues sans l'amour
 ne sont que des sons sans signification.


\section*{Édifier, adorer, prier}

\begin{specialpar}{\tolerance=500}
Le parler en langues est une expérience très édifiante pour le croyant.
 Paul nous dit dans \ibibleverse{ICo}(14:4)~:
 \og Celui qui parle en langues s'édifie lui-même. \fg{}
 Le terme \og édifier \fg{} signifie construire, et il est utilisé
 dans le Nouveau Testament pour parler de la constitution de Christ
 dans la vie de l'Église ou du croyant. Le but du rassemblement de l'Église
 est de s'édifier en Christ, et quand je suis à l'église,
 je devrais chercher à édifier le corps du Christ dans son entier.
 Les dévotions individuelles ont pour but de m'édifier moi-même en Christ ;
 quand je parle en langues dans mes dévotions personnelles,
 c'est une des manières par lesquelles Christ est édifié en moi.
\end{specialpar}

Le parler en langues est également un excellent moyen d'adorer le Seigneur.
 Je trouve souvent que j'ai du mal à exprimer à Dieu les sentiments
 que j'ai en moi. Dieu a été si bon, et m'a tant béni, que simplement dire~:
 \og Oh Dieu, je Te remercie pour tout ce que Tu as fait \fg{}
 est bien loin de mon sentiment de gratitude profonde et d'adoration.
 J'ai du mal à exprimer ces sentiments profonds de mon esprit.
 Il est merveilleux d'être capable, par l'Esprit, d'exprimer mes louanges
 à Dieu sans être obligé de me limiter au canal étriqué
 de mon propre intellect. Paul nous dit que, quand nous parlons en langues,
 nous bénissons Dieu avec l'Esprit. Cependant, si nous le faisons
 dans l'église sans interprète, la personne qui est parmi les simples auditeurs
 ne peut pas dire \og Amen \fg{} en réponse à mon action de grâces,
 car elle ne comprend pas ce que je dis. Paul fait remarquer~:
 \og Tu rends, il est vrai, d'excellentes actions de grâces. \fg{}
 En d'autres termes, Paul déclare que c'est une bonne manière
 de rendre grâce à Dieu et de Lui exprimer votre adoration et vos louanges.

Écrivant aux Éphésiens au sujet du combat spirituel, Paul parle de l'armure
 que nous devons porter. Puis il continue en indiquant comment
 nous devons livrer bataille à l'ennemi~:
 \og Priez en tout temps par l'Esprit, avec toutes sortes de prières
 et de supplications \fg{} (\ibibleverse{Ep}(6:18)).
 Dans les versets \ibiblevs{Ep}{6:20} et \ibiblevs{Ep}(6:21) de cet épître,
 Jude nous exhorte à demeurer dans l'amour de Dieu.
 Il nous dit que l'une des manières de demeurer dans Son amour est de prier
 par l'Esprit-Saint. Dans \ibibleverse{Rm}(8:26), Paul nous dit que l'une
 des faiblesses dont nous faisons l'expérience dans notre démarche
 chrétienne vient de notre vie de prière, en ne sachant pas toujours
 comment nous devrions prier dans une situation donnée.
 Bien souvent, nous ne connaissons pas la volonté de Dieu.
 Je veux prier selon la volonté de Dieu car je sais que la prière en dehors
 de la volonté de Dieu n'a aucune valeur. Nous savons que si nous demandons
 quoi que ce soit selon Sa volonté, Il nous entend.
 Mais c'est là que le problème émerge, et c'est là notre faiblesse~:
 nous ne savons pas toujours quelle est la volonté de Dieu.

Paul nous dit dans \ibibleverse{Rm}(8:) que l'Esprit-Saint vient au secours
 de notre faiblesse quand nous ne savons pas comment nous devrions prier,
 car l'Esprit lui-même intercédera pour nous par des soupirs inexprimables.
 Il recherche le cœur et Il sait quelle est la pensée de l'esprit,
 parcequ'Il intercède pour les saints selon la volonté de Dieu.

Donc, quand je ne sais pas comment prier pour un problème en particulier,
 je peux simplement soupirer en mon esprit, et bien que je ne comprenne pas
 les soupirs, Dieu les interprète lui-même comme une intercession
 et une prière selon Sa volonté pour cette personne ou cette situation
 particulière pour laquelle je soupire. Maintenant, si Dieu comprend
 les soupirs inexprimables de l'esprit comme une intercession
 et une prière selon Sa volonté, certainement ces mots articulés
 dans une autre langue, bien que cette langue me soit inconnue,
 sont toutefois compréhensibles à Dieu.


\closechapter
