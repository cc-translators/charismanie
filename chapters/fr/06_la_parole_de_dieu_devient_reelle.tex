\chapter{La Parole de Dieu devient réelle}

\chlettrine{A}{lors que nous considérions} le travail de l'Esprit-Saint
 dans la vie du croyant, nous avons vu comment Il nous donne la puissance
 d'être tout ce que Dieu voudrait que nous soyons.
 Ensuite, nous avons vu comment Il nous conforme à l'image de Jésus-Christ~:
 \og Nous tous, qui le visage dévoilé, reflétons comme un miroir la gloire
 du Seigneur, nous sommes transformés en la même image, de gloire en gloire,
 comme par le Seigneur, l'Esprit. \fg{}
 Ensuite, nous avons vu comment Il nous apporte l'amour agapé de Dieu,
 nous aidant à le recevoir et nous donnant ensuite la capacité d'aimer
 avec l'amour agapé de Dieu, car l'Esprit-Saint a répandu l'amour de Dieu
 dans nos cœurs et dans nos vies.

Maintenant, nous voudrions étudier le travail de l'Esprit-Saint dans la vie
 du croyant, alors qu'il rend les choses de Dieu et la Parole de Dieu
 réels pour nous.


\section*{Ce que l'œil n'a pas vu}

Dans \bibleverse{ICo}(2:9), Paul a écrit~:
 \og Ce que l'œil n'a pas vu, ce que l'oreille n'a pas entendu,
 et ce qui n'est pas monté au cœur de l'homme, tout ce que Dieu
 a préparé pour ceux qui l'aiment. \fg{}
 Cette écriture a souvent été mal interprétée. En fait,
 je pense que de tous les passages de la Bible qui ont été mal interprétés
 ou cités hors de leur contexte, ce verset fait partie des principaux.
 Vous entendez généralement cette écriture citée en lien avec le paradis.
 Le paradis, nous dit-on, va être si glorieux, si extraordinaire,
 si beau que \og l'œil ne l'a pas vu, l'oreille ne l'a pas entendu,
 et ce n'est pas monté au cœur de l'homme, tout ce que Dieu a préparé
 pour ceux qui l'aiment. \fg{}

Pendant toute mon enfance à l'église, c'est de cette façon que j'ai entendu
 ce verset cité, et c'est pourquoi je l'ai interprété ainsi moi aussi
 durant les dix premières années de mon ministère.
 Et puis un jour, j'ai lu le contexte dans son intégralité et j'ai réalisé
 que Paul ne parlait pas du paradis. Paul parlait des choses que Dieu
 a pour Son peuple \emph{dès maintenant},
 ces choses que Dieu a pour nous parcequ'Il
 nous aime et que nous L'aimons. Que dit Paul?
 L'homme naturel ne peut pas voir, il ne peut pas connaître,
 il ne peut pas comprendre ces choses que Dieu a pour \emph{nous}.
 Voilà de quoi il parle. L'œil de l'homme naturel ne peut pas voir,
 son oreille ne peut pas entendre, et ces choses que Dieu a pour \emph{nous}
 car nous L'aimons ne sont pas entrées dans son cœur.
 Pierre nous dit que même les anges désirent voir les choses que Dieu
 a en réserve pour Son Église.
 Sans aucun doute, ils trouvent cela extraordinaire que Dieu vienne
 et demeure parmi nous.

Remarquez que le verset suivant dit~:
 \og À nous, Dieu \emph{nous} l'\emph{a} révélé. \fg{}
 Paul ne parle pas du paradis.
 Il parle \emph{des gloires présentes de la vie vécue dans la puissance
 de l'Esprit-Saint}.
 \og Dieu nous l'a révélé par l'Esprit, car l'Esprit sonde tout,
 même les profondeurs de Dieu. Qui donc, parmi les hommes,
 sait ce qui concerne l'homme, si ce n'est l'esprit de l'homme
 qui est en lui ? \fg{} (\bibleverse{ICo}(2:10-11)).
 En d'autres termes, qui sait vraiment ce qui est dans votre cœur sinon vous?
 Vous pouvez construire une belle façade ; il est possible que vous trompiez
 beaucoup de gens.
 Je n'ai vraiment aucune idée de ce qui est dans votre cœur.
 Vous le \emph{savez}.
 Vous savez ce qui est caché et voilé et ce que vous cachez à tous.
 Ainsi, Paul dit~:
 \og Qui donc, parmi les hommes, sait ce qui concerne l'homme,
 si ce n'est l'esprit de l'homme qui est en lui ?
 De même, personne ne connaît ce qui concerne Dieu,
 si ce n'est l'Esprit de Dieu \fg{} (\bibleverse{ICo}(2:11)).


\section*{L'Esprit de révélation}

Il y a des choses que Dieu sait
 (par exemple, des aspects de l'amour de Dieu)
 et que l'homme ne peut simplement pas comprendre ;
 seul l'Esprit de Dieu comprend ces choses. Paul continue~:
 \og Or nous, nous n'avons pas reçu l'esprit du monde,
 mais l'Esprit qui vient de Dieu, afin de savoir ce que Dieu
 nous a donné par grâce \fg{} (\bibleverse{ICo}(2:12)).
 L'Esprit-Saint nous fait connaître ces choses que Dieu
 nous a librement données.
 \og Et nous en parlons, dit Paul, non avec des discours qu'enseigne
 la sagesse humaine, mais avec ceux qu'enseigne l'Esprit,
 en expliquant les réalités spirituelles à des hommes spirituels.
 Mais l'homme naturel ne reçoit pas les choses de l'Esprit de Dieu,
 car elles sont une folie pour lui, et il ne peut les connaître,
 parce que c'est spirituellement qu'on en juge \fg{}
 (\bibleverse{ICo}(2:13-14)).

Beaucoup de gens ont fait l'erreur d'essayer de découvrir ce que recèle
 la Bible par eux-mêmes. Ils ont lu la Bible pour découvrir son message,
 mais ils l'ont fait avec leur intelligence humaine uniquement.
 Cela se termine en général par un échec. Ils essaient de lire,
 mais n'arrivent nulle part. Ils disent~:
 \og J'ai essayé de lire la Bible, mais je ne peux pas voir ce qu'on peut
 en retirer. Je ne comprend vraiment pas. \fg{}
 C'est exactement ce que Paul dit~:
 \og L'homme naturel \emph{ne peut pas} comprendre les choses de l'Esprit,
 ni même les connaître. \fg{}
 Il est \emph{impossible} à l'homme naturel ou à l'esprit naturel
 de comprendre les choses de l'Esprit car il lui manque la capacité
 de le faire. Vous pourriez dire avec la même logique qu'un aveugle
 ne peut pas apprécier la beauté d'un coucher de soleil ou qu'un sourd
 ne peut pas apprécier la musique d'un concert, car il lui manque la capacité
 par laquelle ces choses sont appréciées et comprises.
 À moins que l'Esprit-Saint n'ouvre notre cœur et notre esprit à ces choses,
 nous ne pouvons simplement pas les comprendre.


