\chapter{La Parole de Dieu devient r\'eelle}

\lettrine{A}{lors} que nous considérions
 le travail de l'Esprit-Saint
 dans la vie du croyant, nous avons vu comment Il nous donne la puissance
 d'être tout ce que Dieu voudrait que nous soyons.
 Nous avons vu ensuite comment Il nous conforme à l'image de Jésus-Christ~:
 \og Nous tous, qui le visage dévoilé, reflétons comme un miroir la gloire
 du Seigneur, nous sommes transformés en la même image, de gloire en gloire,
 comme par le Seigneur, l'Esprit. \fg{}
 \ibiblephantom{IICo}(3:18)Ensuite, nous avons vu comment
 Il nous apporte l'amour agapé de Dieu,~nous aidant à le recevoir
 et nous donnant ensuite la capacité d'aimer
 avec l'amour agapé de Dieu, car l'Esprit-Saint a répandu l'amour de Dieu
 dans nos cœurs et dans nos vies.

Nous souhaiterions maintenant étudier l'œuvre de l'Esprit-Saint dans la vie
 du croyant, lorsqu'Il rend les choses de Dieu et la Parole de Dieu
 réelles pour nous.


\section{Ce que l'\oe{}il n'a pas vu}

Dans \ibibleverse{ICo}(2:9), Paul a écrit~:
 \og Ce que l'œil n'a pas vu, ce que l'oreille n'a pas entendu,
 et ce qui n'est pas monté au cœur de l'homme, tout ce que Dieu
 a préparé pour ceux qui l'aiment. \fg{}
 Ce passage des Écritures a souvent été mal interprété. En fait,
 je pense que de tous les passages de la Bible qui ont été mal interprétés
 ou cités hors de leur contexte, ce verset fait partie des principaux.
 On entend généralement ce passage cité en lien avec le paradis.
 Le paradis, nous dit-on, va être si glorieux, si extraordinaire,
 si beau que \og l'œil ne l'a pas vu, l'oreille ne l'a pas entendu,
 et ce n'est pas monté au cœur de l'homme, tout ce que Dieu a préparé
 pour ceux qui l'aiment. \fg{}

Pendant toute mon enfance à l'église, c'est de cette façon que j'ai entendu
 ce verset cité, et c'est pourquoi je l'ai moi-même interprété ainsi
 durant les dix premières années de mon ministère.
 Et puis un jour, j'ai lu le contexte dans son intégralité et j'ai réalisé
 que Paul ne parlait pas du paradis. Paul parlait des choses que Dieu
 a pour Son peuple \emph{dès maintenant},
 ces choses que Dieu a pour nous parcequ'Il
 nous aime et que nous L'aimons. Que dit Paul?
 L'homme naturel ne peut pas voir, il ne peut pas connaître,
 il ne peut pas comprendre ces choses que Dieu a pour \emph{nous}.
 Voilà de quoi il parle. L'œil de l'homme naturel ne peut pas voir,
 son oreille ne peut pas entendre, et ces choses que Dieu a pour \emph{nous}
 car nous L'aimons ne sont pas entrées dans son cœur.
 Pierre nous dit que même les anges désirent voir les choses que Dieu
 a en réserve pour Son Église.
 Sans aucun doute, ils trouvent cela extraordinaire que Dieu vienne
 et demeure parmi nous.

Remarquez que le verset suivant dit~:
 \og À nous, Dieu \emph{nous} l'\emph{a} révélé. \fg{}
 Paul ne parle pas du paradis.
 Il parle \emph{des gloires présentes de la vie vécue dans la puissance
 de l'Esprit-Saint}.
 \og Dieu nous l'a révélé par l'Esprit, car l'Esprit sonde tout,
 même les profondeurs de Dieu. Qui donc, parmi les hommes,
 sait ce qui concerne l'homme, si ce n'est l'esprit de l'homme
 qui est en lui ? \fg{} (\ibibleverse{ICo}(2:10-11)).
 En d'autres termes, qui sait vraiment ce qui est dans votre cœur sinon vous?
 Vous pouvez édifier une belle façade ; il est possible que vous trompiez
 beaucoup de gens.
 Je n'ai vraiment aucune idée de ce qui est dans votre cœur.
 Vous le \emph{savez}.
 Vous savez ce qui est caché et voilé et ce que vous cachez à tous.
 Ainsi, Paul dit~:
 \og Qui donc, parmi les hommes, sait ce qui concerne l'homme,
 si ce n'est l'esprit de l'homme qui est en lui ?
 De même, personne ne connaît ce qui concerne Dieu,
 si ce n'est l'Esprit de Dieu \fg{} (\ibibleverse{ICo}(2:11)).


\section{L'Esprit de r\'ev\'elation}

Il y a des choses que Dieu connaît
 (par exemple, des aspects de l'amour de Dieu)
 et que l'homme ne peut tout simplement pas comprendre ;
 seul l'Esprit de Dieu comprend ces choses. Paul continue~:
 \og Or nous, nous n'avons pas reçu l'esprit du monde,
 mais l'Esprit qui vient de Dieu, afin de savoir ce que Dieu
 nous a donné par grâce \fg{} (\ibibleverse{ICo}(2:12)).
 L'Esprit-Saint nous fait connaître ces choses que Dieu
 nous a librement données.
 \og Et nous en parlons, dit Paul, non avec des discours qu'enseigne
 la sagesse humaine, mais avec ceux qu'enseigne l'Esprit,
 en expliquant les réalités spirituelles à des hommes spirituels.
 Mais l'homme naturel ne reçoit pas les choses de l'Esprit de Dieu,
 car elles sont une folie pour lui, et il ne peut les connaître,
 parce que c'est spirituellement qu'on en juge \fg{}
 (\ibibleverse{ICo}(2:13-14)).

Beaucoup de gens ont fait l'erreur d'essayer de découvrir ce que recèle
 la Bible par eux-mêmes. Ils ont lu la Bible pour découvrir son message,
 mais ils l'ont fait avec leur intelligence humaine uniquement.
 Cela se solde en général par un échec. Ils essaient de lire,
 mais n'aboutissent à rien. Ils disent~:
 \og J'ai essayé de lire la Bible,~mais~je ne peux pas voir ce qu'on peut
 en retirer. Je ne comprend vraiment pas. \fg{}
 C'est exactement ce que Paul dit~:
 \og L'homme naturel \emph{ne peut~pas} comprendre les choses de l'Esprit,
 ni même les connaître. \fg{}
 Il est \emph{impossible} à l'homme naturel ou à l'esprit naturel
 de comprendre les choses de l'Esprit car il lui manque la capacité
 de le faire. Vous pourriez dire avec la même logique qu'un aveugle
 ne peut pas apprécier la beauté d'un coucher de soleil ou qu'un sourd
 ne peut pas apprécier la musique d'un concert, car il lui manque la capacité
 par laquelle ces choses sont appréciées et comprises.
 À moins que l'Esprit-Saint n'ouvre notre cœur et notre esprit à ces choses,
 nous ne pouvons tout simplement pas les comprendre.


\section{Voir et ne pas voir}

Une des choses les plus difficiles qui soient est d'avoir une compréhension
 claire d'un problème et de se demander pourquoi une autre personne
 n'a pas cette même compréhension claire. C'est si simple!
 C'est si évident! Comment ne peux-tu pas le voir?
 C'est juste là, regarde! Mais s'ils ne sont pas éveillés spirituellement,
 si l'Esprit de Dieu ne demeure pas en eux, ils peuvent regarder
 toute la journée et ne pas le voir pour autant.
 Cette compréhension et cet éveil doivent venir par la puissance
 de l'Esprit-Saint afin d'ouvrir les choses de Dieu à nos cœurs.
 Paul priait pour les croyants Éphésiens dans \ibiblechvs{Ep}(1:17)
 que Dieu leur donne l'Esprit de sagesse et de révélation~dans la
 connaissance qu'ils avaient de Lui.
 L'œuvre de l'Esprit-Saint est de nous faire
 connaître cette grâce glorieuse de Dieu déversée sur nous,
 cette œuvre de Dieu pour nous et ce que~Dieu veut faire dans nos vies.
 Quelle tragédie que les gens cherchent à vivre la vie chrétienne
 et à comprendre la démarche chrétienne sans l'aide de l'Esprit-Saint!
 C'est tout simplement impossible.

Dieu a mis à notre disposition tout ce dont nous avons besoin pour la vie
 et la sanctification. Dieu n'a rien oublié ; quelle que soit la situation
 que nous rencontrions, Dieu a un moyen de nous en faire sortir.
 Dieu a déjà tout préparé. L'Esprit-Saint nous rend conscients
 de ces choses que Dieu nous a déjà données gratuitement,
 afin que nous nous appropriions l'œuvre de Dieu pour nos besoins
 particuliers et nos situations particulières.

Je ne m'assied pas pour lire la Parole de Dieu sans dire d'abord~: \linebreak
 \og Oh~Esprit-Saint, ouvre mon esprit et mon cœur pour recevoir
 et comprendre la Parole de Dieu. \fg{}
 Je n'ose pas m'approcher de la Parole de Dieu avec mon propre intellect.
 Ce serait trop flou. J'ai besoin de l'aide de l'Esprit-Saint
 pour m'enseigner ce que Dieu a dit.
 Dans \ibibleverse{IJn}(2:27), nous lisons~:
 \og Vous n'avez pas besoin qu'on vous enseigne, mais cette onction
 que vous avez reçue va vous enseigner. \fg{}
 Nous pouvons donc nous tourner vers l'Esprit-Saint
 pour nous enseigner et nous guider lorsque nous étudions
 la Parole de Dieu.
 \nowidow[3]


\section{Recevoir les instructions de J\'esus}

Quand nous parlons avec des personnes qui viennent de donner leur vie
 à Jésus-Christ, nous cherchons à mettre en avant l'im\-por\-tance d'apprendre
 à connaître Jésus. Jésus a dit trois choses \ocadr tout d'abord~:
 \og Venez à moi \fg{} ; puis~: \og Prenez mon joug sur vous \fg{} ;
 et enfin~: \og Recevez mes instructions. \fg{}
 \ibiblephantom{Mt}(11:28-29)Le salut est plus que le simple fait
 de venir à Christ~:
 c'est prendre Son joug sur nous ;
 c'est soumettre nos vies à Jésus-Christ comme maître ;
 c'est Lui donner les rênes de nos vies.
 Mais alors si je dois grandir, je dois recevoir Ses instructions.

Bien des gens ont échoué dans la poursuite de la grâce de Dieu
 car ils n'ont pas reçu les instructions de Jésus-Christ ;
 ils n'ont pas grandi par leur connaissance de Lui.
 Il n'y a qu'une seule manière de recevoir les instructions de Jésus-Christ,
 c'est de lire la Parole~de Dieu.
 La Parole de Dieu est la seule nourriture spirituelle pour le nouveau
 chrétien, le nouveau-né spirituel ; vous ne pouvez pas grandir en dehors
 de la Parole de Dieu et de la connaissance de Dieu et de Jésus-Christ
 par la Parole.
 Elle est vitale pour votre expérience \index{expérience} chrétienne
 et pour votre croissance chrétienne.


\section{Quelle est la bonne mani\`ere?}

J'enseigne toujours aux gens de prier simplement
 avant de commencer à lire~:
 \og Seigneur, ouvre mes yeux et fais-moi voir,
 ouvre mes oreilles et fais-moi entendre ce que l'Esprit
 voudrait me dire par la Parole. \fg{}
 Certaines personnes disent~:
 \og Il y a tellement d'in\-ter\-pré\-ta\-tions que je suis perdu. \fg{}
 Je ne vous demande pas de lire l'interprétation de la Bible
 selon une personne en particulier.
 Certaines personnes demandent~:
 \og Comment sais-tu que ta manière est bonne? Peut-être que Jean Dupont
 avait raison, ou peut-être que les Témoins de Jéhovah ont raison,
 ou peut-être que Mary Baker Eddy avait raison.
 Tout le monde a sa propre interprétation.
 Comment sais-tu que tu as raison? \fg{}
 Laissez-moi vous dire ceci~: je ne suis pas du tout inquiet
 de ce que vous croirez si vous vous contentez de lire la Bible.
 Je crois que l'Esprit-Saint a la possibilité de vous enseigner
 \ocadr directement depuis la Bible elle-même \fcadr{}
 tout ce que vous avez besoin de savoir.
 Je ne vous encourage pas à lire l'interprétation d'une personne
 en particulier. Je vous encourage à lire simplement la Bible
 et à laisser l'Esprit-Saint vous enseigner ce que Dieu a dit.

Je ne suis pas du tout inquiet de la manière dont vous allez interpréter
 la Bible si vous vous contentez de lire la Bible seule.
 Je ne suis pas inquiet que vous tombiez dans une fausse doctrine
 ou un délire hallucinatoire si vous lisez simplement la Bible.
 Je \emph{suis} inquiet des délires que vous avez lorsque vous lisez certains
 des torchons qui circulent et prétendent intepréter la Bible.


\section{Apprendre et se rappeler}

Jésus a dit~: \og L'Esprit-Saint viendra à vos côtés pour vous aider ;
 Il vous enseignera toutes choses et ensuite vous rappellera toutes choses,
 tout ce que je vous ai commandé. \fg{}
 Notez qu'Il va vous \emph{rappeler} toutes choses.
 Qu'est-ce que cela signifie? Cela signifie que les choses
 doivent être plantées là afin de pouvoir être remémorées.
 Il ne peut pas vous rappeler des choses sans que cela ait été planté
 dans votre esprit au préalable. Il est important que vous lisiez
 la Parole de Dieu. David a dit~: \og Je serre ta promesse dans mon cœur,
 afin de ne pas pécher contre toi. \fg{}
 \ibiblephantom{Ps}(119:11)Il est possible que vous ne vous rappeliez
 plus de ce que vous avez lu
 cinq minutes après l'avoir lu, pourtant une situation de crise
 peut surgir dans votre vie, et tout à coup une Écriture vous saute
 à l'esprit. Que s'est-il passé? L'Esprit-Saint vous a rappelé
 ce que vous y aviez placé, et au moment d'urgence l'Esprit-Saint
 vous a aidé.

Dans \ibibleverse{Jn}(16:), Jésus parlait avec Ses disciples juste avant
 de partir au jardin de Gethsémané, où Il allait être arrêté.
 C'était la dernière nuit que Jésus partageait avec Ses disciples
 avant Sa crucifixion.
 Il leur dit au verset~\ibiblevs{Jn}(16:12)~:
 \og J'ai encore beaucoup de choses à vous dire,
 mais vous ne pouvez pas les comprendre maintenant. \fg{}
 Cela faisait trois ans et demi qu'Il était avec eux et les instruisait,
 et à ce moment-là il y avait beaucoup de choses qu'Il devait encore
 leur dire, mais ils n'étaient pas capables de les recevoir.
 Alors Il a dit~: \og Quand il sera venu, lui, l'Esprit de vérité,
 il vous conduira dans toute la vérité ; car ses paroles ne viendront pas
 de lui-même, mais il parlera de tout ce qu'il aura entendu et vous annoncera
 les choses à venir \fg{} (\ibibleverse{Jn}(16:13)).
 Jésus disait~: \og J'ai beaucoup de choses à vous dire ; vous ne pouvez pas
 les comprendre maintenant, mais lorsque l'Esprit de vérité sera venu,
 Il vous conduira dans toute la vérité, et Il vous montrera les choses
 à venir. \fg{}
 L'Esprit de Dieu nous est donné pour nous enseigner les choses de Dieu ;
 l'Esprit de Dieu nous est donné pour nous conduire dans la vérité de Dieu,
 et ensuite pour nous montrer les choses à venir.


\section{Savoir ce qui vient}

Au sujet du fait qu'Il nous montre les choses à venir,
 l'apôtre Paul à écrit à l'église de Thessalonique~:
 \og Mais vous, frères, vous n'êtes pas dans les ténèbres,
 pour que ce jour [c'est-à-dire le jour du Seigneur,
 l'enlèvement de l'Église] vous surprenne comme un voleur ;
 vous êtes tous fils de la lumière \fg{}.
 \ibiblephantom{ITh}(5:4-5)Que dit-il ?
 Il dit que le jour du Seigneur, la venue de Christ
 pour Son Église, ne devrait pas nous prendre par surprise.
 Cela ne devrait pas nous surprendre. Si vous marchons dans l'Esprit,
 si nous sommes conduits par l'Esprit, Il nous montre les choses à venir
 et nous garde en éveil au sujet du jour dans lequel nous vivons.

Je suis vraiment choqué de voir combien il y a de personnes aveugles
 dans cette époque où nous vivons. Je participais à une émission de radio
 une nuit à Los Angeles lorsque j'ai mentionné la seconde venue
 de Jésus-Christ \ocadr Son retour prochain.
 Tous les autres participants à l'émission
 (qui étaient des responsables religieux) trouvaient qu'il était horrible
 de penser que le Seigneur revenait bientôt, et que nous devrions être plus
 intéressés par le fait de rendre ce monde meilleur. J'ai répondu~:
 \og Cela fait longtemps que vous essayez de faire ça, mais cela n'a jamais
 été pire qu'aujourd'hui. \fg{}
 Je suis surpris qu'ils ne soient pas découragés ;
 il faut leur reconnaître ça!
 Lorsque vous essayez autant de rendre le monde meilleur et qu'il ne fait
 qu'empirer, si vous n'êtes pas encore découragé, vous êtes
 particulièrement persévérant.
 Mais je pense qu'une personne doit être complètement aveugle pour regarder
 autour d'elle aujourd'hui et dire~:
 \og Oh magnifique! Les choses s'améliorent! \fg{}

Ils ne font pas face à la réalité. Je suis un réaliste.
 L'Église, dans son effort pour faire du monde un meilleur endroit
 pour vivre, n'a fait que se disqualifier.
 Les méthodes par lesquelles ils essaient de~faire du monde un meilleur
 endroit pour vivre sont des choses que je ne comprend pas. 
 Lorsque des groupes d'église soutiennent des organisations terroristes
 africaines et l'OLP \NdT{L'Organisation de Libération de la Palestine.},
 j'ai du mal à voir comment cela contribue à rendre le monde meilleur.

\section{La trag\'edie de l'aveuglement}

Ce qui est frustrant lorsqu'on marche en étant guidé par l'Esprit,
 c'est que lorsque l'Esprit vous montre des choses qui sont si évidentes,
 vous ne pouvez vraiment pas comprendre comment quelqu'un d'autre
 peut ne pas les voir. Mais la raison pour laquelle il ne peut pas les voir
 est que l'homme naturel ne peut pas comprendre les choses de l'Esprit~:
 \og [\dots{}] et il ne peut les connaître, parce que c'est spirituellement
 qu'on en juge. \fg{}
 \ibiblephantom{ICo}(2:14)Certaines personnes sont des imposteurs manifestes,
 et pourtant les gens se laissent tromper par eux.
 C'est comme voir un portrait \emph{souriant} d'Abraham Lincoln
 sur un billet de cinq dollars ; c'est évident que c'est un faux,
 et pourtant les gens se laissent duper aveuglément et se font prendre.
 Vous vous dites~: \og Vous ne voyez pas ? Ils vous arnaquent.
 C'est un faux. \fg{}
 Mais ils n'ont pas le don de discernement, et c'est pourquoi ils sont
 complètement trompés.
 C'est la chose la plus difficile qui soit que de voir si clairement
 et de ne pas comprendre pourquoi d'autres personnes ne peuvent
 pas voir les problèmes aussi clairement que vous le pouvez.

Le retour de Jésus-Christ est si proche ; nous sommes juste à la fin
 des temps des nations, et il est évident que la venue du Seigneur
 est à portée de main. Les Écritures sont réalisées de façon si évidente,
 et pourtant les gens sont complètement ignorants de cette image prophétique
 dans son ensemble ; ils sont aveugles et continuent leurs vies
 comme s'ils allaient être ici pour toujours.
 À moins que l'Esprit-Saint ne nous révèle la seconde venue,
 nous ne l'aurions jamais su.


\section{La vraie source de v\'erit\'e}

Lorsque Jean a dit~:
 \og Vous n'avez pas besoin qu'on vous enseigne ;
 mais [\dots{}] son onction vous enseigne toutes choses \fg{},
 cela veut-il dire que ça ne sert à rien d'aller à l'église,
 mais qu'il vaut mieux rester à la maison et lire ma Bible,
 en permettant à l'Esprit-Saint de m'y enseigner ? Non.
 Paul nous dit que l'Esprit-Saint a placé des pasteurs
 et des enseignants dans l'église pour la perfection des saints
 et pour le travail du ministère.
 Mais la vérité reste que seul l'Esprit-Saint peut \emph{vous} enseigner,
 et si vous recevez une quelconque vérité de Dieu,
 vous la recevez uniquement parce que l'Esprit-Saint
 l'a rendue vraie et vous l'a donnée, en ouvrant votre cœur
 pour la comprendre.
 Il est possible que je vous apporte la vérité de Dieu
 et que d'un seul coup vous disiez~: \og Je vois!
 Oh, merci Chuck! \fg{} Non, ne me remerciez pas,
 car vous n'auriez jamais vu si l'Esprit-Saint
 ne l'avait vous l'avait révélé.


Une autre personne lisant la même page ne l'a pas vu.
 Elle est encore dans les mêmes ténèbres où elle a toujours été.
 Elle ne la comprend pas du tout ; c'est passé au-dessus de sa tête.
 \og Bon, pourquoi l'ai-je vu et pas elle ? \fg{}
 Parce que l'Esprit-Saint vous a enseigné la vérité ;
 vous étiez prêt à la recevoir.
 C'était le bon moment pour vous de savoir, et l'Esprit-Saint
 a ouvert votre cœur et vous a enseigné la vérité.
 Peut-être que j'ai déclaré la vérité, mais vous ne pouvez pas la recevoir
 ou la comprendre à moins que l'Esprit-Saint ne vous la donne.
 Votre compréhension des choses spirituelles peut seulement se faire
 par l'Esprit-Saint.


\section{Deux types d'enseignants}

Il s'ensuit également que vous ne pouvez pas enseigner
 la vérité de Dieu autrement que par l'Esprit-Saint,
 car comment pouvez-vous enseigner ce que vous ne comprenez pas?
 Cela signifie que, bien qu'un homme puisse connaître le grec et l'hébreu
 couramment, et bien qu'il puisse avoir mémorisé l'Ancien
 et le Nouveau Testaments dans leurs langues d'origine,
 et bien qu'il puisse connaître tous les commentaires,
 s'il n'est pas rempli de l'Esprit-Saint,
 il ne peut pas être un véritable guide pour vous
 au sujet des choses de Dieu.
 Vous feriez mieux d'écouter un jeune homme sans diplôme
 du séminaire qui est rempli de l'Esprit.
 Une personne sans éducation mais qui est un serviteur de Dieu
 rempli de l'Esprit-Saint est un guide plus véritable
 dans les Écritures qu'un docteur qui n'est pas né de nouveau,
 car personne ne peut comprendre les choses de l'Esprit
 à moins que l'Esprit ne les lui enseigne,
 et personne ne peut vraiment enseigner les choses de l'Esprit
 à moins que l'Esprit ne l'oigne pour enseigner.

Connaître le grec est un grand bénéfice pour la compréhension
 du Nouveau Testament, mais le fait est que c'est l'œuvre
 de \emph{l'Esprit-Saint} de nous enseigner les choses de Dieu,
 de nous les rappeler, de rendre les choses de Dieu réelles
 dans notre vie, pour ouvrir notre compréhension
 à ces choses spirituelles et à l'œuvre que Dieu a opérée pour nous.
 C'est l'Esprit-Saint qui nous rend conscients des choses
 qui se réalisent dans le monde autour de nous et nous alerte
 au sujet du jour et de l'heure dans laquelle nous vivons.
 Remerciez Dieu pour l'Esprit-Saint dans nos vies,
 qui rend la Parole de Dieu vivante et nous rend bénéficiaires
 de ces choses que Dieu nous a données gratuitement!
\closechapter

