\chapter[L'\oe{}uvre de l'Esprit-Saint dans le monde]{L'\oe{}uvre de l'Esprit-Saint\\ dans le monde}

\lettrine{L}{'Esprit-Saint} a une œuvre importante
 à accomplir dans le monde.
 Nous avons regardé l'œuvre de l'Esprit-Saint dans la vie du croyant~:
 nous conformer à l'image de Christ, nous ouvrir aux choses de Dieu
 et nous amener à l'amour agapé.

Mais quelle est l'œuvre de l'Esprit-Saint dans la monde ?
 Jésus a dit dans \ibibleverse{Jn}(16:8-9)~:
 \og Il convaincra le monde de péché, de justice et de jugement. \fg{}
 Puis Jésus a continué en disant~:
 \og De péché, parce qu'ils ne croient pas en moi. \fg{}
 Il y a un péché ultime dont l'homme devra répondre devant Dieu
 \ocadr le péché de n'avoir pas cru en Jésus-Christ.
 Lorsque Christ est mort, Il est mort pour les péchés du monde~:
 \og Nous étions tous errants comme des brebis,
 chacun suivait sa propre voie ;
 et l'Éternel a fait retomber sur lui la faute de nous tous \fg{}
 (\ibibleverse{Is}(53:6)).
 Christ a pris sur Lui les péchés de toute l'humanité.
 Il est mort pour les péchés du monde, afin qu'il n'y ait plus qu'un péché
 ultime qui vous condamne devant Dieu, et c'est le péché qui consiste
 à rejeter le plan de salut de Dieu pour vous par Jésus-Christ.
 L'homme était déjà condamné avant que Jésus ne vienne.
 Jésus a dit~: \og Je ne suis pas venu pour juger le monde ;
 je suis venu pour que le monde soit sauvé.
 Celui qui croit n'est pas condamné, mais celui qui ne croit pas
 est déjà condamné, car \emph{il n'a pas cru dans le Fils
 unique de Dieu}. \fg{}
 Voilà le péché ultime qui va condamner un homme
 \ocadr ne pas croire en Jésus.


\section*{Le plus grand péché}


Le péché impardonnable se résume simplement à ceci~:
 le rejet continu de Jésus-Christ comme notre Sauveur,
 qui est en fait un blasphème contre l'Esprit-Saint
 car l'Esprit-Saint est venu pour vous convaincre de péché.
 Il \og convainc le monde de péché \fg{} car celui-ci
 ne croit pas en Jésus.
 Alors qu'Il nous convainc de péché, nous montrant que Christ
 est notre seul espoir de salut, si nous rejetons continuellement
 le message que l'Esprit-Saint adresse à notre cœur,
 c'est cela le blasphème contre l'Esprit-Saint.
 Si nous persévérons dans ce rejet, il n'y a pas de pardon possible,
 ni dans ce monde, ni dans le monde à venir.

Il n'existe qu'un seul problème~: notre relation à Jésus-Christ.
 Il y a des personnes qui pensent ne pas être trop mauvaises.
 Ce sont des personnes bonnes, morales ; dans les grandes lignes,
 elles sont honnêtes et ont toujours été des personnes fidèles
 dans leur famille. Elles n'ont jamais commis aucun crime majeur,
 et lorsqu'elles se regardent dans le miroir, elles se disent~:
 \og  J'ai mes chances, je suis aussi bon que n'importe qui. \fg{}
 Mais en réalité, si ces personnes n'ont pas reçu Jésus comme leur Sauveur,
 elles sont condamnées ; elles sont coupables du pire péché
 \ocadr le rejet du plan de salut de Dieu et de Son amour infini.
 C'est pourquoi l'Esprit-Saint est dans le monde aujourd'hui,
 afin de convaincre le monde de péché,
 \og car, a dit Jésus, ils ne croient pas en moi. \fg{}

Croyez-vous en Jésus-Christ comme votre Sauveur ?
 Lui avez-vous confié votre vie ? Si vous croyez en Lui et vous vous êtes
 confié en Lui, vous êtes sauvé.
 Si vous ne croyez pas en Lui et que vous ne vous êtes pas confié en Lui,
 vous êtes condamné, et l'Esprit-Saint vous convaincra de péché
 à cause de votre manque de foi en Lui.

\section*{Complètement pur}

La deuxième chose dont l'Esprit-Saint convainc le monde est la justice.
 Jésus a fait une remarque très intéressante à ce sujet. Il a dit~:
 \og de justice, parce que je vais vers le Père,
 et que vous ne me verrez plus \fg{} (\ibibleverse{Jn}(16:10)).
 Après être apparu à Ses disciples à la suite de Sa résurrection
 pendant environ 40 jours, Jésus les a conduit jusqu'à la ville distante
 de Béthanie et leur a dit qu'ils devraient retourner à Jérusalem
 et attendre là-bas la promesse de Dieu.
 Ils recevraient alors la puissance lorsque l'Esprit-Saint viendrait sur eux,
 et ils seraient les témoins de Christ à travers le monde.
 Puis Jésus est monté aux cieux, et un nuage L'a enlevé à leur vue.
 \ibibleverse{Ac}(1:10-11) nous dit que deux hommes se tenaient à leurs côtés
 vêtus de blanc, et dirent aux disciples~:
 \og Vous Galiléens, pourquoi vous arrêtez-vous à regarder au ciel ?
 Ce Jésus, qui a été enlevé au ciel du milieu de vous,
 reviendra de la même manière dont vous l'avez vu aller au ciel. \fg{}

Que nous enseigne l'ascension du Christ au ciel ?
 Que nous disait Dieu par cette ascension ?
 Dieu rendait témoignage que c'est le standard de justice qu'Il va admettre.
 Jésus a vécu le type de vie juste que le Père va accepter au paradis.
 Qu'est-ce que cela vous enseigne si vous voulez aller au ciel par votre propre
 justice ou vos propres bonnes œuvres ?
 Le seul moyen d'aller au ciel est d'être aussi juste, aussi pur,
 aussi saint que Jésus-Christ.
 Si vous arrivez n'importe où en-dessous de ce standard de justice,
 vous ne pouvez pas y entrer. Quand vous le voyez sous cet angle,
 il est possible que vous laissiez tomber, car même si vous êtes une très bonne
 personne, et bien que vous puissiez être la meilleure personne de votre foyer
 et de votre voisinage, à moins que votre justice ne soit aussi parfaite
 que celle de Jésus-Christ, vous ne pourrez pas entrer au ciel.
 Voilà le standard de justice que Dieu va accepter.
 L'Esprit-Saint convainc le monde de justice, a dit Jésus,
 \og parce que je vais vers le Père, et que vous ne me verrez plus. \fg{}


\section*{Satan a été vaincu}

Enfin, Jésus a parlé de l'Esprit-Saint comme convainquant le monde de jugement.
 L'Esprit ne convainc pas le monde du jugement qui vient ;
 ce n'est pas ce que Jésus a dit. Il a dit dans \ibibleverse{Jn}(16:11)~:
 \og De jugement, parce que le prince de ce monde est jugé. \fg{}
 Il ne parlait pas du jugement futur auquel l'homme devra faire face
 lorsque la mort et l'enfer délivrerons leurs morts afin qu'ils se tiennent
 devant le Grand Trône Blanc du jugement de Dieu.
 L'Esprit-Saint ne nous parle pas de cela. Il nous parle du jugement
 \og parce que le prince de ce monde est jugé. \fg{}
 Qu'est-ce que Jésus veut dire par là ?

Lorsque nous avons donné notre cœur à Jésus-Christ et tourné notre vie
 vers Lui, cela n'a pas été la fin de notre problème avec le péché.
 Nous avons continué à avoir des problèmes à cause de la chair.
 Et Satan, connaissant nos faiblesses, était là pour exploiter ces problèmes
 au maximum. Mais ce dont l'Esprit-Saint nous rend témoignage,
 c'est que le prince de ce monde a été jugé.
 Quand Jésus est allé à la Croix, qu'Il y a porté nos péchés et est mort
 à notre place, le prince de ce monde était jugé,
 afin que par notre relation à Christ, nous puissions avoir la puissance
 sur le péché ; il n'a plus besoin de régner sur nos corps.

Paul a dit que le péché ne devrait plus régner en roi sur nos corps mortels,
 mais que Christ devrait régner maintenant.
 Dans le deuxième chapitre de l'Épître aux Colossiens,
 Paul parle de la victoire de Jésus sur les principautés et les pouvoirs
 des ténèbres qui ont été châtiés par la Croix. Il dit~:
 \og Il a effacé l'acte rédigé contre nous et dont les dispositions
 nous étaient contraires ; il l'a supprimé, en le clouant à la Croix ;
 il a dépouillé les principautés et les pouvoirs
 [la hiérarchie des esprits impurs],
 et les a publiquement livrés en spectacle, en triomphant d'eux
 par la Croix.  \fg{} \ibiblephantom{Col}(2:14-15)

Le prince de ce monde a été vaincu à la Croix.
 Son pouvoir de contrôler votre vie et de vous forcer à faire ces choses
 contraires à la justice de Dieu et aux voies de Dieu lui a été retiré,
 afin que vous puissiez entrer dans la victoire de Jésus-Christ.
 Désormais, vous pouvez être libres du péché et avoir le pouvoir
 sur le péché dans votre vie.
 \og De jugement, parce que le prince de ce monde a été jugé \fg{}
 signifie que le Seigneur a fait en sorte que vous soyez libre de tout péché
 qui embourbait votre vie. Paul dit~:
 \og La vieille nature a été crucifiée avec Christ. \fg{}
 Le prince de ce monde, Satan, a été jugé à la Croix.
 Et par notre identification avec Jésus-Christ dans notre nouvelle vie
 \ocadr par le pouvoir de Son Esprit \fcadr{}
 nous pouvons être libres du péché.


\section*{Et si je pèche ?}

Qu'est-ce que cela signifie si je pèche malgré tout ?
 Cela signifie que je n'ai pas profité de ce que Dieu a rendu disponible
 pour moi par le pouvoir du Christ ressuscité et de l'Esprit-Saint.
 Dieu m'a donné tout ce qui est nécessaire pour vivre la vie
 qu'Il veut que je vive, mais il est important que je profite
 de ce que Dieu a fait pour moi \ocadr que j'exerce le pouvoir
 qu'Il m'a donné.
 Cela ne signifie pas que j'ai une perfection exempte de péché ;
 cela ne signifie pas que je ne vais plus jamais pécher.
 Mais cela signifie que si je pèche, je ne peux pas en accuser Dieu
 en disant~:
 \og Bon, c'est comme ça que Dieu m'a fait. \fg{}
 Tout ce que je peux accuser, c'est mon propre échec à mettre en pratique
 le pouvoir de l'Esprit et la victoire de Christ.
 Je suis conscient que dans certaines de vos vies,
 Satan a assis une position importante.
 Certains d'entre vous sont liés par des habitudes qui vous ont tenus
 en échec spirituel durant toute votre vie chrétienne.
 Vous n'avez jamais été capables de vraiment profiter de la joie complète
 en Christ, car il y a eu ce péché ou cet échec qui vous a hanté
 tout au long du chemin. Je sais que certains d'entre vous ont prié
 à ce sujet pendant des mois ; vous avez crié vers Dieu.
 Vous avez cherché l'aide de Dieu par la prière assidue,
 mais vous trouvez que votre chair est si faible.
 Vous vous retrouvez à retomber dans les vieilles habitudes,
 dans le vieux chemin, presque jusqu'à en désespérer.
 Vous êtes découragés, et Satan commence à vous mentir~:
 \og Il n'y a pas de porte de sortie. Tu n'y arriveras jamais.
 Tu ferais mieux de laisser tomber. \fg{}


\section*{Vous pouvez vaincre}

Pourtant, vous pouvez y arriver, car Dieu vous a rendu la victoire
 sur le péché possible par la puissance de l'Esprit-Saint.
 Satan a été jugé à la Croix, donc tout pouvoir que Satan exerce
 dans votre vie est un pouvoir usurpé ; il n'a aucune autorité et aucun droit.
 Mais il est très présomptueux et sans-gêne. Il s'invite là où il n'a aucun
 droit. Il prend ce qu'il peut, par quelque méthode qu'il puisse utiliser,
 bien qu'il n'ait aucun droit légal d'être là puisque vous avez été racheté
 par Jésus-Christ ; vous êtes désormais Sa possession.
 Satan a été jugé à la Croix, et tout pouvoir et toute autorité qu'il cherche
 à exercer dans votre vie est mensonger.
 Lorsque vous lui opposez le nom de Jésus-Christ et le pouvoir
 dans la victoire de Jésus-Christ, Satan doit se rendre.
 Puisqu'il n'a aucune autorité ou droit d'être là, il doit se soumettre.
 Il a été vaincu à la Croix, et il doit se soumettre à la victoire de Christ
 dès le moment où vous vous saisissez de cette victoire qui est aussi la vôtre.
 Puisque Satan a été jugé, il n'a absolument aucune autorité réelle
 dans votre vie. Si vous voulez exiger qu'il s'en aille,
 il n'a aucun moyen de rester.


\section*{Que le vrai roi règne}

Dieu a dit au prophète Samuel de se rendre à la maison d'Isaï et d'oindre
 l'un des fils d'Isaï pour qu'il soit roi sur Israël,
 car Dieu avait rejeté le roi Saül du pouvoir.
 Craignant Saül, Samuel s'est rendu en secret à la maison d'Isaï
 et a demandé à Isaï de faire entrer ses fils.
 Le premier est entré \ocadr d'une belle allure, d'une haute taille \fcadr
 et Samuel s'est dit~: \og C'est sûrement celui-ci ! \fg{}
 Mais le Seigneur a dit~: \og Non, l'homme regarde l'apparence extérieure,
 mais Je regarde au cœur. Ce n'est pas celui-là. \fg{}
 Alors, un par un, Isaï a fait passer ses fils, et le Seigneur a rejeté
 chacun d'eux, jusqu'à ce que finalement Samuel demande~:
 \og Tu n'en as pas d'autres ? \fg{}
 Isaï a dit~: \og Il y en a juste un autre, mais ce n'est qu'un enfant
 et il est dehors à garder les moutons. \fg{}
 Samuel a dit~: \og Appelle-le. \fg{}

Pendant que David accourait en laissant ses moutons,
 Dieu a parlé à Samuel et lui a dit~: \og C'est celui-ci. \fg{}
 David s'est tenu là et Samuel a pris une jarre d'huile
 et la lui a versée sur la tête ; l'huile a coulé sur David
 tandis qu'il se tenait là, et l'onction de Dieu est venue sur Lui
 \ocadr l'onction de Dieu pour régner sur le peuple de Dieu.
 Mais il est intéressant qu'alors même que Dieu oignait David comme roi
 sur Israël et que le trône appartenait à David,
 Saül était toujours assis dessus. Pendant les quelques années qui ont suivi,
 Saül a fait tout son possible pour se débarrasser de David,
 chassant David dans les montagnes comme une perdrix,
 jusqu'à ce que David en vienne à désespérer de sa propre vie.
 Saül a fait de son mieux pour s'accrocher par la force à ce qui
 ne lui appartenait plus. Mais à cause de l'élu de Dieu,
 Saül a fini par tomber sur son épée sur le mont Guilboa,
 et le trône qui avait appartenu à David depuis un certain temps
 est devenu celui de David, et il s'est assis sur le trône et a régné.

Dieu a ordonné que Satan ne règne plus dans votre vie.
 Quand vous avez donné votre vie à Jésus-Christ, vous êtes devenus
 Sa propriété, et Dieu veut régner dans votre vie aujourd'hui.
 Le prince de ce monde a été jugé. Il a essayé de s'accrocher à certains
 d'entre vous pendant longtemps ; bien que vous ayez soumis votre vie
 à Christ, Satan est toujours là. Il est temps que vous vous appropriiez
 l'autorité que Dieu vous a donnée, et que par le pouvoir de l'Esprit-Saint,
 vous demandiez que Satan vous laisse tranquille.
 La Bible nous promet que si nous résistons au diable,
 il va s'enfuir de nous.


\section*{Priez spécifiquement}

Satan ne se contente pas de venir et de prendre ce qui ne lui appartient pas,
 il essaie également de s'accrocher à ce qui ne lui appartient plus.
 Il est très têtu ; il n'abandonne pas facilement ;
 c'est pourquoi nos prières doivent être spécifiques.
 Je crois que Satan apprécie vraiment les prières généralisées du croyant.
 Elles ne font même pas mal. \og Dieu, sauve le monde. \fg{}
 C'est si général que ça ne conduit à aucun accomplissement.
 Vous devez être spécifique. \og Seigneur, je réclame la victoire du Christ
 sur cette partie de ma vie. Seigneur, je Te dédie cette partie ;
 je veux que Christ vienne et s'asseye sur le trône.
 Merci, Seigneur, désarme-le ; entre et assieds-Toi sur le trône,
 et règne sur ma vie. \fg{} Soyez spécifique.


\section*{Prenez et tenez ferme !}

Mais une fois que Satan est chassé, il contre-attaque et essaie de reprendre
 le territoire d'où il a été chassé.
 Jésus a dit que lorsqu'un esprit impur s'échappe d'un homme,
 il traverse des lieux arides à la recherche d'un lieu où habiter,
 et, n'en trouvant pas, il revient.
 \ibiblephantom{Mt}(12:43-44)
 Je trouve qu'il y a toujours cette contre-attaque,
 cette tentative pour rétablir la mainmise.
 Donc ce qui est pris dans le nom de Jésus-Christ doit être tenu
 dans le nom de Jésus-Christ.

Bien des fois, une personne qui a une victoire initiale dit~:
 \og Oh, gloire à Dieu ! Le Seigneur m'a donné la victoire ! \fg{}
 Et il baisse la garde. Il pense~: \og Oh, je l'ai obtenue.
 Je n'ai plus à me soucier de cela. \fg{} Alors Satan revient.
 Il a été chassé par la porte d'entrée, donc il fait le tour
 pour rentrer par la porte de derrière.
 Il se glisse à l'intérieur pendant que vous êtes dans le salon
 en train de crier victoire ; il revient en passant par la cuisine !
 Ce que nous prenons, nous devons le tenir par la puissance de l'Esprit-Saint.

\closechapter


