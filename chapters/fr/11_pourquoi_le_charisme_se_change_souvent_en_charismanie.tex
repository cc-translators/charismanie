\chapter[Pourquoi les charismes se changent souvent en charismanie]{Pourquoi les charismes\\ se changent souvent en charismanie}
% Reset running title
\renewcommand{\chaphead}{\textls[0]{Pourquoi les charismes se changent souvent en charismanie}}

\lettrine{D}{ans \ibibleverse{Ep}(4:),}
 Paul nous dit que Dieu a placé dans l'Église
 des hommes munis de dons, tels que des pasteurs enseignants,
 pour parfaire les saints pour l'œuvre du ministère et édifier le corps
 du Christ. Le résultat final d'un enseignement sain est d'amener
 les croyants dans un état de maturité complète afin qu'ils ne soient
 pas ébranlés par tout vent de doctrine.

\index{expérience|(}
Une des grandes faiblesses du mouvement charismatique est son manque
 d'enseignement sain de la Bible. Il semble y avoir une préoccupation
 indûe pour l'expérience, qui est souvent placée au-dessus de la Parole.
 En conséquence, les charismatiques sont devenus un champ fertile
 pour toutes sortes de doctrines étranges et non scripturaires qui prolifèrent
 dans leurs rangs.

Il est de la plus grande importance que nous laissions la Bible être
 l'autorité ultime de notre foi et de notre pratique.
 Chaque fois que nous commençons à laisser des expériences
 devenir le critère de pertinence d'une doctrine ou d'une croyance,
 nous perdons notre autorité biblique, et le résultat inévitable
 est la confusion. Il y a tant de gens aujourd'hui qui sont témoins
 d'expériences remarquables et excitantes. Les Mormons, par exemple,
 \Og sont témoins \Fg{} de l'expé\-rience de la vérité du livre de Mormon.
 Ils encouragent les gens à prier pour faire l'expérience de la véracité
 ou non du livre de Mormon. Une personne dit avoir fait l'expérience
 de sa véracité, et une autre de sa fausseté. Qui dois-je croire ?
 Chacun jure qu'il a eu une~expérience véritable avec Dieu ;
 pourtant, il faut bien que l'un des deux ait tort.
 Dès lors que vous ouvrez la porte à l'expérience comme fondement
 ou critère pour l'établissement de la vérité doctrinale, vous ouvrez une boîte de Pandore.
 Le résultat est que la vérité est perdue dans les expériences
 conflictuelles, et la conséquence inévitable est la confusion totale.
 Nous savons que Dieu n'est pas l'auteur de la confusion.


\section{Frapp\'es par l'Esprit ?}

Une des expériences qui est assez courante dans les milieux charismatiques
 est la pratique d'être \Og frappé par l'Esprit. \Fg{}
 Je n'ai jamais compris la valeur supposée de cette expérience.
 Et pourtant, elle est assez commune parmi les charismatiques.
 Quand on leur demande de citer une base scripturaire,
 ils mentionnent généralement les soldats qui étaient venus arrêter
 Jésus dans le jardin. Quand Jésus leur a demandé\frcolon{}
 \Og Qui cherchez-vous ? \Fg{}, ils ont rétorqué\frcolon{}
 \Og Jésus de Nazareth. \Fg{} Il leur a répondu\frcolon{}
 \Og C'est moi \Fg{}, et ils sont tombés à la renverse.
 Mais notez que c'étaient des non croyants, et non des membres
 du corps du Christ remplis de l'Esprit (il n'y a aucune indication
 qu'ils se soient jamais converti). Ce n'est certainement pas une base
 scripturaire pour la pratique parmi les croyants d'aujourd'hui.

Les charismatiques font souvent référence à l'apôtre Paul sur la route
 de Damas. Là encore, Paul à l'époque était un ennemi de Christ.
 Il n'y avait aucun pasteur évangélique lui imposant les mains,
 et on ne lit pas que cette expérience s'est répétée après sa conversion.
 Paul a également eu une rencontre personnelle capitale avec Jésus-Christ
 car le Seigneur lui a parlé distinctement au cours de cette expérience.

Quand j'étais jeune, j'ai assisté à beaucoup de services où des personnes
 étaient soi-disant frappées par l'Esprit.
 J'avais souvent des mains imposées sur moi ; assez souvent,
 il y avait une pression discrète exercée sur mon front,
 me poussant vers l'arrière. Avec certains des évangélistes,
 ce n'était pas si discret. Si vous vous tenez debout, les yeux fermés,
 les mains tendues et votre tête penchée vers l'arrière,
 il n'y a pas besoin de pousser bien fort pour vous faire tomber
 à la renverse, surtout si vous savez que quelqu'un se tient derrière
 vous pour vous rattraper!
 \nowidow[5]


\section{Exorciser les d\'emons ?}

Un autre passe-temps au sein de nombreux groupes charismatiques
 est le discernement et l'exorcisation des démons les uns des autres.
 De nombreux livres et articles ont été écrits à ce sujet par leurs chefs de file,
 et une doctrine entière a été développée sur la base d'expériences
 uniquement. On rapporte qu'un des évangélistes qui était considéré
 comme particulièrement doué dans ce ministère a commencé à faire passer
 des mouchoirs en papier pendant le service pour que les gens puissent
 régurgiter les démons dans les mouchoirs ! Si, au cours d'une réunion,
 l'une des personnes du groupe se mettait à bailler, c'était un signe
 qu'elle était possédée par un démon de léthargie. Roter provoquait
 l'exorcisme du démon de gloutonnerie, qui vous envahissait dès l'instant
 que vous mangiez une bouchée de plus que ce dont vous aviez besoin.
 Cette doctrine pernicieuse a causé de grands dommages auprès de personnes
 sensibles, et beaucoup de victimes tragiques de ses conséquences
 sont disséminées dans le monde aujourd'hui.

Dans l'un des livres que j'ai lu à ce sujet, l'auteur parlait
 de la manière dont nous devions expédier ces démons dans la fosse
 quand nous les chassions. Et comment savait-il que nous avons la puissance
 de les envoyer dans la fosse ? Pendant qu'il était en conversation
 avec un démon, avant de le chasser, le démon l'a supplié de ne pas
 l'envoyer dans la fosse. Il a alors demandé au démon s'il en avait
 l'autorité, et le démon lui a répondu que oui.
 Il a donc déclaré, sur~l'autorité de ce que le démon lui avait dit,
 qu'il pouvait commander aux démons d'aller dans la fosse.
 Si Satan \index{Satan} est le père de tous les mensonges,
 comment pouvait-il faire
 confiance à la parole de l'un de ses émissaires ?
 Ici, une doctrine a été basée sur la parole supposée d'un démon.


\section{Doctrine biblique ou doctrine d\'emoniaque ?}

Paul a mis en garde contre les doctrines démoniaques des derniers jours.
 La totalité de cette doctrine et de cette pratique a été développée intégralement
 sur la base d'expériences, sans fondement scripturaire solide.
 Beaucoup de gens m'ont témoigné la grande victoire dont ils ont fait
 l'expérience après avoir été délivrés d'un quelconque démon.
 Devrions-nous donc croire que nous pouvons avoir la victoire
 sur notre vie charnelle en faisant chasser le démon de luxure ?
 La Bible enseigne-t-elle que je peux, en tant qu'enfant~de Dieu,
 être possédé par un démon, et y a-t-il des exemples dans le Nouveau
 Testament où, au cours de rassemblements d'église, ils chassaient
 les démons les uns des autres ? Bien au contraire, il y a des passages
 qui indiquent qu'un enfant de Dieu ne \emph{peut pas} être possédé par des démons.
 \index{expérience|)}

Paul, écrivant aux Corinthiens, a dit que nos corps sont les temples
 de l'Esprit-Saint qui est en nous (\ibibleverse{ICo}(6:19)).
 Il a également demandé quelle communion il y a entre la lumière
 et les ténèbres, et quel accord Christ a avec Belial,
 et quel contrat d'alliance le temple de Dieu a avec les idoles
 (\ibibleverse{IICo}(6:14-16)). Dans \ibibleverse{ICo}(10:20),
 il identifie les idoles avec des démons, et au verset~\ibiblevs{ICo}(10:21),
 il déclare\frcolon{} \Og Vous ne pouvez boire la coupe du Seigneur
 et la coupe des démons. \Fg{} Face à ces Écritures,
 beaucoup de charismatiques impliqués dans ces pratiques
 ont développé la doctrine selon laquelle les démons pourraient envahir
 la raison du croyant mais pas son esprit.
 Ce concept n'a également aucun fondement scripturaire.
 Il n'est fait aucune référence dans les Écritures à un croyant né de nouveau
 en Jésus-Christ qui se soit fait exorciser d'un démon.

Que des démons puissent posséder les corps de non croyants
 et le font effectivement est un fait accepté des Écritures,
 et qu'ils puissent être exorcisés par l'autorité du nom de Jésus
 est également évident. Mais croire qu'un enfant de Dieu peut être libéré
 des problèmes de la chair (tels que la luxure, la colère ou l'envie)
 par l'exorcisme relève de la charismanie.


\section{\'Ecrit ou oral ?}

Beaucoup de charismatiques semblent préférer les paroles orales
 à la Parole écrite, et cherchent à démontrer la puissance
 du \emph{rhéma} sur le \emph{logos}.
 Le ministère de prophétie ou d'exhortation
 est préféré à celui d'enseignement. L'onction de l'Esprit est préférée
 à celle de l'enseignant. L'onction de l'Esprit n'est reconnue
 pas tant par la vérité qui en découle que par la ferveur
 et l'excitation dont fait preuve l'orateur. Si la voix est forte
 et haut perchée, et le flot de parole très forcé et rapide,
 c'est le signe d'une onction véritable, surtout s'il inspire beaucoup
 d'air entre les phrases et glisse des amens et des alléluias
 entre les idées ! Certains des évangélistes les plus engagés
 ont développé de grandes compétences pour amener les gens
 à un état d'excitation proche de l'hystérie en répétant simplement
 une phrase, comme\frcolon{} \Og Louez le Seigneur \Fg{} et en utilisant différentes
 intonations.

À cause de cette préférence pour la parole orale, les langues
 et leurs interprétations, ou les manifestations prophétiques
 sont davantage désirées que les sermons ou l'enseignement des Écritures.
 Dans beaucoup d'assemblées charismatiques, s'il n'y a eu aucune
 manifestation de ces dons bruyants, les gens ne reconnaissent ni
 n'admettent le mouvement de l'Esprit au cours de la réunion.
 J'ai souvent entendu des gens dire que l'Esprit se mouvait d'une manière
 si puissante durant un service que l'orateur n'avait même pas eu
 l'opportunité de parler. Cette idée est utilisée pour décrire le
 \emph{nec plus ultra} du mouvement de l'Esprit de Dieu.


\section{Spirituel ou de nature humaine ?}

\index{expérience|)}
Une grande partie de l'expérience d'adoration de l'homme relève plus
 de la nature humaine que du spirituel.
 Une liturgie d'église typique fait grandement appel à la nature de l'homme.
 Les costumes fleuris, les mélopées des chœurs, les bougies et l'encens
 \ocadr tout me transporte dans une expérience psychologique agréable
 de profond respect. De l'autre côté du spectre, la libération incontrôlée
 des émotions, avec des cris, des applaudissements et des dances,
 transporte d'autres personnes vers d'autres expériences psychologiques.
 Il est possible qu'aucune de ces expériences ne touche vraiment mon esprit.
 Dans \ibibleverse{He}(4:12), nous lisons que la Parole de Dieu
 est plus acérée qu'aucune épée à deux tranchants,
 et est capable de diviser l'âme et l'esprit.
 C'est la Parole de Dieu qui pourvoit aux besoins de l'esprit de l'homme
 et nourrit l'esprit. Donc si le ministre du culte n'a pas l'occasion
 de partager la Parole de Dieu, nous devons légitimement nous demander
 si c'est l'âme de l'homme ou l'Esprit de Dieu qui se mouvait pendant
 le service. Il est triste que des excès non scripturaires soient si
 librement tolérés parmi les charismatiques.
 Souvent, des croyants affamés et sincères qui ressentent un manque
 de puissance dans leurs vies vont aller au service avec un cœur ouvert,
 affamé et rechercher la puissance de Dieu, mais quand ils constatent
 l'absence d'un solide contenu scripturaire et la présence
 de manifestations de la chair d'un mauvais goût, ils tournent le dos
 à toute l'œuvre valide et admirable de l'Esprit de Dieu dont une personne
 peut faire l'expérience dans sa vie.
 \index{expérience|)}


\section{Exalter la chair}

Je sais que rien de bon ne réside dans ma chair.
 Un des plus grands problèmes dans ma démarche spirituelle, c'est ma chair.
 Ma chair veut être reconnue et admirée. La chair va même chercher
 la gloire et l'attention dans une atmosphère spirituelle.
 Ma chair veut faire croire que je suis plus spirituel que je ne le suis
 vraiment, que je prie plus que je ne le fais vraiment.
 Cela me fait me sentir bien quand quelqu'un dit\frcolon{}
 \Og Vous connaissez si bien la Parole ; avez-vous mémorisé la Bible
 en entier ? \Fg{} J'aime les entendre dire\frcolon{}
 \Og Vous êtes un tel homme de prière, \Fg{} même si je sais
 que je ne le suis pas.

Jésus nous a mis en garde dans \ibibleverse{Mt}(6:) de faire attention
 à ne pas accomplir nos actes de justice devant les hommes
 pour être vus des hommes. Ce \Og être vu des hommes \Fg{} est une
 puissante force de motivation, et je dois sans cesse m'en protéger.
 Jésus a ensuite signalé que le désir d'être vu des hommes
 est derrière beaucoup de nos dons, prières, activités spirituelles
 telles que le jeûne. Les Écritures nous disent qu'un jour,
 toutes nos œuvres seront testées à l'épreuve du feu pour déterminer
 de quelle nature elles sont, et leurs motivations sous-jacentes.
 Il est très sage pour nous d'examiner nos motivations,
 car si nous nous jugeons nous-mêmes, nous ne serons pas jugés par Dieu.

Beaucoup des choses faites lors des services charismatiques
 le sont pour attirer l'attention vers l'individu.
 La personne qui crie \Og Alléluia ! \Fg{} et jette ses mains en l'air
 attire l'attention sur elle, et bien souvent distrait ceux qui adorent
 véritablement Dieu. Pendant que le groupe chante des refrains de louange,
 une ou plusieurs personnes vont souvent se lever, leurs yeux fermés
 et leurs mains levées tandis que les autres restent assis.
 Cela paraît très spirituel, tout comme prier au coin d'une rue,
 mais cela attire l'attention sur soi-même, et dès l'instant
 que vous attirez l'attention sur vous-mêmes, vous la retirez à Jésus.

Souvent, les méthodes par lesquelles les offrandes sont reçues
 sont conçues pour honorer la chair, et complètement dépouiller
 la pauvre âme de la récompense de Dieu. J'ai entendu des évangélistes
 annoncer que Dieu leur avait dit que dix personnes allaient donner
 mille dollars ce soir-là, puis pester, tempêter et menacer jusqu'à
 ce que les dix se soient levés. Chaque fois que l'un d'entre eux
 se levait, l'attention lui était donnée, et les applaudissements
 étaient encouragés. Pendant que la foule applaudissait,
 j'étais écœuré et je pensais\frcolon{} \Og Profites-en bien, pauvre âme,
 car ce sera la seule récompense que tu recevras pour ce don. \Fg{}
 Comme l'a dit Jésus\frcolon{} \Og Vous avez votre récompense. \Fg{}
 Je ressentais également de la colère envers le pasteur
 ou l'évangéliste qui encourageait les gens à donner d'une manière
 qui les empêchait de recevoir une récompense de Dieu.
 J'avais aussi l'impression qu'il mentait lorsqu'il déclarait
 que Dieu lui avait dit combien de gens allaient donner mille dollars.
 Tout cela n'est rien de plus qu'un stratagème psychologique.


\section{Des stratag\`emes de mauvais go\^ut}

Sont également de mauvais goût les autres stratagèmes psychologiques
 qui sont utilisés pour solliciter des fonds afin de supporter l'œuvre
 de Dieu. Beaucoup des évangélistes charismatiques ont développé
 des listes de diffusion, et à l'aide de logiciels de traitement
 de texte, envoient des courriers en masse à leurs adeptes crédules,
 dont beaucoup sont dupés en croyant recevoir une lettre personnelle
 d'un cher Frère Untel (le nom d'un quelconque évangéliste guérisseur
 de renom), car l'ordinateur a répété leur nom de nombreuses fois
 dans le corps de la lettre. Les lettres sont truffées de mensonges,
 et disent bien souvent\frcolon{} \Og Le Seigneur t'as mis sur mon cœur
 ce matin afin que je prie particulièrement pour toi.
 Y a-t-il quelque chose qui ne va pas? Écris-moi s'il te plaît
 et fais-moi connaître ton besoin afin que je puisse t'aider. \Fg{}

Dans \ibibleverse{IIP}(2:), on nous dit que l'un des signes
 d'un faux prophète est qu'il utilisera des paroles trompeuses
 dans le but d'ex\-ploi\-ter les gens. Ces lettres correspondent parfaitement
 à la description de Pierre. Elles font le plus souvent appel à la chair.
 Si vous voulez des réponses à vos prières, ou une œuvre spéciale de Dieu,
 alors plantez la graine que vous avez reçue.
 Ces méthodes renient la grâce de Dieu, puisque vous êtes sensés acheter
 la faveur de Dieu. J'ai toujours été intrigué par la manière dont
 ces hommes qui ont appris tous les secrets de la foi
 et reçoivent une telle puissance de Dieu semblent n'avoir jamais assez
 de foi pour faire confiance à Dieu afin qu'il pourvoie
 à leurs \emph{propres} besoins, mais mettent en garde que l'œuvre de Dieu
 va échouer à moins que les gens ne Lui viennent en aide
 immédiatement pour Lui éviter la banqueroute.


\section{Suffit-il de le dire pour le recevoir ?}

\index{prospérité!doctrine de la \textasciitilde{}|(}
Le dernier vent de doctrine pernicieuse et non scripturaire à \linebreak
 souffler dans les rangs de certaines assemblées charismatiques
 est l'en\-sei\-gne\-ment du \Og dites-le-recevez-le \Fg{},
 également connu sous le nom de doctrine de la prospérité.
 Parmi les prétentions généralement avancées, on trouve le fait
 que Dieu ne désire jamais que vous soyez malade et que toute maladie
 est le résultat d'une ignorance ou d'un manque de foi.
 De tels enseignements se rapprochent plus de Mary Baker Eddy
 \NdT{Fondatrice du mouvement de la Science chrétienne.}
 que de l'apôtre Paul !

Ces gens parlent beaucoup de faire des confessions positives,
 et mettent en garde contre les confessions négatives.
 Ils enseignent que la parole prononcée devient une force spirituelle
 pour le bien ou le mal selon la confession.
 Ainsi, vous ne devez jamais confesser\frcolon{} \Og Je ne me sens pas bien \Fg{},
 car c'est une confession négative qui va vous conduire à vous sentir mal.
 Vous êtes donc encouragés à mentir sur votre condition réelle
 ou vos sentiments. Lorsque vous entendez cet enseignement,
 vous pourriez jurer que les sermons ont été extraits
 de \emph{Science et santé avec la clef des écritures}
 \NdT{Livre d'étude de la Science chrétienne.} plutôt que de la Bible.

J'ai entendu de telles personnes chercher à apporter l'explication
 de l'épine dans la chair de Paul en disant\frcolon{}
 \Og Dans quels autres passages de la Bible trouve-t-on des épines ?
 Jésus, disent-ils, a parlé d'épines étouffant la Parole afin
 que les graines ne puissent pas porter de fruit. \Fg{}
 Mais qu'étaient ces épines ? Les soucis du monde,
 la séduction des richesses et l'invasion des autres convoitises.
 \ibiblephantom{Mc}(4:7,19)Par conséquent, l'épine dans la chair de Paul
 était les soucis du monde qu'il avait pris sur lui-même.

Si ces gens s'inquiétaient de faire un tant soit peu de recherches,
 ils auraient découvert qu'il y a deux mots grecs très différents
 qui sont traduits par \Og épines \Fg{} dans ces passages.
 Le mot que Paul a utilisé pour parler de son épine dans la chair
 est un mot grec qui se réfère à un piquet de tente,
 pas à une petite irritation lancinante.
 Paul parlait aux Galates de ses infirmités ;
 le mot français a la même racine que \Og infirmerie \Fg{},
 ou ce que nous appelons un hôpital.

Un de ces leaders charismatiques m'a dit\frcolon{}
 \Og Dans la mesure où cela a été donné à Paul de peur
 qu'il ne soit exagérément exalté, ne croyez-vous pas que si Paul
 avait pu simplement conquérir sa chair, l'épine n'aurait pas
 été nécessaire ? \Fg{}
 Je ne peux pas imaginer l'orgueil spirituel insinué par une telle remarque.
 En substance, il déclarait qu'il avait conquis sa chair
 plus complètement que Paul. Cela ne se voyait certainement pas
 aux vêtements, à la voiture et à la maison tous tape-à-l'œil
 qu'il possédait. Pourtant, il disait que tout ce mode de vie
 fastueux n'était qu'un signe de sa foi, car si Dieu pouvait nous faire
 confiance avec l'argent, Il voulait que nous prospérions tous,
 et toute personne avec assez de foi pouvait vivre comme le fils d'un roi.

Quel message cela transmet-il au sujet de Jésus, qui n'avait pas
 d'endroit où poser Sa tête, et devait envoyer Pierre pêcher
 pour avoir une pièce afin de payer ses impôts ? Je connais beaucoup de gens
 qui sont morts au cours de leurs confessions positives ou~de~leurs guérisons.
 Certains d'entre eux auraient pu être aidés par des soins médicaux
 compétents, mais aller chez le médecin aurait été une confession négative
 et une reconnaissance que quelque chose allait mal.
 Dans d'autres cas, je connais des gens qui ont suivis les mensonges
 des évangélistes de la confession positive, et quand la matérialisation
 de leurs confessions a échoué, ils se sont complètement détournés de Dieu.
 Je sais également que certains des évangélistes qui soutiennent
 le plus cette confession positive comme le moyen d'avoir la santé
 et la prospérité en tout temps ont passé du temps à l'hôpital
 pour dépression nerveuse.

Les gens qui semblent avoir tiré le plus de profit de ces enseignements
 sont les évangélistes eux-mêmes. Comment vont-ils répondre à Dieu
 pour avoir escroqué la pauvre petite veuve de la moitié de son chèque
 de la Sécurité Sociale, l'amenant à sauter plusieurs repas par manque
 de moyens financiers pour qu'ils puissent voler dans leurs jets privés
 vers leurs propriétés luxueuses de Palm Springs et dîner dans les
 restaurants les plus cossus ?

Paul signale à Timothée les enseignements pervers
 d'hom\-mes aux esprits corrompus qui sont dénués de vérité, car ils pensent
 que la piété est un chemin vers la prospérité.
 Paul a prévenu Timothée de rester loin d'eux.
 Ceci est une traduction libre mais juste du texte grec
 de \ibibleverse{ITm}(6:5).
 Paul a alors dit a Timothée que la piété est une grande source de gains,
 si l'on se contente de ce qu'on a.
\index{prospérité!doctrine de la \textasciitilde{}|)}
\closechapter

