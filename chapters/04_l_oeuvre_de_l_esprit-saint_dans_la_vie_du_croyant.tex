\chapter{L'œuvre de l'Esprit-Saint dans la vie du croyant}

Quel est le travail de l'Esprit attendu dans la vie du croyant ?
 Comme nous l'avons déjà noté dans \bibleverse{Jn}(14:), Son nom,
 \og consolateur \fg{}, indique Sa venue à nos côtés pour nous aider.
 Je ne trouve pas que ma démarche de chrétien soit facile.
 Je trouve que ma chair lutte avec moi tout le temps.
 Tout comme Pierre, j'ai moi aussi souvent découvert que l'esprit
 est bien disposé, mais que la chair est faible.
 Je comprend ce dont Paul parlait dans \bibleverse{Ga}(5:) lorsqu'il
 mentionnait la bataille entre la chair et l'esprit.
 Si Dieu a de l'aide pour moi, je suis prêt à la recevoir~;
 je veux toute l'aide que je peux avoir !
 Je ne veux jamais mettre des limites à ce que Dieu veut me donner,
 ou à ce qu'Il veut faire dans ma vie. Je ne veux pas être coupable,
 comme les israélites dans le désert,
 de limiter Dieu (\bibleverse{Ps}(78:41)).
 De la même manière, je ne cherche pas l'expérience
 pour l'expérience en soit~; je désire seulement l'œuvre véritable
 de l'Esprit-Saint, mais je la veux complètement.

\section*{Faire confiance à l'Esprit}

Dans le discours de Jésus à Ses disciples, à partir du chapitre 14
 de l'évangile de Jean, Il cherche à les préparer à Son départ.
 Il parle beaucoup du fait qu'Il va les quitter et retourner auprès du Père.
 Il leur parle également beaucoup des réserves que le Père et Lui
 ont faites pour eux par la puissance de l'Esprit-Saint.
 Il serait là pour les aider. Tout comme ils avaient appris à faire confiance
 à Jésus pour toute situation d'urgence qui pourrait arriver,
 ils doivent maintenant apprendre à faire confiance à l'Esprit-Saint.
 Il sera désormais leur soutien.

Jésus a passé Ses trois années avec Ses disciples à leur enseigner
 la vérité sur Dieu. Maintenant, leur enseignant les quitte pour retourner
 auprès du Père, mais les étudiants ne resteront pas seuls~;
 ce soutien, l'Esprit-Saint, va maintenant leur enseigner toutes choses
 et leur rappeler toutes les choses que Jésus leur a dites
 (\bibleverse{Jn}(14:26)).

Peut-être avez-vous eu l'expérience de partager l'évangile avec quelqu'un,
 et il ou elle vous a posé une question qui vous a immédiatement bloqué~;
 mais alors que vous avez commencé à répondre, les Écritures ont commencé
 à vous venir à l'esprit, et vous étiez content et satisfait de la réponse
 que vous avez donné à cette personne. Ceci est le travail de rappel de l'Esprit.

L'Esprit-Saint nous aide à comprendre les choses de Dieu.
 Bien souvent, j'ai été frustré dans ma tentative d'expliquer
 une vérité spirituelle à un non croyant.
 Cela paraît si clair et évident, et pourtant il ou elle ne semble pas
 pouvoir le comprendre.
 Si vous êtes dans ce type de situations au sujet de l'Esprit de Dieu,
 l'homme naturel \og ne reçoit pas les choses de l'Esprit de Dieu,
 car elles sont une folie pour lui, et il ne peut les connaître, 
 parce que c'est spirituellement qu'on en juge \fg{}
 (\bibleverse{ICo}(2:14)).


\section*{L'Esprit mort et l'Esprit vivant}

Lorsque j'étais à l'université, j'ai eu un professeur de sociologie
 qui croyait au dualisme de l'homme, et je croyais au trialisme de l'homme.
 Bien souvent, nous avons exprimé nos vues différentes l'un à l'autre.
 J'étais frustré qu'il ne puisse pas faire la distinction entre l'âme
 et l'esprit de l'homme, mais croyait qu'ils étaient synonymes.
 Un jour, alors que je quittais la classe frustré
 après une nouvelle discussion dans laquelle il paraissait
 avoir un aveuglement délibéré, ce fut comme si l'Esprit-Saint
 m'avait mis \bibleverse{ICo}(2:14) à l'esprit.
 Alors, j'ai réalisé qu'en tant que personne non régénérée,
 son esprit était mort, donc je lui parlais de mystères qu'il ne pouvait
 pas connaître.
 Il ne savait pas et ne pouvait pas savoir quoi que ce soit sur l'esprit
 de l'homme à moins d'être né de l'Esprit-Saint.
 Dans \bibleverse{ICo}(2:15), Paul dit~:
 \og L'homme spirituel, au contraire, juge de tout,
 et il n'est lui-même jugé par personne. \fg{}

Quiconque vit seulement au niveau de la conscience du corps
 vit au niveau animal de l'existence.
 Son esprit est dirigé et dominé par les besoins de son corps~;
 il ne comprend pas les choses de l'Esprit, car son propre esprit est mort.
 Il n'est pas étonnant qu'il cherche à se rapprocher lui-même
 du reigne animal, car il vit comme un animal,
 une conscience dominée par le corps.
 Lorsqu'une personne est née de nouveau par l'Esprit,
 son propre esprit vient à la vie et, ainsi jointe à Dieu par l'Esprit,
 il est encouragé à travers la Parole à vivre une vie dominée par l'Esprit.
 Lorsqu'il le fait, il commence à avoir une conscience dominée par l'Esprit

