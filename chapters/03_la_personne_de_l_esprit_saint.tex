\chapter{La personne de l'Esprit-Saint}

Puisque nous voulons avoir une rencontre personnelle avec l'Esprit-Saint, nous allons maintenant montrer que les Écritures enseignent que l'Esprit-Saint est une personne, plutôt qu'une essence, une force, ou encore une puissance. Vous pouvez avoir une force brute sans personnalité, comme l'électricité, mais il est difficile d'avoir une relation intime et proche avec une telle puissance impersonnelle.

Le mot \og esprit \fg{} en grec est \emph{pneuma}, qui est du genre neutre. À cause de cela, dans l'église primitive, un brillant théologien du nom d'Arius a commencé à promouvoir l'idée que Jésus était moins que Dieu, puisqu'il avait été créé par Dieu, et que l'Esprit-Saint était seulement une \og essence \fg{} de Dieu. Cela devint connu sous le nom d'hérésie arienne, qui existe toujours et attire une large audience. Le conseil de Nicée a retiré sa position à Arius et frappé ses enseignements d'hérésie. L'Esprit-Saint est plus que simplement une essence ou une force ; Il est une personne. Vous ne devriez pas simplement louer une force ou une essence. Pouvez-vous vous imaginer chantant la doxologie \og Louez le Père, le Fils et l'Essence \fg{} ? Il est une personne, et en tant qu'une des personnes du Dieu Unique, Il est digne d'être loué. Si nous ne croyons pas dans la personnalité du Saint-Esprit, nous nions la louange et l'adoration qui lui sont dues. Si nous ne réalisons pas que le Saint-Esprit est une personne, nous nous trouvons dans une position où nous cherchons à nous situer par rapport à une force ou une essence. Nous dirions : \og J'ai besoin de livrer ma vie à cela \fg{} ou \og J'ai besoin de plus de cela dans ma vie .\fg



