\chapter{La personne de l'Esprit-Saint}

\lettrine[lines=3]{P}{uisque} nous voulons avoir une rencontre personnelle avec l'Esprit-Saint, nous allons maintenant montrer que les Écritures enseignent que l'Esprit-Saint est une personne, plutôt qu'une essence, une force, ou encore une puissance. Vous pouvez avoir une force brute sans personnalité, comme l'électricité, mais il est difficile d'avoir une relation intime et proche avec une telle puissance impersonnelle.

Le mot \og esprit \fg{} en grec est \emph{pneuma}, qui est du genre neutre. À cause de cela, dans l'église primitive, un brillant théologien du nom d'Arius a commencé à promouvoir l'idée que Jésus était moins que Dieu, puisqu'il avait été créé par Dieu, et que l'Esprit-Saint était seulement une \og essence \fg{} de Dieu. Cela devint connu sous le nom d'hérésie arienne, qui existe toujours et attire une large audience. Le conseil de Nicée a retiré sa position à Arius et frappé ses enseignements d'hérésie. L'Esprit-Saint est plus que simplement une essence ou une force ; Il est une personne. Vous ne devriez pas simplement louer une force ou une essence. Pouvez-vous vous imaginer chantant la doxologie \og Louez le Père, le Fils et l'Essence \fg{} ? Il est une personne, et en tant qu'une des personnes du Dieu Unique, Il est digne d'être loué. Si nous ne croyons pas dans la personnalité du Saint-Esprit, nous nions la louange et l'adoration qui lui sont dues. Si nous ne réalisons pas que le Saint-Esprit est une personne, nous nous trouvons dans une position où nous cherchons à nous situer par rapport à une force ou une essence. Nous dirions~: \og J'ai besoin de livrer ma vie à cela \fg{} ou \og J'ai besoin de plus de cela dans ma vie .\fg


\section{Connaître, agir, ressentir}

Le fait qu'Il est une personne est montré clairement dans les Écritures. Des caractéristiques qui lui sont assignées ne peuvent être assignées qu'à des personnes. Une personne est un être avec un esprit, une volonté et des sentiments. Si dans les Écritures ces caractéristiques sont assignées à l'Esprit-Saint, alors nous devons conclure que l'Esprit est une personne. Dans \bibleverse{ICo}(2:10-11), nous lisons~: \og À nous, Dieu nous l'a révélé par l'Esprit. Car l'Esprit sonde tout, même les profondeurs de Dieu. Qui donc, parmi les hommes, sait ce qui concerne l'homme, si ce n'est l'esprit de l'homme qui est en lui ? De même, personne ne connaît ce qui concerne Dieu, si ce n'est l'Esprit de Dieu. \fg{} Ici il est fait référence à l'Esprit possédant une connaissance. Une force brute ou une puissance ne possède pas de connaissance. Il serait absurde de remplacer le mot \og Esprit \fg{} par \og essence \fg{} dans le texte, car vous auriez alors une \og essence \fg{} qui recherche toute chose!

Dans \bibleverse{Rom}(8:27), Paul dit~: \og Et celui qui sonde les cœurs connaît quelle est l'intention de l'Esprit~: c'est selon Dieu qu'il intercède en faveur des saints. \fg{} Ici, il est fait référence à l'esprit \NdT{le mental} de l'Esprit, une caractéristique qui n'est pas associée à une simple essence. Dans \bibleverse{ICo}(12:11), Paul, à propos des dons de l'Esprit, dit~: \og Un seul et même Esprit opère toutes ces choses, les distribuant à chacun en particulier comme il veut. \fg{}. L'Esprit-Saint possède donc une volonté, un trait associé à une personnalité.

Dans \bibleverse{Rom}(15:30), Paul associe l'émotion de l'amour avec l'Esprit. Une force ou une puissance ne peut pas aimer. On n'associe pas l'amour à autre chose qu'une personnalité. Il est intéressant que, bien que j'ai lu ou entendu un grand nombre de sermons sur l'amour de Dieu, et l'amour de Jésus Christ pour nous, je n'ai encore jamais entendu de sermon sur l'amour de l'Esprit-Saint. Et pourtant cela doit être l'une des principales caractéristiques de l'Esprit, puisque c'est le fruit qu'Il produit dans nos vies. L'Esprit-Saint possède réellement des émotions et peut être attristé, car Paul dans \bibleverse{Eph}(4:30) exhorte l'Église à ne pas attrister l'Esprit-Saint de Dieu. Réfléchissez à quel point cela paraîtrait abherrant de dire que vous avez attristé l'essence.


\section{Les pronoms personnels}

À travers les Écritures, des pronoms personnels sont utilisés pour se référer à l'Esprit-Saint. Dans \bibleverse{Jn}(14:16-17), Jésus dit~: \og Je demanderai au Père de vous donner quelqu'un d'autre pour vous venir en aide, afin qu'il soit toujours avec vous~: c'est l'Esprit de vérité. \fg{} Ici, le pronom \og il \fg{} est utilisé pour le Père et pour l'Esprit. Si vous croyez en un Dieu personnel, vous devriez également croire en un Esprit personnel. Dans ce même passage, Jésus a continué en disant que le monde ne pouvait pas recevoir l'Esprit car ils ne Le voyaient ni le Le connaissaient. Jésus a dit que vous Le connaissez, car Il habite en vous.

Remarquez le nombre de fois où Jésus utilise le pronom personnel pour désigner l'Esprit-Saint. Dans \bibleverse{Jn}(16:7-14), Jésus utilise plusieurs fois le pronom personnel pour se référer à l'Esprit-Saint. \og Cependant, je vous dis la vérité~: il est préférable pour vous que je parte ; en effet, si je ne pars pas, celui qui doit vous venir en aide ne viendra pas à vous. Mais si je pars, je vous l'enverrai. Et quand il viendra, il prouvera aux gens de ce monde leur erreur au sujet du péché, de la justice et du jugement de Dieu. Quant au péché, il réside en ceci~: ils ne croient pas en moi ; quant à la justice, elle se révèle en ceci~: je vais auprès du Père et vous ne me verrez plus ; quant au jugement, il consiste en ceci~: le dominateur de ce monde est déjà jugé. J'ai encore beaucoup de choses à vous dire, mais vous ne pourriez pas les supporter maintenant. Quand viendra l'Esprit de vérité, il vous conduira dans toute la vérité. Il ne parlera pas en son propre nom, mais il dira tout ce qu'il aura entendu et vous annoncera ce qui doit arriver. Il révélera ma gloire, car il recevra de ce qui est à moi et vous l'annoncera. \fg{} Dans le texte en grec, le pronom personnel \og il \fg{} est utilisé pour l'Esprit de nombreuses fois dans les Écritures.


