\chapter{La personne de l'Esprit-Saint}

\chlettrine{P}{uisque} nous voulons avoir une rencontre personnelle avec
 l'Esprit-Saint, nous allons maintenant montrer que les Écritures enseignent
 que l'Esprit-Saint est une personne, plutôt qu'une essence, une force, ou
 encore une puissance. Vous pouvez avoir une force brute sans personnalité,
 comme l'électricité, mais il est difficile d'avoir une relation intime et
 proche avec une telle puissance impersonnelle.

Le mot \og esprit \fg{} en grec est \emph{pneuma}, qui est du genre neutre.
 À cause de cela, dans l'église primitive, un brillant théologien du nom
 d'Arius a commencé à promouvoir l'idée que Jésus était moins que Dieu,
 puisqu'il avait été créé par Dieu, et que l'Esprit-Saint était seulement
 une \og essence \fg{} de Dieu. Cela devint connu sous le nom d'hérésie
 arienne, qui existe toujours et attire une large audience. Le conseil de
 Nicée a retiré sa position à Arius et frappé ses enseignements d'hérésie.
 L'Esprit-Saint est plus que simplement une essence ou une force ; Il est
 une personne. Vous ne devriez pas simplement louer une force ou une essence.
 Pouvez-vous vous imaginer chantant la doxologie \og Louez le Père, le Fils
 et l'Essence \fg{} ? Il est une personne, et en tant qu'une des personnes
 du Dieu Unique, Il est digne d'être loué. Si nous ne croyons pas dans la
 personnalité du Saint-Esprit, nous nions la louange et l'adoration qui lui
 sont dues. Si nous ne réalisons pas que le Saint-Esprit est une personne,
 nous nous trouvons dans une position où nous cherchons à nous situer par
 rapport à une force ou une essence. Nous dirions~: \og J'ai besoin de livrer
 ma vie à cela \fg{} ou \og J'ai besoin de plus de cela dans ma vie.\fg


\section*{Connaître, agir, ressentir}

Le fait qu'Il est une personne est montré clairement dans les Écritures.
 Des caractéristiques qui lui sont assignées ne peuvent être assignées
 qu'à des personnes. Une personne est un être avec un esprit, une volonté et
 des sentiments. Si dans les Écritures ces caractéristiques sont assignées à
 l'Esprit-Saint, alors nous devons conclure que l'Esprit est une personne.
 Dans \bibleverse{ICo}(2:10-11), nous lisons~: \og À nous, Dieu nous l'a
 révélé par l'Esprit. Car l'Esprit sonde tout, même les profondeurs de Dieu.
 Qui donc, parmi les hommes, sait ce qui concerne l'homme, si ce n'est
 l'esprit de l'homme qui est en lui ? De même, personne ne connaît ce qui
 concerne Dieu, si ce n'est l'Esprit de Dieu. \fg{} Ici il est fait référence
 à l'Esprit possédant une connaissance. Une force brute ou une puissance ne
 possède pas de connaissance. Il serait absurde de remplacer le mot
 \og Esprit \fg{} par \og essence \fg{} dans le texte, car vous auriez alors
 une \og essence \fg{} qui recherche toute chose!

Dans \bibleverse{Rom}(8:27), Paul dit~: \og Et celui qui sonde les cœurs
 connaît quelle est l'intention de l'Esprit~: c'est selon Dieu qu'il
 intercède en faveur des saints. \fg{} Ici, il est fait référence à l'esprit
 \NdT{le mental} de l'Esprit, une caractéristique qui n'est pas associée à
 une simple essence. Dans \bibleverse{ICo}(12:11), Paul, à propos des dons de
 l'Esprit, dit~: \og Un seul et même Esprit opère toutes ces choses, les
 distribuant à chacun en particulier comme il veut. \fg{} L'Esprit-Saint
 possède donc une volonté, un trait associé à une personnalité.

Dans \bibleverse{Rom}(15:30), Paul associe l'émotion de l'amour avec l'Esprit.
 Une force ou une puissance ne peut pas aimer. On n'associe pas l'amour à autre
 chose qu'une personnalité. Il est intéressant que, bien que j'aie lu ou
 entendu un grand nombre de sermons sur l'amour de Dieu, et l'amour de
 Jésus Christ pour nous, je n'ai encore jamais entendu de sermon sur l'amour de
 l'Esprit-Saint. Et pourtant cela doit être l'une des principales
 caractéristiques de l'Esprit, puisque c'est le fruit qu'Il produit dans nos
 vies. L'Esprit-Saint possède réellement des émotions et peut être attristé,
 car Paul dans \bibleverse{Eph}(4:30) exhorte l'Église à ne pas attrister
 l'Esprit-Saint de Dieu. Réfléchissez à quel point cela paraîtrait abherrant
 de dire que vous avez attristé l'essence.


\section*{Les pronoms personnels}

À travers les Écritures, des pronoms personnels sont utilisés pour se
 référer à l'Esprit-Saint. Dans \bibleverse{Jn}(14:16-17), Jésus dit~:
 \og Je demanderai au Père de vous donner quelqu'un d'autre pour vous
 venir en aide, afin qu'il soit toujours avec vous~:
 c'est l'Esprit de vérité. \fg{}
 Ici, le pronom \og il \fg{} est utilisé pour le Père et pour l'Esprit.
 Si vous croyez en un Dieu personnel, vous devriez également croire en
 un Esprit personnel. Dans ce même passage, Jésus a continué en disant
 que le monde ne pouvait pas recevoir l'Esprit car il ne Le voyait ni
 ne Le connaissait. Jésus a dit que vous Le connaissez, car Il habite
 en vous.

Remarquez le nombre de fois où Jésus utilise le pronom personnel pour
 désigner l'Esprit-Saint. Dans \bibleverse{Jn}(16:7-14), Jésus utilise
 plusieurs fois le pronom personnel pour se référer à l'Esprit-Saint.
 \og Cependant, je vous dis la vérité~: il est préférable pour vous que
 je parte ; en effet, si je ne pars pas, celui qui doit vous venir en aide
 ne viendra pas à vous. Mais si je pars, je vous l'enverrai. Et quand il
 viendra, il prouvera aux gens de ce monde leur erreur au sujet du péché,
 de la justice et du jugement de Dieu. Quant au péché, il réside en ceci~:
 ils ne croient pas en moi ; quant à la justice, elle se révèle en ceci~:
 je vais auprès du Père et vous ne me verrez plus ; quant au jugement,
 il consiste en ceci~: le dominateur de ce monde est déjà jugé.
 J'ai encore beaucoup de choses à vous dire, mais vous ne pourriez pas les
 supporter maintenant. Quand viendra l'Esprit de vérité, il vous conduira
 dans toute la vérité. Il ne parlera pas en son propre nom, mais il dira tout
 ce qu'il aura entendu et vous annoncera ce qui doit arriver.
 Il révélera ma gloire, car il recevra de ce qui est à moi et vous
 l'annoncera.
 \fg{} Dans le texte en grec, le pronom personnel \og il \fg{} est utilisé
 pour l'Esprit de nombreuses fois dans les Écritures.

\section*{L'Esprit en action}

Des actes personnels sont assignés à l'Esprit-Saint dans les Écritures. Dans
 \bibleverse{Acts}(13:2), nous lisons \og Le Saint-Esprit dit~:
 Mettez-moi à part Barnabas et Saul pour l'œuvre à laquelle je les ai
 appelés. \fg{}
 Encore une fois, insérer \og puissance \fg{} ou \og essence \fg{} à la place
 de l'Esprit est incompréhensible. Comment une essence ou une puissance
 peut-elle parler ? Dans \bibleverse{Rom}(8:26), on nous dit que l'Esprit-Saint
 lui-même intercède pour nous par des grognements imprononçables.
 Encore une fois, essayez de concevoir une simple force qui intercèderait !
 Si l'Esprit-Saint n'était qu'une essence ou une force lorsqu'il est mentionné
 dans l'Écriture, vous devriez pouvoir insérer les mots \og force \fg{} ou
 \og essence \fg{} et la signification du texte n'en serait pas altérée pour
 autant. Mais une telle chose est bien évidemment impossible, car
 l'Esprit-Saint est une personne. L'Esprit-Saint témoigne de Jésus Christ dans
 \bibleverse{Jn}(15:26), et Il enseigne les croyants et leur rappelle des choses
 dans \bibleverse{Jn}(14:26). Dans \bibleverse{Acts}(16:2-7), l'Esprit-Saint a
 empêché Paul et ses compagnons d'aller en Asie et ne les a pas laissé aller
 en Bithinie. Dans \bibleverse{Gn}(6:3), nous voyons que l'Esprit-Saint lutte
 avec l'homme.

L'Esprit-Saint peut être traité comme une personne. Il peut être offensé.
 Il est impossible de concevoir d'offenser \og la puissance \fg{} ou
 \og le souffle \fg{}. Votre souffle peut-être offensif, mais vous ne pouvez
 pas offenser votre souffle ! Dans \bibleverse{Eph}(4:30), Paul exhorte~:
 \og N'attristez pas le Saint-Esprit de Dieu \fg{}.
 Il est possible de mentir à l'Esprit-Saint. Voici l'accusation que Pierre a
 formulée à l'encontre d'Ananias~:
 \og Vous avez menti à l'Esprit-Saint. \fg{}
 Il est également possible de blasphémer contre l'Esprit-Saint.
 Jésus a dit que c'était un péché tellement horrible qu'il ne pouvait pas
 être pardonné à la personne qui l'avait commis. Il a dit~:
 \og Vous pouvez blasphémer contre moi et être pardonné, mais pas contre
 l'Esprit-Saint. \fg{}
 Ici, Jésus fait une distinction claire entre Lui-même et l'Esprit-Saint.

L'Esprit-Saint est identifié à des personnes. Paul a dit~:
 \og Cela parût bon au Saint-Esprit et à nous \fg{}
 (\bibleverse{Acts}(15:28)).
 Essayez de remplacer par vent ou puissance dans ce verset et voyez si cela
 a toujours du sens.

L'Esprit-Saint est une personne ; il n'est pas simplement l'essence de Dieu.
 Vous avez besoin d'entrer dans une relation personnelle avec Lui pour
 commencer à faire l'expérience de Son amour et de Sa puissance à l'œuvre
 dans votre vie alors qu'Il vous guide dans votre cheminement spirituel.


\section*{La puissance de l'Esprit}

Vous est-il déjà arrivé de sentir que vous deviez communiquer à quelqu'un
son besoin d'accepter Jésus-Christ, mais sans avoir le cran d'aborder le sujet.
 Vous est-il déjà arrivé, en passant devant une université,
 d'observer les étudiants, de réaliser que la plupart d'entre eux sont perdus,
 et de vous demander s'ils pourraient être gagnés pour Christ ?
 Vous arrive-t-il de penser aux milliards de gens à qui on n'a jamais vraiment
 parlé de l'Évangile, et vous-êtes vous ensuite demandé comment cela pourrait
 être accompli?

Pour Pierre, qui a renié son Seigneur en présence d'une seule servante,
 et pour le reste des disciples (qui se sont enfuis quand ça s'est compliqué),
 le commandement de Jésus d'aller dans le monde entier et de prêcher
 l'évangile à toute créature doit avoir paru complètement impraticable et
 impossible, et cela l'était en effet.
 Il n'était pas possible que 11 hommes insignifiants venant de Galilée
 puissent toucher le monde pour Jésus-Christ.
 C'est pour cela que Jésus leur a dit d'attendre à Jérusalem jusqu'à ce
 qu'ils reçoivent la puissance de l'Esprit-Saint, car c'était par sa puissance
 qu'ils pourraient témoigner jusqu'à extrémités de la Terre.

Dieu a-t-il réservé cette expérience de la puissance du Saint-Esprit à
 l'Église primitive seule? Les Écritures indiquent-elles qu'il viendrait
 un temps où nous n'aurions plus besoin de dépendre de la puissance
 de l'Esprit, mais où nous pourrions, par notre connaissance parfaite
 des Écritures, faire l'œuvre de Dieu par nous-mêmes? L'Église qui a été
 initiée dans l'Esprit doit-elle maintenant être perfectionnée dans
 la chair?
 Quelle est la réponse à l'impuissance de l'Église?
 Pourquoi l'Église a-t-elle échoué dans sa lutte contre la chute folle
 du monde corrompu autour de nous?

Paul nous met en garde dans \bibleverse{Heb}(4:) de craindre de ne pas
 recevoir la promesse de Dieu d'entrer dans Son repos.
 N'est-il pas également approprié pour nous de craindre que, si Dieu
 nous a donné une promesse de puissance dans nos vies personnelles et de
 puissance dans le corps unis de l'Église, nous ne la recevions pas?


\section*{La promesse du Père}

Dans \bibleverse{Acts}(1:), nous lisons que les disciples étaient avec
 Jésus à Béthanie, d'où Il devrait bientôt les quitter pour monter
 aux cieux.
 Les nuages Le recevraient alors hors de leur vue.
 Il leur donnait leurs instructions finales.
 Dans \bibleverse{Acts}(1:4), Jésus leur a dit de \og ne pas s'éloigner
 de Jérusalem, mais d'attendre la promesse du Père dont, leur dit-il,
 vous m'avez entendu parler. \fg{}
 Dans \bibleverse{Lc}(24:49), Jésus a dit~:
 \og Et voici, j'enverrai sur vous ce que mon Père a promis, mais vous,
 restez dans la ville, jusqu'à ce que vous soyez revêtus de la puissance
 d'en haut.\fg{}.

Dans ces deux passages, Jésus se référait à la promesse du Père, qui est
 sans nul doute une référence à \bibleverse{Joel}(3:1-2), où Dieu a promis~:
 \og Après cela, je répandrai mon Esprit sur toute chair ;
 vos fils et vos filles prophétiseront, vos anciens auront des songes,
 et vos jeunes gens des visions.
 Même sur les serviteurs et sur les servantes,
 en ces jours-là, je répandrai mon Esprit. \fg{}
 Cela est confirmé dans le second chapitre des Actes, lorsque la foule qui
 s'était assemblée à cause du phénomène surnaturel qui accompagnait l'envoi
 de l'Esprit-Saint se demandait~:
 \og Qu'est-ce que cela signifie ? \fg{}
 Pierre a répondu en explication~:
 \og C'est ce qui a été dit pas le prophète Joël \fg{},
 et il a cité la prophétie de Joël.
 La promesse de Dieu était qu'un jour viendrait où Il enverrait Son Esprit,
 pas seulement sur des individus en particulier, mais sur toute chair.

\section*{La promesse du Sauveur}

Jésus a également promis son Esprit à Ses disciples dans
 \bibleverse{Jn}(14:16-17), où Il a dit~:
 \og Et moi, je prierai le Père, et il vous donnera un autre Consolateur
 qui soit éternellement avec vous, l'Esprit de vérité, que le monde ne peut
 pas recevoir, parce qu'il ne le voit pas et ne le connaît pas ; mais vous,
 vous le connaissez, parce qu'il demeure près de vous et qu'il sera en vous \fg{}.
 Lorsque Jésus a promis l'Esprit-Saint, Il se référait à Lui comme
 \og un autre Consolateur \fg{}.
 Le mot traduit par consolateur vient du mot grec \emph{parakletos},
 qui signifie littéralement \og venir aux côtés pour aider \fg{}.
 C'est le ministère de base de l'Esprit-Saint auprès du croyant.
 Il est là pour nous aider. Jusqu'à ce moment, Jésus avait été aux côtés
 de Ses disciples, les aidant. Ils étaient, avec raison, devenus dépendants
 de Son aide. Il était le Maître de toute situation.

Lorsque la tempête menaçait de couler leur petite embarcation, Jésus
 a repris le vent et les vagues, et il y a eu un grand calme.
 Lorsque le collecteur d'impôts demandait des taxes injustes,
 Jésus a dit à Pierre d'aller pêcher un poisson et de prendre
 la pièce dans sa bouche afin de payer les taxes.
 Quelle que soit la situation, Jésus était toujours prêt à aider.

Cependant, Il leur a dit qu'Il allait les quitter.
 Il ne serait plus avec eux comme par le passé.
 Leurs cœurs devaient être troublés par Ses mots,
 et ils avaient peur d'envisager l'avenir sans Lui.
 C'est pourquoi Il leur a promis qu'Il ne les laisserait pas sans aide,
 qu'Il demanderait au Père, et qu'Il leur enverrait un autre Consolateur
 ou Défenseur pour rester avec eux éternellement~:
 l'Esprit de vérité.
 Pour notre démarche de chrétiens, nous sommes complètement dépendants de
 l'aide de l'Esprit-Saint.
 Il est impossible de faire quoi que ce soit de valable dans le service
 chrétien sans Son aide.

\section*{L'attente à Jérusalem}

À cause des mots \og restez dans la ville \fg{} utilisés dans
 l'évangile selon Luc, beaucoup de pentecôtistes ont établi des \og réunions
 d'attente \fg{} comme moyen par lequel la puissance de l'Esprit-Saint est
 reçue dans la vie du croyant. Notez que le commandement était de \og rester
 à Jérusalem \fg{} \NdT{la Bible à la Colombe traduit \og restez dans la
 ville \fg{}, alors que la King James précise \og dans la ville de Jérusalem \fg{}.},
 donc pour être vraiment scripturaire, les réunions d'attente devraient toutes
 avoir lieu à Jérusalem!

Il est clair que Jésus n'établissait pas ainsi une méthode universelle par
 laquelle l'Esprit-Saint serait donné aux croyants de tous temps. Il les
 encourageait seulement à attendre quelques jours à Jérusalem jusqu'à ce
 qu'Il envoie l'Esprit-Saint comme don à l'Église. Une fois que
 l'Esprit-Saint était donné au jour de la Pentecôte, il n'était plus
 jamais nécessaire de l'attendre à nouveau, et nous ne trouvons pas dans
 le livre des Actes de réunion d'attente, et elles ne sont pas recommandées
 non plus dans le Nouveau Testament comme méthode pour recevoir le don de
 l'Esprit-Saint.


\section*{Une puissance dynamique pour vous}

Dans \bibleverse{Acts}(1:8), Jésus a promis à ses disciples
 qu'ils recevraient une puissance lorsque l'Esprit-Saint serait
 sur eux, et qu'à travers cette puissance ils seraient témoins
 du Christ jusqu'aux extrémités de la terre.
 Le mot grec traduit par \og puissance \fg{} est \emph{dunamis}.
 Le mot français \og dynamique \fg{} vient directement de ce mot,
 et il décrit que l'Esprit-Saint est en nous \ocadr la dynamique par laquelle
 nous vivons et servons Dieu. Sans cette dynamique, la vie chrétienne est
 impossible et le service ne porte pas de fruit.
 Quelles nouvelles dimensions glorieuses la puissance de l'Esprit-Saint
 apporte dans la vie du croyant \ocadr la puissance d'être et de faire
 tout ce que Dieu veut !

Ce n'est pas la volonté de Dieu que votre vie en Christ soit ennuyeuse
 et terne, ou que votre service soit une corvée.
 Dieu veut que votre marche avec Lui soit pleine de joie.
 Il veut que vous ayez la puissance et la victoire dans votre vie.
 Si votre vie en Christ n'est pas dynamique et victorieuse,
 Dieu a quelque chose de plus pour vous.
 La promesse du don de l'Esprit-Saint est \og pour vous, pour vos enfants,
 et pour tous ceux qui sont au loin,
 en aussi grand nombre que le Seigneur notre Dieu les appellera. \fg{}

