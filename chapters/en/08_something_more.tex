\chapter{Something More}

Recently a young man came up to me and said, "I accepted Christ several years ago, but I was never too excited about it. I found reading the Bible uninteresting. In fact, my mind would wander, and I couldn't really concentrate on the Word. I never really knew what it was to worship God, and my prayer life was erratic. But since I was filled with the Spirit a few months ago, my life has completely changed. I have a great love for the things of God. I can't seem to get enough of the Word, and now I love to fellowship with the believers. What a great change has happened in my life since I was filled with the Spirit!" 

This story, with variations, has been told to me hundreds of times over by those who have found that there is something more than just having the Spirit indwelling their life at conversion. We do recognize that every born-again believer has the Spirit dwelling in him. Writing in I Corinthians 6:19, Paul declares that our bodies are the temples of the Holy Spirit who dwells in us. He also declares in I Corinthians 12:3 that you cannot call Christ Lord except by the Spirit. 

\section*{The Spirit and the Believer}

There are three Greek prepositions used in the New Testament to designate the different relationships of the Spirit to the believer: \emph{para}, \emph{en}, and \emph{epi}. In John 14:17 Jesus said to His disciples concerning the Holy Spirit, "Ye know Him, for He dwells with [para] you and shall be in [en] you." Here a twofold relationship is expressed: \emph{para} (with) and \emph{en} (in). The Holy Spirit was \emph{with} us prior to our conversion. He is the One who brought us conviction of sin and revealed Christ as the answer. When we accepted Jesus as our Savior and invited Him into our lives, the Holy Spirit began to indwell us. 

But God has something more - the beautiful empowering through the \emph{epi} relationship. Note that this is what Jesus was promising His disciples just prior to His ascension. In Luke 24:49 He said, "Behold, I send the promise of my Father upon [epi] you" or "over you." In Acts 1:8 He said, "But ye shall receive power when the Holy Spirit comes upon [epi] you." 

We read in Acts 10:44 that the Holy Spirit descended "upon" the Gentile believers in the house of Cornelius: "While Peter yet spoke these words, the Holy Spirit fell upon [epi] all of them which heard the word." In Acts 19:6, when Paul laid hands upon the Ephesian believers, the Holy Spirit came upon [epi] them. 

We read in Acts 8 that Philip had gone to Samaria and preached Christ unto them; many people believed Philip's preaching of the things of the kingdom of God and the name of Jesus Christ, and they were baptized. If there is just one baptism (Ephesians 4:5), then we must accept that at this point the Samaritan believers were baptized by the Spirit into the body of Christ (I Corinthians 12:13), and the Holy Spirit began to indwell them. It is obvious, however, that there was yet a further relationship to the Holy Spirit to be received, for when the church in Jerusalem heard that the Samaritans had received the gospel, they sent Peter and John unto them that they might pray for them to receive the Holy Spirit, for as yet He had fallen upon [epi] none of them. 

\section*{The Overflowing Life}

When Paul came to the church in Ephesus and found that the believers' experience was lacking, possibly in love or joy and zeal, he asked them, "Did you receive the Holy Spirit when you believed?" If the full relationship with the Spirit is attained simultaneously with conversion, the question makes no sense. The question itself acknowledged a relationship deeper and beyond the conversion experience. What they were lacking was the \emph{epi} relationship with the Holy Spirit, for that is what resulted when Paul laid his hands upon them in Acts 19:6: "and the Spirit came upon [epi] them." 

Being filled with the Spirit adds new dimensions of love, joy, and exuberance to the Christian life. If Paul the Apostle would meet you and begin to share the glories of Christ with you, would he be apt to ask, "Did you receive the Spirit when you believed?" God wants your life not to just be indwelt or even filled with the Spirit. He wants your life to overflow. 

\section*{The Eight-Day Feast}

In John 7:37 we read, "In the last day, that great day of the feast, Jesus stood and cried, saying, 'If any man thirst, let him come unto me and drink.'" This was the Feast of the Tabernacles, the feast in which God's people remembered His divine preservation of their fathers as they wandered for 40 years in the wilderness. In Leviticus 23 we read what when they observed this feast they were to make little booths and were to move out of their houses and dwell in the booths for the eight days of the feast. As tradition developed, they were to leave enough space in the roof thatches so that they could see the stars at night, to remind them that their forefathers had slept under the stars for 40 years. Also, enough space was to be left in the walls so that the wind could blow through, so they would remember that even though their fathers were exposed to the elements for 40 years, God miraculously preserved them. 

At the temple during this feast, each day the priests would make a procession to the Pool of Siloam, where they would fill their large water jugs and then come in procession up the many steps to the temple mount. As the people were singing the glorious hallel psalms, the priests would pour out the water on the pavement; this was to remind the worshipers of the water that came out of the rock in the wilderness when it was smitten by Moses, and of God's supernatural preservation of their fathers in that dry wilderness. 

It is said that on the eighth day, the last day (which was known as the great day of the feast), the priests did not make a procession to the pool to fill the jugs with water. On this day there was no pouring out of water on the pavement. This also was significant, for it was an acknowledgment that God had fulfilled His promise; He had brought them into the land that was well-watered and flowing with milk and honey, and they no longer needed that miraculous supply of water out of the rock. 

It was on this day, the great day of the feast, that Jesus stood and cried, saying, "If any man thirst, let him come unto me and drink!" Jesus is talking of that universal spiritual thirst that every person experiences. 

\section*{Basic Human Needs}

Mankind is a trinity of body, soul, and spirit. It is difficult, if not impossible, for man to separate himself into these three entities because we are totally integrated - body, soul, and spirit - so that anything that affects me physically will affect me mentally, and can also affect me spiritually. Anything that affects me mentally also affects me physically. More and more the psychologists are discovering the close relationship between our emotions and our physical health. By the same token, whatever affects me spiritually will also affect me emotionally and physically, so that when a person is born again, it has an effect upon his whole being: spirit, soul, and body. 

Abraham Maslow has identified and catalogued in order of strength our body drives, which are known as the homeostasis. These are the beautiful built-in mechanisms that God created to monitor our bodies to keep the proper balance that will sustain and perpetuate life. Maslow has identified the strongest of these drives as the air drive: the body monitors the oxygen levels in the blood and demands that oxygen be replenished when it gets too low. The body's response is to start panting as the rate of the heartbeat increases. Next in order is the thirst drive, then hunger, then bladder, then sex, and so on in a declining order. These drives all involve man's physical needs. 

The sociologists have also listed what they call our sociological drives. Man thirsts, or has a drive, for love. There is also a need for security. And there is the need to be needed. 

Down in the deepest part of man, in the area of his spirit, there is also a very strong thirst or drive. This is the thirst of man's spirit for a meaningful relationship with God. The attempt of the psychologist to understand human behavior will always be limited until he recognizes the spiritual dimension of man. The strongest drive and the deepest need of man is to know God. In Psalm 42:1,2 David said, "As the heart panteth after the water brooks, so panteth my soul after Thee, O God. My soul thirsteth for God, for the living God." Paul declared in Philippians 3:7,8 that everything that was once important to him he counted loss for the excellency of the knowledge of Jesus Christ. Paul explains in Romans 8 how God has made man subject to emptiness; he was deliberately designed that way so that man could never be complete apart from God. Nature seeks to fill the vacuum, so man by nature has sought to fill this spiritual void with a variety of physical or emotional experiences. 

\section*{The Needs Are Distinct}

The thirsts we experience are separate and distinct, so that we cannot satisfy a physical thirst with an emotional experience. If you were lost in the desert, walking across the hot sand, and the moisture level of your body was getting dangerously low, you would feel a tremendous physical thirst. As your body dehydrated, you would lose your strength. Let us say that you are finally lying in the hot sand, instinctively digging for water, when someone comes over the sand dune, spies you there, and says, "Oh, I know who you are. I want you to know that I have had a secret love for you. I think you are the greatest person in the world, and I love you tremendously." 

While this person might be fulfilling your emotional need for love, you are dying of thirst in his arms, for you cannot satisfy a physical thirst with an emotional experience. In the same way, you cannot satisfy an emotional thirst with a physical experience, and this has created some problems in our present society. 

\section*{One-Sided Fulfillment}

We are living in a culture in which most of the physical needs of the person are being adequately cared for; yet quite often there is a tragic lack of fulfillment of the emotional needs. Many times parents have difficulty understanding their children's actions of rebellion against the home. I have heard them say, "I've given my child everything; I cannot understand how he can do what he is doing." When they declare, "I've given him everything," they are usually talking about physical things: the child has received several bicycles and his own television set, stereo, and car. 

But so often these things were given to the child in order to push him away. The purpose was to keep the child entertained by these things so that parental time would not have to be given him - time which would cause him to feel the closeness of love within the family. The mother so often says, "Why don't you go into your room and watch television? Don't you know you're making me nervous? Don't ask so many questions. Why don't you go outside and ride your bicycle?" The child, thirsting for love and security, is pushed away to the material things, and one day he ultimately rebels against the material world, as we saw in the counterculture revolution known as the hippie movement. 

You cannot satisfy an emotional thirst with a physical experience. It is also true that, down deep, man has a deep spiritual thirst for God. One of the problems of our present age is that man has endeavored to satisfy that deep thirst for God with physical or emotional experiences. This deep thirst for God is one of the reasons behind the pleasure mania in the world today. People are trying to satisfy that deep need for God with emotional and physical experiences. It also partially explains drug abuse, as people often have pseudo-spiritual experiences through the use of drugs. Many of the people using LSD thought they were having true experiences with God. 

\section*{The Deep Universal Thirst}

When Jesus said, "If any man thirst," He was referring to that deep universal thirst of man's spirit for God. It is interesting to me that some psychology textbooks identify frustration as one of the root causes of neurotic behavior. They declare that a person's problem often begins with frustration, that feeling that you have not attained what life is all about, that there must be more to life than what you have experienced - but what is it and how do I attain it? It is reaching out for something I am not sure of, and not finding what I am hoping for. What is frustration but thirst, spiritual thirst, that deep thirst in man's spirit for God? 

The psychology books show how frustration leads to an inferiority complex, which is nothing more than my rationale to myself as to why I have not achieved this satisfaction or fulfillment that I long for. I say, "If only I had money," or "If only I had blue eyes instead of brown," or "If only I had attained a better education." With these or a thousand other excuses I explain to myself the reason for my frustration. 

\section*{Two Kinds of Escapes}

According to the textbooks, I then move from my inferiority complex to an escape. This can be overt or invert. The inverted escapes are manifested in attempts to build a wall around your true self. You will often display to other people a facade which is far different from the real you. You act as if it does not hurt when it really does; you act very confident when in reality you are scared. You begin to keep people at a distance; you are afraid they might find the true you. You avoid the person whom you feel is moving in too close to you. You do not want to speak to him when he calls. You get to the point where you do not want to answer the doorbell. In its final form the invert escape is manifested in the hermit living alone in his shack in the desert, firing shotgun blasts at anyone who comes beyond his gate with the "Keep Out - No Trespassing" signs. 

The overt escapes are manifested in many forms, such as alcoholism, drug abuse, compulsive eating or gambling, nomadism, extramarital affairs, etc. I cannot bear to face the reality of my failure to find true fulfillment, so I escape into unreality. These escapes then bring me to a guilt complex. I know that what I am doing is wrong. I know it is destroying me and those around me who love me, yet I do not seem to have the capacity to stop. I begin to hate myself for what I am doing to myself and others. 

The guilt complex then moves into a subconscious desire for punishment. This is usually manifested in a neurotic behavior pattern designed to bring the disapproval of my associates, which I interpret as punishment, which in turn relieves me from my feelings of guilt. When I was a child, my father took care of my guilt complex by punishing me. In my case it usually took the form of a spanking. Once I had been punished I no longer felt guilty and I could take my place as a member in good standing in the family. Prior to the punishment, I felt a strained relationship and a sense of alienation. 

As we grow older there is no parental authority over us, so to be relieved of guilt we must behave in an unacceptable way to bring rebuke or rejection, which we interpret as punishment. Once punished, we feel free from our guilt complex and then return to our frustration and begin the cycle over again. When Jesus said, "If any man thirst," He was referring to this thirst in man's spirit for God, that which the psychologists classify as frustration. 

\section*{True Quenching of Thirst}

When Jesus was talking to the woman of Samaria, He asked her for a drink, and she challenged Him for asking, since He was a Jew and she was a Samaritan. By tradition they were not to have any dealings with each other. Jesus responded to her, "If you knew who it was that was asking you for a drink, you would have asked Him for a drink." She answered rather smartly, "Why would I ask you for a drink when you have nothing to draw with, and this well is very deep?" Jesus then said to her, "He who drinks of this water will thirst again." I believe that this verse should be inscribed over every goal, ambition, or pursuit of pleasure that a man has. You may drink of that water, achieve your goal, realize your ambition, and fulfill your fantasies, but you are going to thirst again. It will not satisfy, for way down deep your spirit is thirsting after God, and nothing can satisfy that thirst except a meaningful relationship with God. 

When Jesus said, "If any man thirst, let him come unto me and drink," He was expressing the gospel in its simplest terms. He was saying to all of mankind, "Deep down within your life you need God. You are reaching out for a meaningful relationship with God. Come unto me, and your thirst will not only be fully satisfied and fulfilled, but out of your life there will gush torrents of living water." Only Christ can satisfy your spiritual thirst, for He brings you into a meaningful relationship with God. 

\section*{Torrents of Water}

In John 7:38 Jesus went on to say, "He that believeth on me, as the Scripture hath said, out of his belly shall flow rivers of living water." The Greek words here are a little more intense than what is reflected in the King James Version. The Lord is literally declaring that, if a person believes on Him, "Out of his belly there will gush torrents of living water" - not just a gentle little stream flowing out, but a tremendous torrent of water, as that which cascades down a mountain ravine during a cloudburst. 

To what was Jesus making reference when He spoke of "torrents of living water" coming forth from our life? When John wrote this Gospel, it was several years after the fact. His was one of the last New Testament books to be written, and John was writing with the advantage of hindsight. At the actual time Jesus was talking about the torrents of living water, John was probably confused as to what Jesus meant or what He was promising the people. But because John wrote the Gospel with understanding gained through the advantage of hindsight, he added his own commentary expressed in the brackets of verse 39, in which he explained that Jesus was speaking of the Holy Spirit, "which they that believe on Him should receive; for the Holy Spirit was not yet given, because Jesus was not yet glorified." So Jesus was talking about the empowering of the believer's life by the Holy Spirit. 

\section*{God's Desire for You}

I think we have to accept without question the fact that this description is much more than just the indwelling presence of the Spirit within a believer's life at conversion. It is one thing to have the Holy Spirit indwelling your life; it is another thing to have that glorious, dynamic power of God's Spirit flowing forth from your life like a torrent of living water. 

God has a fuller relationship for you than just the indwelling of the Spirit. It is God's desire that the Spirit flow forth from your life. It really makes little difference what you term it. Some call it the baptism of the Holy Spirit, some call it the filling of the Holy Spirit, and some call it the empowering by the Holy Spirit. It really does not matter what you call it; what is important is that you have that glorious outpouring of the power of the Spirit flowing forth from your life. 

God always looks at man in two ways. First, God looks at him \emph{subjectively}, as He seeks to do His work in your life. But God's purposes are never culminated in His subjective work. God also looks toward that \emph{objective} work which He seeks to do through you. He works in you \emph{subjectively} that He might work through you \emph{objectively}. He desires to do a work \emph{in} you and \emph{for} you in order that He might work \emph{through} you to touch others. Our relationship to the Spirit is never complete when He is just indwelling us. We are more than vessels to contain the Spirit of God. God desires that we might be \emph{channels} through which His Spirit might flow. 

\section*{The Power in Action}

As you look at your own experience and relationship to the Holy Spirit, if you cannot say that the powerful dynamic of God's Spirit is gushing forth from your life like a river or torrents of living water, then God has a fuller relationship to His Spirit that He desires to bring into your life, and I would encourage you to seek this power of God's Spirit until it flows forth from your life. There is a needy world around us which needs to be touched by the power of God's Spirit flowing forth from us. If you object to calling it the baptism of the Holy Spirit, call it whatever you like, but what Jesus is describing is far more than the mere indwelling of the Holy Spirit in the life of the believer experienced at the time of his conversion. That beautiful flowing of the Spirit from a person's life is the true \emph{charisma}. 



