\chapter{A More Excellent Way}

We read in Galatians 5 about the spiritual battle that is going on in every Christian. This battle does not take place in the non-Christian; he does not know anything about it, because the spirit of the non-Christian is dormant. 

But once your spirit has come alive, once you have been born again, there is an internal conflict. In Galatians 5:17 Paul said, "For the flesh lusts against the spirit and the spirit against the flesh, and these are contrary to each other, so that you cannot do the things that you would." The battle that is going on is the spirit versus the flesh; the flesh is keeping you from doing those things that you want to do for the Lord. "Now the works of the flesh are manifest, which are these: adultery, fornication, uncleanness, lasciviousness, idolatry, witchcraft, hatred, variance, emulations, wrath, strife, seditions, heresies, envyings, murders, drunkenness, revelings, and the like: of which I tell you before, as I have also told you in times past, that those who do such things shall not inherit the kingdom of God" (verses 19-21). 

In marked contrast to the works of the flesh are the results of the Spirit: "The fruit of the Spirit is love [agape]" (verse 22). The method of God is fruit in contrast to works. Any time you get in the realm of works, you are getting in the realm of the flesh. But fruit, indicating to us the method of God, is the natural consequence of a relationship. You do not see an apricot tree struggling and straining to produce apricots, nor do you see the apricots struggling and straining to get ripe; it is just a natural process. 

\section*{How to Bear Fruit}

God's method for you is just the natural process of God working in your life - it is not something that you can do, that you strain and struggle to try to develop. The minute you get into that straining and struggling, you are in the realm of works again. God's method is fruit, and the fruit is the natural consequence of the relationship of abiding in Christ. 

Jesus said, "I am the vine and you are the branches. Every branch in me that brings forth fruit, the Father purges it [or cleanses it, washes it] that it might bring forth more fruit. Now you are cleansed through the word that I have spoken to you; abide in me and let my words abide in you." As you abide in Christ, you are going to bear fruit. Jesus said, "You cannot bear fruit unless you abide in the vine." The branch cannot produce fruit by itself. You have got to abide in the vine if you are going to bear fruit. Jesus also said, "Apart from me you can do nothing." But as you abide in Christ, the natural result is that His love will begin to flow forth from your life. The method of God is fruit - the very easy, very natural consequence of just abiding in Jesus. 

The word "Spirit" in "the fruit of the Spirit" indicates to us the dynamic of God - the work of the Holy Spirit within the life of the believer. It is no accident that I Corinthians chapter 13 is sandwiched between Paul's discussion of the gifts of the Spirit in chapters 12 and 14. In chapter 12 he lists many of the gifts of the Spirit; in chapter 14 he describes how some of these gifts operate, and the purposes of some of these gifts. But at the end of chapter 12 he says, "I will show you a more excellent way" - something even more excellent than the possession of these marvelous gifts. 

\section*{The Excellent Way}

Often we say, "Oh God, I want the gift of miracles" or "I want the gift of faith" or "I want the gift of healing" or "I want the gift of discernment of spirits" or "I want the gift of the word of knowledge." We would like to have these supernatural gifts operating in our lives. But Paul said, "I'll show you a more excellent way." Even more than having the supernatural gifts operating in my life, it is preferable to have God's love flowing from my life, and if that love is not flowing, these supernatural gifts become meaningless. 

"The fruit of the Spirit" indicates to us the dynamic of God. Jesus said, "You shall receive power [dunamis] after the Holy Spirit is come upon you." The Holy Spirit is God's dynamic within our lives. He is that power within us who gives us the ability to be what we could not be apart from Him, and to have what we cannot have apart from Him, and to do what we could not do apart from Him. You cannot have the agape without the Holy Spirit, and you cannot express the agape without the Holy Spirit. The fruit of God's Spirit within your life is that this love will come forth. The natural outflow of the Spirit of God within you will be this love, for the Spirit of God is the dynamic force of God within you that produces this agape from your life. The method of God is fruit; the dynamic of God is the Spirit. 

\section*{The Real Fruit of the Spirit}

You have probably heard that there are nine fruits of the Spirit. I have heard messages on the nine gifts of the Spirit and the nine fruits of the Spirit. But I want you to look closely at Galatians 5:22: "The fruit of the Spirit is agape." As I understand English grammar and as I understand Greek grammar, if there were nine fruits Paul would have said, "But the fruits of the Spirit are love, joy, peace, etc." 

But that is not what he said. It is in the singular: "The fruit of the Spirit is love." Then what are these other things in the verse? What about the joy, the peace, the long suffering, the gentleness, the goodness, the faith, the meekness, and the temperance? These are all defining the agape. Our English usage of the word "love" is so weak that it can mean almost anything. So Paul defines this agape by using these other words. 

Joy is love's consciousness. Have you ever seen a person really in love? The chief characteristic of that person is the joy he or she has. Oh what joy there is in true love! You can face some of the toughest situations and still have true joy. You can be doing some of the most miserable tasks, but if there is true love, there is glorious joy. Talking with a young girl a while back I asked, "Well, how are you doing?" She replied, "Oh, I'm doing great! I just got married and I don't have to work anymore." She meant that she did not have to work from 8 to 5 behind a desk. She was probably doing more work than she had ever done, but now there was such love that she did not even count it as work. Love makes every task a pleasure. When you truly love, you do not get uptight over things you do for those you love. You enjoy doing them. Joy is love's consciousness. 

The second characteristic of the Spirit's agape is peace. There can be no true peace apart from agape love. Someone said, "There is now peace in the Middle East." Don't you believe it! There is no peace at all. There is so much hatred; there is so much bitterness. There isn't any real peace there. At a moment's notice the whole thing could blow into a full-scale war. The only true basis for peace is love. You can have a cessation of hostilities; you can have truces; you can have agreements; but the only true basis for peace is real love. When I love you so much that I would not want to do anything that would harm you in any way, then we have peace between us. 

Longsuffering. Paul used this word in his definition of agape in I Corinthians 13: "Love is longsuffering and kind." If you truly love someone, you do not keep track of how many times you have been offended. You are longsuffering. You take and take and take, and then you are kind. 

Another characteristic of agape is its gentleness: Oh how gentle is love! What a beautiful quality, what an admirable quality, is the gentleness of true love! 

Then goodness. I believe that love is the only real motive for goodness. A lot of people are good only because they fear the consequences of being bad. But that is not true goodness. "I'd like to murder you but I'd get jailed." "I'd like to rob that bank but I might get caught." A lot of people are restrained from evil only for the fear of the consequences. That is not goodness. The only true motivation for goodness is love. Because of love, I would not want to hurt and I would not want to offend. I would do nothing to cause a person to stumble, because I love that person. That's the true motivation for goodness. 

Another characteristic of agape is faith. This is not the same faith that we find as a gift of the Spirit in I Corinthians 12, but it is a faith or trust in people. It is just a "trustingness." If you say, "I don't trust anybody," you are just saying, "I don't love anybody." If you truly love a person, you will trust him, because trusting is a part of the quality of agape love. 

And then there is meekness. True love does not seek its own way, does not vaunt itself, and is not puffed up. One of the chief characteristics of Jesus Christ is His beautiful meekness. He could have been on a big power trip when He was here. After all, look who He was. Often Jesus referred to Himself with the title "the Son of Man." He had many glorious titles that He could have taken: the Son of Righteousness, the Son of Glory, the Bright and Morning Star, the Fairest of Ten Thousand, the Lily of the Valley, the Wonderful Counselor, the Mighty God, the Everlasting Father, the Prince of Peace. All of these are proper titles that Jesus could have taken. He could have said "The Bright and Morning Star saith unto thee" or "The Anointed of God declares" But instead He often referred to Himself as the Son of Man: "The Son of Man has come to seek and to save that which was lost." 

Many people today honor one another with their fancy titles. How people love titles! But Jesus disdained titles; He put down those who loved to stand in the marketplace being called "Rabbi, Rabbi!" Someone once said that titles are only distinctions by which we tell one worm from another. What am I? Nothing apart from God. Love's characteristic is that of meekness. 

Finally, temperance. The best way I can think of to define temperance is to give the opposite: intemperance. Unfortunately we know all too well what that is, and it is the opposite of temperance. Temperance is moderation, not going overboard. It is that beautiful evenness of love. 

\section*{The Fruit in Your Life}

This agape love is what the Holy Spirit is seeking to bring into your life; agape is the true fruit of the Spirit. This will be the final result of God's Spirit dwelling in you. As God's Spirit works within you, and as you yield yourself to the Spirit, the fruit of the Spirit is agape love. The purpose of the work of the Holy Spirit within your life as a believer is to do for you what you cannot do for yourself - bring to you that agape love of God for the family of God. 

This will be a sign by which the world will know that you are Christ's disciple, and a sign by which you will know that you have passed from death to life. It is a sign because you will see God's love working in your life. 

We need to be filled with the Spirit. We need to yield to the Spirit so that His fruit might come forth abundantly from our lives. Then we will have His agape love ruling us, flowing forth from our lives like a torrent of living water. 


