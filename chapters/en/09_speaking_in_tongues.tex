\chapter{Speaking in Tongues}

One of the areas of sharpest controversy within the body of Christ today involves speaking with other tongues, "glossalalia." At one extreme are people who label any exercise of tongues as satanic. At the other extreme are people who declare that you are not filled or baptized with the Holy Spirit unless you speak with other tongues. They declare that speaking in tongues is the initial evidence of the baptism of the Holy Spirit. In I Corinthians 13:1 Paul declares that tongues in themselves are not a valid evidence of the Holy Spirit within the life of a believer, for, "Though I speak with the tongues of men and of angels, and have not love [agape], I am become as sounding brass or a clanging cymbal." In other words, the tongues are only meaningless sounds and have no validity if there is not that accompanying agape love. 

\section*{Tongues Versus No Tongues}

In the book of Acts, speaking in tongues often accompanied the \emph{epi} relationship to the Holy Spirit. Such is the case in Acts chapters 2, 10, and 19. However, in the eighth chapter of Acts, when the Samaritan believers received the Holy Spirit, there is no mention that they spoke in tongues. However, it is obvious that there must have been some kind of phenomena that accompanied their receiving the Holy Spirit, for Simon the sorcerer was seeking to buy the power that Peter and John possessed; he desired that he might also be able to lay hands upon people that they might receive the Holy Spirit. It is evident that some kind of phenomena accompanied their receiving the Spirit because Simon wanted to buy the power so he could duplicate the feat. Later, in Acts 9:17, when Ananias laid his hands upon Saul (Paul) that he might receive the gift of the Holy Spirit, there is no mention of Paul speaking in tongues. However, we do know that subsequently, as Paul was writing to the Corinthians, he thanked God that he spoke in tongues more than all of the Corinthians. When Paul first experienced the gift of tongues is not divulged. 

We must point out that a person who speaks in tongues and lacks agape love has less valid evidence of the indwelling or the filling of the Spirit in his life than a person who has never spoken in tongues and yet manifests love and other dynamic qualities of God's Spirit. I cannot deny the validity of the Spirit-filled lives of many of those dynamic leaders and laymen in the church today who have never enjoyed the experience of speaking in tongues, and I prefer their fellowship over many who promote the speaking in tongues as the only true evidence of the Spirit-filled life, but whose personal lives are marred by strife or pride and often even heresy. 

When Paul wrote to the Galatians he declared, "The fruit of the Spirit is love." The real proof of God's Spirit filling a person's life is love. Love is the most valid evidence that a man is truly filled with the Spirit, and tongues without love are just meaningless sounds. 

\section*{Building, Praising, Praying}

Speaking in tongues is a very edifying experience to the believer. Paul tells us in I Corinthians 14:4, "He who speaks in an unknown tongue edifies himself." The term "edify" means to build up, and it is used in the New Testament for the building up of Christ within the life of the church or of the believer. The purpose of the church assembling together is to be built up in Christ, and when I am in church I should seek to build up the whole body of Christ. Personal devotions are intended to build up myself in Christ; when I speak in tongues in my personal devotions it is one of the ways by which Christ is built up within me. 

Speaking in tongues is also an excellent way to praise the Lord. I often find that I have difficulty in expressing to God the feelings I have within. God has been so good, and has blessed me so much, that to merely say, "Oh God, I thank You for all that You have done," falls so far short of my feeling of deep gratitude and praise. I find difficulty in expressing these deepest feelings of my spirit. It is wonderful to be able by the Spirit to express my praises to God without having to limit it to the narrowed channel of my own intellect. Paul tells us that, when we speak in tongues, we are blessing God with the Spirit. However, if we do it in church without an interpreter, the person who is occupying the position of the unlearned cannot say "Amen" at my giving of thanks, as he does not understand what I am saying. Paul points out, "You indeed give thanks well." In other words, Paul is declaring that it is a good way to give thanks to God and to express your worship and praise to Him. 

As Paul writes to the Ephesians about the Christian's spiritual warfare, he tells of the armor that we are to put on. Then he goes on to tell how we are to wage warfare against the enemy: "Praying always with all prayer and supplication in the Spirit" (Ephesians 6:18). In verses 20 and 21 of his epistle, Jude exhorts us to keep ourselves in the love of God. He tells us that one of the ways by which we keep ourselves in His love is by praying in the Holy Spirit. In Romans 8:26 Paul tells us that one of the weaknesses that we experience in our Christian walk comes in our prayer life, by not always knowing how we ought to pray in a given situation. Many times we do not know God's will. I want to pray according to God's will because I know that prayer apart from God's will has no value. We do know that if we ask anything according to His will, He hears us. But that is where the problem arises, and that is our weakness: we do not always know what the will of God is. 

Paul tells us in Romans 8 that the Holy Spirit helps these infirmities when we do not know how we ought to pray, for the Spirit Himself will make intercession for us with groanings which cannot be uttered. He is searching the heart and He knows what is the mind of the spirit, because He makes intercession for the saints according to the will of God. 

So when I do not know how to pray for a particular problem, I can just groan in my spirit, and though I do not understand the groanings, yet God interprets them Himself as intercession and prayer according to His will for that person or particular situation over which I am groaning. Now if God understands the inarticulate groanings of the spirit as intercession and prayer according to His will, surely those articulated words in another tongue, though an unknown tongue to me, are nevertheless understandable by God. 

\section*{Tongues in Private}

We cannot challenge the statement of Paul that he thanked God that he spoke in tongues more than all of the Corinthian believers. Yet Paul declared that, when he was in church, he would rather speak five words in a known tongue than 10,000 words in an unknown tongue. There are those who declare that, since the gifts of the Spirit were given to benefit the whole body, as Paul declares in I Corinthians 12:7, any private use of any of the gifts of the Spirit is forbidden and is wrong. Since Paul exercised the gift of tongues more than all the Corinthians (14:18) yet refused to exercise the gift in church (14:19), it must be assumed that he prayed and sang with the Spirit in his own private devotions. 

Since the gift of speaking in tongues builds up the believer who is exercising the gift, and it is preferable that he not exercise the gift in a public assembly and is even forbidden to do so if there is no interpreter, the only place left for the exercise of this gift is in his own personal devotions. Paul said, "Speak unto yourself and unto God," so that it is proper to exercise this gift for one's own edification, as Paul did. 

As you are built up in Christ, you will become an instrument through which the whole body may be edified. For when one member of the body is honored, all the members rejoice with it. 

\section*{The Abuse of Tongues}

Some people say they have no control over their outbursts in tongues, and so many times they just start speaking in tongues in a public service, interrupting the sermon. Sometimes these erratic outbursts come in conversations with friends. I had a lady tell me that when she received the gift of tongues, she had no control over it. The next day, when the gas man came to read her meter, she went out to talk to him about a problem with the service. Then she started speaking to him in tongues. He looked at her rather weirdly, and finally turned and hurried out of the yard. She told me she could not control her speaking in tongues. Scripture tells us in I Corinthians 14:32 that the spirits of the prophets are subject to the prophets. I believe this is declaring that we are always in control of ourselves when exercising any of the gifts of the spirit. Paul instructs us in I Corinthians 14:28 that if there is no interpreter present, the person should keep silent in the church, and speak to himself and to God. Paul is calling for control over the gift; he is saying that a person does not have to speak out, that it is possible for him to just speak to himself and to God. In verse 15 Paul also declared that he would pray with the Spirit, and he would pray with understanding also, showing that the speaking in tongues was actually controlled by the exercise of his own will. When he willed, he could speak in tongues; when he willed, he could speak in one of the languages that he understood and knew. 

In some church services the sermon is interrupted by an utterance in tongues. But there is no Scriptural basis at all for these types of interruptions. In fact, Paul said, "Let all things be done decently and in order." I cannot see that these kinds of interruptions are ever in order. They are, on the other hand, very rude and extremely distracting. There is really no need for the Holy Spirit to bring forth an utterance in tongues during the ministry of the Word of God, for the minister himself should be speaking by the anointing of the Holy Spirit and exercising, as it were, the gift of prophecy as he is speaking forth God's truth to the people. When a person stands up and interrupts God's messenger, he is putting the Holy Spirit in the awkward position of interrupting Himself to interject another thought or idea. Such unscriptural uses of the gift of tongues is another form of \emph{charismania}.

\section*{Tongues and Interpretations}

Unquestionably, Paul is seeking to restrict the use of the gift of tongues in the church. There has developed what I feel to be a false concept of "messages" in tongues, as though God has a special message for the church to be given through tongues and interpretation. This is how they are usually referred to when they are given in church - as a message in tongues. There is not one single instance in the New Testament that we can point to as an example of where God spoke to anyone through tongues and interpretations, or just through tongues themselves. 

Most often, when there is a public utterance of tongues which is to be followed by an interpretation, rarely is a true interpretation of the tongues given. 

I grew up in a Pentecostal church, and I am convinced that during all of my years in the church, I rarely heard a true interpretation of the multitudes of utterances in tongues. If I ever did in those early years, I am not aware of it. There were long utterances in tongues followed by short interpretations. There were short utterances in tongues followed by long interpretations. It was always explained to me that there is a difference between interpreting and translation, which I readily accept. However, I would note that, in the utterance in tongues, oftentimes it would be just one phrase repeated over and over again. Yet the supposed interpretation would have no repetition of a phrase. 

\section*{To Whom Are Tongues Addressed?}

In I Corinthians 14:2 Paul tells us that he who speaks in an unknown tongue "does not speak to men, but unto God, for no man understands him; howbeit, in the Spirit he is speaking mysteries [or divine secrets]." Here he points out that tongues are definitely addressed to God. In all cases of the use of tongues in the New Testament, we find that they were addressed to God. On the Day of Pentecost in Acts 2:11, those who could understand the languages were remarking that these people were declaring the wonderful works of God. They were not using tongues to preach, but were using them to glorify God as they were declaring His glorious works. In I Corinthians 14:14 Paul declares that tongues are used in his prayers to God. In 14:16 he declares that they are used to bless God, and finally, to give thanks unto God. 

But there is not a single reference where the gift was used to address man by either preaching or teaching; we find that they were always addressed to God, it would of necessity follow that a true interpretation would also be addressed to God. The interpretation would be of the prayer, thanksgiving, praise, or declaring of the glory of God. It would often sound as one of David's Psalms declaring God's glory. Paul said, "If you give an utterance in tongues and there is no interpreter, how is the person sitting in the seat of the unlearned going to say yes and amen at your giving of thanks, seeing he does not understand what you say?" 

Notice that Paul declares that you are giving thanks to God, not giving a message to the church; but I cannot even say amen to your giving of thanks if I do not understand what you are saying. Thus there is the necessity of the interpretation if there is a public utterance of tongues, so that the whole body might be edified. 

\section*{Tongues Versus Prophecy}

In contrast to this, Paul tells us that he who prophesies is speaking to men for edification, exhortation, and comfort (I Corinthians 14:3). As I have studied this definition of prophecy, I have concluded that most of the so-called interpretations that we hear in the Pentecostal or charismatic services are actually the exercise of the gift of prophecy, because they so often run along the line, "Thus saith the Lord, 'My little children, call upon my name' or 'Praise me.'" They are exhorting the people to praise, give thanks, worship, or they are comforting the people in the goodness and grace of God. When the words are addressed to the church to edify, or comfort, this falls in the category of prophecy rather than interpretation of tongues. 

I have concluded that when a person gives an utterance in tongues, rather than praying that there might be an interpretation, so often the prayer is, "O God, speak to us." If God speaks to us through a gift of the Spirit, it is usually through the gift of prophecy, word of wisdom, or knowledge. We find so often that the utterance in tongues gives faith to the person with the gift of prophecy, and he stands up and exercises his gift of prophecy rather than giving an interpretation of what was said in the tongues. 

Paul declares that, if everyone is speaking in tongues in church and a stranger would come in, he would say that everybody is crazy. Paul also restricts the use of tongues in church to two or at the most three utterances, and those in turn. If there is no one with the gift of interpretation present, Paul completely forbids the public use of tongues, telling the person that he should keep silent and speak to himself and God, which also implies a person's control over the exercise of the gift. 

\section*{One Result of Tongues}

Several years ago, when Calvary Chapel of Costa Mesa was quite small, we were meeting on Sunday nights in a clubhouse. On a particular Sunday evening (which was Pentecost Sunday), at the close of the lesson as we were softly worshipping God together, I asked one of the ladies in the fellowship if she would worship God in the Spirit, since I knew that when she spoke in tongues she usually spoke in French. As she began to worship God, I could understand enough of her French to know that she was thanking God for her new life in Christ and the beautiful new song of love He had given her. I thought this was especially beautiful, as she used to be a nightclub singer prior to her conversion. At the conclusion of her worship in the Spirit, my wife began to give the interpretation to the group, and knowing that she does not know French, I was particularly blessed to hear how accurately the worship with the Spirit was being interpreted for the fellowship. 

After the meeting one of the young men in the fellowship brought a Jewish girl from Palm Springs for counseling. When we sat down together, she said, "Before we get to my problems, explain to me what was happening here tonight. Why did the one lady speak to God in French, and the other lady translate to the group what she said?" I said, "Would you believe that neither of those ladies knows French?" I told her that I knew for a fact that neither knew French, since one of them was a close friend and the other was my wife. I then showed her in I Corinthians where it speaks of the gift of tongues and interpretation. She then told me that she had lived in France for six years, and that the French spoken was in the perfect accent of what she called the Aristocratic French. She also stated that the translation was perfect. She then said, "I must accept Jesus Christ now, before we go any further." 

It was my joy to see her find her Messiah and become a member of the body of Christ. There was a demonstration of the gift of tongues, followed by the true interpretation, which was glorious praise and worship of God. The result was the edifying of the body and in this case the conversion of this Jewish girl. 


