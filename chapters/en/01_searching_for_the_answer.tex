\chapter{Searching for the Answer}

I spent much of my childhood and adolescent years
trying to prove I was normal even though I didn’t go to
movies or dances. In the Pentecostal church I attended,
movies and dancing were considered horrible sins.

Since I couldn’t join my friends in their worldly activities,
I asked them to attend church with me, for we were
constantly being exhorted to witness for Christ by bringing
friends to church. The problem was that almost every
Sunday the pastor would warn of the evils of Hollywood,
dancing, drinking, and smoking. He used to say, “If God
wanted man to smoke, He would have put a chimney on
top of his head.” Besides this, the service was always interrupted
by two or three “messages in tongues” and interpretations.

Many times as I was seated with my unsaved friends
that I had brought to church, Mrs. Newman would start
breathing funny. I had learned that this was the prelude
to her speaking in tongues, so I would quickly pray, “Oh,
God, please don’t speak in tongues today; my friends
won’t understand.” Either God wasn’t hearing me or Mrs.
Newman wasn’t listening to God, because she would stand
up, shaking all over, and deliver God’s message for the
day in a loud, high-pitched voice. I would die inside as my
friends giggled beside me. I hoped they weren’t committing
the unpardonable sin.

I was always tense after the service as I waited for my
friends to ask the inevitable question, “What was that?” I
had a hard time explaining it because I didn’t fully understand
it myself.

As a child I couldn’t help but wonder about these “messages
in tongues” that I heard. Sometimes a short message
was followed by a long interpretation or else a long message
was followed by a short interpretation. At other times
I would notice repeated phrases in the message in tongues
and wonder why there weren’t correspondingly repeated
phrases in the interpretations.


\section*{The Mounting Questions}

There were other things that bothered me about the
church I attended. I wondered why, if we were the most
spiritual church in town and had the most power, the other
churches had so many more members. I was told that most
people were looking for an easy way to heaven, and that
the other churches were larger because they told the people
what they wanted to hear. If our church did that, it would
be full too—of people bound for hell.

Another problem I had with our church was its lack
of love. I knew that the fruit of the Spirit is love, so I
couldn’t understand why there were so many church splits.
It seemed that there were always some members who
wanted to get rid of the pastor and others who supported
him. People left our church so often that, if all the former
members of the church returned, we would have had the
largest church in town!

Somehow, leaving our church was tantamount to leaving
the Lord. Those who had left had surely backslidden in
their search for an easier way to heaven. However, I often
found myself wishing I could go to the Community Church
or the Presbyterian Church. Then on Sunday night I would
feel convicted for my desire to “backslide,” and I would go
forward to the altar and get “saved” again.

I tried to prove that I was normal by excelling in school.
I worked to be the smartest kid in the class, the fastest
runner in the school, and the one who could hit the ball
farther than anyone else. Unfortunately, most of the other
kids in my Sunday school class tried to prove they were
normal by smoking, drinking, and running around with
the tough gangs at school. Very few of them remained in
Sunday school past junior high. Through the grace of God,
and with deeply committed parents, I somehow survived.


\section*{The Results of My Quest}

As strange as it may seem, I am convinced today that
the dead orthodoxy of many churches could be enhanced
by the gifts of the Holy Spirit in operation within the body.
Not the unscriptural excesses I observed as a child, but the
gifts exercised in a solid, scriptural way, with the Word of
God as the final authority guiding our faith and practice.

With this in mind, I began a search of the Scriptures
for a sound, balanced approach to the Holy Spirit and His
work in the church today. There must be a middle position
between the Pentecostals, with their overemphasis on experience,
and the fundamentalists, who, in their quest to be
right, in too many cases have become dead right. The results
of my quest are recorded in part in this book, which I pray
that God might use to lead you into the fullness of the
Spirit-filled life.

\emph{Charisma} is a beautiful, natural anointing of God’s Spirit
upon a person’s life, enabling him or her to do the work of
God. It is that special dynamic of God’s Spirit by which a
person seems to radiate God’s glory and love.

\emph{Charismania} is an endeavor in the flesh to simulate charisma.
It is any effort to do the work of the Spirit in the
energies or abilities of the flesh—the old, selfish nature of
a person. It is a spiritual hype that substitutes perspiration
for inspiration. It is the use of the genius, energy and gimmicks
of man as a substitute for the wisdom and ability
of God. It can be demonstrated in such widely divergent
forms as planning and strategy sessions, devising programs
for church growth, raising funds for the church budget, or
wild and disorderly outbursts in tongues that disrupt the
Sunday morning message. Whatever lacks a sound biblical
basis and demonstrates a lack of trust in the Holy Spirit
to accomplish His purposes in the church apart from the
devices and abilities of man is the work of the flesh.


\section*{The Balanced Position}

This book will seek to present a scripturally balanced
position between the detractors who say, “The devil makes
them do it” and the fanatics who say, “The Holy Spirit made
me do it.” It will also show who the Holy Spirit is and will
describe His proper work in the world, the church, and the
life of the believer.

We do not ask you to blindly accept all the premises,
but we encourage you to search the Scriptures to see if
these things are so. “Prove all things; hold fast that which is
good” (1 Thessalonians 5:21).















