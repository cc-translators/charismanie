\chapter{Why Charisma Often Becomes Charismania}

In Ephesians 4 Paul tells us that God has placed in the church certain gifted men, such as pastor teachers, to perfect the saints for the work of the ministry and to build up the body of Christ. The end result of sound teaching is to bring the believers into a fully matured state so they will not be carried about with every wind of doctrine. 

One of the greatest weaknesses of the charismatic movement is its lack of sound Bible teaching. There seems to be an undue preoccupation with experience, which is often placed above the Word. As a consequence, charismatics have become a fertile field for strange and unscriptural doctrines proliferating through their ranks. 

It is of utmost importance that we allow the Bible to be the final authority for our faith and practice. Any time we begin to allow experiences to become the criteria for doctrine or belief, we have lost Biblical authority, and the inevitable result is confusion. There are so many people today who witness of remarkable and exciting experiences. The Mormons, for example, "bear witness" to the experience of the truth of the Book of Mormon. They encourage people to pray in order to experience whether or not their Book of Mormon is true. One person says he has experienced that it is true, and another says he has experienced that it is false. Which one am I to believe? Each swears he has had a true experience from God; yet one has to be wrong. Whenever you open the door for experience to become the foundation or criterion for doctrinal truth, you are opening a Pandora's box. The result is that the truth is lost in the conflicting experiences, and the inevitable consequence is total confusion. We know that God is not the Author of confusion. 

\section*{Slain in the Spirit?}

One of the experiences that is quite common among charismatics is the practice of being "slain in the Spirit." I have never discovered the supposed value of this experience. Yet it is quite a common occurrence among charismatics. When pressed for a Scriptural basis, they usually mention the soldiers who came to arrest Jesus in the garden. When Jesus asked them, "Who are you looking for?" they responded, "Jesus of Nazareth." He answered them, "I am He," and they fell backward to the ground. But note that they were unbelievers, not Spirit-filled members of the body of Christ. (There is no indication that they ever became believers.) This is certainly not a Scriptural basis for the practice among believers today. 

Often charismatics refer to the Apostle Paul on the road to Damascus. Again, Paul at that time was an enemy of Christ. There was no evangelist or pastor laying hands on him, nor do we ever read of the experience being repeated after his conversion. Paul also had a vital personal encounter with Jesus Christ as the Lord spoke to him audibly during his experience. 

When I was young I attended many services where people were supposedly being slain in the Spirit. I often had hands laid on me; quite often there was a gentle pressure exerted on the forehead, pushing me backward. With some of the evangelists, it was not quite so gentle. If you stand with your eyes closed, hands lifted, and your head tilted back, it does not take much pressure for you to go over backward, especially if you know that someone is standing behind to catch you! 

\section*{Exorcising Demons?}

Another popular pastime among many charismatic groups is the discerning and exorcising of demons out of each other. Numerous books and articles have been written on this subject by their recognized leaders, and a whole doctrine developed upon the basis of experiences alone. One of the evangelists who was thought to be especially gifted in this ministry reportedly began to pass out Kleenex tissues in his services so that the people might regurgitate the demons into the Kleenex! If in a meeting one of the group began to yawn, this was a sign that he was possessed by a demon of lethargy. To burp would invite the exorcism of the demon of gluttony, which would invade you the moment you ate one bite more food than you needed. Much damage has been done to sensitive people through this pernicious doctrine, and scattered across the world today are many tragic victims of its aftermath. 

In one of the books I read on this subject, the author spoke of how we were to consign these demons to the pit when we cast them out. And how did he know we had the power to send them to the pit? While he was in conversation with a demon, prior to casting it out, the demon begged him not to send it to the pit. He then asked the demon if he had the authority, and the demon answered yes. He thus declared on the authority of what the demon told him that he could command all demons to go to the pit. If Satan is the father of all lies, how could you trust the word of one of his emissaries to be true? Here a doctrine was being based upon the supposed word of a demon. 

\section*{Bible Doctrine Versus Demon Doctrine}

Paul warned against doctrines of demons in the last days. This whole doctrine and practice was developed entirely upon experiences, with no solid Scriptural foundation. Many people have testified to me of the great victory they experienced after having been delivered from some demon. Should we then believe that we can have victory over our flesh life by having the demon of lust cast out? Does the Bible teach that I as a child of God can be possessed by a demon, and are there instances in the New Testament where in the church gatherings they cast demons out of each other? To the contrary, there are passages that indicate that a child of God cannot be possessed by demons. 

Paul, writing to the Corinthians, said that our bodies are the temples of the Holy Spirit which is in us (I Corinthians 6:19). He also asked what communion light has with darkness, and what concord Christ has with Belial, and what agreement the temple of God has with idols (II Corinthians 6:14-16). In I Corinthians 10:20 he identifies idols with demons, and in verse 21 he declares, "You cannot drink the cup of the Lord and the cup of devils." When faced with these Scriptures, many charismatics who were following these practices developed the doctrine that the demons could invade the believer's mind but not his spirit. This concept is also without Scriptural basis. There is no account in the Scriptures of a born-again believer in Jesus Christ being exorcised of a demon. 

That demons can and do possess the bodies of unbelievers is an accepted fact of Scripture, and that they can be exorcised through the authority of the name of Jesus is also evident. But to believe that a child of God can be freed from the problems of the flesh (such as lust, anger, and envy) by exorcism is charismania. 

\section*{Written Versus Spoken}

Many charismatics seem to prefer the spoken word over the written Word, and they seek to show the power of the rhema over the logos. The ministry of the prophet or exhorter is preferred above that of the teacher. The anointing of the Spirit is preferred above that of the teacher. The anointing of the Spirit is recognized not so much by the truth that is unfolded as by the fervency and excitement displayed by the speaker. If the voice is loud and pitched high, and the speech very forceful and rapid, this is the sign of the true anointing, especially if he sucks a lot of air between phrases and throws in amens and hallelujahs between thoughts! Some of the more adept evangelists have developed great skills in whipping the people into a high state of excitement bordering upon hysteria by just repeating a single phrase, such as "Praise the Lord," using different voice intonations. 

Because of the preference for the spoken word, tongues with interpretation or a prophetic utterance is desired above the preaching or teaching of the scriptures. In many charismatic fellowships, if there have not been manifestations of these vocal gifts, the people do not recognize or acknowledge the moving of the Spirit in that service. I have often heard people say that the Spirit moved in such a powerful way in the service that the preacher did not even have an opportunity to speak. This idea is used to express the "ultimate" in the moving of God's Spirit. 

\section*{Spiritual Versus Soulish}

Much of man's worship experience is more soulish than spiritual. Within typical church liturgy there is much that appeals to the soulish nature of man. The ornate robes, the choral chants, the candles and incense - all move me to a pleasant psychic experience of reverence. On the other end of the spectrum, the uncontrolled release of the emotions, with shouting, clapping, and dancing, moves other people to strong psychic experiences. It is possible that neither truly touches my spirit. In Hebrews 4:12 we read that the Word of God is sharper than a two-edged sword, and is able to divide between the soul and the spirit. It is God's Word that ministers to the spirit of man and feeds the spirit. So if the minister is not given the opportunity to share God's Word, we must legitimately question if the soul of man or the Spirit of God was moving in the service. It is sad that unscriptural excesses are so freely tolerated among charismatics. Often, hungry and sincere believers who sense a lack of power in their lives will go to their services with an open, hungry heart seeking God's power, but when they observe the absence of solid Scriptural content and the presence of distasteful fleshly manifestations, they turn their backs on the whole valid and beautiful work of God's Spirit that a person can experience in his life. 

\section*{Exalting the Flesh}

I know that in my flesh dwells no good thing. One of the greatest problems in my spiritual walk is my flesh. My flesh wants to be recognized and admired. The flesh will even seek glory and attention in a spiritual atmosphere. My flesh wants people to think that I am more spiritual than I really am, that I pray more than I really do. It makes me feel good when someone says, "You know the Word so well; have you memorized the whole Bible?" I like to hear them say, "You are such a great man of prayer," even though I know I am not. 

Jesus warned us in Matthew 6 to take heed that we do not perform our righteous acts before man to be seen of man. This "to be seen of man" is a strong motivating force, and I must continually guard against it. Jesus then mentioned that the desire to be seen of man is behind much of our giving, praying, and spiritual activities, such as fasting. We are told in the Scriptures that one day all our works are to be tested by fire to be determined of what sort they are, or the motives behind them. It is very wise for us to examine our motives, for if we will judge ourselves, we will not be judged by God. 

Many of the things done in charismatic services are done to draw attention to the individual. The person who shouts "Hallelujah!" and thrusts his hand upward is drawing attention to himself, and many times distracts those who are truly worshipping God. As the group sings choruses of praise, often one or more persons will stand with eyes closed and hands upraised while the others are still seated. This looks very spiritual, as does praying on a street corner, but it is drawing attention to oneself, and the moment you draw attention to yourself, you are taking it away from Jesus. 

The methods by which the offerings are often received are designed to honor the flesh, and totally rob the poor soul from the reward of God. I have heard evangelists say that God told them that ten people were going to give a thousand dollars that night, then would rant and rave and threaten the people until the ten were standing to their feet. As each would stand, attention would be given to him or her, and applause would be encouraged. As the crowd applauded I felt sick in my heart, and I thought, "Enjoy it and take it in, poor soul, for this will be the only reward you will receive for that gift." As Jesus said, "You have your reward." I also felt anger toward the pastor or evangelist who would encourage people to give in such a way as to receive no reward from God. I also felt that he lied when he declared that God had told him how many people were going to give a thousand dollars. That is nothing more than a psychological ploy. 

\section*{Distasteful Ploys}

Equally distasteful are the other psychological ploys that are used to solicit funds to support the work of God. Many of the charismatic evangelists have developed mailing lists, and with the use of computer typewriters they send out their mass mailings to their gullible followers, many of whom are deceived into thinking that they are receiving a personal letter from dear Brother so-and-so (some famous-name healing evangelist), for the computer has repeated their name many times in the body of the letter. The letters are filled with lies, as they often say, "The Lord laid you on my heart this morning to pray especially for you. Is something wrong? Please write me and tell me your need so I can help you." 

In II Peter 2 we are told that one sign of a false prophet is that he will use feigned words to make merchandise of the people. These letters fit Peter's description perfectly. They so often appeal to the flesh. If you want answers to your prayers, or a special work of God, then plant your seed gift. These methods deny the grace of God, since you are supposedly buying God's favor. It has always puzzled me how these men who have learned all the secrets of faith and have such great power with God never seem to have enough faith to trust God to take care of their own needs, but warn that God's work is going to fail unless the people come to His aid immediately and save Him from bankruptcy. 

\section*{What You Say Is What You Get?}

The latest wind of pernicious, unscriptural doctrine to blow through the ranks of some charismatics is the "what-you-say-is-what-you-get" teaching, otherwise known as the prosperity doctrine. Among the claims that are commonly made is that God never wills that you should be sick and that all sickness is the result of ignorance or lack of faith. Such teachings sound more like Mary Baker Eddy than the Apostle Paul! 

These people speak much of making positive confessions, and warn against any negative confession. They teach that the spoken word becomes a spirit force for good or evil according to the confession. Thus you are never to confess, "I don't feel well," for that is a negative confession and is bound to cause you to feel bad. You are thus encouraged to lie about your true condition or feelings. As you hear this teaching you would swear that the sermons came from Science and Health with Key to the Scriptures rather than the Bible. 

I have heard such people seek to explain away Paul's thorn in the flesh by saying, "Where else do we find thorns in the Bible?" "Jesus," they said, "spoke of the thorns choking out the Word so that the seed could not produce." Now what were these thorns? The cares of this life, the deceitfulness of riches, and the lust for other things. Therefore Paul's thorn in the flesh was the cares of life that he took upon himself. 

Had these people bothered to do the slightest amount of research, they would have discovered that there are two entirely different Greek words translated "thorns" in these passages. The word that Paul used about his thorn in the flesh was a Greek word that referred to a tent stake, not some nagging little irritation. Paul spoke to the Galatians about his infirmities; the English word has the same root as "infirmary," or what we call a hospital. 

One of these charismatic leaders said to me, "Inasmuch as this was given to Paul lest he be exalted above measure, don't you think that if Paul could have just conquered his flesh, the thorn would not have been necessary?" I cannot imagine the spiritual pride insinuated by such a remark. In essence, he was declaring that he had conquered over his flesh more thoroughly than Paul. This surely was not apparent by the flashy clothes, car, and home that he possessed. Yet he said that all of this lavish living was just a sign of his faith, for if God could trust us with money, He wanted us all to prosper, and anyone with enough faith could live like the King's kid. 

What does that say about Jesus, who had no place to lay His head, and had to send Peter fishing in order to get a coin to pay the taxes? I know of many people who have died while making their positive confessions of healing. Some of them could have been helped by competent medical care, but to go to a doctor would be a negative confession and an admission that something was wrong. In other cases I know of people who followed the lies of the positive-confession evangelists, and when their confessions failed to materialize, they turned their backs on God completely. I also know that some of the evangelists who are the chief exponents of this positive confession as the way to constant health and continuous prosperity have spent time in the hospital for nervous exhaustion. 

The people who seem to have prospered the most from these teachings are the evangelists themselves. How will they answer to God for conning the poor little widow out of half her Social Security check, causing her to miss several meals for lack of funds so they can fly in their private jets to their luxury condominiums in Palm Springs and dine in the plushest restaurants? 

Paul writes to Timothy about the perverse teachings of men of corrupt minds who are destitute of the truth, for they suppose that godliness is a way to prosperity. Paul warned Timothy to stay away from them. This is a free but accurate translation of the Greek text in I Timothy 6:5. Paul then told Timothy that godliness with contentment was great riches. 


