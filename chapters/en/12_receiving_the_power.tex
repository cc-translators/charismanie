\chapter{Receiving the Power}

There are often obstacles to receiving this special anointing or power of the Holy Spirit in your life which must be overcome! First is a general sense of unworthiness. Satan, fulfilling his role as the accuser of the brethren, will try to exaggerate our failures and mistakes, and he intimates that we are not worthy to receive anything from God. In one sense this is true; however, God does not give us His gifts as a reward for good behavior, but to enable us to live a life pleasing to Him. The power of the Holy Spirit is the very power that I need to help me live a victorious life in Christ. Also, inasmuch as it is a gift from God, He gives it on the basis of His grace and not my merit. 

Another obstacle is unscriptural anticipations that we may have developed from preconceived ideas which are often planted by the testimony of someone else's experience. For years I thought that, when I received the empowering of the Spirit, I would lapse into an unconscious state or some kind of trance. I had heard the testimonies of those who were filled with the Spirit, and they often declared, "...and when I came to, I was surprised to discover I had been there for four hours." Thus, while tarrying for the Holy Spirit, I often waited in vain to lapse into some unconscious state. Others would testify of various sensations, such as "ten thousand volts of electricity passing through my body" or "a warm sensation came over me." Still others described the continuous waves of glory sweeping over them or the tingling sensation down their spine. Some told of uncontrollable weeping, while others spoke of violent shaking. 

All of these may be valid reactions to the work or power of the Spirit upon a person's life, but the wide variety only shows that God is not bound to any pattern in bestowing the gift of the Holy Spirit upon our lives. We should not anticipate any special kind of sensation as proof that God has filled us with His Spirit apart from that full flooding of love, for the fruit of the Spirit is love. 

Quite often, if I am anticipating some special kind of reaction or sensation, I am disappointed when I do not receive it, and I feel that God has not bestowed His gift upon me. I am prone to doubt my own experience, or lack of it, and take it as God's refusal to bless me. 

\section*{Ask and Receive}

If we are to receive the gift of the Holy Spirit, we must ask. In Luke 11:13 Jesus said, "How much more will your heavenly Father give the Holy Spirit to those who ask Him?" Asking is a very important part of receiving. In James 4:2 we are told that we have not because we ask not. Many people are lacking the power of God's Spirit in their lives today simply because they have never asked. In John 15:16 Jesus said to His disciples, "You did not choose me, but I chose you and ordained you, that you should go and bring forth fruit, and that your fruit should remain; that whatever you shall ask the Father in my name, He may give it to you." Notice that Jesus said, "He may give," not "He shall give." It is something that God has purposed to do already, and the asking just opens the door for Him to do what He is longing to do for you. 

In John 16:24 Jesus said, "Ask and you shall receive, that your joy may be full." The Spirit-filled life is full of joy. Joy is the first word Paul uses to define the love which is the fruit of the Spirit (Galatians 5:22). In I John 5:14,15 we are told that, if we ask anything according to God's will, He hears us, and if He hears us, then we have the petitions that we desired of Him. Is it God's will that we be filled with the Spirit? We know it is, for God commanded it in Ephesians 5:18: "Be ye being filled with the Spirit." When I ask that God fill me with the Spirit, I have that confidence of knowing that I am asking according to His will. 

Anything I ask of God I must ask in faith, believing that God will do it. So my next step must be the step of faith; I must believe that God has done it. Faith is the substance of things hoped for, the evidence of things not seen. Faith is all the evidence you need; believe that you receive it, and you shall have it. We need not look for any immediate sign such as tongues, hot flashes, or waves of glory. They may occur, but not necessarily, and I should not look for some feeling as proof that God has granted my prayer. Our faith must always rest in the sure Word of God, and never in a feeling. Our feelings often change, but never God's Word. 

Paul asked the Galatian believers, "Received ye the Spirit by the works of the law or by the hearing of faith?" The same is true in our lives. The filling of the Spirit is not some reward that God gives me for meritorious service, but just the pure gift of His grace. In Romans 4:20 we read that, because Abraham was strong in faith, he gave glory to God. Ask God now to fill you with His Holy Spirit, and begin to exercise your faith by praising God now for that new dynamic of love He is pouring into your life. 


