\chapter{The Work of the Spirit in the Life of the Believer}

What is the intended work of the Spirit in the life of the
believer? As we have already noted in John 14, His name
“Comforter” indicates His coming alongside us to help us.
I have not found my Christian walk to be easy. I find that
my flesh fights me all the way. With Peter, I too have often
discovered that the spirit indeed is willing, but the flesh is
weak. I understand what Paul was talking about in Galatians
5 as he spoke of the warfare between the flesh and the
spirit. If God has help for me, I’m ready for it; I want all the
help I can get! I never want to set limits on what God wants
to give me, or what He wants to do in my life. I do not want
to be guilty, as were the Israelites in the wilderness, of limiting
God (Psalm 78:41). By the same token, I am not looking
for experience for experience’s sake; I want only the genuine
work of the Holy Spirit, but I want it all.


\section*{Trusting the Spirit}

In the discourse of Jesus to His disciples, beginning in
John chapter 14, He is seeking to prepare them for His
departure. He is talking much about His leaving them and
returning to the Father. He also speaks much to them of the
provisions which the Father and He had made for them by
the power of the Holy Spirit. He would be there to help
them. As they had learned to trust in Jesus for any situation
or emergency that might arise, they must now learn to trust
in the Holy Spirit. He will now be their Helper.

Jesus spent His three years with His disciples teaching
them the truth of God. Now their Teacher is departing to
return to the Father, but the students will not be on their
own; this Helper, the Holy Spirit, will now teach them all
things and will recall to their remembrance all the things
that Jesus had said to them (John 14:26).

Perhaps you’ve had the experience of sharing the gospel
with someone when he or she asked a question that immediately
stumped you; but as you began to answer, the Scriptures
started coming into your mind, and you were pleased
and satisfied with the answer you gave the person. This is
the \emph{recall work} of the Spirit.

The Holy Spirit helps us to understand the things of
God. Many times I have been frustrated in an attempt to
explain some spiritual truth to a non-believer. It seems so
clear and obviously evident, yet he or she cannot seem to
grasp it. If you are dealing with things concerning the Spirit
of God, the natural man “receiveth not the things of the
Spirit of God: …neither can he know them, because they are
spiritually discerned” (1 Corinthians 2:14).


\section*{Dead Versus Living Spirit}

While attending college I had a sociology professor who
believed in the dichotomy of man, and I believed in the trichotomy
of man. Many times we expressed our differing
views to each other. I was frustrated that he could make no
distinction between the soul and spirit of man, but believed
them to be synonymous. One day as I was leaving class
frustrated after another discussion in which he seemed to
have deliberate blindness, it was as though the Holy Spirit
brought 1 Corinthians 2:14 to mind. Then I realized that as
an unregenerate person his spirit was dead, so I was speaking
to him of mysteries which he could not know. He did
not and could not know of the spirit of man until he was
born of the Holy Spirit. In 1 Corinthians 2:15 Paul says, “He
that is spiritual [understands] all things, yet he himself is
[understood] of no man.”

Anyone who lives only on the body conscious plane is
living on the animal level of existence. His mind is ruled
and dominated by his body needs; he does not understand
the things of the Spirit, for his own spirit is dead. No
wonder he seeks to relate himself to the animal kingdom,
for he is living as an animal, a body-dominated consciousness.
When a person is born again by the Spirit, his own
spirit comes alive and, now joined to God by the Spirit, he
is encouraged through the Word to live a Spirit-dominated
life. As he does, he begins to have a Spirit-dominated consciousness.


\section*{Letting the Spirit Lead}

As we live after the Spirit, our thought patterns are different
because we are now thinking of God and how we
might please Him and serve Him. The mind of the Spirit is
life and peace (Romans 8:6). What a tremendous help the
Holy Spirit is to us as He teaches us the things of God and
helps us to understand them! The Bible seems to come alive
with meaning and excitement as Scripture after Scripture
seems to almost jump off the page to minister to us.

In John 16:13 Jesus promised His disciples that when
the Spirit of truth had come He would guide them into all
truth, and would show them things to come. It is so necessary
to have that guidance of the Holy Spirit into all truth.
Jesus warned of false prophets that would be wolves, yet
appear in sheep’s clothing (Matthew 7:15). There are men
who come among the flock of God and appear to be part of
it, but whose main motive is to prey upon the flock. They
bring in damnable heresies and seek to draw men after
themselves. In Acts 20:29–30 Paul warned the elders of the
church in Ephesus, “For I know this, that after my departing
shall grievous wolves enter in among you, not sparing
the flock. Also of your own selves shall men arise, speaking
perverse things, to draw away disciples after them.”

Peter warns in 2 Peter 2:1, 3, “But there were false prophets
also among the people, even as there shall be false teachers
among you, who privily shall bring in damnable heresies,
even denying the Lord that bought them, and bring
upon themselves swift destruction.… And through covetousness
shall they with feigned words make merchandise
of you: whose judgment now of a long time lingereth not,
and their damnation slumbereth not.”


\section*{Sensing the Phony Prophets}

Note one of the marks of the false prophet: “…with
feigned words make merchandise of you.” I regularly
receive computer typed letters from noted evangelists that
fit Peter’s description perfectly. These letters will say such
things as, “Benny, God laid you on my heart this morning,
and I have been in prayer for you. I just can’t get you off my
mind, Benny; is everything all right? Is there some special
need that I can pray about? Please write me immediately,
because I love you, Benny, and I want to help you! Incidentally,
my ministry is in one of its greatest financial crises
ever. We are going to have to close down some of our great
works for God unless we get your help immediately. If
you don’t have fifty dollars to send me, maybe you could
borrow it elsewhere and help me keep God’s faith work
going. Plant your seed of faith today. God will help you
repay the loan you get. Your partner in faith.”

My name is not Benny, but somehow that is the way it
has gotten onto their mailing lists. These types of deceitful
letters seek only to make merchandise of the people, and on
the authority of God’s Word I do not hesitate to call their
authors false prophets.

This is charismania in one of its most blatant forms and
is practiced by most of the charismatic evangelists, especially
those who emphasize divine healing. I always marvel
that they can have such faith for my healing and so little
faith for their own financial needs.

It is beautiful to see how the Spirit will warn you when
someone starts to get off in his doctrine. Quite often you
cannot pinpoint the error immediately, but you know that
something isn’t quite right. The Spirit has been given to the
believer to guide him into all truth.


\section*{Learning Things to Come}

The Holy Spirit also shows us things to come. When
Daniel was seeking a fuller understanding of the time of the
end and the things he had written, he was commanded to
“shut up the words, and seal the book, even to the time of
the end: many shall run to and fro, and knowledge shall
be increased” (Daniel 12:4). As Daniel persisted in his questioning,
again the Lord said, “Go thy way, Daniel: for the
words are closed up and sealed till the time of the end.
Many shall be purified, and made white, and tried; but
the wicked shall do wickedly: and none of the wicked
shall understand; but the wise shall understand” (Daniel
12:9, 10).

It is by the help of the Holy Spirit that a clearer understanding
of the coming again of Jesus Christ has been given
to the church. Paul the apostle was shown by the Spirit
some of the things that were to come upon his life as he said
to the elders of Ephesus in Acts 20:22–23: “And now, behold,
I go bound in the spirit unto Jerusalem, not knowing the
things that shall befall me there: save that the Holy [Spirit]
witnesseth in every city, saying that bonds and afflictions
[await] me.” Later, as Paul continued his journey toward
Jerusalem, Agabus the Prophet took Paul’s belt, bound himself
with it, and said, “Thus saith the Holy [Spirit], So shall
the Jews at Jerusalem bind the man that owneth this girdle,
and shall deliver him into the hands of the Gentiles” (Acts
21:11). Here is a classic example of the Holy Spirit showing
Paul the things that were to come into his life.

Another example from Paul’s life of this work of the
Spirit is found in Acts 27:21–24: “But after long abstinence
Paul stood forth in the midst of them, and said, Sirs, ye
should have hearkened unto me, and not have loosed from
Crete, and to have gained this harm and loss. And now I
exhort you to be of good cheer: for there shall be no loss of
any man’s life among you, but of the ship. For there stood
by me this night the angel of God, whose I am, and whom
I serve, saying, Fear not, Paul; thou must be brought before
Caesar: and, lo, God hath given thee all them that sail with
thee.”


\section*{The Mighty Hand of God}

In a small prayer group we decided to pray one for
another; we had the person to be prayed over sit in a chair
in the center of the group. When it was my turn to be prayed
for, someone spoke a word of prophecy by the Holy Spirit
declaring that God’s hand of blessing was going to come
upon my ministry in a great way, that the people would
come to hear the Word in such numbers that there would
not be room in the church to contain them. The prophecy
went on to declare that I was being given a new name which
meant shepherd, for the Lord was going to make me a shepherd
of many flocks.

Up until this time I had been struggling for almost seventeen
years in the ministry with such limited success that
I was contemplating leaving the ministry for some other
type of work. The church I was then pastoring was running
around one hundred in attendance in spite of all our efforts
to increase its size through giving free hamburgers to everyone
who brought a friend to Sunday school. As these words
were being spoken, I was in my heart much like the man
upon whom the king leaned—who, when he heard Elisha’s
promise of God’s bountiful provision to come upon the
starving inhabitants of Samaria, said, “If God should open
windows in heaven, could such a thing be?”

Fortunately, God was merciful to me, for my fate was
not the same as his; I have both seen the fulfillment of the
prophecy and have been able to partake of it as we see the
greatly expanded church facility filled to over-capacity, not
just once, but three times on Sunday mornings, and as we
minister by cassette tapes and videotapes to hundreds of
Bible study groups all over the world.


\section*{Power to Conquer}

The work of the Holy Spirit in your life if you are a
believer is to give you the power to be a witness for Jesus
Christ, to give you the power to be all that God wants you to
be. One of the most frustrating things in the world is trying
to live the Christian life in the energy of the flesh. The Bible
speaks of the frustration in Romans 7, where Paul speaks of
how, when he tried to keep the law of God and to do good,
he found that “when I would do good, evil is present with
me. And the good that I would I do not. And that which I
would not I do. Oh wretched man that I am!” Paul describes
in Galatians 5 how the flesh wars against the spirit and the
spirit against the flesh, and how these two are contrary to
each other. Jesus said to Peter, “Watch and pray, that ye
enter not into temptation: the spirit indeed is willing, but
the flesh is weak” (Matthew 26:41). Because of the weakness
of our flesh, we cannot live the kind of life that the Lord
would have us to live and that we ourselves would like to
live before the world.

God desires that your life be a true representation of
Him in this world. God wants the world to see Jesus Christ
in you. He wants your actions and reactions to reflect Him.
He wants you to be His witness, representing Him. But if
you attempt to be His witness, if you try to react like Christ,
you’ll find how difficult and frustrating it is to do this—in
fact, how \emph{impossible} it is because of the weakness of the
flesh.


\section*{The Perfect Witness}

Many Christians find themselves in that frustration of
knowing what is right and wanting what is right, but somehow
not doing what is right. The Bible says of Jesus Christ
that He was the true and faithful witness; He was a witness
of the Father. If you want to know what God is like, just look
at Jesus Christ, because He was the true and faithful Witness.
When Philip cried out, “Lord, if You’ll just show us
the Father, we’ll be satisfied,” Jesus replied, “Have I been so
long with you, and yet hast thou not known me, Philip? He
that hath seen me hath seen the Father; and how sayest thou
then, Shew us the Father? Believest thou not that I am in the
Father, and the Father in me? …Believe me that I am in the
Father, and the Father in me, or else believe me for the very
works’ sake. Verily, verily, I say unto you, He that believeth
on me, the works that I do shall he do also” (John 14:9–12).
Jesus then gave the promise of the Holy Spirit to them.

Jesus faithfully represented God in every action. He
demonstrated to us that God is interested in the physical,
emotional, and spiritual welfare of man. God is interested
in your sufferings; God is interested in your sorrows. God
is interested in your pains; God is interested in your weaknesses.
Jesus never came upon a sorrowing scene without
bringing victory and joy to it. He never faced the weakness
of humanity without imparting the strength of God.


\section*{The Great Helper}

God wants to help you in your weakness, so He has
sent another Comforter, One to come alongside you and
help you. Jesus said, “Ye shall receive power when the Holy
Ghost comes upon you.” You will receive this dynamic.
When I think of the power of the Holy Spirit, it is first the
power to be what God wants me to be, and this extends into
every area of my life: power in my prayer life, power for a
holy walk, power to be and do. Here the promise is power,
and it is related to being a witness for Jesus Christ: “Ye shall
be witnesses.”

We make a mistake when we think of witnessing as
something we do; in reality it is something that we are. So
often witnessing is associated with passing out tracts on the
street corner, or going door to door to declare the gospel,
or sharing the four spiritual laws with our neighbor over
a cup of coffee. These are all forms of sharing our faith,
but doing them does not make us witnesses of Jesus Christ.
Being a witness is more than speaking words; it is living a
life. The word witness comes from the Greek word martus,
which transliterated into English is martyr. We think of a
martyr as one who dies for his faith; however, it is really one
whose life is so totally committed to his faith that nothing
will dissuade him from it, not even the threat of death. His
death does not make him a martyr; it only confirms that he
was truly a martyr. Many Christians testify of Jesus Christ
without ever being a true witness.


\section*{More Than Just Words}

What a person \emph{says} is often meaningless because he is
not living a life that backs up what he is saying. If you are
trying to share with someone the love that Jesus brings, yet
your life is filled with hatred, bitterness, and jealousy, he
will not respond to what you are saying because your life is
contradicting what you are saying. If you go around saying,
“You really need to know the joy of Jesus Christ; He’ll give
you such joy,” but you’re always depressed and pessimistic,
your life isn’t a witness of what you’re saying. People will
observe your depression and discount what you said. If you
say, “You need to know the Lord so you can have real peace
in your heart, the peace that passes all human understanding.
Receive Jesus and have peace”—but your life is torn up
and you’re constantly nervous and worried and filled with
anxiety, people will look at your anxiety and worry and
won’t hear what you’re saying about peace. Your words can
be totally drowned out by your actions. It is more important
that the activities of your life be a witness for Jesus Christ,
and then your words become meaningful. If your words
aren’t backed up by your life, your words really have no
good effect at all.

A lot of people think, “I’m a witness for Jesus—I go
down to the beach and pass out tracts. I share the four spiritual
laws wherever I go.” That doesn’t make you a true
and faithful witness. Your life must be in full harmony with
God, so that when people look at your life they say, “There’s
something different about that person.” \emph{Saying} it doesn’t
make you a witness; \emph{living} it does.

The Lord wants to give us the power to be a witness.
He will empower us through the Holy Spirit, for in our own
selves we are weak and failing. God wants us to be strong.
God wants us as witnesses for Him.


\section*{Peter as a Failure}

In Mark 14:53–54 we read, “And they led Jesus away to
the high priest: and with him were assembled all the chief
priests and the elders and the scribes. And Peter followed
him afar off, even into the palace of the high priest: and he
sat with the servants, and warmed himself at the fire.” As
we follow the story through to verse 66 we read, “And as
Peter was beneath in the palace, there cometh one of the
maids of the high priest: and when she saw Peter warming
himself, she looked upon him, and said, And thou also wast
with Jesus of Nazareth. But he denied, saying, I know not,
neither understand what thou sayest. And he went out into
the porch; and the cock crew. And a maid saw him again,
and began to say to them that stood by, This is one of them.
And he denied it again. And a little after they that stood by
said again to Peter, Surely thou art one of them, for thou
art a Galilean, and thy speech agreeth thereto. But he began
to curse and swear, saying, I know not this man of whom
ye speak. And the second time the cock crew. And Peter
called to mind the word that Jesus said unto him, Before the
cock crow twice, thou shalt deny me thrice. And when he
thought thereon, he wept” (verses 66–72).

Earlier that evening Jesus had said, “All of you are
going to be offended tonight because of me.” But Peter had
replied, “Lord, though they all be offended, I will never
be offended.” And Jesus had responded, “Peter, before the
cock crows twice you will deny me three times.” At this
point Peter began to get very vehement, and he said, “Lord,
even though they \emph{slay me}, I would never deny You!”

Peter thought he was a true martyr, and I believe he was
perfectly sincere. I know exactly how he felt when he made
his boast to the Lord, for his spirit was willing and he really
felt he had all it took to die for Jesus if necessary. But when
the chips were down, Peter didn’t have it; when this young
maid asked, “Weren’t you with Jesus?” Peter responded, “I
don’t know what you’re talking about!” Later she said to a
group standing there, “He is one of them,” and Peter denied
Jesus again. Then those standing by said, “Surely you are
one of them; you have a Galilean accent.” Then Peter began
to curse and swear, saying, “I don’t know the man!” Then
came the reminder of what the Lord had predicted: the
rooster began to crow. when Peter heard this he went out
and wept. How many times I have wept over my own
weakness and my own failure! I didn’t want to fail the Lord;
I didn’t want to let Him down; I really wanted to stand for
Him. But the pressure was too great, and I was not a witness—
I failed. How bitter is that failure; how hard it is to
realize, “Oh Lord, I failed You again.” We get to the place
where we don’t even want to promise Him anything anymore,
because we just know we’ll fail Him again.

I can identify with Peter; I know exactly how he felt
when he heard that rooster crowing. I know exactly that
misery—“Oh, God, I’m so sorry I failed You again.” Must
we go on forever in our Christian experience failing our
Lord? No. Thank God we don’t have to go on failing—He
has promised to us the power to be what we could never be
through our own strength or strong wishes.


\section*{Peter as a Witness}

A few weeks later Peter faced the same group of men
who had incited the murder of Jesus: “It came to pass on
the morrow, that their rulers, and elders, and scribes, and
Annas the high priest, and Caiaphas, and John, and Alexander,
and as many as were of the [family] of the high priest,
were gathered together at Jerusalem. And when they had
set them in the midst [that is, Peter and John and the lame
man] they asked, By what power or what name have ye
done this? Then Peter, \emph{filled with the Holy [Spirit]}, said unto
them, Ye rulers of the people, and elders of Israel, if we this
day be examined of the good deed done to the impotent
man, by what means he is made whole, be it known unto
you all, and to all the people of Israel, that by the name of
Jesus Christ of Nazareth, whom ye crucified, whom God
raised from the dead, even by Him doth this man stand here
before you whole. This is the stone which was set at nought
of you builders, which is become the head of the corner.
Neither is there salvation in any other: for there is no other
name under heaven given among men whereby we must be
saved” (Acts 4:5–12).

When they saw the boldness of Peter they marveled
(Acts 4:13). This is a different fellow from the one who a few
weeks earlier stood on the porch of the palace and denied
his Lord. What a different man! Reading the two accounts,
you wouldn’t believe it was the same person. What made
the difference? The difference lies in that little phrase “filled
with the Holy Spirit.” Jesus had said to His disciples, “They
are going to bring you before the magistrates and the
judges, and when they do, do not worry about what you
are going to say. Don’t make any little prepared speeches,
for in that hour the Holy Spirit will come upon you, and it
will be the Holy Spirit who speaks through you. You shall
receive power; you shall be witnesses.” The Holy Spirit is
the Helper, the One who helps you to be all that God wants
you to be—a true and faithful witness of Him.


\section*{The Only Source of Power}

The Holy Spirit gives us the power to be a true and
faithful witness of Jesus Christ—the power of truly representing
Him on the job, in the home, or in the classroom—so
that when people look at us they will see the love, the peace,
and the beauty of Jesus Christ in our actions and attitudes.
They will see a person who is at peace in the storm. That is
the power that we need if we are to be His witnesses, for we
cannot be a true witness of Him in our own strength or ability;
without the power of the Holy Spirit we will fail every
time the real issue arises and the pressure is on. It is not
until we learn to rely completely on the Holy Spirit that we
experience this power.

One of our most common mistakes is that when we see
an area of weakness in our life we immediately try to compensate
for it and to correct it ourselves. We say, “I’m sorry,
Lord; I’ll never do that again. I promise you, Lord.” We
mean what we say, yet we do it again. The problem is that
we are trying to correct the issue ourselves, thinking that
somehow, if we will only work at it a little harder or try
a different approach, we can change and correct the weaknesses
of our own character and nature.

It is not until we are brought to the total desperation
of the helplessness of ourselves, and give up and surrender,
that we know the joy of \emph{His} victory. It wasn’t until Paul
cried out, “Oh wretched man that I am!” that he recognized
the truth about himself and no longer looked for “Who has
another program that I can try?” or “Who has another formula?”
Paul gave up and cried out for a power outside himself:
“Oh wretched man that I am! Who shall deliver me? I
can’t deliver myself.” He gave up trying to deliver himself
and he recognized that he was wretched.

Then he answered his own question: “Thanks be unto
God that, through the promise of Jesus Christ and the power
of the Holy Spirit, God has provided for my victory.” As we
move into Romans 8 we read all about the Spirit-led, Spiritfilled,
Spirit-directed, Spirit-empowered life. Paul concludes
the chapter by saying that we are “more than conquerors
through Him who loved us.”

What a different story from the defeat and sad despair
of the weakness of the flesh in chapter 7! What a glorious
cry of victory—“More than conquerors through him that
loved us. For I am persuaded, that neither death, nor life,
nor angels, nor principalities, nor powers, nor things present,
nor things to come, nor height, nor depth, nor any other
creature, shall be able to separate us from the love of God,
which is in Christ Jesus our Lord” (Romans 8:37–39).

That glorious cry of victory is possible because of the
empowering of the Spirit when I give myself up and turn
myself over to God and receive that power, that dynamic
from God. At this point I allow the Holy Spirit to do His
work within my life, the work that God has designed for
Him to do.


\section*{Not My Own Power}

The effect of this is that I cannot stand up and boast
to you about what a wonderful person I am or about the
wonderful witness I am for my Lord or about the wonderful
way I react in tough situations. All boasting is now in
God’s work through His Spirit. I am still a wretched man,
but thank God I have been delivered from my wretchedness
through the power of the Holy Spirit. When I face a
pressured situation now, and things are pushing in on every
side, thank God the pressure doesn’t even build up anymore.
It’s almost like sitting on the outside and watching
the Spirit work rather than being involved. All of a sudden I
say, “Thank God! That’s not me; that’s not the way I react!”

A retired naval officer accepted Jesus as his Lord a while
ago. He had a foul tongue, as many military people do.
After he accepted the Lord he was really all-out for Jesus.
When he was about six months old in the Lord, he was out
in the backyard mowing the lawn with his power mower,
whistling as he bubbled in the joy of the Lord. As he was
busy mowing and not watching very closely, he went under
a tree and a limb caught him right in the forehead and laid
him on his back.

As he lay there on his back, all of a sudden he got
excited. He went running into the house, grabbed his wife,
and said, “Honey, guess what happened to me!” She looked
at his bloodied face and asked, “What happened to you?
What did you do?” He replied, “That’s not what it is; it’s
what I \emph{didn’t} do! When it happened I didn’t cuss! Not even
one bad word!” She responded, “Honey, do you know that
I haven’t heard you use a bad word in six months?” He
asked, “You haven’t?” The Lord had taken away his foul
tongue without his even realizing it, until a situation came
along that was so apt to arouse the old nature that suddenly
he realized that God had given him the victory.


\section*{Changed on the Inside}

That’s the beautiful way of the Holy Spirit; He works
in such a way that many times the work is already accomplished,
and we don’t even realize it. We’re changed from
within; that’s the method of the Spirit. It’s the change from
within that comes out, which is exactly the opposite of the
method by which we have been trying to do it. We have
been trying to force the changes from the outside in. We can
sometimes be successful in changing the outside, but if the
inside isn’t changed, what is \emph{in} will eventually come \emph{out}.

It is important that the Spirit do the changing from
\emph{within}. When this happens, only God can receive the glory.
Where is my boasting? It is excluded. There is no way I can
boast, because what I was I still am. But thanks be to God
for His grace: through the power of His Holy Spirit I am
now a new creature in Christ Jesus. The old nature I count
as dead. Does this mean that I never get angry? No, I wish it
did. But it does mean this: whenever I do get angry and fail,
I say, “Lord, let Your Spirit work. Give me the power, Lord; I
can’t do it. You will have to do it, Lord; give me the power.”
In one area after another in my life, when I yield that area
of weakness to the power of the Holy Spirit, I begin to experience
real changes as the Spirit works within me and conforms
me into the image of Christ.


\section*{Closing the Door?}

In the Sermon on the Mount Jesus made a very remarkable
statement, one that must have puzzled those who
heard it. In Matthew 5:20 He said, “For I say unto you, That
except your righteousness shall exceed the righteousness of
the scribes and Pharisees, ye shall in no case enter into the
kingdom of heaven.” It is our desire, goal, and prayer that
we might enter into the kingdom of heaven. But it would
seem that Jesus was actually closing rather than opening
the door of the kingdom of heaven when He made this
astounding statement, for the Pharisees practiced being
righteous. They spent their lives trying to interpret the right
action, then sought to do it. When Jesus said to His disciples,
“Except your righteousness exceed that of the scribes
and Pharisees, you shall in no wise enter the kingdom of
heaven,” I imagine that a sorrowful sigh went through them
as they gave up hope of ever entering.

Then Jesus went on to close the door even tighter,
because He began to give a series of illustrations to explain
what He meant—illustrations of how the law was wrongly
interpreted by the scribes and Pharisees. Then He contrasted
that with how the law was originally meant to be
understood. The basic flaw in the Pharisees’ interpretation
of the law was that they were interpreting it so they could
fulfill it and feel good about it. They were interpreting the
law to live comfortably with it—but you can’t live comfortably
with the law. They had begun to feel that they had fulfilled
the law, and were going around doing their little righteous
acts and thinking they were righteous.

But Jesus showed them that, though their actions were
correct, their attitudes were wrong, and thus they were sinners,
for the law was spiritual. The law wasn’t intended to
deal with just the outward actions of man; it was intended
to deal with the \emph{inner attitudes} of man. When the law says,
“Thou shalt not murder,” you can’t really sit back comfortably
and boast in yourself by saying, “Well, I’ve never killed
anybody.” If you feel smug and self-righteous that you have
kept that law, remember that Jesus said, “The way God
meant this is that you are not even to hate your brother.”
The attitude of hatred, Jesus said, was equivalent to the
action of murder, as far as the violating of the law.

So it would seem that Jesus was closing the door to the
kingdom of God. Finally we get to the last verse of Matthew
5, where it seems like He bolted and locked it, for there He
said, “Be ye therefore perfect, even as your Father in heaven
is perfect.”


\section*{I Give Up}

All of a sudden I realize I cannot attain what God
requires of me, because no matter how hard I try, I can’t be
perfect. I have failed, and there is no way I can fulfill God’s
requirement or the command of Jesus Christ. It’s not that I
don’t want to be perfect. The Lord knows I’d love to be perfect,
especially when I’m wrong. It would be nice to always
do the right thing; it would be nice to always have the right
reaction; but I don’t. Many times I have a very wrong reaction
to things, and that’s when I wish I were right.

This is what psychologists call our superego—the picture
of our ideal self, what we really want to be, and what
we would be if circumstances were only different. In contrast
to this is our actual self, our real self, our ego—that
which we really are. Psychologists tell us that our mental
problems are sometimes caused by the disparity between
the two. If the real you is far removed from the ideal
you, then you may have great mental conflicts. The closer
together the ideal you and the real you are, the betteradjusted
person you are.

If you go to a psychologist because you are mentally
disturbed, he will attempt to discover what you really think
you should be—the ideal you—and where in your actual
self you are failing. Often he will then try to bring down
your concept of the ideal you. He will seek to show you
that your values are so high and pure that they are impractical.
Often he will seek to lower your standards in order to
remove your inner conflicts.

However, when the Lord works on us, He does the
opposite. He tries to bring the real you closer to the ideal
you. Man working on the problem would bring the ideal
you down; the Lord working on the problem would bring
the real you up to match the ideal. But God requires of us
that which we cannot attain, that which we cannot give.


\section*{God's Provision}

There is no way that I can fulfill the divine ideal for
my life, so God, realizing that, has made provision for me.
Knowing that I cannot attain to His divine ideal, God sent
His only begotten Son to take all of my failures, all of my
sin, all of my shortcomings—and to accept the responsibility
for them and to die in my place. God, knowing that I
cannot fulfill the divine ideal, has inaugurated a substitute
plan so that now what God requires of me is only that I
believe in His Son, Jesus Christ.

I can do that! Though I cannot be perfect as God has
\emph{ideally} required of me, I can believe in Jesus Christ, which
is God’s \emph{actual} requirement for me. You see, God has now
made the kingdom of God open and available to all of us
because all it takes is for us to believe in Jesus Christ. When
people came to Jesus and asked, “What shall we do that we
might work the works of God?” Jesus answered, “This is the
work of God, that you believe on Him whom He hath sent.”
You cannot stand before God in that day of judgment and
try to excuse yourself by saying, “Well, God, I just couldn’t
be perfect. I’m just human; I just had all these faults, and
I just couldn’t keep your requirements, so I just gave up
because I figured there’s no sense trying.” God will reject
your excuse because God has only required that you believe
in Jesus Christ, the provision that He made for your failing
and sinful self. God has made the kingdom of heaven available
for everyone. You don’t have to be perfect to get there.
All you have to do is believe on God’s provision through
Jesus Christ.

But when you believe on Jesus Christ, and you open the
door of your heart and invite Him to come in, then as the
Spirit of God comes in He begins to work within you and
change you. The Bible says, “If any man be in Christ, he
is a new creature, old things are passed away; behold, all
things are become new” (2 Corinthians 5:17). God’s Spirit
begins to work in your life to do in you what you couldn’t
do for yourself. God’s Spirit begins His work of change in
you, strengthening you, helping you, and conforming you
into the image of Jesus Christ.


\section*{God's Ideal}

As we look around us today and seek to understand
God by His creation—God’s purpose for man and what His
intention was when He created man and placed him on the
earth—we cannot discover this truth, for we do not see man
fulfilling that ideal. The only place we can discover what
God really intended man to be is in Jesus Christ. He is what
God intended when in that divine council they said, “Let us
make man in our image, after our own likeness.”

What did God intend? Look at Jesus Christ and you’ll
know, for Jesus said, “I do always those things that please
the Father.” The Father said about Christ, “This is my
beloved Son, in whom I am well pleased.” As we look at
Jesus Christ we see what God intended man to be. We
cannot look at Adam, because Adam fell, even though God
did not intend for man to fall. We cannot look at ourselves,
because we have fallen, even though God didn’t intend for
us to fall. But if we look at Jesus Christ, there we find the
divine ideal, that which God intended when He created
man. It is the purpose of God that, through the work of His
Holy Spirit in your life, and through the power of the Spirit
to bring about changes in you, He will bring you into the
likeness or image of Jesus Christ.

In Ephesians 4:13 we see what God wants to do in us.
Paul declares, “Till we all come in the unity of the faith and
of the knowledge of the Son of God, unto a perfect man,
unto the measure of the stature of the fullness of Christ.”
That’s what God is working in us today. That’s the work
that God seeks to accomplish in our lives through His Holy
Spirit, bringing us unto the perfect man, unto the measure
of the stature of the fullness of Christ. In Romans 8:29 we
read of the work of the Holy Spirit within us, conforming us
into the image of God’s Son. It is God’s predestined purpose
for us that He, by the Holy Spirit, might conform us into the
image of His Son.


\section*{How it Works}

In 2 Corinthians 3:6–18 Paul talks about the Old Testament
period when God first gave the law. When Moses
came down from the mountain having been there in the
glory of God’s presence, his face shone so that he had to put
a veil over his face when he talked with the people (verse
13). But then in verse 18, in contrast to this veil, Paul said,
“We all with [unveiled] face beholding as in a [mirror] the
glory of the Lord, are changed into the same image from
glory to glory, even as by the Spirit the Lord.”

It is when I look at God’s divine ideal in Jesus Christ
that the Holy Spirit works in me, changing me from glory
to glory into the image of Christ. I believe that this is a lifelong
work. The Holy Spirit has not completed His work
in my life by a long shot. But, praise the Lord, He’s working.
And, praise the Lord, I’m not what I was—I’m being
changed! Those changes are taking place, though I confess
that they’re taking place too slowly for my own desire. I
would love to have them made all at once.

Whenever the Holy Spirit shows me an area that needs
working on, whenever He opens up the light and causes me
to see my true self and how far it is from what God would
have me to be, immediately I think, “Oh, let’s get to it; let’s
conquer.” I step in and try to be better, and I keep trying to
be better, but the harder I try the worse I get, until I get to
the place of defeat and giving up. Then I cry out, “Oh
God, I’m so wretched. I cannot do it.” He replies, “Good.
Now will you step aside and let Me work? You’ve been in
My way.” He is not interested in my self-righteousness; He
is not interested in my help. He wants to do His work in
me unhindered by my fumbling efforts, because even if He
used my fumbling efforts to help give me the victory, I’d be
going around boasting in my fumbling rather than in my
God. God lets me fail until in despair I cry out for help. As
I yield myself to the Spirit of God and allow Him to do His
work, He conforms me into the image of Jesus Christ.


\section*{I'm No Match}

I must be brought to the place of acknowledgment and
recognition that I cannot rid myself of the flesh or its desires
or weaknesses. I’m no match. As long as I’m struggling and
trying, I cannot make it; I will fail.

We're sinners; we need to recognize that fact; there is
nothing we can do about it ourselves. We have to call out
for a power greater than our own. That’s what Paul was
doing in Romans 7 when he said, “O wretched man that
I am! Who shall deliver me from the body of this death?”
(verse 24). He called out for a power greater than himself,
and when he did so, he found the power.

When we with open face behold the glory of the Lord
we are changed from glory to glory. God is changing
us—changing our attitudes. By our Adamic nature we are
very selfish and self-centered. It begins very early in life;
you can see it in small children as they say “mine”; it’s one
of the first words they learn outside of “mama” and “dada.”
You see them clutching their possession, and you don’t dare
try to take it away from them or you will hear about it in no
uncertain terms. If you take their bottle away, you’ve got a
fight on your hands—screaming, hollering, and kicking. It’s
just fortunate that they are as small and weak as they are,
or they would tear the crib to pieces! They are blessed little
children, but they have the Adamic nature.

As long as I am selfish and self-centered, I am not what
God wants me to be. God does not want me to be self-centered.
God does not want me to be interested in my own
welfare first. The Lord wants me to be interested in other
people and to share what I have with them in their need.
That is what Jesus was talking about in Matthew 5 when He
said, “Be ye therefore perfect; even as your Father in heaven
is perfect.” But this isn’t natural; it’s supernatural, and we
can attain it only by the supernatural power of the Holy
Spirit coming in and changing our attitude concerning ourselves
and our possessions.

Not only does He change the attitude (which is the most
important thing), but the changed attitude changes action.
Too often we try to do it the other way around. It seems that
our philosophy is to change a person’s actions and hope
that by changing his actions we can change his attitude.
Psychologists say that if we act out emotion we will get
the corresponding emotion. But God is interested in truly
changing the attitude of our heart, and this changed attitude
brings the changed action of our life.


\section*{The Change from Within}

The gospel and the Holy Spirit work from the inside
out. My heart is changed and my attitude is changed, and
thus my actions reflect the changed attitudes within. The
Holy Spirit working in me is changing me from glory to
glory, bringing me into the image of Jesus Christ. How? By
my looking at Him with unveiled face. How do I see Him?
I can only see Him in the Word, and the Spirit makes the
Word alive to my heart.

Peter tells us in his second epistle that God has given us
exceedingly rich and precious promises, and that by these
we are made partakers of the divine nature. It is there in the
Book, but you must behold Jesus in the Book; you must look
for Him there. Many people read the Bible with their faces
veiled. It takes the Holy Spirit to open the Bible, to take the
veil off their eyes so they can understand. The work of the
Holy Spirit within us is so important. We cannot be what
God would have us to be apart from the working of the
Holy Spirit within our life.

No one knows me as well as I know myself except the
Lord, and He knows me better than I know myself. A lot
of things that I thought about myself I discovered were not
true. Many of the things in my ideal self-image did not turn
out like I thought they would. When I looked at myself
through rose-colored glasses, I looked very rosy! But when
the Holy Spirit broke my glasses, I was surprised. But He
had to. He had to destroy my illusions of myself in order
to deal with those areas of my life that I refused to acknowledge
to Him. He had to bring them forth and reveal them in
all of their ugliness so that He could then work to rid me of
them.


\section*{A Son of God}

I know that I am now a son of God, not through any
righteousness of my own, but by my faith in Jesus Christ.
“As many as received him, to them gave he power to
become the sons of God, even to them that believe on his
name” (John 1:12). Because I believed on Jesus Christ, God
gave me the power to become a son of God. So now I know
I am a son of God, and that to me is absolutely glorious. If
I am a son, then I am an heir. I am an heir of God and a
joint heir with Jesus Christ, and I do not know anything that
could be more glorious than that.

God is working on my inner man by His Holy Spirit, but
my problem is that though I am new on the inside, I am still
the old Chuck on the outside. But the old Chuck is actually
dead, so I have got to drag this old corpse around until the
day that God finally delivers me from it. With my mind and
with my heart I serve the Lord, but so many times with my
body I am controlled by my own selfish desires. This old
corpse gets heavy and hard to carry around. There are times
when I groan, desiring to be delivered, not that I would be
an unembodied spirit, but that I might be clothed upon with
that new body which is from heaven, that I might be like
Him, as I see Him as He is.

I am a son of God. I have a renewed spirit in an unredeemed
body. God is not going to take this body into
heaven, praise the Lord! “This corruption must put on incorruption.”
A change is starting to take place now: “We with
open face beholding the glory of the Lord are changed…”
This particular word changed in the Greek is “metamorphoo.”
The word is used to describe a change of body, as when a
caterpillar is changed into a butterfly. Paul said that all creation
groans and travails together until now, waiting for the
manifestation of the sons of God, namely, the redemption of
our bodies.


\section*{Like Christ Forever}

I should never be satisfied with myself or with my present
state of development until I am like Jesus Christ. David
said, “I shall be satisfied when I awake in Thy likeness.” The
morning I wake up and take a deep breath of air, and there
is no smog, and I feel so different, and I realize that this corruption
has put on incorruption—then I will be satisfied, for
I will be like Him, for I will see Him as He is. That is what
the Spirit is bringing me to; that is the purpose of the work
of the Holy Spirit in my life. He will not be satisfied until He
is finished and brings me into complete conformity to the
image of Christ.

The final change will take place at the coming again of
Jesus Christ for me, whether it be by death or the rapture of
the church. At that time the old nature will be put off and
the final change made, but I should not wait for that day.
Even now, as I look to Jesus, the process of change is taking
place. We should be closer to the image of Christ this year
than we were last year, and next year than we are this year,
for we grow in grace and in the knowledge of Him, and the
Spirit working in us should be bringing us more and more
into His likeness.


\section*{Mature of Just Old?}

I love to be around saints who have been walking with
the Lord for fifty, sixty, seventy years. I mean those who
have really been developing in their walk. I know there are
some who have been around fifty, sixty, seventy years but
are still in their spiritual crib, and that is tragic. If you see
a child that is just four or five months old, and his arms
sort of wave excitedly, and he says, “Da da da…” you think,
“That’s beautiful; look how smart he is—what a beautiful
child!” But if your child were twenty-one years old, and
when you walked into the room he was lying on the bed
and started smiling and saying, “Da da,” it would no longer
be a thrilling, exciting emotion—it would be very tragic.

That is the tragedy about so many people in the church
today. After fifteen or twenty years they are still at the same
level of development. They are still banging their cribs.
They have still got the same petty gripes. They are still
upset with the message of last Sunday, and they are still
divided in their little factions. They have not progressed at
all. They are spiritual monstrosities because there has never
been any development, and the problem is that there are so
many of these Christians that it isn’t even novel enough to
be a curiosity. They are all over the place. They just have not
delved into the Word of God to really behold the face of the
Lord. They have not allowed the Word of God to really penetrate
and the Spirit of God to really teach and instruct them
in the things of the Lord or to reveal Jesus unto them in the
Book.


\section*{How Far to Go?}

Oh, that we would yield ourselves to the Holy Spirit
now, that He might do His work in us, conforming us into
the image of Jesus Christ! How far does He have to go in
your life? Have you ever taken one of these little self-evaluation
tests to discover if you are appealing or a social bore?
In 1 Corinthians 13 is a very simple little self-analysis test
that you can take to find out how far the Spirit has come in
your life in just one area: the area of love, one of the most
important areas. Beginning with verse 4, the definition of
this word love is given: “Love suffers long and is kind; love
envies not, love vaunts not itself, is not puffed up, does not
behave itself unseemly, seeks not its own, is not easily provoked,
thinks no evil, rejoices not in iniquity, but rejoices
in the truth; bears all things, believes all things, hopes all
things, endures all things. Love never fails.”

You say, “What’s that got to do with me?” Take out the
word love and put your name there instead. Then read the
list again. “Chuck suffers long and is kind. Chuck envies
not, Chuck vaunts not himself, is not puffed up, does not
behave himself unseemly, seeks not his own, is not easily
provoked, thinks no evil, etc.” To the degree to which the
text seems to be incongruous, to that degree I have failed
from attaining what God wants me to attain. God help me
to yield myself to the Holy Spirit so that He might do His
work in me, so that what I have tried to do and failed—
what I want but cannot achieve or attain—might be accomplished
for me by His power.


