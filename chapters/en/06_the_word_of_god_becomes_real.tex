\chapter{The Word of God becomes real}

As we have been considering the work of the Holy Spirit 
in the life of the believer, we have seen how He gives us 
the power to be all that God would have us to be. Next we 
saw how He conforms us into the image of Jesus Christ: 
“We, with open faces beholding the glory of the Lord, are 
changed from glory to glory, even into this same image by 
His Spirit working in us.” Next we saw how He brings to us 
the agapé love of God, helping us receive it and then giving 
us the capacity to love with God’s agapé love, for the Holy 
Spirit has shed the love of God abroad in our hearts and in 
our lives.

Now we would like to study the work of the Holy Spirit 
in the life of the believer, making the things of God and the 
Word of God real to us.


\section*{Eye Has Not Seen}

In 1 Corinthians 2:9 Paul wrote, “Eye hath not seen, nor 
ear heard, neither have entered into the heart of man, the 
things which God has prepared for those that love him.” 
This Scripture has often been misinterpreted. In fact, I think 
that of all the Scriptures in the Bible that have been misinterpreted or quoted out of context, this verse ranks near 
the top. You usually hear this Scripture quoted in regard 
to heaven. Heaven, we are told, is going to be so glorious, 
so marvelous, so beautiful that “eye hath not seen, nor ear 
heard, neither have entered into the heart of man, the things 
that God has prepared for those that love him.”

All the time in church as I was growing up, this is the 
way I heard this passage quoted, and so the first ten years 
in my ministry I interpreted it like this too. Then one day I 
read the whole context and realized that Paul was not talking about heaven. Paul was talking about the things that 
God has for His people  \emph{right now}—those things that God 
has for us because He loves us and we love Him. What is 
Paul saying? The natural man cannot see, he cannot know, 
he cannot understand those things that God has for \emph{us}. That 
is what he is talking about. The natural man’s eye cannot 
see, his ear cannot hear, and neither has it entered into his 
heart those things that God has for \emph{us} because we love Him. 
Peter tells us that even the angels desire to look into the 
things that God has in store for His church. It no doubt 
amazes them that God would actually come and dwell 
within us.

Notice what the next verse says: “But God \emph{has} revealed 
them unto \emph{us}.” Paul is not talking about heaven. 
He is talking about \emph{the present glories of the life empowered by the Holy Spirit.}“
God hath revealed them unto us by his Spirit: for 
the Spirit searcheth all things, yea, the deep things of God. 
For what man knoweth the things of a man, save the spirit 
of man which is in him?” (1 Corinthians 2:10–11). In other 
words, who really knows what is in your heart except you? 
You may be putting on a pretty good front; it could be that 
you are deceiving a lot of people. I really do not know 
what is in your heart. You \emph{know}. You know what is covered 
and veiled and what you are hiding from everybody else. 
So, Paul is saying, “What man knows the things of a man 
except the spirit of man which is in him? Even so the things 
of God knows no man, but the Spirit of God” (1 Corinthians 
2:11).


\section*{The Revealing Spirit}

There are things God knows (for example, aspects of 
God’s love) that man does not understand; only the Spirit 
of God understands. Paul continues, “We have received, not 
the spirit of the world, but the spirit which is of God; that 
we might know the things that are freely given to us of 
God” (1 Cor. 2:12). The Holy Spirit reveals those things that 
God has freely given us. “Which things,” Paul says, “also 
we speak, not in words which man’s wisdom teacheth, but 
which the Holy [Spirit] teacheth, comparing spiritual things 
with spiritual. But the natural man receiveth not the things 
of the Spirit of God: for they are foolishness unto him: neither can he know them, because they are spiritually discerned” (1 Cor. 2:13–14).

A lot of people have made the mistake of trying to discover what the Bible is all about on their own. They have 
read the Bible to discover its message, but they did it with 
only their human intellect. This usually ends in failure. They 
try to read, but they get nowhere. They say, “I’ve tried reading the Bible, but I can’t see what anybody gets out of it. I 
just can’t understand it.” That’s exactly what Paul is saying: 
“The natural man \emph{cannot} understand the things of the Spirit, 
neither can he know them.” It is \emph{impossible} for the natural 
man or the natural mind to comprehend the things of the 
Spirit because he lacks the capacity to do so. You could 
say with equal logic that a blind man cannot appreciate the 
beauty of the sunset or a deaf man cannot enjoy the music 
of a concert, because he lacks the capacity by which these 
things are appreciated and understood. Unless the Holy 
Spirit opens our heart and our mind to these things, we just 
cannot understand them.


\section*{Seeing and Not Seeing}

One of the most difficult things is to have a clear understanding of an issue and wonder why someone else does 
not have the same clear understanding. It is so plain! It is so 
obvious! Why can’t you see it? It is right there—look! But if 
they are not spiritually awakened, if the Spirit of God is not 
dwelling in them, they can look all day and still not see it. 
This understanding and enlightenment must come through 
the power of the Holy Spirit to open up the things of God 
to our hearts. Paul prayed for the Ephesian believers in 1:17 
that God might give to them the Spirit of wisdom and revelation in the knowledge of Him. The work of the Holy 
Spirit is to let us know that glorious grace of God bestowed 
unto us, that work of God in our behalf and that which God 
wants to do within our lives. How tragic it is that people 
are seeking to live the Christian life and understand the 
Christian walk without the help of the Holy Spirit! It simply 
cannot be done.

God has made available unto us all that is necessary for 
life and godliness. God has not left a single thing out; no 
matter what the situation is that we might be facing, God 
has the way out for us. God has already made His provision. The Holy Spirit makes us aware of those things that 
God has already freely given us, so that we can then appropriate the work of God for our particular need and situation.

I do not sit down to read the Word of God without 
first saying, “Oh Holy Spirit, open my mind and my heart 
to receive and understand the Word of God.” I dare not 
approach the Word of God with my own natural intellect. It 
will just be a blur. I need the help of the Holy Spirit to teach 
me what God has said. In 1 John 2:27 we read, “Ye need not 
that any man teach you: but as the same anointing teacheth 
you of all things.” So we can look to the Holy Spirit to teach 
and guide us as we study the Word of God.


\section*{Learning of Jesus}

When we are talking with people who have just committed their lives to Jesus Christ, we seek to emphasize the 
importance of learning about Jesus. Jesus said three things: 
First, “Come unto me”; next, “Take my yoke upon you”; 
and then, “Learn of me.” Salvation is more than just coming 
to Christ: it is taking His yoke upon us; it is submitting our
lives to the mastery of Jesus Christ; it is turning the reins of 
our life over to Him. But then if I am to grow, I must learn 
of Him.

Many a person has failed to continue in the grace of God 
because he has not learned of Jesus Christ; he has not grown 
in his knowledge of Him. There is only one way you can 
learn of Jesus Christ, and that is by reading the Word of 
God. The Word of God is the only spiritual food for the 
new Christian, the spiritual babe; you cannot grow apart 
from the Word of God and the knowledge of God and Jesus 
Christ through the Word. It is vital for your Christian experience and for your Christian growth.


\section*{Which Way Is Right?}

I am always instructing people, “Before you start to 
read, just pray, ‘Lord, open my eyes and let me see, open 
my ears and let me hear what the Spirit would have to say 
through the Word.’” Some people say, “There are so many 
interpretations that I just get confused.” I am not asking you 
to read some man’s interpretation of the Bible. Just read the 
Bible itself. Some people ask, “How do you know your way 
is right? Maybe Joe Smith was right, or maybe the Jehovah’s 
Witnesses are right, or maybe Mary Baker Eddy was right. 
Everyone has his own interpretation. How do you know 
you’re right?” Let me say this: I am not at all worried what 
you will come to believe by just reading the Bible. I believe 
that the Holy Spirit is able to teach you—right out of the 
Bible itself—all that you need to know. I do not encourage 
you to read some interpretation of man. I encourage you to 
just read the Bible and let the Holy Spirit teach you what 
God has said.

I am not worried about how you are going to interpret 
the Bible if you just read the Bible alone. I am not worried 
about you getting off into some false doctrine or some weird 
trip if you are just reading the Bible. I \emph{am} worried about the 
weird trips that you get on when you read some of the junk 
that is going around supposedly interpreting the Bible.


\section*{Learning and Recalling}

Jesus said, “The Holy Spirit will come alongside you 
and help you; He will teach you all things and then bring all 
things to your remembrance, whatever I have commanded 
you.” Notice that He will bring all things to your  \emph{remembrance}.
 What does that mean? This means that the things 
have to be planted there in order to be brought to remembrance. He cannot bring something to your remembrance 
that has not been planted in your mind first. It is important 
that you read the Word of God. David said, “Thy Word, O 
LORD, have I hid in my heart, that I might not sin against 
Thee.” You may not remember what you read five minutes 
after you read it, yet a crisis situation may arise in your life 
and all of a sudden a Scripture pops into your mind. What 
has happened? The Holy Spirit has brought to your remembrance that which had been placed there, and now in this 
moment of emergency the Holy Spirit has helped you.

In John 16 Jesus was talking with His disciples just 
before going to the Garden of Gethsemane, where He was 
to be arrested. This was the last night that Jesus was sharing 
with His disciples before His crucifixion. He said to them in 
verse 12, “I have yet many things to say unto you, but ye 
cannot bear them now.” He had been with them for three 
and a half years teaching them, and now there were many 
things that He still had to say to them, but they were not 
capable of receiving them. So He said, “When he, the Spirit 
of truth, is come, he will guide you into all truth: for he 
shall not speak of himself; but whatsoever he shall hear, that 
shall he speak; and he will shew you things to come” (John 
16:13). Jesus was saying, “I have a lot of things to say to you; 
you can’t handle them now, but when the Spirit of truth is 
come, He will lead you into all truth, and He will show you 
things to come.” The Spirit of God is given to us to teach us 
the things of God; the Spirit of God is given to us to lead us 
into the truth of God, and then to show us things that are 
going to come.


\section*{Knowing What's Coming}

Concerning His showing us things to come, the apostle 
Paul said in writing to the Thessalonian church, “Ye, brethren, are not in darkness, that day [that is, the day of the 
Lord, the rapture of the church] should overtake you as a 
thief. Ye are all the children of light” (1 Thessalonians 5:4–5). 
What is he saying? He is saying that the day of the Lord, 
the coming of Christ for His church, should not take us by 
surprise. It should not catch us unaware. If we are walking 
in the Spirit, if we are being led by the Spirit, He is showing 
us things to come and is keeping us alert as to the day in 
which we live.

I am really shocked at how blind many people are to 
this age in which we live. I was on a radio show panel one 
night in Los Angeles when I mentioned the second coming 
of Jesus Christ—His soon return. All the other panel members (who were religious leaders) thought that it was horrible to think that the Lord is coming soon, and that we 
ought to be more interested in making this a better world. I 
replied, “You’ve been trying to do that for a long time, but 
we’re in the worst mess ever.” I am surprised that they are 
not discouraged; you have to credit them for that! When 
you try so hard to make a better world but it just gets worse 
and worse, if you are not discouraged by now, you have got 
some kind of grit. But I think a person has to be totally blind 
to look around today and say, “How lovely! Things are getting better!”

They are not facing reality. I am a realist. The church, in 
its endeavor to make the world a better place to live, has just 
about wiped itself out. The methods by which they are seeking to make the world a better place to live are things I do 
not understand. When some church groups support African
terrorist organizations and the PLO, I hardly see how this 
can contribute to a better world.


\section*{The Tragedy of Blindness}

The frustrating thing about walking and being led by 
the Spirit is that when the Spirit shows you things that are 
so obvious and plain, you just cannot understand why the 
next fellow cannot see them. But the reason he cannot see 
them is because the natural man cannot understand the 
things of the Spirit: “…neither can he know them, for they 
are spiritually discerned.” There are those who are such 
obvious counterfeits, yet people are deceived by them. It 
is like seeing the portrait of a \emph{smiling} Abraham Lincoln on 
a five dollar bill; it is so obvious that it is phony, and yet 
people are just as blindly duped and taken in. You think, 
“Can’t you see? They’re ripping you off. He’s a phony.” But 
they do not have the gift of discernment, and thus they are 
totally fooled. It is the hardest thing to see so clearly and 
not to understand why other people cannot see the issues as 
clearly as you can.

The return of Jesus Christ is so near; we are right at the 
end of the times of the Gentiles, and it is so obvious that 
the coming of the Lord is at hand. The Scriptures are so 
clearly fulfilled, and yet people are totally oblivious to this 
whole prophetic picture; they are blinded and continuing 
their lives as if they are going to be here forever. Unless 
the Holy Spirit had revealed the Second Coming to us, we 
would never have known it.


\section*{The Real Source of Truth}

When John said, “You have need that no man should 
teach you, but the anointing which you have received will 
teach you all things,” does this mean that I am not to fellowship at church, but to stay home and read my Bible, 
allowing the Holy Spirit to teach me there? No. Paul tells us 
that the Holy Spirit has placed pastors and teachers in the
church for the perfecting of the saints for the work of the 
ministry. But the truth still stands that only the Holy Spirit 
can teach \emph{you}, and if you receive any truth at all from God, 
you receive it only because the Holy Spirit has made it true 
and brought it to you, opening your heart to understand it. 
I might be bringing God’s truth to you, and all of a sudden 
you say, “I see it! Oh, thank you, Chuck!” No, do not thank 
me, because you would never have seen it unless the Holy 
Spirit had revealed it to you.

Another person reading the same page did not catch it. 
He is still in as much darkness as he ever was. He does not 
understand it at all; it went right over him. “Well, why did 
I see it and he didn’t?” Because the Holy Spirit taught you 
the truth; you were ready for it. It was time for you to know, 
and the Holy Spirit opened your heart and taught you the 
truth. I perhaps declared the truth, but you cannot receive it 
or understand it unless the Holy Spirit brings it to you. Your 
understanding of spiritual things can only come through 
the Holy Spirit.


\section*{Two Kinds of Teachers}

It also follows that you cannot teach the truth of God 
except by the Holy Spirit, for how can you teach what you 
do not understand? This means that, though a man may 
know Greek and Hebrew fluently, and though he may have 
committed the Old and New Testaments to memory in the 
original languages, and though he may know all the commentaries, if he is not filled with the Spirit, he cannot be 
a true guide in the things of God. You would do better to 
listen to a young man without a seminary degree who is 
filled with the Spirit. An uneducated, but Spirit-filled servant of God is a truer guide to scriptural truths than a Ph.D. 
who is not born again. No man can truly understand the 
things of the Spirit except the Spirit teach him, and no man 
can truly teach the things of the Spirit except the Spirit 
anoint him to teach.

Knowing the Greek language is very beneficial to understanding the New Testament, but the fact remains that it is 
the work of the \emph{Holy Spirit} to teach us the things of God, to 
bring them to our remembrance, to make the things of God 
real in our life, to open up our understanding of these spiritual things and to the work that God has done for us. It 
is the Holy Spirit who makes us aware of the things that 
are coming to pass in the world around us and alerts us to 
the day and the hour in which we live. Thank God for the 
Holy Spirit in our lives, making God’s Word come alive and 
making us the partakers of those rich things that God has 
freely given us!



