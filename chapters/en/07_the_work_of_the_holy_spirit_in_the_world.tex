\chapter{The Work of the Holy Spirit in the World}

The Holy Spirit has an important work to accomplish in the world. We have been looking at the work of the Holy Spirit in the life of the believer: conforming us into the image of Christ, opening unto us the things of God, and bringing us the agape love. 

But what is the work of the Holy Spirit in the world? Jesus said in John 16:8,9: "He will reprove the world of sin, of righteousness, and of judgment." Then Jesus continues by saying, "Of sin, because they believe not on me." There is one ultimate sin that man will have to answer to God for - the sin of not believing in Jesus Christ. When Christ died, He died for the sins of the world: "All we like sheep have gone astray; we have turned every one to his own way, and the Lord has laid on Him the iniquities of us all" (Isaiah 53:6). Christ took upon Himself the sins of all mankind. He died for the sins of the world, so that there is one ultimate sin that will condemn you before God, and that is the sin of rejecting God's plan of salvation for you through Jesus Christ. Man was already condemned before Jesus came. Jesus said, "I didn't come to condemn the world; I came that the world through me might be saved. He who believes is not condemned, but he who believes not is condemned already, because \emph{he has not believed on the only begotten Son of God}." That is the ultimate sin that will condemn a man - not believing in Jesus. 

\section*{The Greatest Sin}

The unpardonable sin basically boils down to this: the continual rejection of Jesus Christ as our Savior, which is actually blasphemy against the Holy Spirit because the Holy Spirit has come to reprove you of sin. He "reproves the world of sin" because it does not believe on Jesus. As He reproves us of sin, showing us that Christ is our only hope of salvation, if we continually reject the Holy Spirit's message to our heart, that is blasphemy against the Holy Spirit. If we continue in that rejection there is no forgiveness, not in this world nor in the world to come. 

There is only one issue, and that is our relationship to Jesus Christ. There are some people who fancy themselves as not too bad. They are good, moral people; for the most part they are honest and have always been a faithful family person. They have never committed any great crime, and as they look at themselves they say, "I'll take my chances; I'm about as good as anybody." But in reality, if they have not received Jesus as their Savior, they are condemned; they are guilty of the worst sin - the rejection of God's plan of salvation and of His infinite love. So the Holy Spirit is in the world today to reprove the world of sin, "because," Jesus said, "they believe not on me." 

Do you believe in Jesus Christ as your Savior? Have you committed your life to Him? If you believe in Him and have committed yourself to Him, you are saved. If you do not believe in Him and have not committed yourself to Him, you are condemned, and the Holy Spirit will reprove you of your sin because of your disbelief in Him. 

\section*{Totally Pure}

The second thing the Holy Spirit is reproving the world of is righteousness. Jesus made a very interesting statement about this. He said, "Of righteousness, because I go to my Father, and you see me no more" (John 16:10). After appearing to His disciples after His resurrection for about 40 days, Jesus led them as far as the city of Bethany and then told them that they were to go back to Jerusalem and wait there for the promise of God. They would receive power when the Holy Spirit came upon them, and they would become Christ's witnesses throughout the world. Then Jesus ascended into heaven, and a cloud received Him out of their sight. Acts 1:10,11 tells us that two men stood by them in white apparel, and they said to the disciples, "Ye men of Galilee, why stand ye gazing up into heaven? This same Jesus will come again in like manner as ye have seen Him go." 

What does the ascension of Christ into heaven tell us? What was God saying to us by the ascension? God was bearing witness that this is the standard of righteousness that He will receive. Jesus lived the kind of righteous life that the Father will accept into heaven. What is it saying to you if you want to go to heaven by your own righteousness or your own good works? The only way you will ever make it to heaven is to be as righteous, as pure, as holy as Jesus Christ. If you come anywhere beneath that standard of righteousness, you will not make it in. When you look at it that way, you might as well give up, because even though you may be a very good person, and even though you may be the best person in your house and in your neighborhood, unless your righteousness is as perfect as that of Jesus Christ, there is no getting into heaven. That is the standard of righteousness that God will accept. The Holy Spirit is reproving the world of righteousness, Jesus said, "because I go to the Father and you see me no more." 

\section*{Satan Defeated}

Finally Jesus spoke of the Holy Spirit as reproving the world of judgment. The Spirit is not reproving the world of the judgment that is to come; that is not what Jesus said. He said in John 16:11, "Of judgment because the prince of this world \emph{is judged}." He was not talking about the future judgment that man will have to face when death and hell deliver up the dead that are in them and they all stand before the Great White Throne judgment of God. The Holy Spirit is not talking to us about that. He is talking to us about judgment "because the prince of this world is judged." What did Jesus mean by that? 

When we gave our heart to Jesus Christ and turned our life over to Him, it was not the end of our problem with sin. We still had problems because of our flesh. And Satan, knowing our weaknesses, was there to exploit these problems to their fullest. But what the Holy Spirit is testifying to us is that the prince of this world is judged. When Jesus went to the cross, and there bore our sins and died in our place, the prince of this world was judged, so that through our relationship to Christ we can have power over sin; it no longer needs to reign in our bodies. 

Paul said that sin should no longer reign as king in our mortal bodies, but that Christ might now reign. In the second chapter of Colossians, Paul speaks of the victory of Jesus over the principalities and powers of darkness that were spoiled through the cross. He said, "having nailed the handwriting and ordinances that were against us to the cross, He spoiled the principalities and the powers [rankings of evil spirits] which are against us, having triumphed over them, making an open display of His victory in the cross." 

The prince of this world was defeated at the cross. His power to control your life and to force you to do those things contrary to God's righteousness and God's way was taken away from him, so that you can enter into the victory of Jesus Christ. Now you can know freedom from sin and can have power over sin in your life. "Of judgment because the prince of this world is judged" means that the Lord has made provision for you to be free from whatever sin has been blighting your life. Paul says, "The old man [the old nature] was crucified with Christ." The prince of this world, Satan, was judged at the cross. And through our identification with Jesus Christ in our new life - through the power of His Spirit - we can be free from sin. 

\section*{What If I Sin?}

What does it mean if I do sin? It means that I have not taken advantage of what God has made available to me through the power of the risen Christ and the Holy Spirit. God has given to me all that is necessary to live the life that He wants me to live, but it is important that I take advantage of what God has done for me - that I exercise the power that he has given to me. This does not mean that I have sinless perfection; it does not mean that I am never going to sin again. But it means that if I do sin I cannot blame God for it and say, "Well, that's just the way God made me." All I can blame is my own failure to yield to the power of the Spirit and the victory of Christ. I am aware that in some of your lives Satan has taken a very strong foothold. Some of you are bound by habits that have kept you in spiritual defeat all your Christian life. You have never been able to really enjoy your full joy in Christ, because there has been this sin or failure that has plagued and hounded you all the way along. I know that some of you have been praying about this for months; you have been crying out to God. You have been seeking God's help through diligent prayer, but you find that your flesh is so weak. You find yourself falling back into the old patterns, into the old path, almost to the point of despair. You are discouraged, and Satan begins to lie to you: "There's no way out. You'll never make it. You might as well give up." 

\section*{You Can Conquer}

But you \emph{can} make it, for God has made victory over sin available to you through the power of the Holy Spirit. Satan was judged at the cross, so any power that Satan exercises in your life is usurped power; he has no authority or right. But he is very bold; he is very brash. He moves in where he has no right. He takes what he can, by whatever method he can, even though he has no legal right of being there because you have been purchased by Jesus Christ; you are now His possession. Satan was judged at the cross, and any power or authority that he seeks to exercise in your life is false. When you come against him in the name of Jesus Christ and in the power in the victory of Christ, Satan must yield. Because he has no legal authority or right to be there, he must submit. He has been defeated at the cross, and he must yield to the victory of Christ as you take that victory that is yours. Because Satan was judged, he has no real authority in your life at all. If you want to demand that he go, there is no way he can stay. 

\section*{Let the True King Reign}

God told Samuel the Prophet to go down to the house of Jesse and anoint one of Jesse's sons to be king over Israel, because God had rejected King Saul from ruling. Fearing Saul, Samuel went secretly down to the house of Jesse and had Jesse bring his sons in. The first one came in - good-looking, a big fellow - and Samuel thought to himself, "This must be the one!" But the Lord said, "No, man looks on the outward appearance, but I look on the heart. This isn't the one." So one by one, Jesse had his sons pass through, and every one the Lord rejected, until finally Samuel said, "Don't you have any more?" Jesse said, "There is just one more, but he's just a lad and he's out watching the sheep." Samuel said, "Call him." 

As David came running in from following the sheep, God spoke to Samuel and said, "That's the one." David stood there as Samuel took a cruse of oil and poured it over his head; oil ran down over David as he stood there, and the anointing of God came upon him - God's anointing to reign over God's people. But it is interesting that even though God anointed David as king over Israel and the throne belonged to David, Saul was still sitting on it. For the next few years Saul did his best to drive David out, chasing David over the mountains like a partridge, until David even despaired of his own life. Saul did his best to hold on by force to that which was no longer his. But because of God's edict, Saul finally fell on his sword at Mount Gilboa, and the throne that had belonged to David for quite a while became David's, and he sat upon the throne and reigned. 

God has ordained that Satan should no longer reign in your life. When you gave your life to Jesus Christ, you became His property, and God wants to reign in your life today. The prince of this world has been judged. Some of you he has been trying to hold onto for a long time; though you have submitted your life to Christ, Satan is hanging in there. It is time that you take the authority that God has given to you, and through the power of the Holy Spirit demand that Satan leave you alone. The Bible promises us that if we resist the devil, he will flee from us. 

\section*{Pray Specifically}

Satan not only goes in and takes what is not his, but he tries to hold onto what no longer belongs to him. He is very stubborn; he does not let go easily; so our prayers must be specific. I think that Satan really enjoys the generalized prayers of the believer. They do not even hurt. "God, save the world." That is so general that it does not get anything accomplished. You have got to be specific. "Lord, I claim the victory of Christ over this area in my life. Lord, I dedicate this area to you; I want Christ to come and sit on the throne. Thank You, Lord, for judging Satan. Thank You, Lord, for victory over him. And now, Lord, move him off; You move in and sit on the throne, and rule and reign in my life." Be specific. 

\section*{Take and Hold!}

But once Satan is driven out, he counterattacks and tries to take back the territory from which he has been driven. Jesus said that when an evil spirit goes forth from a man, he goes out through dry places seeking a house to inhabit, and, finding none, he comes back. I find that there is always that counterattack, that attempt to reestablish the foothold. So that which is \emph{taken} in the name of Jesus Christ must be \emph{held} in the name of Jesus Christ. 

Many times a person having initial victory says, "Oh, praise the Lord! The Lord has given me the victory!" And he lets his guard down. He thinks, "Oh, I've got it made. I don't have to worry about that anymore." Then Satan comes right back. He has been kicked out the front door, so he comes right around to the back door. He slips right in while you are in the living room shouting the victory; he comes back in through the kitchen! What we \emph{take} we must \emph{hold} through the power of the Holy Spirit. 


