\chapter{When are Tongues to Cease?}

Those religious doctrines which forbid speaking with tongues usually point to I Corinthians 13:8, where we are told, "Whether there be tongues, they shall cease," as the basis for their prohibition. When tongues would cease, however, is dependent on how the phrase "that which is perfect is come" is interpreted. Those who use this verse to prohibit tongues interpret "that which is perfect" as the full revelation of the Canon of Scripture, concluding with the Revelation of Jesus Christ given to John. Their argument usually assumes that, until the Canon of Scripture was complete, these gifts were used to instruct the early church. But once Scripture was complete, they no longer needed to depend on these gifts. Thus tongues ceased when Scripture was completed. 

\section*{Answers to the Argument}

This argument at first sounds plausible; however, it is nothing more than hypothetical speculation, and it is not only void of Scriptural foundation but it would appear to be against all Scriptural use of the gift in the New Testament. Not once do we find speaking in tongues used to instruct believers in the New Testament. On the contrary, we read in I Corinthians 14:2 that those who spoke in tongues were not speaking to man but to God. They also were not to speak in tongues in the church unless someone with the gift of interpretation were present, so that everyone present might be able to say yes and amen to the blessing and giving of thanks offered to God. 

The use of the gift of tongues in Scripture has never had any association with teaching God's truth to the church. There could therefore be no relationship between the gift of tongues ceasing and the arrival of the full Canon of Scripture. One of the cardinal rules of Scripture interpretation is to examine the text in light of its context. To know the text, read the context. The context of I Corinthians 13 is the supremacy of love. It is supreme over the exercise of the gifts of the Spirit, which are void without love (verses 1-3). Love is next defined in verses 4-7, and then the never-failing quality of love is declared in verses 8-12, showing that it will outlast tongues, prophecies, and knowledge. Finally, in verse 13 the abiding trilogy of faith, hope, and love is presented, claiming love as supreme. The immediate context is the unfailing nature of love in contrast to prophecies (which would fail), and tongues (which would cease), and knowledge (which would vanish away). Prophecies and knowledge are partial, but when that which is perfect is come, we will no longer have obscured vision, but will see face to face. Our knowledge will no longer be partial knowledge, but complete, because we then will know even as we are known. 

The idea that the Greek word \emph{teleios}, translated "perfect," referred to the full Canon of Scripture did not occur to some of the greatest of all Greek scholars from the past century. It is more of an invention or creation of recent vintage to counteract the modern tongues movement. Thayer, in his \emph{Greek-English Lexicon}, says of \emph{teleios} as used in I Corinthians 13:10, "The perfect state of all things to be ushered in by the return of Christ from heaven." Alford, in his \emph{New Testament for English Readers}, says of it, "At the Lord's coming and after." When the only Scriptural basis for rejecting the validity of speaking in tongues rests on such a questionable and tenuous interpretation of the Greek word \emph{teleios}, which was wrested from the context in which it is used, one has to sincerely challenge the expositional honesty of such scholarship. To be kind, I will say that at best it is prejudicial blindness - not at all scholarly or conclusive. 

It should also be noted that, simultaneously with the tongues ceasing in I Corinthians 13:8, it declared that prophecies would fail and knowledge would vanish away. Is anyone willing to admit that God no longer speaks to the church to edify, exhort, or comfort it? Has knowledge vanished away? The Scriptures declare that we know in part. Some seem to pretend to perfect knowledge, but I seriously doubt their pretensions. We will not know even as we are known until Christ comes again. 

\section*{The Spirit in Church History}

Since there is no solid Scriptural basis for denying the validity of speaking in tongues today, what other basis do we have to challenge the exercise of this gift? There is always the alleged absence of its use in the subsequent history of the church. This is not true, however, for throughout the history of the church the issue seemed to crop up every now and then. There are reports of speaking in tongues among zealous groups throughout church history. Its seeming absence of practice during much of church history is not a strong witness against its validity. 

I personally am not proud of the traditional church history. It seems to me that it is the story of failure. The New Testament church thrived during the apostolic age; Paul was able to report to the Colossians that the truth of the gospel had gone into all the world and was bearing fruit (Colossians 1:6). With the guidance and empowering of the Spirit they were able to take the gospel to all the world in the first century. This is a feat which the traditional church has not been able to duplicate in all the subsequent ages. It is tragic that many people seek to relegate the special power of the Holy Spirit to the apostolic period only, and have now substituted the genius and programs of man to accomplish Christ's Great Commission. The result has been the dismal failure of the church. One must seriously question if it was God's plan or man's pride to set aside the dependency upon the guidance and the power of the Holy Spirit to reach the lost world for Jesus Christ. 

Paul said to the Galatians, "Are you so foolish? Having begun in the Spirit, are you now made perfect by the flesh?" (Galatians 3:3). This is precisely what is being declared by those who would relegate the operations of the gifts of the Spirit to the apostolic age only. The church, they say, was begun in the Spirit to help them overcome all the obstacles of the pagan, hostile world. But once seminaries and organizational structures were established, they no longer needed that power of the Spirit. The church could now be perfected by the trained men. An honest look at church history should dispel that fallacy once and for all. 

\section*{Joel's Promise}

As we consider the promise of the Spirit in Joel 2:28, and as we read the whole context of that promise, we see that he was referring to the last days. The prophecy actually carries right into the tribulation period, with the sun turning to darkness and the moon to blood, and onto the coming great day of the Lord, when that which is perfect has come. What began at Pentecost was obviously to continue to the coming again of Jesus Christ. Peter confirmed this when he spoke to the inquiring throng on the Day of Pentecost who were asking, "What shall we do?" He commanded, "Repent and be baptized, every one of you, in the name of Jesus Christ for the remission of sins, and you shall receive the gift of the Holy Spirit. For the promise [the promise in Joel] is unto you and to your children, and to all who are afar off, as many as the Lord our God shall call" (Acts 2:38,39). Nothing was said about a cutoff date at the end of the apostolic period. That idea is an invention of man to excuse the lack of power in their churches and in their lives today. 

We certainly are not advocating that everyone speak in tongues. Paul, through his rhetorical question, "Do all speak in tongues?" (I Corinthians 12:30), expected a "no" answer, even as all do not have the gifts of healing. On the other hand, I feel that it is wrong to forbid, or even discourage, the speaking in tongues by those who want to use the gift to assist them in their prayer life or personal devotions to God. 


