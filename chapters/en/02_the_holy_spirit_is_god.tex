\chapter{The Holy Spirit Is God}

Since the purpose of this book is to bring you into a full,
personal, and soundly biblical relationship with God the
Holy Spirit, we need to first show that the Holy Spirit is one
of the three Persons of the Godhead.

The church has accepted throughout its history that
there is one God who exists in three Persons: the Father, the
Son, and the Holy Spirit. In the Scriptures we find Them
working together in total harmony for the redemption of
man. Paul confessed to Timothy that the Godhead was a
great mystery; for us to try to \emph{fully} comprehend it is a futile
expenditure of mental energy.

Many cult groups (such as Jehovah’s Witnesses) take
advantage of this gulf between the finite and the infinite
to attack the triunity of God by denying the deity of Jesus
Christ and passing off the Holy Spirit as an essence. Other
groups deny the existence of the Father and the Holy Spirit,
and say that Jesus alone is God. One of the common marks
of every cult is a denial of the deity of Jesus Christ and the
Person of the Holy Spirit.

\section*{The Triune God}

Because this is one of the areas that the enemy constantly
attacks, we must affirm not only the fact of the deity
of the Holy Spirit, but also why we believe in His deity. The
word trinity is not found in the Bible, but it is a convenient
term that theologians use to describe the three Persons of
the one God. Perhaps the term \emph{triunity} would more accurately
describe God. He is not 1 + 1 + 1 = 3, but 1 x 1 x 1 =
1.

In Genesis 1:1 we read, “In the beginning God…” The
Hebrew word translated God is “Elohim,” which is plural for
El (God in the singular). In Hebrew there is a singular, dual,
and plural tense. “God” in the singular is El, in the dual is
Elah, and in the plural is Elohim. There can be no denying
that the word Elohim at least suggests the triunity of God.

Continuing in Genesis 1:2 we read, “…the Spirit of God
moved upon the face of the waters.” The Holy Spirit is the
first Person of the Godhead to be identified separately in the
Bible. In Genesis 1:26 we read, “And [Elohim] said, Let \emph{us}
make man in our image, after our likeness.” He did not say,
“\emph{I} will make man after \emph{my} image.” In other words, the three
Persons of the Godhead were speaking jointly.


\section*{The Attributes of the Holy Spirit}

To establish that the Holy Spirit is God, we will first
show that attributes which can be ascribed to God alone
are ascribed to the Holy Spirit. One of the divine attributes
is the eternal nature of God. He has always existed. In
Hebrews 9:14 we read that Christ through the \emph{eternal} Spirit
offered Himself without spot to God. If the Spirit is eternal,
and this is an attribute that can only be ascribed to deity,
then the Spirit is God. Notice also how the three Persons of
the Trinity are linked in the verse.

Another attribute of God is His omniscience. God knows
all things, as James said in Acts 15:18: “Known unto God
are all his works from the beginning of the world.” This
attribute is also ascribed to the Holy Spirit. In 1 Corinthians
2:10–11 we read, “But God hath revealed them unto us by
his Spirit: for the Spirit searcheth all things, yea, the deep
things of God. For what man knoweth the things of a man
save the spirit of man which is in him? Even so the things
of God knoweth no man, but the Spirit of God.” Here the
knowledge of God is attributed to the Spirit of God.

Another attribute of deity is omnipresence. God exists
everywhere in the universe at once. In Psalm 139:7 David
asks, “Whither shall I go from thy spirit? Or whither shall I
flee from thy presence?” God exists in the heavens, in hell,
and in the uttermost parts of the sea. The Spirit is with me
now where I am, and at the same time He is with you wherever
you may be reading this book right now. God is omnipotent.
This is a word used to express that He is all powerful.
When Sarah laughed at the announcement that she was to
have a son in her old age, the angel of the Lord asked, “Is
any thing too hard for the LORD?” (Genesis 18:14). Jesus said,
“With God all things are possible” (Mark 10:27). In Luke
1:37 He said, “With God nothing shall be impossible.” The
angel said to Mary when she questioned him on how she, a
virgin, could bear a child, “The Holy Spirit shall come upon
thee, and the \emph{power of the Highest} shall overshadow thee”
(Luke 1:35). Here the Holy Spirit and the power of the highest
are used synonymously.


\section*{The Works of the Spirit}

Not only are divine \emph{attributes} ascribed to the Holy Spirit,
but so are divine works. One of the divine \emph{works} is that of
creation. The entire Trinity was active in creation. In Genesis
1:1 we read, “In the beginning [Elohim] created the heavens
and the earth.” In John 1:1–3 we read, “In the beginning was
the Word, and the Word was with God, and the Word was
God.… All things were made by him; and without him was
not any thing made that was made.” The Spirit was also an
active force in creation. In Genesis 1:2 the Spirit is described
as moving over the face of the waters. The Spirit was in conference
with the Father and the Son when God said, “Let us
make man in \emph{our} likeness” (Genesis 1:26). In Psalm 104:30
we read, “Thou sendest forth thy Spirit, they are created.”

Another work of God is that of giving life. We recognize
that God is the giver and sustainer of life. In 2 Corinthians
3:6, as Paul was referring to the letter of the law, he said,
“The letter killeth, but the spirit giveth life.” In John 6:63
Jesus said, “It is the spirit that [maketh alive].”

The Bible was written by the inspiration of the Holy
Spirit, yet we properly refer to the Bible as the Word of God.
Second Peter 1:21 tells us, “For the prophecy came not in
old time by the will of man: but holy men of God spake as
they were moved by the Holy [Spirit].” In 2 Timothy 3:16
Paul declares, “All scripture is given by inspiration of God.”
Peter says that the writers were moved by the Holy Spirit
and Paul says that they were inspired by God. Thus the
Spirit is recognized to be God.

This is why many Scriptures in the Old Testament which
declare that \emph{the Lord} spoke are attributed to the Holy Spirit
when quoted in the New Testament. In Isaiah 6:8–9 the
prophet said, “I heard the voice of the Lord, saying, Whom
shall I send, and who will go for \emph{us}? Then said I, Here am
I; send me. And he said, Go, and tell this people, hear ye
indeed, but understand not; and see ye indeed, but perceive
not.” When Paul quoted this passage in Acts 28:25–26 he
said, “Well spake the Holy [Spirit] by Isaiah the prophet
unto our fathers, saying, …hearing ye shall hear, and shall
not understand; and seeing ye shall see, and not perceive.”
Isaiah said the Lord spoke; Paul said the Holy Spirit spoke.
They can both be right only if the Holy Spirit and the Lord
are one.


\section*{The Trinity Working Together}

In Acts 5:1–11 we have an interesting account of discipline
in the infant church as God was seeking to preserve
its purity. Motivated by love, many Christians attempted
to establish a Christian community by selling all their possessions
and turning the proceeds over to the apostles, so
that the Christians might have all things in common. A certain
couple, Ananias and Sapphira, sold their property, but
together decided to hold back a share of the price for themselves.
When Ananias brought his portion to Peter, Peter
asked, “Why hath Satan filled thine heart to lie to the Holy
[Spirit], and to keep part of the price of the land? While it
remained, was it not thine own? And after it was sold, was
it not in thine own power? Why hast thou conceived this
thing in thine heart? Thou hast not lied unto men, but unto
God” (Acts 5:3–4). Peter said that Satan had filled the heart
of Ananias to lie to the Holy Spirit, then declared that he
had lied to \emph{God}, thereby making the Holy Spirit and God
one.

Throughout the New Testament we see the Trinity working
together or coupled together. When Jesus commissioned
the disciples to go out and teach all nations (Matthew
28:19–20), He told them to baptize in the name of the Father,
and of the Son, and of the Holy Spirit. These three names
distinguish the three Persons of the one God.

In 2 Corinthians 13:14, in his apostolic benediction, Paul
said, “The grace of the Lord Jesus Christ, and the love of
God, and the communion of the Holy [Spirit] be with you
all. Amen.” Here again the three Persons of the one God are
linked together.

In 1 Corinthians 12:4–6 Paul says, “Now there are diversities
of gifts, but the same \emph{Spirit}. And there are differences
of administrations, but the same \emph{Lord}. And there are diversities
of operations, but it is the same \emph{God} which worketh
all in all.” In verse four he refers to the Spirit, in verse five
to the Lord (Jesus), and in verse six to God (the Father).
So, though there may be diversities in the gifts and in their
operation and administration, there is a unity because God
is behind it all.


\section*{Access Through the Spirit}

At this point you may be thinking, “Well, what difference
does it all make whether the Spirit is God or just an
essence from God?” Because the Spirit is a part of the Godhead,
it is proper to worship Him, and we are correct when
we sing, “Praise Father, Son, and Holy Ghost.” God has
ordained that we relate to Him through the Spirit. It is in the
realm of the Spirit that man can touch God. It is my spirit
brought into union with the Holy Spirit. Jesus said, “God
is a Spirit, and they that worship him \emph{must} worship him in
spirit and in truth” (John 4:24). Paul also said, “The Spirit
itself beareth witness with our spirit” (Romans 8:16). If I am
to have communion with God, I must recognize the Holy
Spirit and realize that He is the One that makes this fellowship
possible.

Man has never had direct access to the Father; this is
a common fallacy among people who forget the awesome
holiness of God. When God manifested Himself on the Holy
Mount to the Jewish people (Exodus 19), He had them set
boundaries around the mountain so they wouldn’t get too
close to the manifestation of God and be put to death. When
the people saw from afar the awesome demonstration of
God, they said to Moses, “You speak to us and we will hear,
but don’t let God speak to us lest we die.”

The veil in the tabernacle demonstrated the separation
that must exist between the Holy God and an unholy
people. This veil could only be penetrated after an elaborate
cleansing and sacrifices by the high priest, and this only one
day in the year, and by only one man, the high priest.

Jesus said, “No man cometh unto the Father, but by me”
(John 14:6). Jesus told the Jews that they really didn’t know
the Father. He also told them that Moses would be the witness
against them. They do not follow the prescribed way
to God that was given to Moses by God, but today seek
to approach Him on the basis of their good works without
sacrifice. Sin has always been the barrier between man and
God, and until something is done about man’s sin, there can
be no approach to God. In Isaiah 59:1–2 we read, “Behold,
the LORD’s hand is not shortened, that it cannot save; neither
his ear heavy, that it cannot hear: but your iniquities
have separated between you and your God, and your sins
have hid his face from you, that he will not hear.” Jesus
provided a way to cleanse us from our sins, thus making
the approach to God possible. Through faith in Jesus Christ
my spirit is made alive, and thus can be united with God’s
Spirit. In this way God and man are joined in the Spirit.





