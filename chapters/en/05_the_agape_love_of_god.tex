\chapter{The Agapé Love of God}

If a person did not make it to heaven he could blame 
a lot of people or circumstances, but there is one Person 
he will never be able to blame, and that is God. Someone 
might say, “It’s the church’s fault—I tried the church but 
it just didn’t do anything for me.” He might blame some 
poor example of Christianity that he saw: “Well, he said he 
was a Christian; but I saw the way he lived, and I decided I 
wanted nothing to do with that.”

But one Person you can never blame is God. When I 
think of all that God has done to bring us salvation, I realize what a fight it is not to be saved. The Bible says, “If 
God is for us, who can be against us?” God so loved the 
world that He gave His only begotten Son, and then God 
sent His Spirit into the world to convict us of our sins and 
to draw us to Jesus Christ. He points out our own helplessness, and then points to Jesus Christ as the Answer—the 
Way, the Truth, and the Life.


\section*{Starting the Real Work}

Once we have been brought to this point of coming to 
Jesus, and we say, “Okay, Lord, take over my life”—the 
moment we surrender ourselves to God—the Holy Spirit 
begins His work in us in earnest. The moment the door 
of our heart is open to receive salvation, the Holy Spirit 
comes into our life, and He begins to make those necessary 
changes within, conforming us into the image of Christ and 
empowering us to be the kind of person the Lord would
have us to be. He gives us knowledge and understanding 
in the things of God, so that suddenly the Bible becomes a 
totally new book to us. As we start to read, it begins to come 
alive, because the Spirit begins to open up our understanding
 and fill our heart with God’s \emph{agapé} love. But it first takes 
our coming to Jesus and submitting our life to Him.

Jesus said in Revelation 3:20, “Behold, I stand at the 
door, and knock: if any man hear my voice, and open the 
door, I will come in to him, and will sup with him, and he 
with me.” It actually means “eat supper with him.” In the 
Orient, the greatest method of communion with a person 
was to eat with him. In eating together with a person, you 
have created a unity or a bond. Since you have partaken 
of the same food, it now becomes part of both of you, and 
so you become part of each other. The people of the Orient 
placed great significance on the breaking of bread together 
and the drinking from the same cup, because it created an 
affinity, a unity. It is interesting to note that Jesus always 
liked to eat supper with people. He enjoyed that oneness, 
that identity with people.

It is significant that Jesus says, “I’m standing at the door 
knocking; if you will open the door, I will come in and 
eat supper with you.” He will come into your life and you 
can begin that beautiful, intimate relationship with Him in 
which you become a part of each other. This is all done 
through the work of the Holy Spirit. The moment I open the 
door and believe on Jesus, the Spirit does a marvelous work 
for me and in me. In Ephesians 1:13 and following, we read 
about this work of the Holy Spirit in sealing the believer. 
Paul describes for us all the fantastic blessings that we have 
as children of God. He begins in 1:3 by saying, “Blessed 
be the God and Father of our Lord Jesus Christ, who has 
blessed us with all spiritual blessings.” Then he begins to 
list some of those glorious blessings with which God has 
blessed us.


\section*{Blessed Beyond Description}

The Christian is actually the most blessed person in the 
world. God has just blessed us until we cannot take any 
more. And after He is all through blessing us here, He is 
going to receive us into the eternal glory, where He is going 
to bless us forever. Paul talks about the blessings of God: 
He has chosen us, predestined us, accepted us, redeemed 
us, forgiven us, made known to us the mystery of His will, 
and given us an inheritance. In Ephesians 1:13 Paul says, 
“In whom ye also trusted after ye heard the word of truth, 
the gospel of your salvation: in whom also after that ye 
believed.”

It all begins with hearing “the word of truth, the gospel 
of your salvation.” Paul said, “How can they believe on 
Him whom they have not heard?” It is necessary for faith 
that you first hear the message that God loves you with an 
everlasting love, and that because He loves you He sent His 
Son to take your sin and die in your place, so that which 
kept you from God could be put aside and nothing would 
hinder your fellowship with God.

God said through the prophet Isaiah, “The LORD’s hand 
is not shortened, that he cannot save; neither is his ear 
heavy, that he cannot hear: but your sins have separated 
between you and your God” (Isaiah 59:1–2). That is always 
the tragic by-product of sin—separation from God. Sin in 
my life separates me from God. God did not want that separation, but sin had to be dealt with, so God sent His Son to 
take my sins—to die in my place—so that I would not have 
to be separated, and could come back into fellowship with 
God.

If you are born again, you heard God’s good news for 
you, and you trusted after you heard. First there was the 
\emph{hearing} and then the \emph{believing} of what God said. Then, after 
you believed, after you opened the door, you were “sealed 
with that Holy Spirit of promise.”


\section*{God's Seal of Ownership}

The seal was used in ancient days primarily as a stamp 
of ownership. The city of Ephesus was a major port where 
goods from Asia were brought and then shipped to other 
places, including Rome. The merchants from Rome would 
come to Ephesus to buy their goods there. Then they would 
seal those goods with their signet ring in wax. When the 
ship arrived at Puteoli (the port where they would land the 
goods going to Rome), the merchant would claim his goods, 
proving his ownership by his own personal seal. If someone else started to claim the goods, the merchant could say, 
“Those are mine. That’s my stamp of ownership.”

The beautiful biblical truth is that, once I believed, God 
sealed me with His own personal stamp of ownership. He 
actually claimed me as His own possession, so that when 
the enemy would try to claim me, God would say, “Keep 
your hands off—he’s mine!” That seal of God’s ownership 
is the Holy Spirit. When you believe in Jesus Christ, the 
Holy Spirit comes into your life, and the indwelling of the 
Holy Spirit in your life is God’s seal, God’s mark of ownership by which He claims you as His own.


\section*{Treasured by God}

I do not understand why God prizes me so highly, but 
He does. In Ephesians 1:18 Paul says, “…that ye may know 
what is the hope of his calling, and what the riches of the 
glory of his inheritance in the saints.” He’s saying, in other 
words, “May God open up your eyes to realize how much 
He treasures you!” I pray that God would open my eyes by 
the Spirit so that I might find how much God treasures me. 
That to me is glorious—that God would treasure me. For 
His own reasons God treasures us, and He has placed His 
seal on us. The Holy Spirit within us is God’s seal. We 
are also told this in 2 Corinthians 1:22: “Who hath also 
sealed us, and given the earnest of the Spirit in our hearts.” 
In Ephesians 4:30 we are commanded, “Grieve not the 
Holy Spirit of God, whereby ye are sealed unto the day of 
redemption.”


\section*{The Future Claim}

God has put his stamp of ownership on you at this 
point because He is claiming you as His own, even though 
your redemption is not yet complete. That is why the merchants put their stamp of ownership on their merchandise—
because they had not yet claimed their goods in the home 
port. So wherever these goods went they were marked; they 
had the seal of ownership. In the same way God has put His 
seal of ownership on you, though He has not yet claimed 
His purchased possession. Our redemption is not yet complete, but the Holy Spirit is that seal and “the earnest.”

In Ephesians 1:14 Paul declares, “Which is the earnest of 
our inheritance until the redemption of the purchased possession, unto the praise of his glory.” The Holy Spirit is not 
only the \emph{seal} of God’s ownership, but He is also the \emph{earnest}, 
which means a deposit or down payment. God has every 
intention of completing your redemption. He has made the 
deposit or down payment—the Holy Spirit. God is declaring His intention of completing His transaction for you.

This redemption will not be complete until we are freed 
from this body. This body is the thing that is still dragging 
us down. Paul said, “We who are in this body do groan.” 
In Romans 8:22 he describes how we “groan and travail.” 
All of creation is actually groaning and travailing together 
until now. Romans 8:23 says, “And not only they, but ourselves also, which have the firstfruits of the Spirit, even we 
ourselves groan within ourselves, waiting for the adoption 
[literally being placed as sons], …the redemption of our 
body.”


\section*{The End of the Old Body}

Robert Service, in his poem “The Cremation of Sam 
Magee,” speaks of hurrying on “with a corpse half hid that 
he couldn’t get rid.” It was lashed to the sleigh, but he 
couldn’t get rid of it because he had made this promise on 
Christmas Eve. That is the way we Christians are. We have 
a redeemed spirit which is alive unto God, but we have to 
carry around this old corpse of our body. It hangs on wherever we go, until someday we finally get rid of our load. 
Paul said, “We who are in this body do groan, earnestly 
desiring to be delivered, not that we would be unclothed, 
but that we might be clothed with that body which is from 
heaven” (see 2 Corinthians 5:1–4). That will be the completion of our redemption. That is what I am waiting for.

Some people are troubled to think they are going to get 
rid of this body. It doesn’t trouble me. The apostle Paul said, 
“We know that when the earthly body of this tent is dissolved, we have a building of God not made with hands, 
eternal in the heavens.” This body that I now possess has 
been passed down to me through my ancestry; all the gene 
factors have been passed down the line, so I am a composite of my ancestry. I have picked up all the inherited characteristics of man’s failure, and so here I am in my groaning 
body.


\section*{The New Body from Heaven}

The new body that I am going to have will not be passed 
down through failing man; it will be given to me directly 
from God. It is not going to be subject to pain or fatigue 
or gimpy football knees or so many of the other things I 
have experienced in this body. It is going to be coming to me 
directly from God. Paul calls this present body a “tent” in 
2 Corinthians 5:4. You never think of a tent as a permanent 
place to live. If you have to live in a tent, it is all right for 
a few weeks in the mountains on vacation, but you do not 
like to think of it as a permanent place to live. It is so much 
better to move out of the tent into a real house. God is planning to redeem us completely.

The redemption includes not just a \emph{redeemed} body, but a 
\emph{new} body. With my mind I want to do the will of God. With 
my mind I want to turn it all over to God. Completely and 
fully I want to live the kind of life that God wants me to live. 
There is no problem with my heart and my mind. My problem is that
 my \emph{body} keeps dragging me down.
 It keeps pulling me back and pulling me down, so that I do not always 
do those things that I want to do. I am pulled down by my 
body appetites. I cannot be all that I want to be, so I groan. 
All creation around us is groaning, waiting for the day of 
redemption, when God lays claim to that which is His. He 
has His stamp of ownership on it, and one day He is going 
to come down and say, “This is it.” He is going to release my 
soul and spirit from my body and immediately incorporate 
it into that new body which is from heaven.


\section*{The Suffering World}

The same is true of this world. Right now the whole 
world is suffering as the result of sin: “All creation groans 
and travails.” Every thorn, they say, is an undeveloped blossom. Thorns have come as a result of the curse. And a thorn 
is just a mark of groaning creation—wanting to blossom 
out, but unable to. All creation suffers under the curse of 
sin, waiting for that day of deliverance, waiting for that day 
when God redeems what He has purchased.

Jesus purchased the world, but He has not yet claimed 
it. It belongs to Him, but He has not claimed it. It is still 
under Satan’s control. But one of these days very soon He 
will come back to claim what He purchased. We read about 
this in Revelation chapter 5. There is a scroll in the right 
hand of Him who sits upon the throne. An angel declares 
in a loud voice, “Who is worthy to take the scroll and loose 
the seals?” John responds, “I began to weep because no one 
was found worthy.” The elder replies, “Don’t weep, John;
the Lion of the tribe of Judah has prevailed to take the scroll 
and to loose its seals.” So John says, “I turned and saw 
Him like a Lamb that had been slaughtered, and He took 
the scroll out of the right hand of Him who sits upon the 
throne.” This scroll is the title deed to the earth. Who is 
worthy to take it? Who is worthy to claim it? No one but 
Jesus, for He purchased it on the cross, and He is coming 
back to lay claim to His purchased possession. And \emph{I} am His 
purchased possession!


\section*{My Assurance}

The work of the Holy Spirit today in my life is that of 
having sealed me. So His indwelling gives me real assurance. When Satan comes and begins to badger me because 
of the weakness of my flesh and my failures, and begins 
to tell me that God is not interested in me and God does 
not love me and is not going to save me, I say, “Satan, 
you are \emph{wrong}! I have God’s seal; He has marked me; He 
has stamped me with His seal of ownership. The Holy 
Spirit indwells me. God has sealed me! He has given 
me the deposit, and He is coming to claim what He has 
redeemed.”

When we get to heaven (either when Jesus comes for us 
at the rapture or when we die), we will have our redemption complete. The work of Christ will be finished in us, and 
we will share forever in the glorious kingdom of God without any further hindrances of this body. We will be able to 
love, to share, to give, and to relate with one another without any restrictions or limitations.

What a glorious day! What a glorious work of God in 
sealing us and giving us the earnest of the Spirit until that 
day of the redemption of the purchased possession!


\section*{The Agape Love of Christ}

Another work of the Holy Spirit in the life of the believer 
is to bring us the agapé love of Jesus Christ. Jesus said to 
His disciples, “By this sign shall men know that ye are my 
disciples: that ye  \emph{love} one another.” The word translated 
\emph{love} is the Greek word “\emph{agapé,}” which is rarely found in 
Greek outside the Bible. It is a word that was used by our 
Lord Jesus Christ to define a quality of love above an ordinary experience of love. The English language is restricted 
in some ways, and perhaps most limited in its ability to 
express love. The French say, “You Englishmen have one 
way to tell a woman you love her. We have a hundred.” 
They are expressing how much freer and fuller the French 
language is than the English language in this respect. Within 
the Greek language are several words for love, but we are 
limited to just the word \emph{love} in English. I \emph{love} peanuts, I \emph{love}
Cracker Jacks, and I \emph{love} my wife. I have to use the same 
word to describe my feelings for hot fudge sundaes as my 
feelings for my children. Yet what I feel toward hot fudge 
sundaes is entirely different from what I feel toward my 
children or my wife. I am stuck with the one word “love.”


\section*{Love's Three Words}

In the Greek language there is a word for love on the 
physical plane—the word \emph{Eros}. It is easy to see the English 
words that we get from this Greek word, such as “erotic.” It 
is love on the purely physical plane. This word has become 
vogue among young people today, and I guess older people, 
too. They say, “Let’s make love,” and by this they are 
referring to an Eros experience, which does not necessarily 
involve true love at all.

The Greeks have a second word for love, a love on a 
higher plane than the physical—the mental plane,
 an emotional relationship. The word is  \emph{Phileo}. This is far deeper 
than Eros, because this involves deeper interaction with 
another person. Phileo is developed by conversing and 
finding out that we like the same things. We have a lot in 
common; we appreciate each other, and through a mutual 
sharing we come into the Phileo experience of love.

\emph{Agapé} love is total love. It is love in the deepest area; it is 
true spiritual love. Eros is not true love. If I say “I love you” 
in the Eros realm, what I am really saying is “I love \emph{me}, and 
I want you because I’m in love with me, and I need you.” 
If someone says, “I can’t live without you,” that does not 
express deep love for you—it only shows that he is thinking 
of himself. Eros is extremely selfish. It is self-love.

Phileo love is reciprocal: “I love you because you love 
me; I love you because you laugh at my jokes; I love you 
because we like so many of the same things; I love you 
because we get along well together and have a lot of fun 
when we’re together. I love you because you’re a pleasant 
person, and we have a great time together.”


\section*{The Highest Love}

Agapé love remains loving even if there is not a return 
of love. It is a deep love that gives and asks nothing in 
return. The love is so deep and so great that it just keeps on 
giving. In fact, that is the chief concern of agapé: giving. The 
word \emph{agapé} is such a vast, broad word that it is difficult for 
us to even define it in the English language. It is impossible 
for us to understand it apart from the Spirit of God and His 
revelation to our heart, because it is not a natural love; it is 
a supernatural love. The Bible says, “God is \emph{agapé}.” It is a 
divine love, and it is probably best defined for us in 1 Corinthians 13.

First of all, Paul points out the supremacy of this agapé 
love. It is more important that you have this kind of love 
than that you have spiritual gifts. Paul said, “Though I 
speak with the tongues of men and of angels, and have not 
love (agapé), I am become as sounding brass or a [clanging] cymbal.” You may have very powerful oratory; you
may have a silver tongue; you may be able to express yourself extremely well. But if you do not have agapé love, it is 
no more meaningful than just clanging on a cymbal. It is a 
meaningless sound.

Agapé is more important than the gifts of prophecy or 
the word of knowledge or the gift of faith, for “though I 
could prophecy and understand all things, and though I 
had all knowledge and faith so that I could remove mountains, if I have not agapé, I am nothing.” Paul continues, 
“and though I bestow all my goods to feed the poor…” 
agapé love is even more important than sacrifice. You may 
sell everything you have and feed the poor, and give your 
body to be burned, making the supreme sacrifice, but if you 
do not have agapé love, it profits you nothing.

Now Paul goes on to define this kind of love. “Love suffers long and is kind.” This means that agapé love receives 
abuse and suffers long. It takes and takes and at the end 
of the taking is still kind. You have heard people say, “All 
right, I’ve taken and taken, and that’s it; now I’m going 
to get even.” That is not agapé. Agapé takes and takes 
and then is still kind. It is not crying for vengeance. “Love 
envieth not; love vaunteth not itself, is not puffed up, doth 
not behave itself unseemly, seeketh not her own [does not 
insist on its own way], is not easily provoked.” (The word 
\emph{easily} was inserted by the translators. They could not quite 
understand “is not provoked.”) Love “thinketh no evil; 
rejoiceth not in iniquity, but rejoiceth in the truth; beareth all 
things, believeth all things, hopeth all things, endureth all 
things.” Agapé love “never fails.”


\section*{Love's Two Signs}

Jesus said, “By this sign shall men know that ye are my 
disciples, because ye love [agapé] one another.” This really 
becomes the most powerful evidence to the world that we 
are the disciples of our Lord Jesus Christ. This agapé love 
should be working within our lives and making us one with
each other, giving preference to one another, not exalting 
ourselves or forming cliques, but just sharing that oneness 
of love that makes us all one together. We should share 
together with one another the goodness and grace of God, 
freely giving as we have freely received of God’s love and 
grace. As this agapé love works within our lives, it becomes 
the sign to the world that we are truly Christ’s disciples.

In 1 John 3:14 we read, “We know that we have passed 
from death unto life, because we \emph{love} the brethren.” Again, 
the word \emph{agapé} is used. Not only is it a sign to the \emph{world} that 
we are Christ’s disciples, but it is a sign to us that we have 
passed from death to life. As God’s love begins to work in 
my life, it becomes a sign to  \emph{me} that I have passed from 
death to life, because I have this love for the brethren, for 
those within the body of Christ.


\section*{The Source of True Love}

Because this is a divine love, its source is in God. It is not 
something I can generate. It is not something I can work up 
within myself. This has been one of the difficulties within 
the Christian community—knowing that we are to love all 
believers, but also knowing that there are those whom we 
do not truly love. So we try to work up an artificial love. 
We try to talk ourselves into it. But agapé love does not 
originate with me; agapé’s origin is in God. God is agapé. I 
cannot develop it; it is something that has to come to me as 
a work of God within my life. If I find that I am lacking in 
this love, I cannot really do anything about it myself; I must 
just confess this lack to God and ask Him to plant that agapé 
within me.

Many Christians have been totally frustrated because 
they have tried to produce this agapé. They have sought so 
hard to love with this divine love, but they cannot do it. 
Its origin is in God, and it has to come from God as a gift 
to you, and then it goes forth from your life. If you find 
yourself lacking in this agapé, the only thing you can do is 
to ask God to fill your heart with agapé through the Holy 
Spirit. Do not browbeat yourself and become defeated in 
your spiritual walk because you find that you do not have 
this agapé as you should; just ask the Lord for it.


