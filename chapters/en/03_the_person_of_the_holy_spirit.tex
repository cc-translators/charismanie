\chapter{The Person of the Holy Spirit}

Because we want you to have a personal encounter
with the Holy Spirit, we will next show that the Scriptures
teach that the Holy Spirit is a Person, rather than merely an
essence, force, or power. You can have raw power without
personality, such as electricity, but it is difficult to have an
intimate, close relationship to such impersonal power.

The word \emph{spirit} in Greek is “\emph{pneuma},” which is in the
neuter gender. Because of this, in early church history a brilliant
theologian named Arius began to promote the idea
that Jesus was less than God, having been created by God,
and that the Holy Spirit is just the “essence” of God. This
became known as the Arian heresy, and it still exists and
attracts a wide following. The Nicaean Counsel stripped
Arius from his position and branded his teachings as heresy.
The Holy Spirit is more than just an essence or force; He
is a Person. You shouldn’t worship a force or essence. Can
you imagine singing the doxology, “Praise Father, Son, and
Essence”? He is a Person, and as one of the Persons of the
Godhead He is worthy to be praised. If we do not believe in
the personality of the Holy Spirit, we deny Him the praise
and worship due Him. If we do not realize that the Holy
Spirit is a Person, we find ourselves in the position of seeking
to relate to a force or essence. We would be saying, “I
need to yield my life to it,” or, “I need more of it in my
life.”


\section*{Knowing, Acting, Feeling}

That He is a Person is clearly shown in the Scriptures.
Characteristics are ascribed to Him that can only be ascribed
to persons. A person is a being with a mind, will, and feelings.
If in the Scriptures these characteristics are ascribed
to the Holy Spirit, then it must be concluded that the Spirit
is a Person. In 1 Corinthians 2:10–11 we read, “But God
hath revealed them unto us by his Spirit: for the Spirit searcheth
all things, yea, the deep things of God. For what man
knoweth the things of a man, save the spirit of man which is
in him? Even so the things of God knoweth no man, but the
Spirit of God.” Here reference is made to the Spirit possessing
knowledge. Raw force or power possesses no knowledge.
It would be absurd to replace the word essence for
Spirit in the text, for you would have the “essence” searching
all things!

In Romans 8:27 Paul says, “And he that searcheth the
hearts knoweth what is the mind of the Spirit, because he
maketh intercession for the saints according to the will of
God.” Here reference is made to the mind of the Spirit, a
characteristic not associated with just an essence. In 1 Corinthians
12:11 Paul, concerning the gifts of the Spirit, says,
“But all these worketh that one and selfsame Spirit, dividing
to every man severally as he will.” So the Holy Spirit
possesses a will, a trait associated with personality.

In Romans 15:30 Paul associates the emotion of love
to the Spirit. A force or power cannot love. You do not
associate love apart from personality. It is interesting that,
although I have read or heard scores of sermons on God’s
love, or the love of Jesus Christ for us, I have yet to hear a
sermon on the love of the Holy Spirit. Yet this must be one
of the chief characteristics of the Spirit, as this is the fruit He
produces in our lives. The Holy Spirit does possess feelings
and can be grieved, because Paul in Ephesians 4:30 admonishes
the church not to grieve the Holy Spirit of God. Think
how foolish it would sound to say you have grieved the
essence.


\section*{The Personal Words}

Throughout the Scriptures personal pronouns are used
to refer to the Holy Spirit. In John 14:16 Jesus said, “And
I will pray the Father, and he shall give you another Comforter,
that he may abide with you for ever.” Here the pronoun
“he” is used for both the Father and the Spirit. If you
believe in a personal God, you should also believe in a personal
Spirit. In that same passage Jesus went on to say that
the world could not receive the Spirit because they did not
see Him or know Him; Jesus said that you know Him, for He
dwells with you.

Notice how many times Jesus uses the personal pronoun
for the Holy Spirit. In John 16:7–14 Jesus repeatedly
uses the personal pronoun to refer to the Holy Spirit. “Nevertheless
I tell you the truth; It is expedient for you that I go
away: for if I go not away, the Comforter will not come unto
you; but if I depart, I will send him unto you. And when he
is come, he will reprove the world of sin, and of righteousness,
and of judgment: of sin, because they believe not on
me; of righteousness, because I go to my Father, and ye see
me no more; of judgment, because the prince of this world
is judged. I have yet many things to say unto you, but ye
cannot bear them now. Howbeit when he, the Spirit of truth,
is come, he will guide you into all truth: for he shall not
speak of himself; but whatsoever he shall hear, that shall he
speak: and he will shew you things to come. He shall glorify
me: for he shall receive of mine, and shall shew it unto
you. All things that the Father hath are mine: therefore said
I, that he shall take of mine, and shall show it unto you.” In
the Greek, the personal pronouns “he” and “him” are used
for the Spirit over and over in Scripture.


\section*{The Spirit in Action}

Personal acts are ascribed to the Holy Spirit in the Scriptures.
In Acts 13:2 we read, “The Holy [Spirit] said, Separate
me Barnabas and Saul for the work whereunto I have
called them.” Again, to insert “power” or “essence” for the
Spirit is incomprehensible. How can an essence or power
speak? In Romans 8:26 we are told that the Holy Spirit Himself
makes intercession for us with groanings which cannot
be uttered. Again, try to conceive a mere force making intercession!
If the Holy Spirit were just an essence or force
where He is mentioned in Scripture, you should be able to
insert the words “force” or “essence” and do no damage
to the meaning of the text. But such a thing is obviously
impossible, because the Holy Spirit is a Person. The Holy
Spirit testifies of Jesus Christ in John 15:26, and He teaches
the believers and brings things to their remembrance in
John 14:26. In Acts 16:2, 7 the Holy Spirit forbade Paul and
his companions to go into Asia and would not allow them
to go into Bithynia. In Genesis 6:3 we find that the Holy
Spirit strives with man.

The Holy Spirit can receive treatment as a Person. He
can be offended. It is impossible to conceive of offending
“the power” or “the breath.” Your breath can be offensive,
but you can’t offend your breath! In Ephesians 4:30, Paul
exhorts, “Grieve not the Holy Spirit of God.” The Holy
Spirit can be lied to. This is the accusation that Peter brought
against Ananias: “You have lied to the Holy Spirit.” It is also
possible to blaspheme the Holy Spirit. Jesus said that this
was such a heinous sin that there was no forgiveness for the
person who did it. He said, “You can blaspheme me and be
forgiven, but not the Holy Spirit.” Here Jesus makes the distinct
separation between Himself and the Holy Spirit.

The Holy Spirit is identified with persons. Paul said, “It
seemed good to the Holy [Spirit], and to us” (Acts 15:28).
Try making sense by inserting wind or power in this verse!

The Holy Spirit is a Person; He is not just the essence
of God. You need to come into a personal relationship with
Him so that you might begin to experience His love and His
power working in your life as He guides you in your spiritual
walk.


\section*{The Power of the Spirit}

Have you ever felt that you should share with a person
his or her need to accept Jesus Christ, yet you didn’t have
the nerve to bring up the subject? Have you ever gone past
a college, observed the thousands of students, realized that
most of them are lost, and then wondered how they could
possibly be reached for Christ? Do you ever think of the billions
of people who have never received a true presentation
of the gospel, and then wonder how it might be accomplished?

To Peter, who denied his Lord on a one-to-one basis
with the young maid, and to the rest of the disciples (who
fled when the going got tough) the commission of Jesus to
go into all the world and preach the gospel to every creature
must have seemed a totally impractical as well as an
impossible command, and indeed it was. There is no way
that eleven insignificant men from Galilee could reach the
world for Jesus Christ. That is why Jesus told them to wait
in Jerusalem until they received the power of the Holy Spirit,
for it was by this power that they were to be witnesses to the
uttermost parts of the world.

Is this experience of the power of the Holy Spirit something
that God intended only for the early church? Do the
Scriptures indicate that the time would come when we did
not need to depend on the power of the Spirit, but through
our perfected knowledge of the Scriptures we could do
God’s work on our own? Is the church that was begun in the
Spirit now to be perfected in the flesh? What is the answer
to the church’s impotence? Why has the church failed to
stop the mad downward plunge of the corrupted world
around us?

Paul warns us in Hebrews 4 to fear that we do not come
short of receiving the promise of God to enter into His rest.
Is it not also appropriate for us to fear that, if God has given
us a promise of power in our personal lives and power in
the corporate body of the church, we might come short of
it?


\section*{The Promise of the Father}

In Acts 1 we read that the disciples were with Jesus
in Bethany, from where He would soon be departing from
them and ascending into heaven. The clouds would receive
Him out of their sight. He was giving to them their final
instructions, which were of the utmost importance. In Acts
1:4 Jesus told them not to depart from Jerusalem, but to wait
for the promise of the Father, which, He said, “Ye have heard
of me.” In Luke 24:49 Jesus said, “And, behold, I send the
promise of my Father upon you: but tarry ye in the city of
Jerusalem, until ye be endued with power from on high.”

In both places Jesus referred to the promise of the Father,
which is no doubt a reference to Joel 2:28–29, where God
promised, “And it shall come to pass afterward, that I
will pour out my spirit upon all flesh; and your sons and
your daughters shall prophesy, your old men shall dream
dreams, your young men shall see visions: And also upon
the servants and upon the handmaids in those days will I
pour out my spirit.” This is confirmed in the second chapter
of Acts, when the crowd that had assembled as a result of
the supernatural phenomena accompanying the outpouring
of the Holy Spirit was asking the question, “What does
this mean?” Peter in explanation replied, “This is that which
was spoken of by the prophet Joel,” and he quoted the
prophecy of Joel. The promise of God was that the day
would come when He would pour out His Spirit, not just
upon special individuals, but upon all flesh.


\section*{The Promise of the Savior}

Jesus had also promised the Spirit to His disciples in
John 14:16–17, where He said, “And I will pray the Father,
and he shall give you another Comforter, that he may abide
with you for ever; Even the Spirit of truth, whom the world
cannot receive, because it seeth him not neither knoweth
him; but ye know him; for he dwelleth with you, and
shall be in you.” When Jesus promised the Holy Spirit, He
referred to Him as “another Comforter.” The word translated
comforter comes from the Greek word parakletos, which
literally means “to come alongside to help.” This is the basic
ministry of the Holy Spirit to the believer. He is there to help
us. Up to this time Jesus had been alongside His disciples,
helping them. They had rightly come to depend upon His
help. He was the Master of every situation.

When the storm threatened to sink their little boat, Jesus
rebuked the winds and the waves, and there was a great
calm. When the tax collector was demanding unjust taxes,
Jesus told Peter to go down and catch a fish and take the
coin out of its mouth and pay the taxes. No matter what
situation arose, Jesus was always alongside to help.

Now He has told them that He is leaving them. He
won’t be with them as in times past. Their hearts must have
been troubled by His words, and they were afraid to face
the future without Him. So He promised that He would
not leave them without help, that He would ask the Father,
and He would send them another Comforter or Helper to
abide with them forever: the Spirit of truth. For our Christian
walk, we are completely dependent upon the help of
the Holy Spirit. It is impossible to do any worthwhile Christian
service apart from His help.


\section*{Waiting in Jerusalem}

Because the words “tarry in Jerusalem” are used in
Luke’s Gospel, many Pentecostals have established “tarrying
meetings” as the way by which the power of the Holy
Spirit is to be received in the believer’s life. It should be
noted that the command was to “tarry in Jerusalem,” so to
be entirely scriptural, the tarrying meetings should all be in
Jerusalem!

It is obvious that Jesus was not establishing a universal
method by which the Holy Spirit would be imparted to
believers in all ages. He was only encouraging them to wait
for just a few days in Jerusalem until He sent the Holy Spirit
as a gift to the church. Once the Holy Spirit was given on
the Day of Pentecost, it was never necessary to tarry for
Him again, and we do not find in the book of Acts any tarrying
meetings, nor are they advocated in the New Testament
as the method by which the gift of the Holy Spirit is to be
received.


\section*{Dynamic Power for You}

In Acts 1:8 Jesus promised His disciples that they would
receive power when the Holy Spirit had come upon them,
and that through this power they would bear witness of
Christ to the uttermost parts of the earth. The Greek word
translated power is “\emph{dunamis.}” Our English word \emph{dynamic}
comes directly from this word, and that describes what
the Holy Spirit is to be in us—the dynamic by which we
live and serve God. Without this dynamic the Christian life
is impossible and service is fruitless. What glorious new
dimensions the power of the Holy Spirit brings into the
believer’s life—the power to be and to do all that God
wants!

It is not God’s will that your life in Christ be dull and
drab, or that your service be a chore. God intends that your
walk with Him be full of joy. He wants you to have power
and victory in your life. If your life in Christ is not dynamic
and victorious, God has something more for you. The promise
of the gift of the Holy Spirit is “to you and to your children
and to all that are afar off, even as many as the Lord
our God shall call.”



