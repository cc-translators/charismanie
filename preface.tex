\addchap{Pr\'eface}
\cohead[\headerfont\textsc{Préface}]{}

\begin{preface}

\lettrine[ante=\Og]{Q}{uoi?} \Fg{}, me direz-vous, \Og encore un livre sur le sujet des dons du Saint-Esprit? \Fg{}\dots{}\\[1ex]
Ce à quoi je vous répondrai d'abord que l'original a été rédigé en anglais voilà plus de vingt ans, et ensuite que je reconnais bien volontiers avec vous que le sujet à déjà fait couler beaucoup d'encre! Trop peut-être! Les traitements du sujet ont, en effet, amené plus souvent de la \Og chaleur \Fg{} que de la lumière, avec une conséquence éminemment regrettable, celle d'attrister le Saint-Esprit en divisant le Corps de Christ qu'est Son \'Eglise\dots{}

Les positions sur le sujet sont généralement bien établies et ressemblent davantage à des retranchements où règnent malaise, méfiance, hostilité, voire mépris des autres points de vue plutôt qu'à des points de départ d'un dialogue fraternel et constructif\dots{}

Selon notre arrière plan ecclésial, dénominationel, nous avons tous adopté un point de vue plus ou moins catégorique sur ce sujet \Og clivant \Fg{}.

Ceux qui viennent d'un arrière-plan pentecôtiste ou charismatique, font souvent de la pratique régulière d'un ou plusieurs dons, le test incontournable de la foi chrétienne authentique, la condition indispensable pour savoir et montrer que l'on est \Og vraiment sauvé \Fg{}.

Ceux qui estiment que les dons ont cessé à la fin de la période apostolique ne veulent pas entendre parler de ces dons. Ces \Og cessationistes \Fg{} sont convaincus que ceux qui disent pratiquer ces dons aujourd'hui, sont au mieux des chétiens mal éclairés et au pire des charlatans mal intentionnés.

Le terme \Og cessationiste \Fg{} ne doit pas bien sûr être confondu avec le mot \Og sécessioniste \Fg{} qui évoque la triste guerre civile qui a ensanglanté les \'Etats-Unis dans les années 1860 et que l'on connaît sous le nom de \Og Guerre de Sécession \Fg{}.
 Malheureusement, le débat sur les dons de l'Esprit peut souvent mener à un combat fratricide!

Le grand mérite de l'auteur est, à mon avis, de ne pas adopter et défendre \emph{a priori} un parti pris,
 mais \Og d'examiner les \'Ecritures \Fg{} pour découvrir avec des yeux,
 des pensées et un c\oe{}ur neufs \Og ce que l'Esprit dit aux \'Eglises \Fg{}
 sur Lui-même et Ses dons.

Pourrais-je vous demander, cher lecteur, d'aborder la lecture de ce petit livre en vous efforçant de faire\dots{}
 table rase de tout ce que vous pensez savoir sur le Saint-Esprit et de vous placer sous l'autorité de la Parole
 et non sous celle de telle ou telle tradition ecclésiale\dots{} Vous risquez d'être surpris!

Ma Prière est que ce message puisse contribuer à libérer \Og la puissance du Saint-Esprit survenant sur l'\'Eglise \Fg{}
 pour l'annonce et l'avancée du message de l'\'Evangile en France et dans la Francophonie.
 Puissions-nous être remplis d'amour les uns pour les autres,
 pour montrer au monde que nous sommes les disciples de Jésus
 et qu'Il puisse être glorifié par une abondante moisson d'âmes\dots{}

\begin{quote}
\Og \`A ceci tous connaîtront que vous êtes mes disciples,
 si vous avez de l'amour les uns pour les autres! \Fg{}
\end{quote}

\begin{flushright}
Pierre Petrignani\\
Pasteur de l'église évangélique\\
\Og Calvary Chapel Nice \Fg
\end{flushright}


\end{preface}


